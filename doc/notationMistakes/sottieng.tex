\documentclass[12pt,twoside]{article}
\usepackage{a4}
\usepackage{musixtex}
\usepackage[english]{babel}
\usepackage{randtext}
\input coulhack
\raggedbottom
\setlength{\textwidth}{18cm} \setlength{\textheight}{26cm}
\setlength{\oddsidemargin}{0pt}\setlength{\evensidemargin}{0pt}
\setlength{\parindent}{0cm}\setlength{\parskip}{.5cm}
\setlength{\marginparwidth}{1in} \setlength{\marginparsep}{0pt}
\setlength{\hoffset}{-1cm}
\setlength{\voffset}{-2.5cm}
%\setlength{\headsep}{2cm}
\setlength{\itemsep}{0cm}

\newcommand{\milieu}[1]{\begin{center}{#1}\end{center}}
\newcommand{\pruneau}{$\bullet$~}
\newcommand{\subsec}{\vspace{-1cm}\subsection{~}\vspace{-.5cm}}
% \today should work the same in theory
\def\mytoday{\ifcase\month\or
  January\or February\or March\or April\or May\or June\or
  July\or August\or September\or October\or November\or December\fi
  \space\number\day, \number\year}

\def\notess{\vnotes2.4\elemskip}
\def\notesss{\vnotes1.7\elemskip}
\def\notesm{\vnotes1.5\elemskip}
\def\appog{\multnoteskip\tinyvalue\tinynotesize}
\newcommand{\zcn}[2]{\zcharnote{#1}{#2}}
\newcommand{\lcn}[2]{\lcharnote{#1}{#2}}
\newcommand{\ccn}[2]{\ccharnote{#1}{#2}}
\newcommand{\nxst}{\nextstaff}
\def\hhqsk{\off{.7\elemskip}}

\newcommand{\twoextr}[2]
%{\begin{music}\let\extractline\hbox
{\let\extractline\hbox
\hbox to \hsize{
\hfill
\startextract #1
\zendextract\hfill
\startextract #2
\zendextract
\hfill}

%\let\extractline\centerline\end{music}}
\let\extractline\centerline}

\accshift=0mm
\geometricskipscale
\nostartrule
\nobarmessages\nobarnumbers
\let\extractline\centerline

\begin{document}
\milieu{\ppfftwentynine Repository of music-notation mistakes}
\milieu{\medtype or}
\milieu{\ppfftwentyfour Essay on the true art of music engraving}
\milieu{\Huge Jean-Pierre Coulon}
%\centerline{\tt
%\kern3.5em{obs-nice}\kern-4.5em{@}\kern-3.5em{coulon}\kern4.5em{.fr}}

\centerline{\tt \randomize{coulon@obs-nice.fr}}


\milieu{\mytoday}
%\milieu{\today}

\milieu{\large intended for:}
\begin{itemize}
\item users of music-typesetting software packages,
\item developers of such packages,
\item traditional music-engravers,
\item sheet-music collectors,
\item those keen on  problems of \textit{semantics}, \textit{semiology},
\textit{philology} , etc. 
\end{itemize}

{\footnotesize NB: These examples are certainly musically worthless:
do not read them with your instrument :-)\\
To limit myself to the essential, and lacking sufficient expertise,
I do not deal with any of these neighboring, exciting topics:

\pruneau music theory, harmony, composition, etc.\\
\pruneau comparative test of various typesetting packages,\\
\pruneau how to interpret the quoted symbols according to epochs,\\
\pruneau copyright issues,\\
\pruneau percussion notation, and plucked-string instrument tablatures,\\
\pruneau very-early-music notation, and avant-garde music
notation.\\
\textit{ I apologize to readers of some countries for having adhered
to the U.S. terminology}

}


\section{General issues}
\subsec
When simultaneous notes are present on the same staff, two notations
are posssible: chord notation, or multiple-voice, a.k.a. \textit{polyphonic}
notation.

\twoextr
{\NOtes\qu c\zq c\qu e\en
\Notes\ibu0e2\zq c\qb0e\zq d\tqh0f\en}
{\NOtes\zql c\qp\zql c\qu e\en
\Notes\ibl0c2\ibu1e2\zqb0c\qb1e\tbl0\zqb0d\tqh1f\en}

Of course, if parts have distinct rythms, the polyphonic notation is
required.

\eject\milieu{\ppfftwenty from here:}
{\ppfftwentyfour This side: incorrect. \hfill This side: correct. }

\subsec
Do your best to place \textit{page-turns} at places
acceptable for the musician, otherwise he will either
require a ``page-turner'', or labor to arrange 
chunks of photocopies. Since modern musical scores are smaller 
than before, this demands more efforts from the music engraver.

{\footnotesize The actual print size, i.e. omitting margins,
of most scores from former epochs, almost matched the usual format
of most of modern scores including the margins.

}

An easy solution consists in using a small engraving size.
It is better to use a bigger size, at the price of more effort
to manage the spacing rationally.
These two excerpts require the same horizontal space, but that on the
right is easier to read:

\bigaccid
\begin{music}\let\extractline\hbox
\hbox to \hsize{
\hfill
\smallmusicsize
\setclef1{\treble}
\startextract%
\notess\ibbu0c1\qb0c\qb0{^c}\qb0d\tqh0{^d}\en
\notess\ibbu0f{-1}\qb0{g}\qb0{_g}\qb0f\tqh0e\en
\bar%
\notess\ibbu0c0\qb0{^cc}\qb0{=c}\tqh0c\en
\zendextract\hfill
\normalmusicsize
\startextract%
\notesss\ibbu0c1\qb0c\hqsk\qb0{^c}\qb0d\hqsk\tqh0{^d}\en
\notesss\ibbu0f{-1}\qb0{g}\hqsk\qb0{_g}\qb0f\tqh0e\en
\bar%
\notesss\ibbu0c0\qb0{^cc}\hqsk\qb0{=c}\tqh0c\en
\zendextract
\hfill}
\let\extractline\centerline\end{music}

Moreover, you also have to vary the distances between the staves
of systems, to avoid wasting some vertical space
(see my edition of Dussek's piano Sonata op.35-3).




\subsec
Choose horizontal spacings that increase according to note durations,
but not proportional to these durations. Personally I am happy with a
spacing ratio
of \hbox{$\sqrt{2}\approx 1.414$} to represent a duration-ratio of 2,
but this is not an absolute rule, especially if thirty-second or
sixty-fourth notes are present, because this would bring them 
too close to each other.
Do not change the spacing of a specific duration within a line without
a good reason.

\nostartrule
\setclef1\treble
\hsize82.1mm
\startpiece
\NOTEs\hu c\sk\en
\NOTEs\qu c\en
\NOtes\cu c\en
\notes\ibbu0c0\qb0c\en
\notes\qb0{cc}\tqh0c\en
\setemptybar
\endpiece

\hsize18cm

\setclef1\treble
\twoextr
{\notes\hu c\en
\NOtes\qu c\en
\NOTes\cu c\en
\notes\ibbu0c0\qb0c\en
\Notes\qb0{cc}\tqh0c\en}
{\NOTes\hu c\en
\NOtes\qu c\en
\Notes\cu c\en
\notes\ibbu0c0\qb0{cc}\en
\notes\qb0c\tqh0c\en}

{\footnotesize If a system has several staves, the part
with the \textbf{shortest} durations governs the overall spacing.
Lyrics, if any, may demand still wider spacings.

}

\subsec
Here is a good reason to modify tight note spacing: avoiding clashes:

\twoextr
{\notesss\ccu c\ibbl0k2\qb0{jklm}\tqb0{^n}\en}
{\notesss\ccu c\hqsk\ibbl0k2\qb0{jklm}\hqsk\tqb0{^n}\en}k

\subsec
If no other simultaneous part prevents it, note stems should also be
taken into account for the spacing:

\largemusicsize
\twoextr
{\notes\ibbu0d0\qb0{ded}\tqh0e\ibbl0k0\qb0{jkj}\tqb0k\en}
{\notes\ibbu0d0\qb0{ded}\tqh0e\hqsk\ibbl0k0\qb0{jkj}\tqb0k\en}


\subsec
Never displace the vertical alignment to accommodate accidentals:

\setstaffs1{2}\setclef1\bass\startrule\bigaccid
\twoextr{
\notesm\ibbu0I2\qb0{IJK}\tqh0L\nxst\ibbu0d2\qb0{de}\hhqsk\qb0{^f}\tqh0g\en
\bar%
\NOtes\ql M\nxst\hhqsk\qu{^h}\en}
{\notesm\ibbu0I2\qb0{IJ}\hhqsk\qb0K\tqh0L\nxst\ibbu0d2\qb0{de}\hhqsk\qb0{^f}\tqh0g\en
\bar%
\NOtes\hqsk\ql M\nxst\hqsk\qu{^h}\en}

\nostartrule



\subsec
Write an interval of a \textit{second} in a chord with the lower note
\textbf{left}. On the other hand, in polyphonic writing,
it should be put \textbf{right}.

\largemusicsize\setstaffs1{1}\setclef1{\treble}
\twoextr
{\NOtes\rq b\zq{ce}\qu g\en
\NOtes\rq b\zq{^ce}\qu g\en
\NOtes\rq c\rq e\rq g\qu b\en
\doublebar%
\NOtes\zql b\roffset{.9}{\qu c}\en
\notes\ibbu0c0\loff{\ibl1b{-1}\zqb1b}\qb0{cd}\en
\notes\tbl1\zqb1a\qb0c\tqh0d\en}
{\NOtes\rq c\zq{be}\qu g\en
\NOtes\sh c\rq c\zq{be}\qu g\en
\NOtes\rq c\zq{be}\qu g\en
\doublebar%
\NOtes\zqu c\roffset{.9}{\ql b}\en
\notes\ibbu0c0\roff{\ibl1b{-1}\zqb1b}\qb0c\hqsk\qb0d\en
\notes\tbl1\zqb1a\qb0c\tqh0d\en}

{\footnotesize
I omit cases of voice crossing. 
Do not be perturbed by an accidental associated with the
upper note. Do not perturb notes aside this second.

}


\subsec
In polyphonic writing, notes should be shifted just enough to
distinguish them, anyway less than the spacing for a second:

\Largemusicsize\generalmeter{\empty}\setstaffs1{1}\setclef1\treble
\twoextr
{\NOtes\zql c\roff{\zqu e}\qu g\en
\NOtes\zql c\zqu e\qu g\en
\NOtes\zql c\roffset{.2}{\zqu f}\qu g\en
\NOtes\zqlp c\roff{\zqup f}\qup h\en}
{\NOtes\zql c\loffset{.2}{\zqu g}\qu e\en
\NOtes\zql c\loffset{.2}{\zqu g}\qu e\en
\NOtes\zql c\roff{\zqu f}\qu g\en
\NOtes\loffset{.2}{\zqu h}\pt c\pt f\pt h\zql c\qu f\en}

{\footnotesize Note the case of dotted notes: the augmentation dots
must be aligned vertically.}

\subsec
Sometimes augmentations dots must be placed so as to avoid any
ambiguity:

\twoextr
{\NOtes\ibu0h0\roffset{.9}{\zql g\pt h}\qb0h\en
\notes\tbbu0\tqh0h\en
\NOtes\zqlp g\qu i\en
\Notes\qu j\cl f\en
\Notes\zql e\qu k\en}
{\Notesp\loffset{.3}{\ibu0h0\zqb0h}%
\loffset{.5}{\raise.15\Interligne\hbox{\pt h}}\roffset{.8}{\zql g}\sk\en
%\NOtes\ibu0h0\loffset{.2}{\pt h}\roffset{1.4}{\zql g}\qb0h\en
\notes\tbbu0\tqh0h\en
\NOtes\pt f\zql g\qu i\en
\Notes\qu j\cl f\en
\Notes\zql e\qu k\en}




\subsec
In a two-part polyphonic notation, upper-part note stems should be
upwards, even if the other part has a rest:

\normalmusicsize\setclef1{\bass}
\twoextr
{\NOTesp\zw H\hup a\en
\Notes\qu N\hroff{\liftpause{-2}}\ql{MKLM}\en}
{\NOTesp\zw H\hup a\en
\Notes\qu N\liftpause{-2}\qu{MKLM}\en}

{\footnotesize(same for the lower part.)}


\subsec
Some linkings of notes or rests may be correct
according to music theory,
but impede sight reading:

\normalmusicsize
\generalmeter{\meterfrac44}\setclef1\treble
\twoextr
{\notes\cu c\en
\bar%
\notes\cu d\en
\Notes\qp\en
\notes\ibu0f2\qb0{efgh}\tqh0i\en
\bar
\znotes\en}
%\notes\cl j\en}
{\notes\cu c\en
\bar%
\notes\cu d\ds\ds\cu e\en
\notes\ibu0g2\qb0{fgh}\tqh0i\en
\bar
\znotes\en}

In other words, rests should not act as syncopated notes. In a ternary
beat, separate a rest that affects the second and third time values.
Beams do not demand so much care.

{\footnotesize Some will argue that in early music, flag/beam notation
expresses some degree of articulation. But the rule above almost always
coincides with some \textit{reasonable} articulation.

}


\section{Beaming}
\subsec
Beams should have an appropriate slope. In the past, one would avoid 
too weak a slope, because the printing ink would maliciously attempt
to fill the tiny angle between these beams and the staff lines.
For a scale or an arpeggio, this slope cannot be steeper than that
formed by the notes, nor horizontal. A compromise must be found.
Here is an example, with its solution, by two distinguished publishers:

\generalmeter{\empty}
\setstaffs1{1}\setclef1{\treble}
\twoextr{
\notes\ibbu0a6\qb0{cdefg}\tqh0h\en
\notes\ibbu0g0\qb0{cdefg}\tqh0h\en}
{\notes\zcn q{\smalltype Peters end 19$^{th}$C.}\ibbu0a4\qb0{cdefg}\tqh0h\en
\notes\zcn q{\smalltype Henle end 20$^{th}$C.}\ibbu0e1\qb0{cdefg}\tqh0h\en
\notes\zcn q{recommended}\ibbu0d2\qb0{cdefg}\tqh0h\en}

\footnotesize{If you want to imitate some French publishers :}

\startextract%
\notes\ibu0d2\ibbu1d2\qb1c\qb0{defg}\tbu0\tqh1h\en
\notesss\ibu0i0\ibbbu1i0\qb1j\qb0{eg}\tbu0\tqh1j\en
\zendextract

\subsec
Beam placement should be adapted to the context. This placement varies
depending on whether this group of notes is isolated or in a series:

\startextract
\NOTes\hu c\hu d\en
\bar
\NOtes\qu f\en
\notes\ibbu0f4\qb0{ceg}\tqh0j\en
\NOtes\qu{cd}\en
\bar%
\NOtes\qp\en
\NOTes\hu f\en
\zendextract

\twoextr
{\notes\ibbu0f4\qb0{ceg}\tqh0j\en
\notes\ibbu0f4\qb0{ceg}\tqh0j\en
\notes\ibbu0f4\qb0{ceg}\tqh0j\en
\notes\ibbu0f4\qb0{ceg}\tqh0j\en}
{\notes\ibbu0h1\qb0{ceg}\tqh0j\en
\notes\ibbu0h1\qb0{ceg}\tqh0j\en
\notes\ibbu0h1\qb0{ceg}\tqh0j\en
\notes\ibbu0h1\qb0{ceg}\tqh0j\en}

\subsec
Avoid Z-like beams, as found in former editions:

\normalmusicsize
\setclef1\bass
\twoextr
{\Notes\ibu0E{-4}\qb0G\zq{Nc}\qb0e\en
\Notes\zq{Nc}\qb0e\zq{Nc}\tqb0e\en}
{\Notes\ibl0H2\qb0G\en
\Notes\zq{Nc}\qb0e\zq{Nc}\qb0e\zq{Nc}\tqb0e\en
\doublebar%
\Notes\cu G\en
\Notes\ibl0a0\zq{Nc}\qb0e\zq{Nc}\qb0e\zq{Nc}\tqb0e\en}


\subsec
Beams should never cross ledger lines:

\setclef1{\treble}
\twoextr
{\notes\ibbl0q3\qb0{qrstuvw}\tqb0x\en}
{\notes\ibbl0o1\qb0{qrstuvw}\tqb0x\en}

\subsec
Do not affect beam placement to place rests at their usual height.
Rather move the rests:

\twoextr
{\Notes\ibl0k1\qb0n\ds\tqb0o\en
\Notes\ibl0l0\qb0o\ds\tqb0o\en}
{\Notes\ibl0n1\qb0n\raise2\Interligne\ds\tqb0o\en
\Notes\ibl0o0\qb0o\raise2\Interligne\ds\tqb0o\en}


\section{Ties and slurs}
\subsec
When \textbf{slurred} notes are intended, the slur takes off and lands
above or below the \textbf{centers} of note heads. But for \textbf{tied}
notes,
the same sign takes off and lands vertically aligned with the boundary
of the note head, and not higher than this head.
One breaks from the first rule to
avoid colliding the note stems.


\largemusicsize
\twoextr
{\NOtes\isslurd0c\isluru1c\qu c\tsslur1 d\qu d\tsslur0e\qu e\islurd0f\qu f\en
\bar%
\NOtesp\tslur0f\roff{\itied0f}\qup f\en
\NOtes\ttie0\qu f\en}
{\NOtes\islurd0c\hroff{\isluru1c}\qu c\tslur1d\qu d\tslur0e\qu e\itied0f\qu f\en
\bar%
\NOtesp\ttie0\itied0f\qup f\en
\NOtes\ttie0\qu f\en}

{\footnotesize Note that the tie becomes ``quieter''.}


\subsec
Meanwhile avoid ``acrobatics'' to strictly abide these rules:

\largemusicsize
\twoextr
{\NOtes\islurd0c\qu c\tsslur0j\ql j\en
\NOtes\isluru0j\ql j\ql o\midslur5\tslur0k\ql k\en}
{\NOtes\isslurd0c\qu c\tsslur0j\ql j\en
\NOtes\isluru0l\ql j\ql o\tslur0m\ql k\en}


\subsec
During some epochs a single tie was supposed to refer to \textbf{all}
notes of a particular chord. Particular cases have become so frequent,
that it is better to note all ties explicitly.

\largemusicsize
\twoextr
{\NOtes\itieu0g\zq{ce}\qu g\en
\bar%
\NOtes\ttie0\zq{ce}\qu g\en}
{\NOtes\itied0c\itenl1e\itieu2g\zq{ce}\qu g\en
\bar%
\NOtes\ttie0\tten1\ttie2\zq{ce}\qu g\en}

\subsec
Associations of ties and slurs with dots and ornaments:

\twoextr
{\NOtes\isluru0j\upz k\ql j\tslur0k\upz l\ql k\en
\NOtes\isluru0n\ql n\shake o\tslur0o\ql m\en}
{\NOtes\isluru0l\upz j\ql j\tslur0m\upz k\ql k\en
\NOtes\isluru0n\ql n\shake q\tslur0m\ql m\en}

\newpage
\subsec
Associations of ties and slurs:

\twoextr
{\NOtes\itied0d\qu d\en
\bar
\NOtes\ttie0\islurd1d\qu d\tslur1c\qu c\en
\NOtes\islurd1e\qu e\tslur1f\itied0f\qu f\en
\bar
\NOtes\ttie0\qu f\en}
{\NOtes\islurd1d\itied0d\qu d\en
\bar
\NOtes\ttie0\qu d\tslur1c\qu c\en
\NOtes\islurd1e\qu e\itied0f\qu f\en
\bar
\NOtes\tslur1f\ttie0\qu f\en}

{\footnotesize Bowed-instruments players will find this obvious.}



\subsec
The dotted slur is the best way to emphasize an \textit{editorial slur}:

\setclef1{\treble}
\twoextr
{\NOtes\islurd0c\qu c\tslur0d\qu d\en
\NOtes\zcn a{\smalltype\bf (}\islurd0d\qu e%
\zcn b{\smalltype\bf ~)}\tslur0e\qu f\en
\NOtes\islurd0g\zqu g\hsk\zcn c{\bf /}\hsk\tslur0h\qu h\en}
{\NOtes\islurd0c\qu c\tslur0d\qu d\en
\NOtes\dotted\islurd0e\qu e\tslur0f\qu f\en
\NOtes\dotted\islurd0g\qu g\tslur0h\qu h\en}


\section{Accidentals}
\subsec
When a note with an accidental expands over several measure,
do not repeat the accidental on the next measures, except if there is
a line-break or a page-break:

\twoextr
{\NOTEs\itied0b\itieu1j\zwh{^c}\wh{_i}\en
\bar%
\NOTes\ttie0\ttie1\zwh{^c}\hu{_i}\hu i\en}
{\NOTEs\itied0c\itieu1i\zwh{^c}\wh{_i}\en
\bar%
\NOTes\ttie0\ttie1\zwh c\hu i\hu{_i}\en}

\subsec

When an accidental affects \textit{small notes}, and is desired
to also affect subsequent notes, it must be repeated.

\twoextr
{\notes\appog\ibbu0c2\qb0{^c}\tqh0d\en
\NOTes\wh c\en}
{\notes\appog\ibbu0c2\qb0{^c}\tqh0d\en
\NOTes\qsk\wh{^c}\en}

\subsec
When a note with a single sharp comes after the same note with a
double sharp, you no longer put a natural sign before the single
sharp, unless you want to mimic 19th engraving style. Same for flats:

\twoextr
{\NOtes\qu{>c}\lna c\qu{^c}\qu{<d}\lna d\qu{_d}\en}
{\NOtes\qu{>c}\qu{^c}\qu{<d}\qu{_d}\en}



\subsec
In polyphonic writing, accidentals of either part are not
supposed to affect other parts. If such accidentals are to
affect other parts, they should be written explicitly:


\normalmusicsize
\setstaffs1{1}\setclef1\treble
\twoextr
{\Notes\zql d\qu f\en
\bar%
\NOtes\zql c\qu{_e}\zql d\qu f\zql e\qu g\zql f\qu h\en
\bar\
\znotes\en}
{\Notes\zql d\qu f\en
\bar%
\NOtes\zql c\qu{_e}\zql d\qu f\zql{_e}\qu g\zql f\qu h\en
\bar
\znotes\en}

{\footnotesize Remember, around 1600 a Fugue for organ or harpsichord
would have been written on \textbf{four} staves. Then accidentals where
not dependent between staves.

}

\newpage
\subsec
In the old days, an accidental would also affect corresponding notes
at other octaves:

\startextract
\Notes\ibu0e4\qb0{^ceg}\tqh0j\en
\NOtes\zq{_b}\qu i\en
\zendextract



Nowadays, such accidentals should be explicitly written:

\twoextr
{\Notes\ibu0e4\qb0{^ceg}\uppersh c\tqh0j\en
\NOtes\zq{_b}\upperfl i\qu i\en}
{\Notes\ibu0e4\qb0{^ceg}\tqh0{^j}\en
\NOtes\zq{_b}\qu{_i}\en}


{\footnotesize Note that some software packages, able to convert
between a MIDI file and a score, have kept this former convention. Then
a MIDI file may contradict its source score for some accidentals.

}

\subsec

When changing the key signature to decrease the number
of accidentals, the natural signs should be written \textbf{before} the
remaining accidentals:

\generalsignature5
\twoextr{%
\notes\qu{cde}\en
\doublebar
\notes\sh m\qsk\sh j\qsk\na n\qsk\na k\qsk\na h\qsk\en
\notes\qu{fg}\en}
{\notes\qu{cde}\en
\generalsignature{2}\Changecontext
\notes\qu{fg}\en}
\generalsignature0

\subsec
When such a change comes at a line- or page break, it must be shown
\textbf{before} this break:

\generalsignature2
\twoextr
{\Notes\qu{cdef}\en
\afterruleskip=-4pt\bar}
{\afterruleskip=8pt
\Notes\qu{cdef}\en
\generalsignature{-2}\Changecontext}

\generalsignature0
\twoextr
{\notes\na m\hsk\na j\hsk\en
\doublebar%
\notes\fl i\hsk\fl l\hsk\en
\Notes\qu{ghij}\en}
{\notes\fl i\hsk\fl l\sk\en
\Notes\qu{ghij}\en}


\subsec
When there is a need to shift some accidentals to avoid collisions,
the upper is left at its normal position, and the lower is shifted
to the left, either in chord- or in polyphonic-writing:

\twoextr
{\NOtes\qsk\lsh f\zq{^d}\qu f\en
\bar%
\NOtes\qsk\lsh f\zql{^d}\qu f\en}
{\NOtes\qsk\lsh d\zq d\qu{^f}\en
\bar%
\NOtes\qsk\lsh d\zql d\qu{^f}\en}


\subsec
Accidentals other than well defined ones, get classified in three sets:

\begin{itemize}
\item editorial accidentals \textbf(e),
\item cautionary accidentals \textbf(y),
\item ``courtesy'' accidentals \textbf(c).
\end{itemize}

Put an editorial accidental when you think the source is wrong.
Note it with a small-size accidental above or below the note.
If this note is within a chord, place it left of the note.

{\footnotesize \textit{Thorough bass} also uses such small accidentals above
notes. But the misunderstanding is unlikely most of the time.

}

Put a cautionary accidental when music-theory rules demand its effect,
but you fear the musician misses it, e.g. at the end of a ``crowded'' measure.
It must be parenthesized.

A courtesy accidental is a theoretically redundant accidental, which
confirms the cancelling effect of a barline on previous accidentals. 
The usage is to write it normally.

\twoextr
{\NOtes\qu{^c}\qu d\cfl e\qu e\qu{^c}\en
\bar%
\NOtes\cna c\qu c\en}
{\NOtes\qu{^c}\qu d\ccn q{(e)}\upperfl L\qu e\ccn q{(y)}\csh c\qu c\en
\bar%
\NOtes\ccn q{(c)}\qu{=c}\en}

{\footnotesize It is better to notate \textbf{all} accidentals in
some modern, complex works.

}

\subsec In ancient times some copists thought that when the
first note of a measure is the same as that, with an accidental, of the
previous measure, the accidental was implicit:

\setstaffs12\setclef1{\bass\treble}
\startrule
\interstaff{11}

\startextract
\NOtesp\zhl K\hu a|\zh{^f}\zhl d\qup j\en
\notesp|\triolet r\ibbu3i{-1}\qb3j\en
\notes|\qb3{_i}\tqh3h\en
\NOtesp\pt F\zql G\qu N|\zqp g\zqlp d\qup i\en
\notes\ibbl0H1\triolet C\zcu N\qb0G|\zq g\zcl d\cu i\en
\notesp\qb0H\en
\notes\tqb0{_I}\en
\bar
\NOtes\zhl J\qu N|\pt b\zql c\loffset{.25}{\zql e}\qu i\en
\Notes\qu M|\zhl f\ibu3h{-1}\qbp3h\en
\notes|\cl c\en
\notes|\tbbu3\tqh3g\en
\NOtes\zhl C\qu N|\pt b\zql c\qu g\en
\zendextract

The correction is then obvious, and does not require an editorial
accidental:

\smallmusicsize
\twoextr
{\NOtesp\zhl K\hu a\nextstaff\zh{^f}\zhl d\qup j\en
\notesp\nextstaff\triolet r\ibbu3i{-1}\qb3j\en
\notes\nextstaff\qb3{_i}\tqh3h\en
\NOtesp\pt F\zql G\qu N\nextstaff\zqp g\zqlp d\qup i\en
\notes\ibbl0H1\triolet C\zcu N\qb0G\nextstaff\zq g\zcl d\cu i\en
\notesp\qb0H\en
\notes\tqb0{_I}\en
\bar%
\NOtes\zhl J\qu N\nextstaff\pt b\zql c\loffset{.25}{\zql e}\upperfl n\qu i\en
\Notes\qu M\nextstaff\zhl f\ibu3h{-1}\qbp3h\en
\notes\nextstaff\cl c\en
\notes\nextstaff\tbbu3\tqh3g\en
\NOtes\zhl C\qu N\nextstaff\pt b\zql c\qu g\en}
{\NOtesp\zhl K\hu a\nextstaff\zh{^f}\zhl d\qup j\en
\notesp\nextstaff\triolet r\ibbu3i{-1}\qb3j\en
\notes\nextstaff\qb3{_i}\tqh3h\en
\NOtesp\pt F\zql G\qu N\nextstaff\zqp g\zqlp d\qup i\en
\notes\ibbl0H1\triolet C\zcu N\qb0G\nextstaff\zq g\zcl d\cu i\en
\notesp\qb0H\en
\notes\tqb0{_I}\en
\bar%
\nspace%
\NOtes\zhl J\qu N\nextstaff\pt b\zql c\loffset{.25}{\zql e}\qu{_i}\en
\Notes\qu M\nextstaff\zhl f\ibu3h{-1}\qbp3h\en
\notes\nextstaff\cl c\en
\notes\nextstaff\tbbu3\tqh3g\en
\NOtes\zhl C\qu N\nextstaff\pt b\zql c\qu g\en}

{\footnotesize (a fortiori a cautionary accidental. Your musicologist's
abilities are recognized at this stage!)}


\subsec
Some accidentals in ancient editions may seem redundant according to
our modern rules, like the C sharp at the end of this bar:

\nostartrule
\setstaffs11
\largemusicsize
\setclef1{\bass}
\startextract
\bar%
\NOtes\zw N\zq b\qu d\lsh a\zq a\qu{^c}\en
\NOtes\zq b\qu d\zq{^c}\qu e\en
\bar%
\zendextract


Therefore the following theory must be banned:
\textit{if this accidental is present in my source,
there must be a good reason for this,
but a wrong accidental has been written here. Indeed, if the composer
had wanted a C-sharp at the end of this bar, he wouldn't have
written any sign there. So let me correct this mistake!}

\twoextr{
\bar%
\NOtes\zw N\zq b\qu d\lsh a\zq a\qu{^c}\en
\NOtes\zq b\qu d\zq{=c}\qu e\en
\bar%
}
{\bar%
\NOtes\zw N\zq b\qu d\lsh a\zq a\qu{^c}\en
\NOtes\zq b\qu d\zq c\qu e\en
\bar%
}

{\footnotesize If these thirds of fourth-notes had been written
as polyphonic notes, the explanation would be obvious.
I could also quote similar misunderstandings about accidentals
an octave away, as seen before.

}

\section{Measure numbers}

\subsec
Initial rests, common to all parts, are never notated, especially if
there is a repeat from the first note:

\normalmusicsize\nostartrule\setclef1\treble
\generalmeter{\meterfrac34}

\twoextr
{\NOtes\hpause\en
\Notes\qu c\en
\bar%
\Notes\qu{def}\en
\bar%
\Notes\qu{hi}\en
\rightrepeat
\Notes\ql j\en
\bar
\Notes\ql k\en}
{\Notes\qu c\en
\bar%
\Notes\qu{def}\en
\bar%
\Notes\qu{hi}\en
\rightrepeat
\Notes\ql j\en
\bar
\Notes\ql k\en}



\subsec
The number of the very first measure of a piece is never written.
If this measure is an upbeat, this number is \textbf{zero}.

\normalmusicsize\nostartrule\setclef1\treble
\generalmeter{\meterfrac34}

\twoextr
{\Notes\loffset{8}{\zcharnote o{\eightbf 1}}\qu c\en
\bar%
\Notes\loffset{2}{\zcharnote o{\eightbf 2}}\qu{def}\en
\bar%
\Notes\loffset{2}{\zcharnote o{\eightbf 3}}\qu{hi}\ql j\en}
{\Notes\qu c\en
\bar%
\Notes\loffset{2}{\zcharnote o{\eightbf 1}}\qu{def}\en
\bar%
\Notes\loffset{2}{\zcharnote o{\eightbf 2}}\qu{hi}\ql j\en}

{\footnotesize Measure numbers at system left-tops are enough in practice.

}

\subsec
If there is a double bar within a measure it does not act on measure
numbering.

\twoextr
{\Notes\qu{cde}\en
\bar%
\Notes\loffset{2}{\zcharnote o{\eightbf 2}}\qu{de}\en
\leftrightrepeat%
\Notes\qu f\en
\bar%
\Notes\loffset{2}{\zcharnote o{\eightbf 4}}\qu{hi}\ql j\en
\bar}
{\Notes\qu{cde}\en
\bar%
\Notes\loffset{2}{\zcharnote o{\eightbf 2}}\qu{de}\en
\leftrightrepeat%
\Notes\qu f\en
\bar%
\Notes\loffset{2}{\zcharnote o{\eightbf 3}}\qu{hi}\ql j\en
\bar}



\subsec

When  \textit{first ending-second ending} measures are present, measure
numbers of only the \textit{first ending} measure act on the numbering.
If necessary, corresponding measure numbers have the subscripts
\textit{a} and \textit{b}.

\generalmeter{\null}
\twoextr
{\Notes\loffset{5}{\zcharnote o{\eightbf 30}}\hu c\en\bar
\Notes\loffset{2}{\zcharnote o{\eightbf 31}}\hu d\en
\Setvolta1\bar
\Notes\hu e\en
\bar
\Notes\loffset{2}{\zcharnote o{\eightbf 33}}\hu f\en
\setvolta2\endvoltabox\rightrepeat
\Notes\hu g\en
\bar%
\Notes\loffset{2}{\zcharnote o{\eightbf 35}}\hu h\en
\bar%
\Notes\hu i\en
\bar%
\Notes\loffset{2}{\zcharnote o{\eightbf 37}}\hl j\en}
{\Notes\loffset{5}{\zcharnote o{\eightbf 30}}\hu c\en
\bar
\Notes\loffset{2}{\zcharnote o{\eightbf 31}}\hu d\en
\Setvolta1\bar%
\Notes\hu e\en
\bar%
\Notes\loffset{2}{\zcharnote o{\eightbf 33a}}\hu f\en
\setvolta2\endvoltabox\rightrepeat
\Notes\hu g\en
\bar%
\Notes\loffset{2}{\zcharnote o{\eightbf 33b}}\hu h\en
\bar%
\Notes\hu i\en
\bar%
\Notes\loffset{2}{\zcharnote o{\eightbf 35}}\hl j\en}

{\footnotesize
Indeed many fast movements have repeats bars amid a measure (like the
Allegretto of the \textit{Moonloght sonata}). This repeat could as well had
been noted under the \textit{first ending-second ending} form for this
measure. Hence both notations must yield the same measure numbering.


}


\section{Miscellaneous}

\subsec
No barline should be written at the beginning of an instrument part
involving a single staff, unlike a conductor score.

\setclef1\alto
\startrule\largemusicsize
\begin{music}\let\extractline\hbox
\hbox to \hsize{
\hfill
\startextract
\NOtes\qu{abc}\en
\zendextract\hfill\nostartrule
\startextract 
\NOtes\qu{abc}\en
\zendextract
\hfill}

\let\extractline\centerline\end{music}


\subsec

In ancient times, some staccato notes had a small wedge instead of
the modern dot. If \textbf{both} symbols show up in a piece, they also
represent two degrees of staccato. This notation must then be faithfully
reproduced. On the other hand, if only wedges exist, they must be
replaced with dots.:

\largemusicsize\setstaffs1{1}\setclef1{\treble}\nostartrule
\twoextr{
\Notes\ibu0d1\lppz c\qb0c\lppz d\qb0d\lppz e\qb0e\lppz f\tqh0f\en}
{\Notes\ibu0d1\lpz c\qb0c\lpz d\qb0d\lpz e\qb0e\lpz f\tqh0f\en}

\subsec
Such signs are always located above the note heads, rather than above the
note stems, even if polyphony requires a stem-side location:

\twoextr
{\NOtes\zhl c\hroff{\upz l}\qu e\hroff{\uppz k}\qu f\en
\Notes\zql b\ibu0g0\hroff{\upz m}\qb0g\hroff{\upz m}\tqh0g\en}
{\NOtes\zhl c\upz l\qu e\uppz k\qu f\en
\Notes\zql b\ibu0g0\upz m\qb0g\upz m\tqh0g\en}



\subsec
Appoggiature, or \textit{small notes}, are written \textbf{after}
the bar, regardless of their rythmic interpretation:


\twoextr
{\NOtes\qu N\en
\notes\multnoteskip\tinyvalue\tinynotesize\ibbu0e4\qb0c\tqh0e\en
\bar
\NOtes\hu g\en}
{\NOtes\qu N\en
\bar
\notes\multnoteskip\tinyvalue\tinynotesize\ibbu0e4\qb0c\tqh0e\en
\NOtes\hu g\en}


\subsec
The tempo indication should be written in roman characters, and other marks
should in italic:

\setclef1\treble
\twoextr
{\NOtes\zcn o{\medtype\it allegro}\qu{de}\zcn N{Cres}\qu{fg}\en}
{\NOtes\zcn o{\medtype Allegro}\qu{de}\zcn N{\medtype\it cresc.}\qu{fg}\en}

{\footnotesize Also note the capitalization, and the abbreviation dot.}


\subsec
In the old days one would write \textbf{loco} (from Latin \textit{at
the place})
to emphasize the end
of an octave sign (\textit{8}\raisebox{2mm}{- - - - -}).
Now such signs have become common,
and this \textbf{loco} is well redundant.

\newpage

\section{Some moral advice if you make engrave a score}

\subsec
If you elect to using a specific notation practice, keep up with it 
along the entire piece, instead of toggling between two practices
according to your whim, even if we have seen that this first
practice is questionable.

\subsec
Pitch errors of a third are much more frequent than those of
a second, for obvious visual-perception reasons. Be tolerant
with the authors of your source.

\subsec
What to do when you think you've done with your edition? Of course,
you rush to your instrument, and you play the piece with this edition.

But since you are familiar with the piece, you will very likely
miss quite a few misprints, overwhelmed by your enthusiasm.
Quit your instrument, put your edition and its source on a desk,
and take a pencil. Compare measures one at a time, and if there are
multiple staves, one staff at a time. End up with the miscellaneous
signs (tempo, interpretation, piano pedalling, etc.) common to these
several staves.

But this is not enough yet. Indeed a good musician like you has the ability
to correct some misprints (like accidentals) when playing, without noticing.
The previous test will not detect such misprints, because they are the
same in both versions.

If your typesetting software is able to convert your piece into
MIDI format, the obtained file will be helpful. Otherwise you will
have to submit your edition to a musician less experienced than you,
who would not yet be prone to such automatic corrections. 


\subsec
\textbf{Remember: Noting music correctly is as difficult as playing
an instrument correctly.}

\end{document}

