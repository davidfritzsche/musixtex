%
%   This file is part of MusiXTeX
%
%   MusiXTeX is free software; you can redistribute it and/or modify
%   it under the terms of the GNU General Public License as published by
%   the Free Software Foundation; either version 2, or (at your option)
%   any later version.
%
%   MusiXTeX is distributed in the hope that it will be useful,
%   but WITHOUT ANY WARRANTY; without even the implied warranty of
%   MERCHANTABILITY or FITNESS FOR A PARTICULAR PURPOSE.  See the
%   GNU General Public License for more details.
%
%   You should have received a copy of the GNU General Public License
%   along with MusiXTeX; see the file COPYING.  If not, write to
%   the Free Software Foundation, Inc., 59 Temple Place - Suite 330,
%   Boston, MA 02111-1307, USA.
%
%
%   musixdoc.tex : the document
%
\documentclass[twoside,11pt,a4paper]{report}
\usepackage{multicol}
\usepackage{musixdoc}
\usepackage{etex}
\usepackage[dvips,bookmarks=true,bookmarksnumbered=true,bookmarksopen=false,colorlinks=true,pdfborder={0 0 0},pdfstartview={FitH},pdfmenubar=false]{hyperref}
\usepackage{url}
\usepackage{indentfirst}
\usepackage[ansinew]{inputenc}
\makeatletter
\def\notesintext#1{% no staff lines, no clefs
  {\let\extractline\relax
   \setlines10\smallmusicsize \nobarnumbers \nostartrule
   \staffbotmarg0pt \setclefsymbol1\empty \global\clef@skip0pt
   \startextract\addspace{-\afterruleskip}#1\zendextract}}
\makeatother
\def\musicintext#1#2{% normal
  {\let\extractline\relax
   \smallmusicsize \nobarnumbers
   \staffbotmarg0pt \setclef1{#1}
   \startextract\addspace{-\afterruleskip}#2\endextract}}
% macro adapted to insert music without clef:
\def\musicintextnoclef#1{% no clef small
  {\let\extractline\relax
   \smallmusicsize \nobarnumbers
   \staffbotmarg0pt \setclefsymbol1\empty
   \startextract\addspace{-\afterruleskip}#1\endextract}}
\def\musicintextnoclefn#1{%  no clef normal
  {\let\extractline\relax
   \nobarnumbers
   \staffbotmarg0pt \setclefsymbol1\empty
   \startextract\addspace{-\afterruleskip}#1\endextract}}

\def\musictex{Music\TeX{}}
\def\onen{{\tt\char123}$n${\tt\char125}}
\def\pitchp{{\tt\char123}$p${\tt\char125}}
\def\itbrace#1{{\tt\char123}$#1${\tt\char125}}
\def\nochange{(\ital{NOT to be changed})}

\ifx\setendvolta\undefined\def\setendvolta{\endvolta}\fi
\ifx\setendvoltabox\undefined\def\setendvoltabox{\endvoltabox}\fi

\startmuflex\makeindex

\begin{document}

%\changebarsep 10 pt
\title{\Huge\bf\musixtex\raise1.5ex\hbox{\large\copyright}\\[\bigskipamount]
\LARGE\bf Using \TeX{} to write polyphonic\\or
instrumental music\\\Large\sl Version 1.15~-- January, 2011}
\author{\Large\rm Daniel \sc Taupin\\\large\sl
 Laboratoire de Physique des Solides\\\normalsize\sl
 (associ\'e au CNRS)\\\normalsize\sl
% b\^atiment 510, Centre Universitaire, F-91405 ORSAY Cedex\\ E-mail : {\tt
%taupin@lps.u-psud.fr}\\\medskip
 b\^atiment 510, Centre Universitaire, F-91405 ORSAY Cedex\\\medskip
 \\\Large\rm Ross \sc Mitchell\\\large\sl
 CSIRO Division of Atmospheric Research,\\\normalsize\sl
 Private Bag No.1, Mordialloc, Victoria 3195,\\ Australia \\\medskip
%   \\\Large\rm Andreas \sc Egler\ddag\\\large\sl
   \\\Large\rm Andreas \sc Egler\\\large\sl
   (Ruhr--Uni--Bochum)\\ Ursulastr. 32\\ D-44793 Bochum}
\date{}
%\let\endtitleORI\endtitlepage
% \def\endtitlepage{\vfill\noindent $^\ddag$ {\it For personal reasons, Andreas
%Egler decided to retire from authorship of this work. Nevertherless, he has
%done an important work about that, and I decided to keep his name on this
%first page. {\sc D. Taupin}}\endtitleORI}
 \maketitle
\clearpage
 %\check

\pagenumbering{roman}\setcounter{page}{2}

%\thispagestyle{empty}

\null
\vfill

\begin{flushright}\it
%.. All the possible is done,\\
%the impossible is being done,\\
%for miracles I ask for some delay.\\[\smallskipamount]
%{\sc N.N.}\\[\bigskipamount]
%...~time after time~...\\[\smallskipamount]
%{\sc Cindy Lauper}\\[\bigskipamount]
 If you are not familiar with \TeX{} at all\\
I would recommend to find another software\\
package to do musical typesetting.\\
Setting up \TeX{} and \musixtex\\
on your machine and mastering it\\
is an awesome job which gobbles up\\
a lot of your time and disk space.\\[\medskipamount]
But, once you master it...\\[\smallskipamount]
Hans {\sc Kuykens}
\end{flushright}

\begin{center}
\musixtex{} may be freely copied, duplicated and used in conformance to
the GNU General Public License (Version 2, 1991, see included file {\tt
copying})\footnote{Thanks to Free Software Foundation for advising us. See
\href{http://www.gnu.org}{\underline{\tt http://www.gnu.org}}}.

You may take it or parts of it to include in other packages, but no packages
called \musixtex{} without specific suffix may be distributed under the name
\musixtex{} if different from the original distribution (except obvious bug
corrections).

 Adaptations for specific implementations (e.g. fonts) should be provided as
separate additional \TeX\ or \LaTeX\ files which override original definitions.
 \end{center}

\clearpage

\chapter*{Preface to Version T.113}
%\thispagestyle{empty}

The main author of \musixtex{}, Daniel Taupin, died all too early in a 2003
climbing accident. The \musixtex\ community was shocked by this tragic and unexpected
event. You may read some tributes to Daniel Taupin which are archived at the
\href{http://icking-music-archive.org}{\underline{Werner Icking Music Archive}}.

Now, almost two years after his death, we would like to help keep his excellent
work alive and current by assembling a new release. This new version corrects various
minor bugs without adding any new functionality. At the same time it incorporates directly
into the distribution some additional packages:
\begin{itemize}\setlength{\itemsep}{0ex}
\item Type 1 (postscript) versions of all the \musixtex{} fonts, created by
Takanori Uchiyama;
\item Postscript Slur Package K (Ver 0.92, 12~May~02) by Stanislav Kneifl;
\item musixlyr (Ver 2.1c, 12 June 03) a \musixtex{} extension package for lyrics
handling, by Rainer Dunker.
\end{itemize}
We would like to thank the authors of these packages for generously agreeing to allow their
contributions to be included in the main \musixtex{} package.

We have left the remainder of this manual largely as Daniel Taupin left it,
except we have updated various references and provided dynamic links to
archived versions where possible.

This documentation is rather technical and is probably not the best way to
begin typesetting music. If you are a beginner, you should visit the
software section of the
\href{http://icking-music-archive.org/software/indexmt6.html}
{\underline{Werner Icking Music Archive}}.
In particular, we recommend
\href{http://icking-music-archive.org/software/pmx/pmxccn.pdf}
{\underline{Cornelius Noack's tutorial}}. It contains helpful information for
getting started with \musixtex, as well as a tutorial for \textbf{PMX}, a
preprocessor for \musixtex\ with a much simpler input language, and a brief
introduction to \textbf{M-Tx}, a preprocessor for \textbf{PMX}~which eases
the inclusion of lyrics.

\bigskip

\begin{flushright}
In the name of the community, Olivier Vogel,
July 30, 2005
\end{flushright}

%\clearpage

\chapter*{Preface to Version T.114}
%\thispagestyle{empty}

In order to expedite the release of version T.113 of \musixtex, only the most
essential modifications were made to the manual. With version T.114 the manual
has been extensively edited from cover to cover. Some material has been
rearranged, and references to \musictex, the immediate predecessor to
\musixtex, have been removed. The software itself remains
very stable. Only the slightest changes have been made since version T.113.

It remains true that this is the definitive reference to all features of
\musixtex, but also that it is not the best place for a novice user to start.
The \href{http://icking-music-archive.org/software/indexmt6.html}
{\underline{Werner Icking Music Archive}}\ contains excellent and detailed
instructions for installing \musixtex{} and the strongly recommended
preprocessors \textbf{PMX}
(for instrumental music) and \textbf{M-Tx} (for vocal) under
\href{http://icking-music-archive.org/software/musixtex/musixtex-for-unix.html}
{\underline{Linux/\unix}} or
\href{http://icking-music-archive.org/software/musixtex/musixwinstall.pdf}
{\underline{Windows 2000}}. Once the software is installed, most common
music typesetting tasks can be accomplished entirely by using one of these
preprocessors to generate the \musixtex\ input file, relieving the user of
learning any of the commands or syntax of \musixtex\ itself. It is only
for out-of-the-ordinary constructions that one must learn these details, so he
may insert the necessary \musixtex\ commands into the preprocessor's input file
as so called inline \TeX.
\href{http://icking-music-archive.org/software/pmx/pmxccn.pdf}
{\underline{Cornelius Noack's tutorial}} is an important resource which,
in addition to gently introducing the
novice to \textbf{PMX} and \textbf{M-Tx}, gives further details on installing
Postscript slur facilities.

\begin{flushright}
Don Simons, Andre van Ryckeghem, Cornelius Noack, August 30, 2005
\end{flushright}

\chapter*{Preface to Version 1.15}

From the time of \musixtex's predecessor \musictex\ until the
previous version of \musixtex~(T.114), 
the maximum numbers of simultaneous appearances of certain basic musical 
elements were limited: 
12 instruments, 12 or 6 beams depending on the type,  
12 font-based slurs%
\footnote{\texttt{musixps.tex} by Stanislav Kneifl provides different limits.}%
, and 6 trills or octave lines,
even with \texttt{musixmad.tex}.
For some purposes these limits are too tight, and because of them,  
developers of open-source GUI music editors may deem \musixtex{} 
unworthy of support in their software. Without such support or enhanced capacity,
\musixtex{} risks consignment to the junk heap of obsolete software. 

It has been the \musixtex{} community's long-cherished desire to have a 
new official version with greater capacity. However, for quite some time
the limits were due not only to the design of 
\musixtex{} itself but also to limitations in older versions of \TeX\ on the maximum 
numbers of available registers---256 each  
for counts (\texttt{\string\count}), dimensions (\texttt{\string\dimen}), 
skips (\texttt{\string\skip}), and tokens (\texttt{\string\toks}). 

But now, an extended variation of 
\TeX{}---$\varepsilon$\hbox{-}\nobreak\TeX---is available.\footnote{
$\varepsilon$\hbox{-}\nobreak\TeX{} (e-\TeX) is a trademark 
of the NTS group.}  
One of $\varepsilon$\hbox{-}\nobreak\TeX's features is that it can handle 
32768 registers. Furthermore, 
 \LaTeXe\ and other well-known extension packages require the  
$\varepsilon$\hbox{-}\nobreak\TeX{} engine, 
so it has spread widely among many \TeX\ users. 
In fact, the current executable file called \texttt{latex} 
relies on the $\varepsilon$\hbox{-}\nobreak\TeX{} engine.  
Many other advanced branches based on $\varepsilon$\hbox{-}\nobreak\TeX{} 
are also available. 

Here we provide the new \musixtex{} 1.15 as the solution for typesetting larger 
musical scores, 
taking advantage of $\varepsilon$\hbox{-}\nobreak\TeX's 32768-register capacity. 
We have made the new version 1.15 as compatible as possible with 
earlier versions of \musixtex, existing scores, extension packages, 
and Prof.~Knuth's original (non~$\varepsilon$\hbox{-}\nobreak) \TeX. 
With it, you can use many more instruments and other musical elements within the capacity of 
$\varepsilon$\hbox{-}\nobreak\TeX{} or even with original \TeX.
Also, as of version 2.6, the preprocessor \textbf{PMX} has been enhanced to 
allow up to 24 lines of music, and requires both \musixtex{} 1.15 and
$\varepsilon$\hbox{-}\nobreak\TeX.\footnote{This should pose no problems for
users, owing 
to the free availability of all the software, and to capacities of all modern 
computer hardware.}
Unfortunately, we have found a few rarely used old extensions 
not compatible with 1.15; 
we will provide information for adapting them to the new version.

We hope the new version 1.15 will keep \musixtex{} relevant well into the twenty-first century.


\begin{flushright}
Hiroaki Morimoto, Don Simons, January 2011
\end{flushright}

\clearpage

%\pagenumbering{roman}
\renewcommand{\baselinestretch}{.8}\footnotesize\normalsize
\tableofcontents \setcounter{secnumdepth}3
\renewcommand{\baselinestretch}{1}\footnotesize\normalsize

%avrb page 4 was on top
\clearpage
%avre

\setcounter{page}{1}
\pagenumbering{arabic}
\renewcommand{\thepage}{\arabic{page}}

\pagestyle{headings}



\chapter{Introduction to \musixtex}

This chapter is not a tutorial on the use of \musixtex, but instead serves as
an overview of some of its capabilities, quirks, and history.

\musixtex\ is a set of macros and fonts which enables music typesetting 
within the \TeX\ system. It requires as a prerequisite a working
installation of \TeX\footnote{See section \ref{installation}~for guidance on
installing \TeX.}.
\musixtex\ might be regarded as the digital equivalent of a box of type. It
contains symbols for staves, notes, chords, beams, slurs and ornaments,
ready to be arranged to form a sheet of music. But it must be told how to position
those symbols on the page. This could be done by the typesetter
himself, if he elects to proceed by entering \musixtex\ commands manually
into an input file. However most users will find it far less taxing to let
such decisions be made largely by the preprocessor
\href{http://icking-music-archive.org/software/indexmt6.html#pmx}
{\underline{\textbf{PMX}}}, which in addition uses a much simpler input
language than \musixtex.

Lyrics can also be handled by \musixtex. There is a set of primitive
commands for this which are described later. But there is also a far
more adaptable set of macros contained in the extension file
\verb|musixlyr.tex|, and there is the preprocessor
\href{http://icking-music-archive.org/software/indexmt6.html#mtx}
{\underline{\textbf{M-Tx}}}\ which provides easy, transparent access to
these macros.

Most users of \TeX\ are familiar with \LaTeX, a set of \TeX\ macros
which eases document layout. In fact many may only use \LaTeX.
Until recently, \LaTeX\ and \musixtex\ coexisted only grudgingly, owing
primarily to the limited availabily of storage registers. But with modern
versions of \TeX\ and with the use of e\LaTeX, only a modest increase in
complexity is incurred with the addition of musical excerpts to a \LaTeX\
document. Still, fortunately, for typesetting a musical score there is
rarely if ever any advantage to using \LaTeX. Only if one wanted to create a
text document with embedded musical examples would there be much use for it.
Even in that case there is a perfectly fine way to avoid using \musixtex\
directly in the document file, namely, by using \musixtex\ to create
\verb|eps| files for
each of the examples, then embedding references to those in the file for
the book. But for anyone who still wants to use both together, there is
no better example than this manual, as generated with the files
\verb|musixdoc.tex| and \verb|musixdoc.sty|. A few further details about such
nonstandard applications are given in section \ref{excerpts}.

\section{Primary features of \musixtex}
 \subsection{Music typesetting is two-dimensional} Written music is not
usually a linear sequence of symbols like a literary text. Rather,
except for unaccompanied single-note instruments like clarinets,
trumpets and human voices, it has the form of a two-dimensional matrix.
 Thus, a logical way of coding music consists
in horizontally accumulating a set of \ital{vertical combs} with
\ital{horizontal teeth} as depicted in Table~\ref{readtable}.
\def\hboxit#1{\boxit{\rlap{#1}\hphantom{note sequence three}}}
 \begin{table}
 \begin{center}
%avrb
 \small
 %avre
 \begin{tabular}{|ll|ll|}\hboxit{note sequence one}
  &\hboxit{note seq.\ four}
  &\hboxit{note seq.\ seven}
  &\hboxit{note seq.\ ten}\\\hboxit{note sequence two}
  &\hboxit{note seq.\ five}
  &\hboxit{note seq.\ eight}
  &\hboxit{note seq.\ eleven}\\\hboxit{note sequence three}
  &\hboxit{note seq.\ six}
  &\hboxit{note seq.\ nine}
  &\hboxit{note seq.\ twelve}\\
 \end{tabular}
 \end{center}
 \caption{A logical way of coding music}\label{readtable}
 \end{table}
Accordingly, in \musixtex\ the fundamental macro used to represent
one of those vertical combs (or one of the columns in Table~\ref{readtable})
is of the form
\begin{center}
\verb|\notes ... & ... & ... \enotes|\footnote{The abbreviation
\keyindex{en} can be used in place of \keyindex{enotes}.}
\end{center}\index{"&@{\tt\char'046}}
\noindent where the character \verb|&| is used to separate the notes to be
typeset on respective staves of the various instruments, starting from the
bottom.

In the case of an instrument whose score has to be written with
several staves, they are separated by the
character \|\index{\tt\char'174@{\tt\char'174}}. 

Thus, a score written for a keyboard instrument and
a single-line instrument (e.g., piano and violin)
will be coded as follows:
\begin{center}
\verb=\notes ... | ... & ...\enotes=
\end{center}
\noindent for each column of simultaneous \ital{groups of notes}.
Each of those groups, represented by a single box in Table~\ref{readtable}\
and by a sequence of three dots in the the two example macros above, may contain
not only chord notes to be played simultaneously, but short sequences of
consecutive
notes or chords. As we'll soon see, this implies the need for two fundamentally different
kinds of elemental macros in \musixtex, those that are automatically followed
by some amount of space (called \ital{spacing macros}, and those that are not.
The former type, for example,
would be used to represent all the notes and rests in a single-line score.
The latter would be used for example for chord notes and ornaments.

\subsection{Horizontal spacing}
Deciding upon the proper horizontal spacing of notes is a very complicated matter
that we will not address in any detail here. Obviously short-duration notes
should be closer together than longer ones. Almost as obviously, the spacing
cannot be linearly proportional to the duration; otherwise  for example a whole
note would occupy 32 times as much horizontal space as a thirty-second note. And
in polyphonic scores the spacing in one staff is often influenced by the notes
in another. This is a decision that the typesetter or preprocessor
must make. Once the decision is made, \musixtex\ can provide the desired spacing.
The main mechanism is through a set of macros described in
section~\ref{newspacings}. At this point we shall only mention that to control
spacing, one of those
macros will be selected to replace the symbol \verb|\notes| in the two examples
above, and it will imply that whenever a spacing macro in encountered within
a group of notes, a
certain specific amount of horizontal space will be inserted.

 \subsection{Music tokens}
The tokens provided by \musixtex\ include
\begin{itemize}\setlength{\itemsep}{0ex}
 \item note symbols without stems;
 \item note symbols with stems, and flags for eighth notes and
shorter;
 \item beam beginnings and endings;
 \item beginnings and endings of ties and slurs;
 \item accidentals;
 \item ornaments: arpeggios, trills, mordents, turns, staccatos,
pizzicatos, fermatas, etc.;
 \item bar lines;
 \item meters, key signatures, clefs.
\end{itemize}

\smallskip

 Thus for example, \verb|\wh a| produces a whole note at nominal frequency 222.5~Hz,
\verb|\wh h| produces one an octave higher, \verb|\qu c|
produces an up-stemmed quarter note C (250~Hz), and \verb|\cl J| produces a down-stemmed C
eighth note an octave lower.

 To generate chords with solid note heads, the macro \keyindex{zq}
can be used. It produces a solid note head at the specified pitch,
the vertical position of which
is memorized and recalled whenever the next stemmed note (possibly with a flag) is
coded. The stem length is automatically adjusted to link all simultaneous notes. Thus, the
C-major chord

\begin{music}\nostartrule
\startextract\NOtes\zq{ceg}\qu j\en\zendextract
\end{music}
 \noindent is coded \verb|\zq c\zq e\zq g\qu j| or more concisely,
\verb|\zq{ceg}\qu j|. Here the \verb|u| in the spacing note macro
\verb|\qu| is what causes the upstem.

 \subsection{Beams}
Each beam \index{beams} is generated by a pair of macros. The first defines
the beginning horizontal position (implicitly the current position), altitude,
direction (upper or lower), multiplicity (number of lateral bars), slope and
reference number. This latter feature is needed so independent beams
can overlap. The second macro of the pair specifies the termination
location (again implicitly) and the reference number.

 \subsection{Setting anything on the score}
A general macro (\keyindex{zcharnote}) provides a means of
putting any sequence of symbols (possibly contained in an \verb|\hbox{...}|) at any
pitch of any staff of any instrument. This allows any symbol defined in a font
(letters, math symbols, etc.) to be placed in the score at a position keyed
to the music both in time (horizontally) and pitch (vertically) on the staff,
\section{A simple example}
 Before going into more detail, we give below an example of the two first
bars of the sonata in C-major KV545 by {\sc Mozart}\index{Mozart, W.A.@{\sc
Mozart, W.A.}}:

\begin{music}
\parindent11mm
\setname1{Piano}
\setstaffs12
\generalmeter{\meterfrac44}
\startextract
\Notes\ibu0f0\qb0{cge}\tbu0\qb0g|\hl j\en
\Notes\ibu0f0\qb0{cge}\tbu0\qb0g|\ql l\sk\ql n\en
\bar
\Notes\ibu0f0\qb0{dgf}|\qlp i\en
\notes\tbu0\qb0g|\ibbl1j3\qb1j\tbl1\qb1k\en
\Notes\ibu0f0\qb0{cge}\tbu0\qb0g|\hl j\en
\zendextract
\end{music}
 The coding is as follows:
\begin{quote}\begin{verbatim}
\begin{music}\nostartrule
\parindent10mm
\instrumentnumber{1}       % a single instrument
\setname1{Piano}           % whose name is Piano
\setstaffs1{2}             % with two staffs
\generalmeter{\meterfrac44}% 4/4 meter chosen
\startextract              % starting real score
\Notes\ibu0f0\qb0{cge}\tbu0\qb0g|\hl j\en
\Notes\ibu0f0\qb0{cge}\tbu0\qb0g|\ql l\sk\ql n\en
\bar
\Notes\ibu0f0\qb0{dgf}|\qlp i\en
\notes\tbu0\qb0g|\ibbl1j3\qb1j\tbl1\qb1k\en
\Notes\ibu0f0\qb0{cge}\tbu0\qb0g|\hl j\en
\zendextract                 % terminate excerpt
\end{music}
 \end{verbatim}\end{quote}
\begin{itemize}\setlength{\itemsep}{0ex}
 \item \verb|\ibu0f0| begins an upper beam, aligned on the
\ital{f}, reference number 0, slope 0
 \item \verb|\tbu0| terminates this beam before writing
the second \ital{g} by means of \verb|\qb0g|
 \item \verb|\qb..| indicates a note belonging to a beam.
 \item \verb|\sk| sets a space between the two quarters in the right
hand, so that the second one is aligned with the third eighth in the left hand.
 \item\verb|\qlp| is a dotted quarter note.
 \item\verb|\ibbl1j3| begins a double beam, aligned on the \ital{C}
(\verb|j| at this pitch) with slope 15\%.
\end{itemize}

\section{The three pass system}
\TeX's line-breaking procedure implicitly assumes
that a normal line of text will contain many words, so that inter-word
glue need not stretch or shrink too much to justify the line.
One might at first consider extending this to music, treating each bar like a
word with no stretchable internal space. But typically this would lead to unsightly
gaps before each bar rule, simply because the number of bars per line
is normally many fewer than the number of words in a line of text.
\musixtex\ needs a more sophisticated horizontal spacing algorithm
than is used in \TeX.

To understand how \musixtex\ solves this problem, we have to
recognize two different kinds of horizontal space, \ital{hard}\ and
\ital{scalable}. Hard space is fixed and always represents the same
physical distance.
Examples of hard space are the widths of bar rules, clefs,
and key signatures. Scalable space can be stretched as needed. It
is what is normally used for the space after notes or rests. At the
outset it is only defined in a relative sense. In other words,
scalable spaces are defined as multiples of \verb|\elemskip|, an
initially undefined elemental spacing unit. For example, in \textbf{PMX}\
all sixteenth notes
are typically assigned a scalable width of 1.41\verb|\elemskip|.
One main job of \musixtex\ is to compute the physical value of
\verb|\elemskip|, often expressed in points (72nds of an inch).
The correct value is that which makes all the scalable space on
a line just fill up what's not occupied by hard space. Obviously
it may vary from line to line.

To this end a three pass system was developed. To start the
first pass on the file \verb|jobname.tex|, you would enter \verb|tex jobname|.
Information about each bar is written to
an external file named \verb|jobname.mx1|.
This file begins with a header containing parameters
such as line width and paragraph indentation. Then the hard and
scalable space is listed for each bar.
\index{scalable width}\index{hard width}

The second pass, which is started with \verb|musixflx jobname|,
determines optimal values of the elemental spacing unit \verb|\elemskip|
for each line, so as to properly fill each line, and to spread
the piece nicely over an integral number of full lines. This routine was written
in FORTRAN and now converted to C rather than \TeX, the main reason
being the lack of an array handling capability in \TeX.

\verb|musixflx| reads in the file \verb|jobname.mx1|, and writes its
output to \verb|jobname.mx2|. The latter file contains a single entry for
each line of music in the reformatted output. The key piece of information
is the revised value of \keyindex{elemskip} for each line.

Next, the file is \TeX-ed again, by entering \verb|tex jobname|. On this third pass,
the \verb|jobname.mx2| file is read in, and the information
is used to physically define the final score and embed the
page descriptions into a \verb|dvi|\ file.

You may wonder how the same command can
cause something different to happen the second time it is issued. The explanation
is that \musixtex\ checks for the presence of \verb|jobname.mx2|. If it's
not present, the first pass is executed; if it is, the third pass. Obviously if
you've made some corrections that affect any horizontal spacing, you must remember
to delete the old \verb|jobname.mx2| and then rerun all three passes, or
build or use a batch script that does so.

Here's an example. Here, no value for \verb|\elemskip| is explicitly specified,
so \musixtex\ assign a single, default value. After the first pass you might get
the following output:

\begin{music}
\hsize=100mm
\generalmeter{\meterfrac24}%
\parindent 0pt
\setsign1{-3}
\startpiece\bigaccid
\NOtes\qu{ce}\enotes
\bar
\NOtes\qu{gh}\enotes
\bar
\NOtes\qu{=b}\enotes
\Notes\ds\cu g\enotes
\bar
\NOtes\qu{^f=f}\enotes
\raggedstoppiece\contpiece% this is cheated, the problem was to get
%                           pass 1 and pass 3 at the same time
\NOtes\qu{=e}\itied0e\qu{_e}\enotes
\bar
\Notes\ttie0\Qqbu ed{_d}c\enotes
\bar
\Notes\ibu0b{-2}\qb0{=b}\enotes
\notes\nbbu0\qb0{=a}\tqh0N\enotes
\Notes\Dqbu cf\enotes
\raggedstoppiece\contpiece
\NOtes\uptext{\it tr}\qu e\uptext{\it tr}\qu d\enotes
\bar
\NOtes\qu c\qp\enotes
\setdoubleBAR\raggedstoppiece
\end{music}

\noindent Note that the space after every quarter note is the same, and
that lines are not justified.
After running \verb|musixflx| and \TeX-ing the second time you'll
get:

\begin{music}
\hsize=100mm
\generalmeter{\meterfrac24}%
\parindent 0pt
\generalsignature{-3}
\startpiece\bigaccid
\NOtes\qu{ce}\enotes
\bar
\NOtes\qu{gh}\enotes
\bar
\NOtes\qu{=b}\enotes
\Notes\ds\cu g\enotes
\bar
\NOtes\qu{^f=f}\enotes
\bar
\NOtes\qu{=e}\itied0e\qu{_e}\enotes
\bar
\Notes\ttie0\Qqbu ed{_d}c\enotes
\bar
\Notes\ibu0b{-2}\qb0{=b}\enotes
\notes\nbbu0\qb0{=a}\tqh0N\enotes
\Notes\Dqbu cf\enotes
\bar
\NOtes\uptext{\it tr}\qu e\uptext{\it tr}\qu d\enotes
\bar
\NOtes\qu c\qp\enotes
\Endpiece
\end{music}

\noindent Now \musixtex\ has determined a number of lines
(which is different from the original number), the lines are justified,
and if you look carefully you
can see that the space after quarters in the first line
is smaller than in the second. This example was coded as:
\begin{verbatim}
\hsize=100mm
\generalmeter{\meterfrac24}%
\parindent 0pt
\generalsignature{-3}
\startpiece\bigaccid
\NOtes\qu{ce}\en\bar
\NOtes\qu{gh}\en\bar
\NOtes\qu{=b}\en
\Notes\ds\cu g\en\bar
\NOtes\qu{^f=f}\en\bar
\NOtes\qu{=e}\itied0e\qu{_e}\en\bar
\Notes\ttie0\Qqbu ed{_d}c\en\bar
\Notes\ibu0b{-2}\qb0{=b}\enotes
\notes\nbbu0\qb0{=a}\tqh0N\enotes
\Notes\Dqbu cf\en\bar
\NOtes\uptext{\it tr}\qu e\uptext{\it tr}\qu d\en\bar
\NOtes\qu c\qp\en\Endpiece
\end{verbatim}

One benefit of the 3-pass system is the quick and easy
alteration to the layout which can be achieved
by changing only one parameter, namely \keyindex{mulooseness}. This value
acts analogously to \TeX's \verb|\looseness| command. For non-\TeX-perts: if you
state \verb|\looseness=-1| somewhere inside any paragraph, then
\TeX\ will try to make the paragraph one line shorter than it normally would.
With \verb|\mulooseness|, \musixtex\ does the same, but for \ital{systems}
and \ital{sections} rather than lines and paragraphs. A system is just a group of
staves treated as a unit, and in this discussion is analogous to a line of text.
What is a \ital{section}? It's any chunk of coding not containing a
forced system break, System breaks can be forced with
\keyindex{stoppiece}, \keyindex{endpiece}, \keyindex{zstoppiece},
\keyindex{Stoppiece}, \keyindex{Endpiece}, \keyindex{alaligne}
\keyindex{zalaligne}, \keyindex{alapage} or \keyindex{zalapage}. If none
of these is present, the section comprises the whole piece.
Somewhere\footnote{Advisably, at the beginning or
at the end of the section, for the sake of clarity.}
before the end of the section, you can change the value of
\keyindex{mulooseness} to something different from the default of 0, and
\musixtex\ will typeset that section with a different number of systems.

To give an easy example, changing the last line in the previous example to:
\begin{verbatim}
\NOtes\qu c\qp\en\mulooseness=1\Endpiece
\end{verbatim}
yields:

\begin{music}
\hsize=100mm
\generalmeter{\meterfrac24}%
\parindent 0pt
\generalsignature{-3}
\startpiece\bigaccid
\NOtes\qu{ce}\enotes
\bar
\NOtes\qu{gh}\enotes
\bar
\NOtes\qu{=b}\enotes
\Notes\ds\cu g\enotes
\bar
\NOtes\qu{^f=f}\enotes
\bar
\NOtes\qu{=e}\itied0e\qu{_e}\enotes
\bar
\Notes\ttie0\Qqbu ed{_d}c\enotes
\bar
\Notes\ibu0b{-2}\qb0{=b}\enotes
\notes\nbbu0\qb0{=a}\tqh0N\enotes
\Notes\Dqbu cf\enotes
\bar
\NOtes\uptext{\it tr}\qu e\uptext{\it tr}\qu d\enotes
\bar
\NOtes\qu c\qp\enotes
\mulooseness1\Endpiece
\end{music}

\noindent On the other hand,
\begin{verbatim}
\NOtes\qu c\qp\en\mulooseness=-1\Endpiece
\end{verbatim}
yields

\begin{music}
\hsize=100mm
\generalmeter{\meterfrac24}%
\parindent 0pt
\generalsignature{-3}
\startpiece\bigaccid
\NOtes\qu{ce}\enotes
\bar
\NOtes\qu{gh}\enotes
\bar
\NOtes\qu{=b}\enotes
\Notes\ds\cu g\enotes
\bar
\NOtes\qu{^f=f}\enotes
\bar
\NOtes\qu{=e}\itied0e\qu{_e}\enotes
\bar
\Notes\ttie0\Qqbu ed{_d}c\enotes
\bar
\Notes\ibu0b{-2}\qb0{=b}\enotes
\notes\nbbu0\qb0{=a}\tqh0N\enotes
\Notes\Dqbu cf\enotes
\bar
\NOtes\uptext{\it tr}\qu e\uptext{\it tr}\qu d\enotes
\bar
\NOtes\qu c\qp\enotes
\mulooseness-1\Endpiece
\end{music}

\noindent which is tighter than you would ever want, but serves to further
demonstrate the use of \keyindex{mulooseness}.

If you want to build up a \musixtex\ input file
manually (which in fact very few users will ever need to do, considering the
availability of \textbf{PMX}), here is a roadmap for one way to proceed:
\begin{enumerate}
 \item Enter the data one \verb|\notes| group at a time, taking care to select the
relative horizontal spacing for each group (via the macros discussed in section~\ref{newspacings})
so as to maintain a consistent relationship between scalable space and note durations.
This will be discussed in a great deal more detail in Chapter 2.
 \item \TeX~$\Longrightarrow$ {\tt musixflx} $\Longrightarrow$ \TeX.
 \item Look at the output and decide if you want to have more or fewer
systems, e.g.~to fill the page or to get an even number of
pages.\index{musixflx@{\tt musixflx}}
 \item If you want to change the number of systems in a section, adjust
\keyindex{mulooseness} accordingly. Keep in mind that each section cannot have fewer
systems than bars.
\item Delete \verb|jobname.mx2| and repeat the process\\
\TeX~$\Longrightarrow$ \verb|musixflx| $\Longrightarrow$ \TeX.

\end{enumerate}

There is an alternate way to proceed if you know at the outset
how many systems
you want in a section. You can specify it directly by assigning a
positive number to \keyindex{linegoal} somewhere within the
section (\verb|\linegoal| requires
version 0.83 or later of \verb|musixflx|).
\keyindex{mulooseness} must be zero
for \keyindex{linegoal} to work. Both are automatically reset to
zero after processing the end of a section e.g. as defined by
\keyindex{stoppiece}.

Finally, for large scores (more than 4 pages or so), having
only one section and an overall value of \verb|\mulooseness| becomes
impractical since one wants not only to have nicely spaced systems, but also
completely filled pages without empty top and
bottom margins on the last page. It is then wise to force the total number of
pages and possibly the line breaks in each page, which can be done using
\keyindex{alapage} and/or \keyindex{alaligne}\footnote{This is the technique
always used by \textbf{PMX} in constructing a \musixtex\ input file.}.
%avrb
%, or more automatically by
%means of the \keyindex{autolines} command borrowed from \musictex\ and
%implemented in the additional {\ttxem{musixcpt.tex}} file
%avre

There is another advantage to \musixtex's way of using scalable space and
the three pass system. In
\TeX nical terms it eliminates the need for \ital{glue}, and enables
every horizontal position in a line to be computed. This in turn enables certain
variable length symbols such as slurs to be specified by macros entered at
their beginning and ending points, rather than having to estimate the
length of the symbol and enter it at the starting point.

 \subsection{External executable \texttt{musixflx}}
One issue that compromises portability between computers is the
need for the executable {\tt musixflx}\index{musixflx@{\tt
musixflx}}.
%avrb
To address this, the C source and compiled versions for various OS's are included in
\verb|musixtex.zip| and are available from the
\href{http://icking-music-archive.org}{\underline{Werner Icking Music Archive}}.
%avre

On most computers, the executable is invoked by typing the name of the program
and the name of the file to be acted upon. \ital{i.e.}

\verb|musixflx jobname.mx1|

Optionally, you can add a letter to indicate one of the debug modes, which are:
\begin{quote}
\verb|d | for debug information to screen\\
\verb|f | for debug information to file \verb|jobname.mxl|\\
\verb|s | to get the computed lines immediately on screen
\end{quote}
To allow for ease of use with a batch file, \verb|musixflx| can either be fed
with \verb|jobname.mx1|, \verb|jobname.tex| or only
\verb|jobname|, any one of which will open \verb|jobname.mx1|.

 \subsection{Unrecorded spaces: the novice's bugaboo}

Because of the way \musixtex\ accounts for hard and scalable space and avoids
using glue, it is absolutely essential that every horizontal space be
properly entered into the input file. The most common error in this regard
is including a blank space in the midst \ital{or at the end} of an input line.
Such a blank space, or for that matter any stray character not entered with
an appropriate \musixtex\ macro, will not be properly
recognized and recorded by \musixtex, but it will still be treated like
ordinary text by \TeX. The symptoms of such a transgression will be an
{\bf Overfull hbox} warning during the third pass, and the appearance of
either excessive blank horizontal space or thick black vertical lines in
the page image.

\medskip
\centerline{\boxit{Considerable discipline is needed to avoid this problem!}}

The best technique for avoiding such unpleasantness is to avoid entering
any nonessential blanks within input lines, and to end every
input line with either \verb|%|\index{%@{\tt\char37}} or
\keyindex{relax}, unless it already ends with a control sequence
ending in a letter.

There are other ways to enter unrecorded space which should be
avoided. Between \verb|\startpiece| and \verb|\stoppiece| or
\verb|\endpiece|, never use \verb|\hskip| or \verb|\kern| except
within \ital{zero}-boxes like \verb|\rlap|, \verb|\llap|,
\verb|\zcharnote|, \verb|\uptext|, etc., and never assign hard
values to scalable dimensions like \verb|\noteskip|,
\verb|\elemskip|, \verb|\afterruleskip| and \verb|\beforeruleskip|
\footnote{Note that {\Bslash hardspace} does not fall in this
category; it is specifically designed to enter hard space in a way
the \musixtex\ can properly record it}.

Here is a checklist of tips related to spacing issues. Because the
foregoing several paragraphs are so important, some of their content
is repeated in the list.

\begin{enumerate}
 \item
Between \verb|\startpiece| and \verb|\stoppiece| or \verb|\endpiece|,
end every input line with a either \verb|%| or a command with no arguments
(including \verb|\relax|).
 \item \verb|\off| must only be used with scalable values, e.g.
\verb|\noteskip|, \verb|\elemskip|, \verb|\afterruleskip|, and
 \verb|\beforeruleskip|.
 \item Remember that \verb|\qsk| and \verb|\hqsk| are scalable, i.e. \verb|\qsk|
  doesn't necessarily mean exactly one note head width (it depends on \verb|\elemskip|).
 \item Lyrics and any other non-\musixtex\ text must be put in zero boxes
such as
  \verb|\zcharnote|, \verb|\zchar|, \verb|\rlap|, \verb|\lrlap|, \verb|\llap|,
\verb|\uptext|, or \verb|\zsong|. Additional specific capabilities for entering
lyrics are provided by
 \verb|\hardlyrics|, \verb|\hsong|, and---most significantly---the
text-emplacement macros defined in \verb|musixlyr.tex| (see \ref{lyrics},
p.~\pageref{lyrics}).
 \item
Between \verb|\startpiece| and \verb|\stoppiece| or \verb|\endpiece|,
don't assign hard values to \verb|\noteskip|, \verb|\beforeruleskip|,
or \verb|\afterruleskip|.
\end{enumerate}

\section{Further highlights}
 %\check
 \subsection{Key signatures}
A single key signature can be assigned to all instruments, for example by 
\keyindex{generalsignature}\verb|{-2}|
which sets two flats on each staff. \keyindex{generalsignature}
can be overridden for selected instruments, for example by
\keyindex{setsign}\verb|2{1}| which puts one sharp on each staff
of instrument number 2. Of course, the current signature as well
as meters and clefs may change at any time.
\subsection{Transposition}
With some extra attention, a score can be input in such a way that it is
fairly easily transposed. There is an internal register
called \keyindex{transpose}, the default value of which is zero, but which may be
set to any reasonable positive or negative value. It defines a number of pitch
steps (lines or spaces on a staff) by which all pitched symbols with be offset, provided
they have been entered with letter values to represent their pitch.
However,
it will neither change the local accidentals nor the key signature.

For
example, suppose a piece were originally input in F major, and it contained a
B natural, and you wanted to transpose it to G. If you simply set
\verb|\transpose| to 1 with no other special considerations, then
the key signature would not change, and
the B natural would appear as a C natural, whereas it should be a C sharp.
So first you must explicitly change the key signature. Then, to
solve the problem with accidentals, you should declare
\verb|\relativeaccid|, which will cause the actual appearance of any
accidental to depend on the pitch of
the accidental as well as the current key signature. But the use of this
facility requires the typesetter to have entered the original set of accidentals
according to a nonstandard convention wherein an accidental does not
specifically refer to the black or white keys on a piano, but to the
amount by which the pitch is altered up or down from what it would naturally
have been, taking the key signature into account.\index{relative accidentals} This
is discussed in more detail in section~\ref{transposeaccids}. Now, finally, you
can enter \verb|\transpose=1| with more or less the desired effect.

More or less, because there is also an issue with stem
and beam directions. Normally a typesetter would want full control over them,
and would exercise that control by entering them with macros that explicitly
assign the direction. Naturally the assigned directions would persist
even after changing \verb|\transpose|. With respect to stems of unbeamed notes,
this matter can also be addressed at
the input level, by using special macros for notes that leave the
decision about stem directions up to \musixtex. These macros do the right
thing in the face of transposition. They are discussed in
section~\ref{autostemdirections}. Unfortunately there is no corresponding
such facility for beams, so the typesetter will have to edit the transposed
score to adjust beam directions as required\footnote{\textbf{PMX}
will automatically adjust both stem and beam directions when transposing, which
is yet another reason to use it. However
if a piece is to be transposed, the typesetter must still explicitly activate relative accidentals
and enter accidentals according to the relative accidental convention.}.

 %\check
 \subsection{Extracting parts from a score} Another question is:
\ital{``Can I write a full score and then extract separate scores for
each individual instrument?''}

The answer is yes, but only with a great deal of special attention---so much,
in fact, that we shall strongly recommend that if you want
to do this, you should use \textbf{PMX}, which makes the process
very easy. If for some reason you choose to ignore this advice, you may
refer to the details that were provided in the prior version of this manual,
\href{http://icking-music-archive.org/software/musixtex/mxdoc112.pdf}
{\underline{mxdoc112.pdf}}.

%After assigning symbolic numbers to instruments,
%there are macros (see \ref{instrum-inhibit}) that permit
% \begin{itemize}\setlength{\itemsep}{0ex}
% \item choosing which instrument the following source code is attached to,
% \item choosing which staff of an instrument the following source code is
%attached to,
% \item hiding one or several instruments by zeroing out their staff sizes and staff
%numbers.
% \end{itemize}

 %\check
 \subsection{Staff and note sizes}
 Although the standard staff size is 20pt, \musixtex\ allows scores with
sizes of 16, 24, or 29pt. Furthermore, any instrument may be assigned its own special staff
size (usually smaller than the overall staff size), and there are special macros
(e.g. \verb|\smallnotesize|, \verb|\tinynotesize|) that cause notes, beams, and
accidentals all to take a different size.

\subsection{Add-in macro libraries}
During the early stages of \musixtex's development, common versions of \TeX\ itself were
very limited in capacity, especially in terms of the numbers of registers that could
be defined for use in macros. For this historical reason, many important enhancements
to \musixtex\ are available only via add-in libraries. The user can thus pick and choose
which to include for any particular compilation. Most of these are included in
\verb|musixtex.zip|, and their uses are discussed in this manual. The
libraries have names like \verb|blabla.tex|, and are activated by including a line
like \verb|\input blabla|\ within the the input file. The most common such files
are \ttxem{musixadd.tex} and \ttxem{musixmad.tex} which respectively increase the
number of instruments from the default 6 to 9 or 12 as well as increasing available numbers
of other features; \verb|musixps.tex| which enables Type K postscript slurs; and
\verb|musixlyr.tex| which greatly eases typesetting lyrics. In fact the latter two,
while now included in \verb|musixtex.zip|, are not documented in this manual but
in separate files inside \verb|musixtex.zip|, namely \verb|musixps.tex| itself and
\verb|mxlyrdoc.pdf| respectively.

\section{Where to get the software and help using it}
The home base for all matters related to \musixtex\ is the
Werner Icking Music Archive, at
\href{http://icking-music-archive.org}{\underline{http://icking-music-archive.org}}.
The most up-to-date versions of \musixtex\ and friends are located in the
\href{http://icking-music-archive.org/software/indexmt6.html}{\underline{software}}
section of the archive. Assuming you already have \TeX\ and only want to install
\musixtex, the file you need to download will be named
\href{http://icking-music-archive.org/software/musixtex/musixtex.zip}
{\underline{\texttt{musixtex.zip}}} . Further details about
the installation process are given in section \ref{installation}.

The Werner Icking Archive also hosts the
\href{http://icking-music-archive.org/mailman/listinfo/tex-music}
{\underline{\TeX-music mailing list}}, where you will always find
someone willing to answer questions and help solve problems.

  \section{A very brief history of \musixtex}
The idea of using \TeX\ to typeset music appears to have originated
around 1987 with the master's thesis of Andrea
{\sc Steinbach} and Angelika {\sc Schofer}\footnote{Steinbach A. \& Schofer
A., \ital{Automatisierter Notensatz mit \TeX}, master's thesis,
Rheinische Friedrich-Wilhelms Universit\"at,
Bonn, Germany, 1987}. They called their package
\mutex\index{mutex@\protect\mutex}. It was limited to a single staff.
It introduced two key concepts: (1) using
a large number of font characters to construct beams and slurs, and (2) using
\TeX\ glue to help control horizontal spacing and justification.

The next major step came around 1991 when Daniel {\sc Taupin} created \musictex.
Its major enhancement was to allow multiple staves. But this came at a
price: some flexibility was lost in controlling horizontal spacing and
a great deal of trial and error became necessary to avoid excessive or
insufficient gaps before and after bar lines.

\musictex\ was a single-pass system. To remedy its shortcomings it became clear
that a multi-pass system would be required. Around 1997 Dr. Taupin along with Ross {\sc Mitchell}
and Andreas {\sc Egler} created the first version of \musixtex. At last a fully
automatic procedure was coded so as to provide pleasing horizontal spacing in
multi-staff scores.

Significant enhancements to \musixtex, which have already been mentioned, have been provided
by Stanislav {\sc Kneifl} (Type K postscript slurs) and
Rainer {\sc Dunker} (Lyrics handling via \verb|musixlyr.tex|).

Since Dr. Taupin passed away in 2003, \musixtex\ has been maintained by a 
varying cast including Olivier {\sc Vogel}, Hiroaki {\sc Morimoto}, 
Bob {\sc Tennent}, Andre {\sc Van Ryckeghem}, 
Cornelius {\sc Noack}, and Don {\sc Simons}.

No discussion of the history of \musixtex\ would be complete without mentioning
the contributions of Werner {\sc Icking}. From the early days of \mutex\ until
his untimely death in 2001, he served this line of software as its most prominent
proponent, beta tester, web site and mailing list editor, consultant,
problem solver, and inspiration for
many third-party enhancements including \textbf{PMX}. In fact he founded the
mailing list and the archive that now is named in his honor. The web site is
currently edited by Christian {\sc Mondrup}, the software page by Don {\sc Simons},
and the mailing list by Maurizio {\sc Codogno}.


 %\check
\chapter{Elements of \musixtex}
\section{Setting up the input file}
\subsection{What makes a \TeX\ file a \musixtex\ file?}
A \musixtex\ input file is a special kind of \TeX\ input file. What makes it
special is that it must contain the command
\verb|\input musixtex|
before any reference to \musixtex\ macros. After that might follow
\verb|\input musixadd| or \verb|\input musixmad| if you want to have
respectively up to nine or twelve instruments or simultaneous beams,
ties, or slurs.\ixtt{musixtex.tex}\ixtt{musixadd.tex}\ixtt{musixmad.tex}
If you want to have greater numbers of these elements, you can assign them 
directly by including one or more of these commands: 
\verb|\setmaxinstruments|, 
\verb|\setmaxcxxviiibeams|, 
\verb|\setmaxcclvibeams|, 
\verb|\setmaxgroups|, 
\verb|\setmaxslurs|, 
\verb|\setmaxtrills|, 
\verb|\setmaxoctlines|.

Since it is still a \TeX\ file, after that, if you wished to,
you could write a whole non-musical book
using normal \TeX\ commands provided that you did not
use \verb|&| as a tab character like in plain \TeX\ (In \TeX\ lingo,
its \keyindex{catcode} has been changed).

 \subsection{Cautions for the non \TeX pert}
When \TeX\ reads anything, it inputs one \ital{token} at a time. A token
may be either a \ital{command} or a character. A command (or \ital{macro},
or \ital{control sequence}) is a
\ital{backslash} (``{\Bslash}'') immediately followed by sequence of
letters with no intervening spaces. For practical purposes, any single symbol (letter, digit,
special character, or space) that is not part of a command counts as a character
and therefore as a token.

Each command expects a specific number of parameters.
The tokens ``\verb|{|'' and ``\verb|}|'' are very special, in that (1) they
must occur in matched pairs, and (2) any matched pair together with
the stuff inside counts as a single parameter.
If the first
parameter expected is a single letter, it must either be
separated from the command by a space or
enclosed in braces (otherwise it would be interpreted as part of the command).
For example the command \verb|\ibu| expects three parameters, so the
following are all OK: \verb|\ibu123|, \verb|\ibu1A3|, \verb|\ibu1{`A}3|,
\verb|\ibu{1}{2}{3}|, \verb|\ibu1{-2}3|, or \verb|\ibu1{23}4|, but
\verb|\ibu1234| is not OK; the first three digits are taken as parameters,
leaving the ``4'' with no purpose other than to cause some of the dreaded
unrecorded space that we have already mentioned.

In the rest of this manual, when describing commands we will write
things like \verb|\qb{|$n$\verb|}{|$p$\verb|}|. It should be understood that
when $n$ and $p$ are replaced by their literal values, the braces may
or may not be necessary. In particular, if both are single digits, no
braces are needed; but if $n$ has two digits, or if $p$ has more than
one character, they must be surrounded by braces.

Spaces (blank characters) in the input file must be handled very carefully.
They are ignored at the beginning of a line, enabling logical
indentation schemes to help make the file human-readable. There are also a few
other places within lines where blank spaces are OK (such as mentioned in the prior paragraph),
but in general is it safest to avoid
any unnecessary blanks between the beginning and end of an input line.
At the end of a line, the truth is that a command with no parameters, such as
\verb|\bar| or \verb|\enotes| will cause no trouble. However if a command with
one or more parameters is the last item in an input line, it will cause
unrecorded space. The way around this is to end the line with either ``\verb|%|''
or \verb|\relax|.

 \subsection{Usual setup commands}\label{whatspecify}

 The first decision is what size type to use. \musixtex\ offers four sizes:
``small'' (16-pt-high staves), ``normal'' (20pt),\index{sizes} ``large''
(24pt), and ``Large'' (29pt). The default is
\keyindex{normalmusicsize}. If you want a different size, then you have to
enter \keyindex{smallmusicsize}, \keyindex{largemusicsize}, or
\keyindex{Largemusicsize}. Each of these commands defines not only the
desired staff size but many other related sizes such as note heads, ornaments,
stem lengths, etc.

 The command \keyindex{instrumentnumber}\onen~defines the
number of instruments to be $n$. If not entered, the default is 1. This number is used in loops
that build staves,
set key signatures, set meters, etc., so if it differs from 1 it must be explicitly
defined before any further commands.

An instrument may have one or more staves (e.g.~a piano would normally have 2 staves).
The differences between one
instrument of several staves and several instruments with one staff each are as follows:
\begin{itemize}\setlength{\itemsep}{0ex}
 \item Different instruments may have different \itxem{key signatures}, while different
staves of an instrument will all have the same key signature.
 \item A \itxem{beam} may include notes in different
staves of the same instrument.
 \item A \itxem{chord} may extend across several staves of the same
instrument.
 \item If an instrument has more than one staff, they will be linked together
with a big, curly brace at the beginning of each line.
\end{itemize}
The default number of staves per instrument is 1. If it is different, then it
must be specified by \keyindex{setstaffs}\verb|{|$n$\verb|}{|$p$\verb|}|
where $p$ is the number of staves and $n$ is the
number of the instrument. In \musixtex, instruments are numbered
\ital{starting with the lowest}.
So for example \verb|setstaffs32|
assigns two staves to the third instrument from the bottom.

 The default clef for every staff is the \ital{treble} clef. To assign
any other clef, the command is
\keyindex{setclef}\verb|{|$n$\verb|}{|$s_1s_2s_3s_4$\verb|}|
where $n$ is the number of the instrument, $s_1$ is a digit specifying the
clef for the first (lowest) staff, $s_2$ for the second staff, and so forth.
Note that like
instruments, staves of a given instrument are numbered starting with
the lowest. The parameters $s_2$, $s_3$ and $s_4$ can be omitted, in which case
any unspecified staves will be assigned a treble clef.

The digits $s$ can range from 0 to 9, with the following meanings:
$s=0$ signifies treble or G clef.
$s=1$ to $4$ mean C-clef, respectively on the first (lowest) through fourth staff line.
$1$ is also called \ital{soprano}, $3$ \ital{alto} and $4$ \ital{tenor}.
$s=5$ to $s=7$ mean F-clef, respectively on the third through fifth staff line. $5$ is
also called \ital{baritone} and $6$ is the normal \ital{bass}.
 $s=8$ is not used.
$s=9$ represents a G clef on the first line, also called \ital{French violin} clef.

The three tokens \verb|\treble|, \verb|\alto|, and \verb|\bass| can be used instead
of a digit for $s$, but only if there would have been but one digit in the string.
So for example the clefs for a standard piano score
could be specified by
\keyindex{setclef}\verb|1{\bass}|.

Treble and bass clefs with the digit 8 above or below are also possible; see 
section~\ref{treblelowoct}.

To set a common key signature for all instruments, use
 \keyindex{generalsignature}\verb|{|$s$\verb|}|,
where $s>0$ is the number of \itxem{sharps} in the
signature and $s<0$ the number of \itxem{flats}\footnote{We once saw a
score in G-minor where the signature consisted of two flats (B and E) plus
one sharp (F). This is not directly supported by \musixtex.}. To override
the common key signature for instrument $n$, use
\keyindex{setsign}\verb|{|$n$\verb|}{|$s$\verb|}|. Note that differing
key signatures cannot be assigned to different staves of the same instrument.

A common \itxem{meter} for all staves can be specified by
 \keyindex{generalmeter}\verb|{|$m$\verb|}|,\label{generalmeter}
where $m$ describes the appearance of the meter indication, and can take several
different forms. If the meter is a \ital{fraction} (e.g.~3/4) the command is
 \verb|\generalmeter{|\keyindex{meterfrac}\verb|{3}{4}}|.
Other possible tokens~$m$ are \keyindex{meterC},
\keyindex{allabreve}, \keyindex{reverseC}, \keyindex{reverseallabreve} and
\keyindex{meterplus}. These are illustrated in the following example:

 %\check
\begin{music}
\generalmeter\meterC
\nostartrule
\parindent0pt\startpiece
\NOtes\qa{cegj}\enotes
\generalmeter\allabreve\changecontext
\NOTes\ha{ce}\enotes
\generalmeter\reverseC\changecontext
\NOTEs\zbreve g\enotes
\generalmeter\reverseallabreve\changecontext
\NOTEs\zwq g\enotes
\generalmeter{\meterfrac{3\meterplus2\meterplus3}8}\changecontext
\Notes\Tqbu ceg\Dqbl jg\Tqbu gec\enotes\setemptybar
\endpiece
\end{music}
\noindent which was coded as:
\begin{quote}\begin{verbatim}
\generalmeter\meterC
\nostartrule
\parindent0pt\startpiece
\NOtes\qa{cegj}\enotes
\generalmeter\allabreve\changecontext
\NOTes\ha{ce}\enotes
\generalmeter\reverseC\changecontext
\NOTEs\zbreve g\enotes
\generalmeter\reverseallabreve\changecontext
\NOTEs\zwq g\enotes
\generalmeter{\meterfrac{3\meterplus2\meterplus3}8}\changecontext
\Notes\Tqbu ceg\Dqbl jg\Tqbu gec\enotes\setemptybar
\endpiece
\end{verbatim}\end{quote}

To override the common meter for any  staff, use
\keyindex{setmeter}\verb|{|$n$\verb|}{{|$m_1$\verb|}{|$m_2$\verb|}{|$m_3$\verb|}{|$m_4$\verb|}}|.\linebreak
This works just like \verb|\setclef|. For example,
\verb|setmeter3{{\meterfrac{12}8}\allabreve}|
sets the meter to 12/8 for the first staff of the third instrument, and
\ital{alla breve} for the second staff.

To insert extra space before the meter is
written, use \keyindex{meterskip}$d$ where $d$ is any hard \TeX\
%dimension\footnote{{\Bslash\tt meterskip} is not a macro but a
dimension\footnote{{\tt\Bslash meterskip} is not a macro but a
dimension register. Whatever follows it
must be a \TeX\ dimension and {\it it must not be enclosed in braces}.}.
The assignment must occur outside \verb|\startpiece...\endpiece| and will be
reset to zero after first meter is posted.

To set an \itxem{instrument name}, use
\keyindex{setname}\verb|{|$n$\verb|}{|\ital{name of the instrument}\verb|}|.
This will place the name in the space to the left of the
first staff or group of staves for instrument $n$. To specify the amount of space
available, use \verb|\parindent|$d$ where $d$ is any hard \TeX\ dimension.
 \subsection{Groupings of instruments}\label{curlybrackets}
By default, all staves in a system will be joined
at the left by a thin, vertical rule. In addition, if an instrument has more than
one staff, they will be joined by a big, curly brace. Now we introduce a way
to delineate groups of instruments or choirs with a square brace containing two parallel
vertical rules, the left one thick and the right one thin. This is commonly used
to group together the voices in a choir.

If there is only one choir, this can be done with
 \begin{quote}
\keyindex{songtop}\onen\\
\keyindex{songbottom}\verb|{|$m$\verb|}|
\end{quote}
\noindent where $m$ and $n$ are the instrument numbers of the first
and last voices. An example is shown in section~\ref{song}.

If there is more than one choir to be set off with
square braces, each one can be specified with
 \begin{quote}
\keyindex{grouptop}\verb|{|$g$\verb|}{|$n$\verb|}|\\
\keyindex{groupbottom}\verb|{|$g$\verb|}{|$m$\verb|}|
\end{quote}
\noindent where $m$ and $n$ are the instrument numbers of the first
and last voices of group number $g$. \musixtex{} allows up to three
groups, numbered from 1 to 3. The command \verb|\songtop| is equivalent
to \verb|\grouptop 1|; \verb|\songbottom| is equivalent to
\verb|\groupbottom 1|.

With \ttxem{musixadd.tex} or \ttxem{musixmad.tex}, the allowable number of 
groups is increased to four.
Alternatively, you can specify the allowable number of groups to $m$ by 
\keyindex{setmaxgroups}\verb|{|$m$\verb|}|%
\footnote{Using $m>4$ may require e-\TeX.}.
\label{musixmad_setmaxgroups}

If any of the instruments grouped this way has more than one staff, the
heavy curly brace will be shifted to the left of the square brace.

Previously defined square braces can be removed by declaring
\verb|\songtop| less than \verb|\songbottom|. The same applies to
\verb|\grouptop| and \verb|\groupbottom| for the same group number.

 An alternate command allows you to specify all choirs at once:
\begin{quote}
 \keyindex{akkoladen}\verb|{{|{\it lower\_1\/}\verb|}{|{\it upper\_1\/}%
   \verb|}{|{\it lower\_2\/}\verb|}{|{\it upper\_2\/}\verb|}{|%
   {\it lower\_3\/}\verb|}{|{\it upper\_3\/}\verb|}}|
\end{quote}
\noindent where {\it lower\_n\/} and {\it upper\_n\/} are instrument
numbers that denote the span of bracket number $n$. For
setting fewer than three brackets, just omit all unneeded
\verb|{|{\it lower\_n\/}\verb|}{|{\it upper\_n\/}\verb|}|~pairs.
For example,
 \verb|\instrumentnumber{5}\akkoladen{{1}{2}{3}{5}}|
\noindent yields the first example below, with five single-staff instruments
divided into two groups.

The second example has 2 instruments, the first (lower) with two staves and the
second with three. Each instrument is set off by default with a curly bracket.

If for some reason you want more than one \ital{instrument}\ grouped
within a curly
bracket, then you can use the extension file
%\ttxem{curly.tex}\footnote{Submitted
\href{http://icking-music-archive.org/software/musixtex/add-ons/curly.tex}
{\underline{\ttxem{curly.tex}}}\footnote{Submitted
by Mthimkhulu {\sc Molekwa} to the mutex list}, which defines the command

 \keyindex{curlybrackets}\verb|{{|{\it lower\_1\/}\verb|}{|{\it upper\_1\/}%
   \verb|}{|{\it lower\_2\/}\verb|}{|{\it upper\_2\/}\verb|}...|

\noindent to be used as illustrated in the third example below.\\
\begin{minipage}[t]{50mm}
\begin{music}
 % just to avoid wasting space ...
 \sepbarrules
 \smallmusicsize
\setsize1\smallvalue\setsize2\smallvalue
\setsize3\smallvalue\setsize4\smallvalue\setsize5\smallvalue
 \instrumentnumber{5} \akkoladen{{1}{2}{3}{5}}
 \startextract\notes\en\bar\notes\en\zendextract
\end{music}
\begin{quote}is coded as:
\begin{verbatim}
 \sepbarrules
 \smallmusicsize
 \setsize1\smallvalue
 \setsize2\smallvalue
 \setsize3\smallvalue
 \setsize4\smallvalue
 \setsize5\smallvalue
 \instrumentnumber{5}
 \akkoladen{{1}{2}{3}{5}}
 \startextract
 \notes\en\bar\notes\en
 \zendextract
\end{verbatim}\end{quote}
\end{minipage}
\begin{minipage}[t]{50mm}
\begin{music}
  \sepbarrules
  \smallmusicsize
\setsize1\smallvalue\setsize2\smallvalue
  \instrumentnumber2 \setstaffs12\setstaffs23
  \startextract \notes\en\bar\notes\en\zendextract
\end{music}
\begin{quote}is coded as:
\begin{verbatim}
 \sepbarrules
 \smallmusicsize
 \setsize1\smallvalue
 \setsize2\smallvalue
 \instrumentnumber2
 \setstaffs12
 \setstaffs23
 \startextract
 \notes\en\bar\notes\en
 \zendextract
\end{verbatim}\end{quote}
\end{minipage}
\begin{minipage}[t]{50mm}
\begin{music}
 \input curly
 \sepbarrules
 \smallmusicsize
 \setsize1\smallvalue\setsize2\smallvalue
 \smallmusicsize
 \setsize3\smallvalue\setsize4\smallvalue\setsize5\smallvalue
 \instrumentnumber5 %\setstaffs12\setstaffs23
 \curlybrackets{1235}
 \startextract \notes\en\bar\notes\en\zendextract
\end{music}
\begin{quote}is coded as:
\begin{verbatim}
 \input curly
 \sepbarrules
 \instrumentnumber5
 \smallmusicsize
 \setsize1\smallvalue
 \setsize2\smallvalue
 \setsize3\smallvalue
 \setsize4\smallvalue
 \setsize5\smallvalue
 \curlybrackets{1235}
 \startextract
 \notes\en\bar\notes\en
 \zendextract
\end{verbatim}\end{quote}
\end{minipage}




\section{Preparing to enter notes}
 \subsection{After the setup, what next?}

The command \keyindex{startmuflex} initiates the serious business of
\musixtex. On the first \TeX\ pass it
opens \ital{jobname}{\tt .mx1} for writing bar-by-bar tabulations of all
hard and scalable space to be fed to \verb|musixflx| on the second pass.
\verb|musixflx| generates \ital{jobname}{\tt .mx2} which defines the
number of bars in each system and the factors relating scalable space
to hard space in each system. On the third pass both files will be opened
and read to define the final spacing. These files should be closed before
leaving \TeX, preferably before
\keyindex{bye} or \keyindex{end}, with \keyindex{endmuflex}. Normally \TeX\
closes all open files on its own when terminating the program, but it is still
cleaner to do this explicitly.

After \verb|\startmuflex|, the command \keyindex{startpiece} will initiate
the first system, containing all instruments you have previously defined.
The indentation will be \keyindex{parindent}, so if you want nonzero
indentation, this register should be set to the desired hard dimension
before issuing \verb|\startpiece|.

 \subsection{Horizontal spacing commands}\label{newspacings}
\subsubsection{Basic note spacing}
\musixtex\ provides a set of macros each of which defines a particular
increment of scalable spacing. The default set is tabulated below:

\begin{center}
\tinynotesize
\renewcommand{\arraystretch}{1.5}% I hate LaTeX
\begin{tabular}{|l|l|l@{~~~~}l|}
\multicolumn{1}{c}{usage}&
\multicolumn{1}{c}{spacing}&
\multicolumn{2}{c}{suggested use for}\\\hline
\keyindex{znotes}\verb| ... & ... & ... \enotes|&(non spacing)
  &&specials\\
\keyindex{notes}\verb|  ... & ... & ... \enotes|&\verb=2=\keyindex{elemskip}
  &\ccu1&16th\\
\keyindex{notesp}\verb| ... & ... & ... \enotes|&\verb=2.5\elemskip=
  &\pt1\ccu1&dotted 16th, 8th triplet\\
\keyindex{Notes}\verb|  ... & ... & ... \enotes|&\verb=3\elemskip=
  &\cu1&8th\\
\keyindex{Notesp}\verb| ... & ... & ... \enotes|&\verb=3.5\elemskip=
  &\cup1&dotted 8th, quarter triplet\\
\keyindex{NOtes}\verb|  ... & ... & ... \enotes|&\verb=4\elemskip=
  &\qu1&quarter\\
\keyindex{NOtesp}\verb| ... & ... & ... \enotes|&\verb=4.5\elemskip=
  &\qup1&dotted quarter, half triplet\\
\keyindex{NOTes}\verb|  ... & ... & ... \enotes|&\verb=5\elemskip=
  &\hu1&half\\
\keyindex{NOTesp}\verb| ... & ... & ... \enotes|&\verb=5.5\elemskip=
  &\hup1&dotted half\\
\keyindex{NOTEs}\verb|  ... & ... & ... \enotes|&\verb=6\elemskip=
  &\wh1&whole\\\hline
\end{tabular}\end{center}
\smallskip

\noindent What each of these macros actually does is to set an
internal dimension register \keyindex{noteskip} to the given multiple
of the fundamental spacing unit \keyindex{elemskip}
(which has dimensions of length, usually given in points).
Normally, every \itxem{spacing note} (e.g.,
\keyindex{qu},
\keyindex{qb}, \keyindex{hl}) will then be followed by a spacing of
width \keyindex{noteskip}.  By selecting a particular note spacing macro
from the above table, the typesetter can thus control the relative spacing
between notes.

The actual spacing will therefore be determined by the value of
\verb|\elemskip|. On the first pass, \TeX\ will set a default value for
\verb|\elemskip| based on the declared music size, or the user can
set it to any hard dimension he chooses. However, the value on the
first pass doesn't matter as much as you might think (more about that later).
On the second pass,
\verb|musixflx| determines where the system breaks will come,
and then computes the final value of \verb|\elemskip| for each system.

If the arthmetic progression of note spacings in the above table does not
meet your wishes, you may activate an alternate set with
the command \keyindex{geometricskipscale}. As implied by the name, this
is a geometric progression, where {\Bslash Notes} is
$\sqrt{2}$ times wider than {\Bslash notes},  {\Bslash NOtes} is $\sqrt{2}$
times wider than {\Bslash Notes}, and so forth.
Then the factors in the middle column
of the above table will be replaced by the sequence
2.00, 2.38, 2.83, 3.36, 4.00, 4.76, 5.66, 6.72, and 8.00. Two additional
macros, \verb|\NOTEsp| and \verb|\NOTES|, will be defined corresponding to
factors 9.52 and 11.32. The original arithmetic progression can be
restored by \keyindex{arithmeticskipscale}.

If neither of the predefined progressions satisfies you, you may define
your own, using the more general macro \verb|\vnotes| in the same manner
that \musixtex\ uses it for the predefined progressions. So for example
\verb|\def\NOtes{\vnotes5.34\elemskip}| will redefine \verb|\NOtes| in
the obvious way, and the extension to the other spacing macros should
likewise be obvious.

In addition, inside any pair \verb|\notes...\enotes| there are two
equivalent ways to
locally redefine \verb|\noteskip| to another scalable value, namely by
issuing a command like
\verb|\noteskip=2.4\noteskip| or \keyindex{multnoteskip}\verb|{2.4}|, which
have the expected effect until the notes group is terminated or
\verb|\noteskip| is further redefined.

Finally, by issuing a command like
\keyindex{scale}\verb|{2.4}| outside any notes group, you can scale all subsequent
\verb|\noteskip|s by any desired factor.

These facilities may be useful, for example,
to control spacing when there are three equal duration notes in one staff against two
 in another.

%\subsubsection{\Bslash{\tt elemskip}, \Bslash{\tt beforeruleskip} and
%\Bslash{\tt afterruleskip}}
%
% avr wants to call these "commands", but das doesn't.
%\subsubsection{Commands {\Bslash\texttt{elemskip}},
%{\Bslash\texttt{beforeruleskip}}
%and {\Bslash\texttt{afterruleskip}}}
\subsubsection{{\Bslash\texttt{elemskip}},
{\Bslash\texttt{beforeruleskip}}
and {\Bslash\texttt{afterruleskip}}}

We've just seen how \verb|\elemskip| is used to scale the spacings between notes.
There are two other spacing units that share some behavior with \verb|\elemskip|.
\verb|\beforeruleskip| is the horizontal space that is automatically inserted
\ital{before} every bar line, while \verb|\afterruleskip| goes \ital{after}
every bar line. (In
practice \verb|\beforeruleskip| is almost aways set to \verb|0pt| because there
will typically already be a space of \verb|1\noteskip| before every barline.)
On the first pass, just as with \verb|\elemskip|, \musixtex\ assigns them default
values according to the following table:

%\begin{quote}\begin{tabular}{lrrr}\hline
%\ital{using}&\keyindex{elemskip}&\keyindex{afterruleskip}
%  &\keyindex{beforeruleskip}\\\hline
%\keyindex{normalmusicsize}&6pt&8pt&0pt\\
%\keyindex{smallmusicsize}&4.8pt&6pt&0pt\\\hline
%\end{tabular}\end{quote}

\begin{center}\begin{tabular}{lrrr}\hline
\ital{using}&\keyindex{elemskip}&\keyindex{afterruleskip}
  &\keyindex{beforeruleskip}\\\hline
\keyindex{normalmusicsize}&6pt&8pt&0pt\\
\keyindex{smallmusicsize}&4.8pt&6pt&0pt\\\hline
\end{tabular}\end{center}

\noindent In the second pass, \verb|musixflx| assigns new values to each
of these dimensions, a different set for each line or system. It does this
in such a way that available scalable horizontal space in each system is
exactly filled up.

The values that are assigned to these dimensions on the first pass, whether by
default or explicitly by the user or in some combination, only matter insofar
as their relative sizes. That's why we earlier stated that the first-pass value
of \verb|\elemskip| didn't matter as much as you might think. For both music sizes
in the table above, it appears that by default \verb|\afterruleskip| is
\verb|1.3333\elemskip|\footnote{Editor's note: It is a mystery why the authors
%of \musixtex\ didn't simply define \Bslash{\tt beforeruleskip} and \Bslash{\tt afterruleskip} as
%specific multiples of \Bslash{\tt elemskip}}.
of \musixtex\ didn't simply define \Bslash\texttt{beforeruleskip} and \Bslash\texttt{afterruleskip} as
specific multiples of \Bslash\texttt{elemskip}}.

Note that if you do want to change any of these
values, you have to do so \ital{after} setting the
music size and before \verb|\startpiece|.

Here is an example that illustrates the various dimensions under discussion:\\
 %\check
\begin{music}\nostartrule
\afterruleskip7pt
\beforeruleskip2pt
\parindent0pt
\setclefsymbol1\empty
% special problems afford special solutions
\makeatletter\global\clef@skip\z@ \makeatother
\startpiece
\zchar{16}{\hbox to\afterruleskip{\downbracefill}}%
\zchar{19}{\hbox to\afterruleskip{\hss a\hss}}%
\addspace\afterruleskip
\zchar{16}{\hbox to2\elemskip{\downbracefill}}%
\zchar{19}{\hbox to2\elemskip{\hss b\hss}}%
\zchar{3}{\hbox to\elemskip{\upbracefill}\hbox to\elemskip{\upbracefill}}%
\zchar{-1}{\hbox to\elemskip{\hss e\hss}\hbox to\elemskip{\hss e\hss}}%
\notes\qa l\en
\znotes\en
\zchar{16}{\hbox to4\elemskip{\downbracefill}}%
\zchar{19}{\hbox to4\elemskip{\hss c\hss}}%
\zchar{3}{\hbox to\elemskip{\upbracefill}\hbox to\elemskip{\upbracefill}%
  \hbox to\elemskip{\upbracefill}\hbox to\elemskip{\upbracefill}}%
\zchar{-1}{\hbox to\elemskip{\hss e\hss}\hbox to\elemskip{\hss e\hss}%
  \hbox to\elemskip{\hss e\hss}\hbox to\elemskip{\hss e\hss}}%
\NOtes\qa l\en
\znotes\en
\zchar{16}{\hbox to\beforeruleskip{\downbracefill}}%
\zchar{19}{\hbox to\beforeruleskip{\hss d\hss}}%
\setemptybar\endpiece
\end{music}
\begin{quote}\begin{tabular}{l@{~$\rightarrow$~}l}
a&\keyindex{afterruleskip}\\
b&\keyindex{notes}\verb| = \vnotes 2\elemskip|\\
c&\keyindex{NOtes}\verb| = \vnotes 4\elemskip|\\
d&\keyindex{beforeruleskip}\\
e&\keyindex{elemskip}
\end{tabular}\end{quote}

\subsection{Moving from one staff or instrument to another}
 \label{movingtostaffs}

When entering notes inside \verb|\notes ... \enotes|, the usual way to suspend
input for one instrument and start the next (higher) is with  the character
``\verb|&|''\index{"&@{\tt\char'046}}. If the instrument has more than one
staff, to switch to the next (higher) one you can use the character
``\verb+|+''\index{\tt\char'174@{\tt\char'174}}.

There are some alternate navigation commands that may be useful in
 special situations. Due to ``catcode
problems'' (see section~\ref{catcodeprobs}) it may sometimes be necessary
to use the more explicit commands
\keyindex{nextinstrument} and \keyindex{nextstaff}, which have the same
meanings as ``\verb|&|'' and ``\verb+|+'' respectively. To switch to the previous
(next lower) staff of the same instrument, use \keyindex{prevstaff}. This might
be useful if a beam starts in a higher staff than where it ends. More
generally, to switch to an arbitrary instrument $n$, use
\keyindex{selectinstrument}\onen, and to switch to an arbitrary staff $n$
of the current instrument, use \keyindex{selectstaff}\verb|{|$n$\verb|}|. In the
latter case if $n$ exceeds the number of staves defined for the instrument, you
will receive an error message. You can enter part of a successive voice on
\ital{same} staff by using \verb|\selectstaff{|$n$\verb|}| with $n$ for
the \ital{current} staff.


 %\check
\section{Note pitch specification}\label{pitchspec}
Note pitches can be specified either by letters or numbers. If no transposition
or octaviation is in effect, letters ranging
from \verb|a| to \verb|z| represent notes starting with the A below
middle C. Upper case letters from \verb|A| to \verb|N| represent pitches
two octaves lower than their lower case counterparts. Any letter can be used in
any clef, but some users may
prefer to use the lower case letters in treble clef, and the upper case ones
in bass clef.

Alternatively, a one- or two-digit, positive or negative integer can always be used.
The number represents
the vertical position on the staff, with \verb|0| for the lowest line and
\verb|1| for the space right above, \ital{regardless of the clef}.
Unlike with letters, the
associated pitch will depend on the clef, and notes entered this way are
immune to transposition and octaviation.

Notes lower than \verb|A| and higher than \verb|z| can be entered, with
either numbers as just described, or with octaviation as will be explained in
section~\ref{octaviation}.

 \section{Writing notes}\label{autostemdirections}
There are two major kinds of note macros, those that include a space (of
length \keyindex{noteskip}) after
the printed symbol, and those that don't cause any space. A single-line melody would be
written using the first type. All notes of a chord except the last would
use the second.

Another distinction concerns stemmed notes. Some macros explicitly set the
stem direction with either ``\verb|u|'' or ``\verb|l|'' contained in the name of
the macro. On the other hand, an ``\verb|a|'' in the macro's name usually signifies
\ital{automatic} stem direction selection. In this case notes below the middle
staff line will get up stems, otherwise down.

 \subsection{Normal (unbeamed) spacing notes}\label{NormalNotes}
In the following, \verb|{|$p$\verb|}| signifies a pitch specification as
described in sections~\ref{pitchspec} and \ref{octaviation}. However it
is understood that if the pitch is a single character, the brackets are
not neecessary, provided that if it is a letter, a space separates the
macro from the letter.


\begin{quote}\begin{description}\setlength{\itemsep}{0ex}
 \item[\keyindex{breve}{\tt\char123}$p${\tt\char125} :]breve (\hbox to 8pt{\zbreve1\hss}) .
 \item[\keyindex{longa}{\tt\char123}$p${\tt\char125} :]longa (\hbox to8pt{\zlonga1\hss}) .
 \item[\keyindex{longaa}{\tt\char123}$p${\tt\char125} :]longa with automatic stem
direction\footnote{Editor's note: Evidently there is no explicit up-stemmed longa} .
 \item[\keyindex{zmaxima}{\tt\char123}$p${\tt\char125} :]maxima(\hbox to16pt{\zmaxima1\hss}) .
 \item[\keyindex{wq}{\tt\char123}$p${\tt\char125} :]arbitrary duration note (\hbox to8pt{\zwq1\hss})
(also used as alternate representation of a \ital{breve}).
 \item[\keyindex{wqq}{\tt\char123}$p${\tt\char125} :]long arbitrary duration note
(\hbox to8pt{\zwqq1\hss}) (also used as alternate
representation of a \ital{longa}).
 \item[\keyindex{wh}{\tt\char123}$p${\tt\char125} :]whole note.
 \item[\keyindex{hu}{\tt\char123}$p${\tt\char125} :]half note with stem up.
 \item[\keyindex{hl}{\tt\char123}$p${\tt\char125} :]half note with stem down.
 \item[\keyindex{ha}{\tt\char123}$p${\tt\char125} :]half note with automatic stem direction
 \item[\keyindex{qu}{\tt\char123}$p${\tt\char125} :]quarter note with stem up.
 \item[\keyindex{ql}{\tt\char123}$p${\tt\char125} :]quarter note with stem down.
 \item[\keyindex{qa}{\tt\char123}$p${\tt\char125} :]quarter note with automatic stem direction.
 \item[\keyindex{cu}{\tt\char123}$p${\tt\char125} :]eighth note\footnote{The ``{\tt c}''
within this macro name stands for the equivalent British term ``crotchet''} with stem up.
 \item[\keyindex{cl}{\tt\char123}$p${\tt\char125} :]eighth note with stem down.
 \item[\keyindex{ca}{\tt\char123}$p${\tt\char125} :]eighth note with automatic stem direction.
 \item[\keyindex{ccu}{\tt\char123}$p${\tt\char125} :]sixteenth note with stem up.
 \item[\keyindex{ccl}{\tt\char123}$p${\tt\char125} :]sixteenth note with stem down.
 \item[\keyindex{cca}{\tt\char123}$p${\tt\char125} :]sixteenth note with automatic stem direction.
 \item[\keyindex{cccu}{\tt\char123}$p${\tt\char125} :]32nd note with stem up.
 \item[\keyindex{cccl}{\tt\char123}$p${\tt\char125} :]32nd note with stem down.
 \item[\keyindex{ccca}{\tt\char123}$p${\tt\char125} :]32nd note with automatic stem direction.
 \item[\keyindex{ccccu}{\tt\char123}$p${\tt\char125} :]64th note with stem up.
 \item[\keyindex{ccccl}{\tt\char123}$p${\tt\char125} :]64th note with stem down.
 \item[\keyindex{cccca}{\tt\char123}$p${\tt\char125} :]64th note with automatic stem direction.
 \item[\keyindex{cccccu}{\tt\char123}$p${\tt\char125} :]128nd note with stem up.
 \item[\keyindex{cccccl}{\tt\char123}$p${\tt\char125} :]128nd note with stem down.
 \item[\keyindex{ccccca}{\tt\char123}$p${\tt\char125} :]128nd note with automatic stem direction.
\end{description}\end{quote}
 As an example, the sequence

 %\check
\begin{music}\nostartrule
\startextract
\Notes\cu c\cl j\enotes\bar
\Notes\ccu c\ccl j\enotes\bar
\Notes\cccu c\cccl j\enotes\bar
\Notes\ccccu c\ccccl j\enotes\bar
\Notes\cccccu c\cccccl j\enotes
\zendextract
\end{music}
 \noindent was coded as
\begin{quote}\begin{verbatim}
\Notes\cu c\cl j\enotes\bar
\Notes\ccu c\ccl j\enotes\bar
\Notes\cccu c\cccl j\enotes\bar
\Notes\ccccu c\ccccl j\enotes\bar
\Notes\cccccu c\cccccl j\enotes
\end{verbatim}\end{quote}

 \subsection{Non-spacing note heads}
These macros are used to create chords. Any number of them can be entered
in sequence, followed by a spacing note. All of the note heads will be
joined to the spacing note and the stem length will automatically be adjusted
as needed.

\begin{quote}\begin{description}\setlength{\itemsep}{0ex}
 \item[\keyindex{zq}{\tt\char123}$p${\tt\char125} :]quarter (or shorter) note head.
 \item[\keyindex{zh}{\tt\char123}$p${\tt\char125} :]half note head.
\end{description}\end{quote}
%das ???
% \begin{remark} Notes of duration longer than whole notes are
%always non-spacing. This saves one useless definition, since these notes are
%always longer than other simultaneous ones. If needed they can be followed by
%\keyindex{sk} to force spacing.
%\end{remark}

 \subsection{Shifted non-spacing note heads}
These symbols are used mainly
in chords containing an interval of a \ital{second}. They provide note
heads shifted either to the left or right of the default position by
the width of one note head.

\begin{quote}\begin{description}\setlength{\itemsep}{0ex}
\item[\keyindex{rw}{\tt\char123}$p${\tt\char125} :]whole note head shifted right.
\item[\keyindex{lw}{\tt\char123}$p${\tt\char125} :]whole note head shifted left.
\item[\keyindex{rh}{\tt\char123}$p${\tt\char125} :]half note head shifted right\footnote{Some may not
have realized that half and whole note heads have different shapes}.
\item[\keyindex{lh}{\tt\char123}$p${\tt\char125} :]half note head shifted left.
\item[\keyindex{rq}{\tt\char123}$p${\tt\char125} :]quarter note head shifted right.
\item[\keyindex{lq}{\tt\char123}$p${\tt\char125} :]quarter note head shifted left.
\end{description}\end{quote}

 \subsection{Non-spacing notes}
These macros provide normal notes, with stems if applicable, but without any
following space.
\begin{quote}\begin{description}\setlength{\itemsep}{0ex}
 \item[\keyindex{zhu}\pitchp~:]half note with stem up but no spacing. It acts like
\verb|\hu| for chord building, i.e., it will join together any immediately
preceding non-spacing note heads.
 \item[\keyindex{zhl}\pitchp~:]half note with stem down but no spacing. It acts like
\keyindex{hl} for chord building.
 \item[\keyindex{zqu}\pitchp~:]quarter note with stem up but no spacing. It acts like
\verb|\qu| for chord building.
 \item[\keyindex{zql}\pitchp~:]quarter note with stem down but no spacing. It acts
like \verb|\ql| for chord building.
 \item[\keyindex{zcu}\pitchp, \keyindex{zccu}, \keyindex{zcccu}, 
 \keyindex{zccccu}, \keyindex{zcccccu}
:]eighth, ..., note with stem up but no spacing. They act like
\verb|\cu| for chord building.
 \item[\keyindex{zcl}\pitchp, \keyindex{zccl}, \keyindex{zcccl}, 
 \keyindex{zccccl}, \keyindex{zcccccl}
:]eighth, ..., note with stem down but no spacing. They act
like \verb|\cl| for chord building.
 \item[\keyindex{zqb}\pitchp~:]note belonging to a beam but no spacing. (DAS: put this later!)
 \item[\keyindex{rhu}\pitchp, \keyindex{rhl}, \keyindex{rqu}, \keyindex{rql},
  \keyindex{rcu}, \keyindex{rcl} :] \verb|\rhu| acts like \verb|\zhu|,
  but the note is shifted one note width to the right; others analogous.
 \item[\keyindex{lhu}\pitchp, \keyindex{lhl}, \keyindex{lqu}, \keyindex{lql},
  \keyindex{lcu}, \keyindex{lcl} :]same
  as above, but the note is shifted one note width to the left.
 \item[\keyindex{zw}\pitchp~:]whole note with no following space.
 \item[\keyindex{zwq}\pitchp~:]arbitrary duration note
  (\hbox to8pt{\zwq1\hss}) with no following space.
 \item[\keyindex{zbreve}\pitchp~:]breve
  (\hbox to8pt{\zbreve1\hss}) with no following space.
 \item[\keyindex{zlonga}\pitchp~:]longa
  (\hbox to8pt{\zlonga1\hss}) with no following space.
 \item[\keyindex{zmaxima}\pitchp~:]maxima
  (\hbox to16pt{\zmaxima1\hss}) with no following space.
\end{description}\end{quote}
 %\check

\subsection{Spacing note heads}

Although not needed in normal music scores, these may be useful in
very special cases.

\begin{quote}\begin{description}\setlength{\itemsep}{0ex}
\item[\keyindex{nh}{\tt\char123}$p${\tt\char125} :]spacing half note head.
\item[\keyindex{nq}{\tt\char123}$p${\tt\char125} :]spacing quarter note head.
\end{description}\end{quote}

\noindent As an example, the sequence

\begin{music}\nostartrule
 \startextract
\notes\nq c\nq j\enotes\barre
\Notes\nh c\nh j\enotes\barre
\notes\nq {cdef}\enotes
\zendextract
\end{music}

\noindent was coded as

\begin{quote}\begin{verbatim}
\notes\nq c\nq j\enotes\barre
\Notes\nh c\nh j\enotes\barre
\notes\nq {cdef}\enotes
\end{verbatim}\end{quote}

Non spacing variants are also provided, namely
\keyindex{znh} and \keyindex{znq}.
% DAS: why???

 \subsection{Dotted notes}\label{dots}

By appending one or two \verb|p|'s (for ``pointed'') to the name, many of the
macros just introduced provide one or two dots after the notehead:
\keyindex{whp}\pitchp, \keyindex{whpp},
\keyindex{zwp}, \keyindex{zwpp},
\keyindex{hup}, \keyindex{hupp},
\keyindex{hlp}, \keyindex{hlpp},
\keyindex{zhp}, \keyindex{zhpp},
\keyindex{qup}, \keyindex{qupp},
\keyindex{qlp}, \keyindex{qlpp},
\keyindex{zqp}, \keyindex{zqpp},
\keyindex{cup}, \keyindex{cupp},
\keyindex{clp}, \keyindex{clpp},
\keyindex{qbp} and \keyindex{qbpp}.
Naturally, the ones that start with ``\verb|z|'' are used in chords.
The dot(s) will be raised if the note is on a line.

A more explicit way uses one of the macros \keyindex{pt}{\tt\char123}$p${\tt\char125},
\keyindex{ppt}, or \keyindex{pppt} right
before any note macro to place one to three dots after the normal note
head at pitch $p$. Again they will be raised if on a line. In fact this is the only
way to get a triple-dotted note.
For example a
quarter note with one dot could be coded \verb|\pt h\qu h|, with two dots
as \keyindex{ppt}\verb| h\qu h| and with three as
\keyindex{pppt}\verb| h\qu h|.

Yet another method for posting a dot is to insert a \ital{period} before
the letter representing the pitch. Thus \verb|\qu{.a}| is equivalent to
either \verb|\pt a\qu a| or \verb|\qup a| . This may be useful when
using \ital{collective coding}, which will be discussed in the next
section.

Non-spacing dotted notes can be produced using
\keyindex{zhup}, \keyindex{zhlp}, \keyindex{zqup}, \keyindex{zqlp},
\keyindex{zcup}, \keyindex{zclp}, \keyindex{zqbp},
and similarly with two \verb|p|'s for
double-dotted notes.

As a matter of style, if two voices share one staff, the dots in
the lower voice should be lowered if the note is on a line. For
this you can use \keyindex{lpt}{\tt\char123}$p${\tt\char125} and
\keyindex{lppt}{\tt\char123}$p${\tt\char125}.

\subsection{Sequences of equally spaced notes; collective coding}
\label{CollectiveCoding}

It isn't necessary to write a separate macro
sequence \verb|\notes...\enotes| for every individual column of notes.
Rather, a single such macro can contain all the notes in all staves
over an extended horizontal range, as long as all spacings are equal or
multiples of a unique value of \keyindex{noteskip}.
The notes in each staff could be entered one after another as normal
spacing notes as already described in section~\ref{NormalNotes}. Then
each spacing note will cause the insertion point to advance horizontally by the
operative value of \verb|\noteskip| defined by the choice of
\verb|\notes|, \verb|\Notes|, \verb|\NOtes|, etc. Of course in such sequences
non-spacing chord notes can be entered right before their associated
spacing note. If you need to skip forward by one \verb|\noteskip|, for
example after a quarter note when there are two eighth notes in
another staff, you can use \keyindex{sk}.

If there are only spacing notes in such a sequence, a further
simplification is available, called \ital{collective coding}. For instance
\verb|\qu{cdefghij}| writes the C major scale in quarters with
up stems. Similarly \verb|\cl{abcdef^gh}| writes the
\ital{A-minor} scale in non-beamed eighths. (Here ``\verb|^|'' represents
a sharp). If necessary a void can be inserted in a collective coding sequence
by using~\verb|*|\index{*}. Not all note generating macros can be
used to perform collective coding, but most of them can.

\section{Beams}
\subsection{Starting a beam}

Each beam must be declared with a macro issued before the first spacing
note under the beam is coded. Two distinct kinds of macros are provided
for this. The first kind initiates a ``fixed-slope'' beam, with an arbitrary
slope and starting height chosen by the user, while the second kind, a
``semi-automatic'' beam,
\ital{computes}~the slope and, in addition, adjusts the starting height in some
cases.

\def\nps{{\tt\char123}$n${\tt\char125}\pitchp{\tt\char123}$s${\tt\char125}}

The basic form of the macros for starting fixed-slope beams is exemplified
by the one for a single upper beam, \keyindex{ibu}\nps. Here
$n$ is the reference number of the beam, $p$ the starting ``pitch'', and
$s$ the slope. 

The reference number is assigned by the user. 
It is needed because more than one beam may be open at 
a time, and it tells \musixtex\ to which beam subsequent beamed notes and 
other beam specification commands are assigned. 
By default, the reference number must be in the range [0-5], 
but the range for 8th to 128th beams will be expanded to [0-8] or [0-11] 
if \verb|musixadd| or \verb|musixmad| respectively has been \verb|\input|. 

Alternatively, you can specify the number of 8th to 64th beams 
directly\footnote{8th to 64th beams are so basic that the maximum 
number of these beams are related with the maximum number of instruments 
by this command. 
Using $m>12$ may require e-\TeX.}
with \keyindex{setmaxinstruments}\verb|{|$m$\verb|}| 
within the range $7<m\leq 100$; the corresponding reference number may then be 
in the range from 0 to $(m-1)$. 
\label{musixmad_setmaxinstruments_ccxviiibeams}  
For 128th beams, use \keyindex{setmaxcxxviiibeams}\verb|{|$m$\verb|}|.

For 256th notes, which can only appear in beams, see section \ref{musixbbm}.

The ``pitch'' parameter $p$ is a pitch that is three
staff spaces \ital{below} the bottom of the heavy connecting bar (\ital{above}
the bar for a lower beam); in many (but not all)
cases it should be input as the actual pitch of the first note. The slope $s$
is an integer in the range [-9,9]. When multipled
by 5\%~it gives the actual slope of the heavy bar. Typically a slope of 2 or 3
is OK for ascending scales, and 6 to 9 for ascending arpeggios.

The full set of fixed-slope beam initiation macros is as follows:
\begin{quote}\begin{description}\setlength{\itemsep}{0ex}

 \item[\keyindex{ibu}\nps~:]initiates an \ital{upper beam}.
 \item[\keyindex{ibl}\nps~:]initiates a \ital{lower beam}.
 \item[\keyindex{ibbu}\nps~:]initiates a \ital{double upper beam}.
 \item[\keyindex{ibbl}\nps~:]initiates a \ital{double lower beam}.
 \item[\keyindex{ibbbu}\nps~:]initiates a \ital{triple upper beam}.
 \item[\keyindex{ibbbl}\nps~:]initiates a \ital{triple lower beam}.
 \item[\keyindex{ibbbbu}\nps~:]initiates a \ital{quadruple upper beam}.
 \item[\keyindex{ibbbbl}\nps~:]initiates a \ital{quadruple lower beam}.
\end{description}\end{quote}

A semi-automatic beam is initiated with a command that has \ital{four}
parameters, the beam number, the first and last pitches, and the total
horizontal extent in
\verb|\noteskip|s, based on the value in effect at the start. For example,
if you input \verb|\Ibu2gj3|, \musixtex\ will understand that you want to
build an upper beam (beam number 2) horizontally extending \verb|3\noteskip|,
the first note of which is a \verb|g| and the last note a \verb|j|.
Knowing these parameters it will choose the highest slope number that
corresponds to a slope not more than $(\hbox{\tt j}-\hbox{\tt
g})/(3\keyindex{noteskip})$. The nominal height of the heavy bar is offset the same as
for fixed-slope beams. However, if there is no sufficiently steep beam
slope available, then \musixtex\ will raise (or lower) the starting point.

 Eight such macros are available: \keyindex{Ibu}, \keyindex{Ibbu},
\keyindex{Ibbbu}, \keyindex{Ibbbbu}, \keyindex{Ibl}, \keyindex{Ibbl},
\keyindex{Ibbbl} and \keyindex{Ibbbbl}.

\subsection{Adding notes to a beam}

Spacing notes belonging to beams are coded with the macro
\keyindex{qb}\verb|{|$n$\verb|}|\pitchp~where $n$ is
the beam number and $p$ the pitch of the note. \musixtex\ adjusts the
length of the note stem to link to the beam.

Chord notes within a beam are entered before the main note with the
non-spacing macro \keyindex{zqb}\verb|{|$n$\verb|}|\pitchp. Again,
the stem length will be automatically adjusted as required.

There are also special macros for semi-automatic beams with
two, three, or four notes:
\keyindex{Dqbu}, \keyindex{Dqbl}, \keyindex{Dqbbu}, \keyindex{Dqbbl},
\keyindex{Tqbu}, \keyindex{Tqbl}, \keyindex{Tqbbu}, \keyindex{Tqbbl},
\keyindex{Qqbu}, \keyindex{Qqbl}, \keyindex{Qqbbu} and \keyindex{Qqbbl}.
  %\check
For example \verb|\Dqbu gh| is equivalent to \verb|Iqbu1gh\qb1 g\tbu1\qb1 h|,
except that the special macros don't require a beam number.
Their use is illustrated in the following example:

\medskip
\begin{music}\nostartrule
\parindent0pt\startpiece
\Notes\Dqbu gh\Dqbl jh\en
\notes\Dqbbu fg\Dqbbl hk\en\bar
\Notes\Tqbu ghi\Tqbl mmj\en
\notes\Tqbbu fgj\Tqbbl njh\en\bar
\Notes\Qqbu ghjh\Qqbl jifh\en
\notes\Qqbbu fgge\Qqbbl jhgi\en\setemptybar\endpiece
\end{music}
\noindent This was coded as\footnote{Editor's note: Most music
typesetting books recommend beam slopes that are \ital{less} than
the slope between the starting and ending note; these macros cannot
provide that.}:
\begin{quote}\begin{verbatim}
\Notes\Dqbu gh\Dqbl jh\en
\notes\Dqbbu fg\Dqbbl hk\en\bar
\Notes\Tqbu ghi\Tqbl mmj\en
\notes\Tqbbu fgj\Tqbbl njh\en\bar
\Notes\Qqbu ghjh\Qqbl jifh\en
\notes\Qqbbu fgge\Qqbbl jhgi\en
\end{verbatim}\end{quote}
 %\check

 \subsection{Ending a beam}
 The termination of a given
beam must be declared \ital{before} coding the last spacing note
connected to that beam. The macros for doing that are
\keyindex{tbu}\verb|{|$n$\verb|}| for an upper beam and
\keyindex{tbl}\verb|{|$n$\verb|}| for a lower one. These work for
beams of any
multiplicity. So for example an upper triple beam with
32nd notes is initiated by
\verb|\ibbbu|\nps\ but terminated by \verb|\tbu{|$n$\verb|}|.

 Since beams usually finish with a \verb|\qb| for the last note, the
following shortcut macros have been provided:

\def\enpee{{\tt{\char123$n$\char125\char123$p$\char125}}}
\begin{quote}\begin{description}\setlength{\itemsep}{0ex}
 \item \keyindex{tqb}\enpee~is equivalent to \verb|\tbl{|$n$\verb|}\qb|\enpee~.
 \item \keyindex{tqh}\enpee~is equivalent to \verb|\tbu{|$n$\verb|}\qb|\enpee~.
 \item \keyindex{ztqb}\enpee~is equivalent to \verb|\tbl{|$n$\verb|}\zqb|\enpee~,
i.e., no spacing afterwards.
 \item \keyindex{ztqh}\enpee~is equivalent to \verb|\tbu{|$n$\verb|}\zqb|\enpee~,
i.e., no spacing afterwards.
\end{description}\end{quote}

\subsection{Changing multiplicity after the beam starts}
Multiplicity (the number of heavy bars) can be increased at any position after
the beam starts. The commands are \keyindex{nbbu}\onen~which increases the multiplicity of
upper beam
number $n$ to two starting at the current position, \keyindex{nbbbu}\onen~to
increase it to three, and \keyindex{nbbbbu}\onen~to increase to four. The
commands \keyindex{nbbl}\onen\dots\keyindex{nbbbbl}\onen~do the same
for lower beams.
Thus, the
sequence

\begin{music}\nostartrule
\startextract
\Notes\ibu0h0\qb0e\nbbu0\qb0e\nbbbu0\qb0e\nbbbbu0\qb0e\tbu0\qb0e\enotes
\zendextract
\end{music}
\noindent has been coded as
\begin{quote}\begin{verbatim}
\Notes\ibu0h0\qb0e\nbbu0\qb0e\nbbbu0\qb0e\nbbbbu0\qb0e\tbu0\qb0e\enotes
\end{verbatim}\end{quote}

To decrease multiplicity to one, use \keyindex{tbbu}\onen~or
\keyindex{tbbl}\onen. To decrease to two or three use
\keyindex{tbbbu}\onen\dots\keyindex{tbbbbl}\onen. For example,

\begin{music}\nostartrule
\startextract
\Notes\ibbbu0h0\qb0e\tbbbu0\qb0e\tbbu0\qb0e\tbu0\qb0e\enotes
\zendextract
\end{music}
\noindent has been coded as

\begin{quote}\begin{verbatim}
\startextract
\Notes\ibbbu0h0\qb0e\tbbbu0\qb0e\tbbu0\qb0e\tbu0\qb0e\enotes
\zendextract
\end{verbatim}\end{quote}

Although at first it may seem counterintuitive,
the macros \keyindex{tbbu} and \keyindex{tbbl} and higher order counterparts
may also be invoked when the multiplicity is one. In this case
a second, third, or fourth heavy bar will be opened one note width \ital{before}
the current stem, and immediately closed \ital{at} the stem.
Thus the following sequences

\begin{music}\nostartrule
\let\extractline\hbox
\hbox to \hsize{%
\hss
  \startextract
  \Notes\ibu0e0\qbp0e\tbbu0\tbu0\qb0e\en
  \zendextract
\hss\hss
  \startextract
  \Notes\ibu0e0\qbpp0e\tbbbu0\tbbu0\tbu0\qb0e\en
  \zendextract
\hss}
\end{music}
\noindent are coded
\hspace*{\fill}\begin{minipage}{.4\textwidth}\begin{verbatim}
\Notes\ibu0e0\qbp0e%
  \tbbu0\tbu0\qb0e\en
\end{verbatim}\end{minipage}\hfill
\begin{minipage}{.4\textwidth}\begin{verbatim}
\Notes\ibu0e0\qbpp0e%
  \tbbbu0\tbbu0\tbu0\qb0e\en
\end{verbatim}\end{minipage}\hfill

\medskip


The symmetrical pattern is also possible. For example:

\begin{music}\nostartrule
\startextract
\Notes\ibbl0j0\roff{\tbbl0}\qb0j\tbl0\qbp0j\enotes
\zendextract
\end{music}
\noindent has been coded as:
\begin{quote}\begin{verbatim}
\Notes\ibbl0j0\roff{\tbbl0}\qb0j\tbl0\qbp0j\enotes
\end{verbatim}\end{quote}

The constructions in this section illustrate some general
properties of
beam initiation and termination commands: To mate properly with the
expected stems, the starting position of the heavy bar(s) (for initiation commands)
and the ending position (for terminations) will be at
different horizontal locations
depending on whether they are for upper or lower beams: The position for
upper beam commands is one note head width to the right of those for
lower beams. In fact this is the \ital{only} difference between upper
and lower termination commands. Both types will operate on whatever kind of
beam is open and has the same beam number.

Recognizing this principle, in the example just given it was necessary to
shift the double termination to the right by one note head width, using the
command \keyindex{roff}\verb|{|\dots\verb|}|, which does precisely that for
any \musixtex\ macro.

Here is another, slightly more complicated example which also uses
\verb|\roff|:

\begin{music}\nostartrule
\startextract
\notes\ibbbu0e0\roff{\tbbbu0}\qb0f\en
\notesp\tbbu0\qbp0f\en
\Notes\tbu0\qb0f\en
\notesp\ibbu0f0\roff{\tbbu0}\qbp0f\en
\Notes\qb0f\en
\notes\tbbbu0\tbbu0\tbu0\qb0f\en
\zendextract
\end{music}
\noindent has been coded as:
\begin{quote}\begin{verbatim}
\notes\ibbbu0e0\roff{\tbbbu0}\qb0f\en
\notesp\tbbu0\qbp0f\en
\Notes\tbu0\qb0f\en
\notesp\ibbu0f0\roff{\tbbu0}\qbp0f\en
\Notes\qb0f\en
\notes\tbbbu0\tbbu0\tbu0\qb0f\en
\end{verbatim}\end{quote}

\noindent Note that the first beam opening command used a pitch one step
below the note. This makes the stem shorter by one pitch unit, since it is always
the \ital{closest} heavy bar that is separated from the given pitch by
three staff spaces.

We close this section with an example showing how to
open a beam of one sense, increase multiplicity, then terminate with
opposite sense:

\begin{music}\nostartrule
\startextract
\Notes\ibl0p0\qb0p\nbbl0\qb0p\nbbbl0\qb0p\tbu0\qb0e\enotes
\zendextract
\end{music}
\noindent which has been coded as
\begin{quote}\begin{verbatim}
\Notes\ibl0p0\qb0p\nbbl0\qb0p\nbbbl0\qb0p\tbu0\qb0e\enotes
\end{verbatim}\end{quote}

 \begin{remark} One may save some typing by defining personalized
\TeX\ macros to
perform any oft repeated sequence of commands. For example,
one could define a set of four sixteenths by the macro:

\verb|\def\qqh#1#2#3#4#5{\ibbl0#2#1\qb#2\qb#3\qb#4\tbl0\qb#5}|

\noindent where the first argument is the slope and the other four
arguments are the pitches of the four successive sixteenths.
\end{remark}

 %\check
 \subsection{Shorthand beam notations for repeated notes}\index{repeated patterns}
Sometimes you may want to indicate repeated short notes with open note heads joined
by a beam. Here's an example of how to do that
using the \keyindex{hb} macro:

\begin{music}\nostartrule
\startextract
\Notes\ibbl0j0\hb0j\tbl0\hb0j\enotes
\Notes\ibbu0g0\hb0g\tbu0\hb0g\enotes
\zendextract
\end{music}
\noindent which has been coded as:
\begin{quote}\begin{verbatim}
\Notes\ibbl0j0\hb0j\tbl0\hb0j\enotes
\Notes\ibbu0g0\hb0g\tbu0\hb0g\enotes
\end{verbatim}\end{quote}

It is also possible to dispense with the stems:

\begin{music}\nostartrule
\startextract
\Notes\ibbl0j3\wh j\tbl0\wh l\enotes
\Notes\ibbu0g3\wh g\tbu0\wh i\enotes
\zendextract
\end{music}
\noindent which was coded as
\begin{quote}\begin{verbatim}
\Notes\ibbl0j3\wh j\tbl0\wh l\enotes
\Notes\ibbu0g3\wh g\tbu0\wh i\enotes
\end{verbatim}\end{quote}
\noindent A different look could be obtained as follows:

\begin{music}\nostartrule
\startextract
\Notes\loff{\zw j}\ibbl0j3\sk\tbl0\wh l\enotes
\Notes\ibbu0g3\wh g\tbu0\roff{\wh i}\enotes\qspace
\zendextract
\end{music}
\noindent which was coded as:
\begin{quote}\begin{verbatim}
\Notes\loff{\zw j}\ibbl0j3\sk\tbl0\wh l\enotes
\Notes\ibbu0g3\wh g\tbu0\roff{\wh i}\enotes\qspace
\end{verbatim}\end{quote}

Yet another way to indicate repeated notes is given in the
following example:

 \begin{music}\nostartrule
\startextract
\Notes\ibl0h0\qb0{hhh}\tbl0\qb0h\bsk\bsk\bsk\bsk
      \ibu0j0\qb0{jjj}\tbu0\qb0j\en
\NOTes\loffset{0.5}{\ibl0j9}\roffset{0.5}{\tbl0}\zhl h%
      \loffset{0.5}{\ibu0g9}\roffset{0.5}{\tbu0}\hu j\en\bar
\notes\ibbl0i0\qb0{hhh}\tbl0\qb0h\bsk\bsk\bsk\bsk
      \ibbu0i0\qb0{jjj}\tbu0\qb0j%
      \ibbl0i0\qb0{hhh}\tbl0\qb0h\bsk\bsk\bsk\bsk
      \ibbu0i0\qb0{jjj}\tbu0\qb0j\en
\NOTes\loffset{0.5}{\ibbl0k9}\roffset{0.5}{\tbl0}\zhl h%
      \loffset{0.5}{\ibbu0f9}\roffset{0.5}{\tbu0}\hu j\en
\zendextract
 \end{music}
 \noindent whose coding (due to Werner {\sc Icking}) is

\begin{quote}\begin{verbatim}
\Notes\ibl0h0\qb0{hhh}\tbl0\qb0h\bsk\bsk\bsk\bsk
      \ibu0j0\qb0{jjj}\tbu0\qb0j\en
\NOTes\loffset{0.5}{\ibl0j9}\roffset{0.5}{\tbl0}\zhl h%
      \loffset{0.5}{\ibu0g9}\roffset{0.5}{\tbu0}\hu j\en\bar
\notes\ibbl0i0\qb0{hhh}\tbl0\qb0h\bsk\bsk\bsk\bsk
      \ibbu0i0\qb0{jjj}\tbu0\qb0j%
      \ibbl0i0\qb0{hhh}\tbl0\qb0h\bsk\bsk\bsk\bsk
      \ibbu0i0\qb0{jjj}\tbu0\qb0j\en
\NOTes\loffset{0.5}{\ibbl0k9}\roffset{0.5}{\tbl0}\zhl h%
      \loffset{0.5}{\ibbu0f9}\roffset{0.5}{\tbu0}\hu j\en
\end{verbatim}\end{quote}

 \subsection{Beams that cross line breaks}

Although careful typesetting can usually avoid it, occasionally
a beam may need to cross a line break. If so, it must be manually terminated at
the end of one line and continued in the next. This
can be done by shifting beam terminations and initiations
using \keyindex{roff} and/or
\keyindex{loff}, or by inserting a spacing command such as
\keyindex{hsk}. We give an example from {\sc Grieg}'s ``Hochzeit auf
Troldhaugen'':\index{Grieg, E.@{\sc Grieg, E.}}\medskip

\begin{music}
\parindent0pt
\def\rqs{\lower\Interligne\rlap\qs}
\def\snotes{\vnotes1\elemskip}
\setstaffs1{2}
\generalsignature{2}
\setclef1{\bass}
\interstaff{12}
\startpiece
%%% bar 1
\addspace\afterruleskip
\snotes|\tinynotesize\ibsluru0n\ibbu0m5\qb0{=m}\tqh0n\en
\qspace
\Notes\zchar{-7}\sPed\loffset{.3}{\fl E}\zq E\qu{_I}%
  |\zql g\ibu2l0\busf2\qb2{=m}\en
\Notes|\tubslur0o\qb2{_l}\en
\Notes\loffset{.3}{\fl L}\zq L\ibl0L0\qb0{_b}%
  |\ibl1h0\zqb1g\bupz2\qb2l\en
\Notes\zq L\tqb0b|\tbl1\zqb1g\bupz2\tqh2l\en
\NOtes\zq L\ql b|\zql g\qu l\en
\notes\zchar{-7}\sPed\zchar{-7}{\eightit ~~~sempre}%
  \zchar{14}{\pp\eightit~sempre}\zq I\ibbu1J0\qb1L|\qs\en
\notes\qs|\zq N\ibbu3d0\qb3{_d}\en
\notes\qb1E|\rqs\en
\notes\qs|\fl e\zq N\rq e\qb3d\en
%%% bar 2
\bar
\notes\loffset{.3}{\fl I}\zq I\qb1{_L}|\rqs\en
\qspace
\notes\qs|\lfl d\zq d\zq {=f}\qb3N\en
\notes\tqh1{_E}|\rqs\en
\notes\qs|\zq d\zq g\tqh3N\en
\notes\zq I\ibbu0J0\qb0L|\rqs\en
\notes\qs|\zq d\ibbu1d0\qb1N\en
\notes\qb0E|\rqs\en
\notes\qs|\fl e\rq e\zq d\qb1N\en
\notes\zq I\qb0L|\rqs\en
\notes\qs|\zq f\zq d\qb1N\en
\notes\tqh0E|\rqs\en
\notes\qs|\zq g\zq d\tqh1N\en
\notes\zq I\ibbu0J0\qb0L|\rqs\en
\notes\qs|\zq d\ibbu1d0\qb1N\en
\notes\qb0E|\rqs\en
\notes\rlap\qs\hsk\tbu0|\rq e\zq d\zqb1N\hsk\tbu1\en
\endpiece
\end{music}
\noindent The prolongation of the two upper beam at the end is
illustrated in the code fragment
\begin{quote}\begin{verbatim}
\notes\rlap{\qs}\hsk\tbu0|\rq e\zq d\zqb1N\hsk\tbu1\en
\end{verbatim}\end{quote}
 %\check

 \subsection{Beams with notes on several different staves}

Here's a simple example from {\sc Brahms}'s
Intermezzo op.~118,1 provided by
Miguel {\sc Filgueiras}:\index{Brahms, J.@{\sc Brahms, J.}}\medskip

\begin{music}
\interstaff{12}
\setstaffs1{2}
\setclef1\bass
\generalmeter\allabreve
\startextract
\NOtes\qp\nextstaff\isluru0q\zq{q}\ql{j}\enotes
\bar
\nspace
\Notes\ibu0L2\qb0{CEJLcL}%
  \nextstaff\roff{\zw{l}}\pt{p}\zh{_p}\pt{i}\hl{_i}\enotes
\Notes\qb0J\itied1a\qb0a\nextstaff\tslur0o\zq{o}\ql{h}\enotes
\bar
\Notes\ttie1\zh{.L.a}\hl{.e}%
  \nextstaff\qb0{chj}\tbl0\qb0l\cl{q}\ds\enotes
\NOtes\qp\nextstaff\zq{q}\ql{j}\enotes
\endextract
\end{music}
 %\check
\noindent The coding is
\begin{verbatim}
\interstaff{13}
\instrumentnumber{1}
\setstaffs1{2}
\setclef1\bass
\generalmeter\allabreve
\startextract
\NOtes\qp\nextstaff\isluru0q\zq{q}\ql{j}\enotes
\bar
\nspace
\Notes\ibu0a1\qb0{CEJLcL}%
  \nextstaff\roff{\zw{l}}\pt{p}\zh{_p}\pt{i}\hl{_i}\enotes
\Notes\qb0J\itied1a\qb0a\nextstaff\tslur0o\zq{o}\ql{h}\enotes
\bar
\Notes\ttie1\zh{.L.a}\hl{.e}%
  \nextstaff\qb0{chj}\tbl0\qb0l\cl{q}\ds\enotes
\NOtes\qp\nextstaff\zq{q}\ql{j}\enotes
\endextract
\end{verbatim}

\noindent (This example also shows that there is no problem in extending
a beam across a bar line.)

The general features that enable this type of coding as well as the
more complex example to follow are
 \begin{itemize}\setlength{\itemsep}{0ex}
%  \item Commands like \Bslash{\tt ibu}, \Bslash{\tt ibl},
%\Bslash{\tt Ibu}, and \Bslash{\tt Ibl} \rm
%define beams whose initial vertical position and slope are fixed
%relative
%to the staff where they begin, but notes in other staves can still be
%connected to them using \Bslash{\tt qb}\onen.
%
%  \item The commands \keyindex{tbu}\onen\rm~or \keyindex{tbl}\onen\rm~terminate
%beam $n$ at the specified position, but \musixtex\ remembers the beam
%parameters until a new beam with the same number is defined.
%Therefore, even after beam $n$ has been ``finished'' by a \verb|\tbu| or
%\verb|\tbl| command, commands like \Bslash{\tt qb}\enpee\rm~will still
%generate notes connected to the phantom extension of this beam,
%\ital{provided they are issued in a different staff}. If the command
%\verb|\qb|\enpee\rm~were issued on the same staff as the beam after
% the beam had ended, an error would result.
%
  \item Commands like {\Bslash\texttt{ibu}}, {\Bslash\texttt{ibl}},
{\Bslash\texttt{Ibu}}, and {\Bslash\texttt{Ibl}}
define beams whose initial vertical position and slope are fixed
relative
to the staff where they begin, but notes in other staves can still be
connected to them using {\Bslash\texttt{qb}\onen}.

  \item The commands {\keyindex{tbu}\onen}  or {\keyindex{tbl}\onen}
terminate
beam $n$ at the specified position, but \musixtex\ remembers the beam
parameters until a new beam with the same number is defined.
Therefore, even after beam $n$ has been ``finished'' by a \verb|\tbu| or
\verb|\tbl| command, commands like {\Bslash{\tt qb}\enpee} will still
generate notes connected to the phantom extension of this beam,
\ital{provided they are issued in a different staff}. If the command
\verb|\qb|\enpee\rm~were issued on the same staff as the beam after
 the beam had ended, an error would result.





 \item If the beam is initiated on one staff,
notes in a lower staff can be connected to it, but only \ital{after} the beam
has been defined. This may require using the command \keyindex{prevstaff}
to go back one staff, as described in section
\ref{movingtostaffs}.

 \end{itemize}

 Here is an example:

 \begin{music}
 \setstaffs13
 \setclef1{6000}
 \startextract
 \notes
 \nextstaff\Ibbbu0Ae7\prevstaff
 \qb0{AEH^JLa}\relax\nextstaff
 \qb0{******^c}\tqh0e\relax
 |\zq{h^jl}\ql o\enotes
 \notes
 \nextstaff
 \Ibbbu0hH6\qb0{hec}\prevstaff
 \qb0{***aLJ}\tqh0H\relax\nextstaff
 |\zq{h^jl}\ql o\enotes \nspace
 \zendextract
 \end{music}

 \noindent which is coded as:

 \begin{quote}
 \begin{verbatim}
 \setstaffs13
 \setclef1{6000}
 \startextract
 \notes
 \nextstaff\Ibbbu0Ae7\prevstaff
 \qb0{AEH^JLa}\relax\nextstaff
 \qb0{******^c}\tqh0e\relax
 |\zq{h^jl}\ql o\enotes
 \notes
 \nextstaff
 \Ibbbu0hH6\qb0{hec}\prevstaff
 \qb0{***aLJ}\tqh0H\relax\nextstaff
 |\zq{h^jl}\ql o\enotes \nspace
 \zendextract
 \end{verbatim}
 \end{quote}

In this example we not only see multiple uses of \keyindex{nextstaff} and
\keyindex{prevstaff}, but also the character \verb|*| to make virtual beam notes
(see section \ref{CollectiveCoding}).

\section{Rests}
 \subsection{Ordinary rests}
 A separate macro is defined for each kind of ordinary rest. They cause
a space after the symbol, just like spacing note commands, but they have
no parameters. A whole rest is coded as
\keyindex{pause}, dotted whole rest \keyindex{pausep},
half rest \keyindex{hpause}, dotted half rest \keyindex{hpausep},
quarter rest \keyindex{qp} or \keyindex{soupir},
eighth rest \keyindex{ds}, sixteenth rest \keyindex{qs},
32nd rest \keyindex{hs}, and 64th rest \keyindex{qqs}.

 Longer rests, normally interpreted as lasting
two or four bars respectively, can be coded as \keyindex{PAuse}
and \keyindex{PAUSe}, which yield:

\begin{music}\nostartrule
\generalmeter{\meterfrac44}
\startextract
\def\atnextbar{\znotes\centerbar\PAuse\en}%
\NOTEs\en\bar
\def\atnextbar{\znotes\centerbar\PAUSe\en}%
\NOTEs\en\zendextract
\end{music}
 %\check


 \subsection{Raising rests}\index{raising rests}
All the
previous rests except \keyindex{pausep} and
\keyindex{hpausep} are \ital{hboxes}, which means that
they can be vertically offset if needed using the
standard \TeX\ command \keyindex{raise}. For example:

 \begin{quote}
 \begin{verbatim}
 \raise 2\Interligne\qp
 \raise 3mm\qq
 \end{verbatim}
 \end{quote}

\noindent where \keyindex{Interligne} is the distance from one staff line to the
next.

In addition, two macros are available to put a whole or
half rest above or below the staff. The ordinary \verb|\pause| or
\verb|\hpause| cannot be used outside the staff because a short horizontal line
must be added to distinguish between the whole and the half rest. The commands,
which are non-spacing\footnote{Editor's note: The reason for having defined these
as non-spacing is not obvious}, are
 \begin{itemize}\setlength{\itemsep}{0ex}
 \item \keyindex{liftpause}~$n$ to get a
  \hbox to10pt{\liftpause{-2}\hss}
  raised from original position by $n$ staff line intervals,
 \item \keyindex{lifthpause}~$n$ to get
  \hbox to10pt{\lifthpause{-1}\hss} raised the same way.
 \item \keyindex{liftpausep}~$n$ to get a
  \hbox to10pt{\liftpausep{-2}\hss}
  raised from original position by $n$ staff line intervals,
 \item \keyindex{lifthpausep}~$n$ to get
  \hbox to10pt{\lifthpausep{-1}\hss} raised the same way.
 \end{itemize}
 %\check

 \subsection{Bar centered rests}\label{barcentered}
Sometimes it is necessary to place a rest (or any other symbol) exactly in the middle
of a bar. This can be done with combinations of the commands
\keyindex{atnextbar},
\keyindex{centerbar},
\keyindex{centerPAUSe},
\keyindex{centerPAuse},
\keyindex{centerpause},
\keyindex{centerhpause},
as demonstrated in the following example:

\begin{music}
\generalmeter\meterC
\setclef1\bass
\setstaffs1{2}
\parindent0pt
\startpiece\addspace\afterruleskip
\NOtes|\qa{cegj}\en
\def\atnextbar{\znotes\centerpause\en}\bar
\NOtes|\qa{jgec}\en
\def\atnextbar{\znotes\centerpause\en}\bar
\Notes\ca{`jihgfedc}\en
\def\atnextbar{\znotes|\centerpause\en}\bar
\NOTes\ha{Nc}\en
\def\atnextbar{\znotes|\centerpause\en}\bar
\addspace{10\elemskip}%
\def\atnextbar{\znotes\centerbar\duevolte|\centerbar\duevolte\en}\endpiece
\end{music}
\noindent with the coding
\begin{verbatim}
\generalmeter\meterC
\setclef1\bass
\setstaffs1{2}
\parindent0pt
\startpiece\addspace\afterruleskip
\NOtes|\qa{cegj}\en
\def\atnextbar{\znotes\centerpause\en}\bar
\NOtes|\qa{jgec}\en
\def\atnextbar{\znotes\centerpause\en}\bar
\Notes\ca{`jihgfedc}\en
\def\atnextbar{\znotes|\centerpause\en}\bar
\NOTes\ha{Nc}\en
\def\atnextbar{\znotes|\centerpause\en}\bar
\addspace{10\elemskip}%
\def\atnextbar{\znotes\centerbar{\duevolte}|\centerbar{\duevolte}\en}\endpiece
\end{verbatim}
 %\check

\section{Skipping spaces and shifting symbols}\label{spacing}
We've already mentioned that when coding a sequence of notes inside a
particular pair \verb|\notes...\enotes|, the command \keyindex{sk} can
be used to skip horizontally by one \keyindex{noteskip}. This would be
used for example to align the third note in one staff with the second note
in another. Skipping in this manner is logically equivalent to inserting
blank space; as such, the space must be recorded by \musixtex. This command
and the others discussed here will
do just that, so that \verb|musixflx| can properly account for
the added space.

To skip by one \verb|noteskip| while in a collective coding sequence,
you may simply insert an asterisk (``\verb|*|''\index{*}). This would have
the same effect as stopping the sequence, entering {\Bslash sk}, then
restarting. For example,

\begin{music}
\setstaffs12
\startextract
\Notes\hu{e*f*g}|\qu{gghhii}\en
\zendextract
\end{music}
\noindent was coded as
\verb|\Notes\hu{e*f*g}|{\tt |}\verb|\qu{gghhii}\en|

To skip forward
by one half of a \verb|\noteskip|, use \keyindex{hsk}.
To insert spacing of approximately one note head width, you can use
\keyindex{qsk}, or for half of that, \keyindex{hqsk}. To skip backward
by one \verb|\noteskip|, use \keyindex{bsk}. More generally,
to skip an arbitrary distance, use \keyindex{off}\verb|{|$D$\verb|}|
where $D$ is any \itxem{scalable dimension}, e.g.~\verb|\noteskip| or
\verb|\elemskip|. Indeed, if you look in the
\musixtex\ source, you will see that \verb|\off| is the basic control
sequence used to define all the other skip commands.

The foregoing commands only work \ital{inside} a \verb|\notes...\enotes| group.
A different set of commands must be used to insert space \ital{outside}
such a group. \keyindex{nspace} produces an
additional spacing of half a note head width;
\keyindex{qspace}, one note head width. These are ``hard'' spacings. To
insert an arbitrary amount of hard space outside a \verb|\notes...\enotes|
group, use \keyindex{hardspace}\verb|{|$d$\verb|}{|\dots\verb|}| where
$d$ is any fixed dimension\label{hardspace}. The foregoing three commands are the only
space-generating commands that insert hard space; all the others insert
scalable spacing: \keyindex{elemskip},
\keyindex{beforeruleskip}, \keyindex{afterruleskip}, \keyindex{noteskip} and
their multiples. Finally, to insert
scalable spacing outside
a \verb|\notes...\enotes| group, use
\keyindex{addspace}\verb|{|$D$\verb|}|. The argument may be negative, in
which case the normal spacing will be reduced. For example, after
\keyindex{changecontext}, many users prefer to reduce the space with
a command like
\verb|addspace{-|\keyindex{afterruleskip}\verb|}|.

There is yet another set of commands for simply shifting a note, symbol, or sequence
inside \verb|\notes...\enotes|
without adding or subtracting any space. To shift by one note head width,
you may write \keyindex{roff}\verb|{|\ital{any macro}\verb|}|
or \keyindex{loff}\verb|{|\dots\verb|}| for a right or left shift respectively.
 To shift by half of a note head width, use
\keyindex{hroff}\verb|{|\dots\verb|}| or
\keyindex{hloff}\verb|{|\dots\verb|}|.
For example, to get
 %\check

\begin{music}\nostartrule
\startextract
\Notes\roff{\zwh g}\qu g\qu h\qu i\enotes
\zendextract
\end{music}
\noindent you would code:

 %\check

\begin{quote}\begin{verbatim}
\Notes\roff{\zwh g}\qu g\qu h\qu i\enotes
\end{verbatim}\end{quote}

\noindent To shift notes or symbols by an arbitrary amount, use
\keyindex{roffset}\verb|{|$N$\verb|}{|\dots\verb|}| or
\keyindex{loffset}\verb|{|$N$\verb|}{|\dots\verb|}|, where
$N$ is the distance to be shifted in note head widths. For example

\begin{music}\nostartrule
\startextract
\Notes\roffset{1.5}{\zwh g}\qu g\qu h\qu i\enotes
\zendextract
\end{music}
\noindent was coded as

\begin{quote}\begin{verbatim}
\Notes\roffset{1.5}{\zwh g}\qu g\qu h\qu i\enotes
\end{verbatim}\end{quote}

An important feature of these shift commands is that the offset,
whether implicit or explicit, is \ital{not} added to
the total spacing amount, but any spacing due to the included commands is.


\section{Accidentals}

Accidentals can be introduced in two ways.

 The first way, using explicit macros, consists for
example in coding \keyindex{fl}\pitchp~to put a \ital{flat} at the
pitch $p$, presumably right before a note at the same pitch. This is a
non-spacing command and will automatically place the accidental an
appropriate distance to the left of the anticipated note head.
Naturals, sharps, double flats and double sharps are coded \keyindex{na}\pitchp,
\keyindex{sh}\pitchp, \keyindex{dfl}\pitchp~and \keyindex{dsh}\pitchp~respectively.

The alternate macros \keyindex{lfl}, \keyindex{lna}, \keyindex{lsh},
\keyindex{ldfl} and \keyindex{ldsh}
place the same accidentals, but shifted one note head width
to the left. These can be used if a note head has been shifted to the left, or
to avoid collision with other accidentals
in a chord. If you want to shift an accidental by some other amount for
more precise positioning, you could use \keyindex{loffset} with the normal
accidental macro as the second parameter.

 The second way of coding accidentals is to modify the parameter of a
note command. Just put the symbol
\verb|^| for a sharp, \verb|_| for a flat, \verb|=|~for a natural,
\verb|>| for a double sharp, or \verb|<| for a double
flat, right before the letter or number representing the pitch.
For example, \verb|\qb{^g}| yields a
$G\sharp$. This may be used effectively in collective coding, e.g.
\verb|\qu{ac^d}|.

 There are two sizes of accidentals. By default they will be large unless there
is not enough space between notes, in which case they will be made small. Either
size can be forced locally by coding \keyindex{bigfl}, \keyindex{bigsh}, etc., or
\keyindex{smallfl}, \keyindex{smallsh}, etc. If you want all accidentals to
be large, then declare \keyindex{bigaccid} near the top of the input file. For
exclusively small ones use \keyindex{smallaccid}. \keyindex{varaccid} will restore
variable sizes.

 For editorial purposes, small accidentals can be placed \ital{above} note
heads. This is done using \keyindex{uppersh}\pitchp, \keyindex{upperna}\pitchp, or
\keyindex{upperfl}\pitchp:

\begin{music}\nostartrule
\startextract
\NOtes\uppersh l\qa l\enotes
\NOtes\upperna m\qa m\enotes
\NOtes\upperfl l\qa l\enotes
\zendextract
\end{music}
 It also possible to introduce \ital{\ixem{cautionary accidental}s},
i.e.\ small accidentals enclosed in parentheses. This done by preceding
the name of the accidental keyword with ``\verb|c|'',\label{cautionary}
 e.g.~\keyindex{cfl}\pitchp~for a cautionary flat.
 Available cautionary accidentals are \keyindex{csh}, \keyindex{cfl},
\keyindex{cna},
\keyindex{cdfl} and \keyindex{cdsh}, which give

\begin{music}\nostartrule
\startextract
\NOtes\csh g\qa g\enotes
\NOtes\cfl h\qa h\enotes
\NOtes\cna i\qa i\enotes
\NOtes\cdfl j\qa j\enotes
\NOtes\cdsh k\qa k\enotes
\zendextract
\end{music}
\noindent

The distance between all notes and accidentals is controlled by
\keyindex{accshift}\verb|=|\ital{any \TeX\ dimension}, where
positive values shift to the left and negative to right, with a
default value of \verb|0pt|.

 \section{Transposition and octaviation}\label{octaviation}
Two different subjects are discussed in this section. First, there
are commands that cause notes to be printed at different pitches than
entered. We shall refer to this as \ital{logical} transposition.
Second, there are notations for octaviation that do
not otherwise alter the appearance of the score, which we'll call
\ital{octaviation lines}.

\subsection{Logical transposition and octaviation}

Logical transposition is controlled by an integer-valued \TeX\ register
\keyindex{transpose}. Its default value is 0. If you enter
\verb|\transpose=|$n$, then all subsequent pitches specified by
letters will be transposed by $n$ positions. Normally this method would be
used to transpose an entire piece. Pitches specified with
numbers will not be affected, so if you think you will ever want to
transpose a piece, you should enter all note pitches with letters.

One way to transpose up or down by one octave would be to set
\keyindex{transpose} to 7 or $-7$. For example, to make a quarter note
octave as a chord, you could define a macro as
\verb|\def\soqu#1{\zq{#1}{\transpose=7 \qu{#1}}}|.
Note that because \verb|\transpose| is altered inside a pair of braces, the
effect of the alteration is only local and does not reach outside the braces.

 Another more convenient way to transpose locally up or down by one octave
makes use respectively of the characters
\verb|'| (\itxem{acute accent}) and \verb|`| (\itxem{grave accent}),
placed immediately before the letter specifying the pitch. So
for example \verb|\qu{'ab}| is equivalent to \verb|\qu{hi}| and
\verb|\qu{`kl}| is equivalent to \verb|\qu{de}|. These characters have
cumulative effects but in a somewhat restricted sense. They will alter the value of
\verb|\transpose|, but only until changing to a different staff or
instrument or encountering \verb|\enotes|, at which time it will be reset to the
value it had before the accents were used. (That value is stored in
another register called \keyindex{normaltranspose}). Thus for example
\verb|\qu{''A'A}| and \verb|\qu{''A}\qu{'A}|
are both equivalent to \verb|\qu{ah}|.

At any point it is possible to reset the \verb|\transpose| register
explicitly to the value it had
when entering \verb|\notes|, by prefacing a pitch indication with
``\verb|!|''. Thus \verb|\qu{!a'a}| always
gives the note \verb|a| and its upper octave \verb|h|, shifted by the
value of {\Bslash transpose} at the beginning of the current
\verb|\notes...\enotes| group, regardless of the number of grave and
acute accents occurring previously within that group.

 \subsection{Behavior of accidentals under logical transposition}\label{transposeaccids}
 The above processes indeed change the vertical position of the note heads
and associated symbols (e.g.~stems and beams), but they don't take
care of the necessary changes of accidentals when transposing. For example,
suppose an F$\sharp$ occurs in the key of C major. If the piece is
transposed up three steps to the key of F, the F$\sharp$ should logically
become a
B$\natural$. But if all you do is set \verb|\transpose=3|, the note will
be typeset as a B$\sharp$. In other words, \musixtex\ will interpret the~\verb|\sh|
or~\verb|^| to mean ``print a $\sharp$''.

Naturally there is a solution, but it requires the typesetter to plan
ahead: To force accidentals to behave well under transposition, they
must be entered according to the \ital{relative
accidental convention}. To alert \musixtex\ that you are using this
convention to enter notes, you must issue the command
\keyindex{relativeaccid}. Once you have done this, the meaning of
accidental macros and characters (accents) in the input file is changed.
Under the convention, when for example a
\verb|\sh| is entered, it indicates a note that is supposed to sound
\ital{one half step higher than what it would normally be under the
current key signature}. Flats and naturals on entry similarly indicate
notes one half step lower or at the same pitch as what the key signature
dictates. \musixtex\ will take account of the key signature, and print
the correct symbol according to the modern notational convention,
provided you have explicitly entered the transposed key signature using
for example \verb|\generalsignature|.

Many people have a difficult time understanding how this works, so
here are two simple examples in great detail. Consider the case already mentioned
of the F$\sharp$ in the key of C major. With
\verb|\relativeaccid| in effect, it should still be entered as \verb|\sh f|, and
with no transposition it will still appear as F$\sharp$. With
\verb|\transpose=3| and \verb|\generalsignature{-1}| it will appear
(correctly) as B$\natural$. Conversely, suppose you want to enter a
B$\natural$ when originally in the key of F. With \verb|\relativeaccid| in effect, it
should be entered as \verb|\sh i|. (That's the part that people
have the most trouble with: ``If I want a natural, why do I have to enter a sharp?'' Answer:
``Go back and re-read the previous paragraph very carefully.'') With no transposition, it will be
printed as B$\natural$. Now to transpose this to C major, set
\verb|\transpose=-3| and \verb|\generalsignature0|, and it will appear
as F$\sharp$.

If you have invoked \verb|\relativeaccid| and then
later for some reason wish to revert to the ordinary convention,
enter \keyindex{absoluteaccid}.

\subsection{Octaviation lines}
The first kind of notation for octave transposition covers a
horizontal range that must be specified at the outset. The sequence

\medskip
\begin{music}\nostartrule
\startextract
\NOTEs\octfinup{10}{3.5}\ql{!'a}\ql b\ql c\ql d\en
\zendextract
\end{music}
\noindent can be coded as
\verb|\NOTEs\octfinup{10}{3.5}\qu a\qu b\qu c\qu d\en|.
\zkeyindex{octfinup}
Here, the dashed line is at staff level 10 and extends 3.5\verb|\noteskip|.
Conversely, lower octaviation can be coded. For example

\begin{music}\nostartrule
\startextract
\NOTEs\octfindown{-5}{2.6}\ql j\ql i\ql h\en
\zendextract
\end{music}
\noindent is coded as
\verb|\NOTEs\octfindown{-5}{2.6}\ql j\ql i\ql h\en|.
\zkeyindex{octfindown}
 To change the text that is part of these notations, redefine one of
the macros \keyindex{octnumberup} or \keyindex{octnumberdown}. The reason for
the distinction between up and down is that, traditionally, upper octaviation
only uses the figure
``8'' to denote its beginning, while lower octaviation uses a more
elaborate indication such as \hbox{\ppffsixteen8$^{va}$ \it bassa}. Thus

\bigskip
\begin{music}\nostartrule
\startextract
\NOTEs\def\octnumberup{\ppffsixteen8$^{va}$}\octfinup{10}{2.5}\qu c\qu d\qu e\en
\zendextract
\end{music}
\noindent is coded

\noindent\verb|\NOTEs\def\octnumberup{\ppffsixteen8$^{va}$}\octfinup{10}{2.5}\qu c\qu d\qu e\en|

\noindent while

\begin{music}
\startextract
\NOTEs\def\octnumberdown{\ppffsixteen8$_{ba}$}\octfindown{-5}{2.5}\ql l\ql
k\ql j\en
\endextract
\end{music}
\noindent is coded as
\begin{quote}\begin{verbatim}
\NOTEs\def\octnumberdown{\ppffsixteen8$_{ba}$}%
  \octfindown{-5}{2.5}\ql l\ql k\ql j\en
\end{verbatim}\end{quote}

The foregoing constructions have the drawbacks that (a) the span must be
indicated ahead of time and (b) they cannot extend across a line break.
Both restrictions are removed with the use of the alternate macros \keyindex{Ioctfinup},
\keyindex{Ioctfindown} and
\keyindex{Toctfin}.

{\Bslash Ioctfinup}~$np$ indicates an upward octave transposition line with reference number
$n$ and with dashed line at pitch $p$. 
By default $n$ must be in the range [0-5], but 
you can specify a larger maximum number 
directly with \keyindex{setmaxoctlines}\verb|{|$m$\verb|}| 
where $7<m\leq 100$\footnote{This may require e-\TeX.}; the
reference number will be in the range between 0 and $(m-1)$.
\label{musixmad_setmaxoctlines}

Usually $p$ will be numeric and $>9$,
but it can also be a letter.
{\Bslash Ioctfindown}~$np$ starts a lower octave transposition line at pitch $p$
(usually $p<-1$). Both extend until terminated with {\Bslash Toctfin}. The
difference between {\Bslash Ioctfinup}~$n$ and {\Bslash Ioctfindown}~$n$ is
the relative position of the figure ``8'' with respect to the dashed line, and
the sense of the terminating hook As shorthand, \keyindex{ioctfinup} is equivalent to
\verb|\Ioctfinup 0| and \keyindex{ioctfindown} is equivalent to
\verb|\Ioctfindown 0|.

For example,

\medskip
 \begin{music}\nostartrule
\instrumentnumber{1}
\setstaffs12
\setclef1{6000}
%
\startextract
\notes\wh{CDEFGH}|\wh{cde}\Ioctfinup 1p\wh{fgh}\enotes
\bar
\notes\Ioctfindown 2A\wh{IJKLMN}|\wh{ijklmn}\enotes
\bar
\Notes\wh{NMLKJI}|\wh{nmlkji}\Toctfin1\enotes
\bar
\Notes\wh{HGFED}\Toctfin2\wh C|\wh{hgfedc}\enotes
\zendextract
\end{music}
\noindent is coded as
\begin{quote}\begin{verbatim}
\begin{music}\nostartrule
\instrumentnumber{1}
\setstaffs12
\setclef1{6000}
%
\startextract
\notes\wh{CDEFGH}|\wh{cde}\Ioctfinup 1p\wh{fgh}\enotes
\bar
\notes\Ioctfindown 2A\wh{IJKLMN}|\wh{ijklmn}\enotes
\bar
\Notes\wh{NMLKJI}|\wh{nmlkji}\Toctfin1\enotes
\bar
\Notes\wh{HGFED}\Toctfin2\wh C|\wh{hgfedc}\enotes
\zendextract
\end{music}
\end{verbatim}\end{quote}
 %\check

 The elevation of octaviation lines may be changed in midstream using
\keyindex{Liftoctline}~$n$$p$, where $n$ is the reference number of the
 octave line, and $p$ a (possibly negative) number of
\keyindex{internote}s (staff pitch positions) by which elevation of the dashed line should be changed.
This may be useful when octaviation lines extend over several systems and the
elevation needs to be changed in a systems after the one where it was initiated.


\section{Slurs and ties}
Two fundamentally different implementations of slurs, ties, and hairpins are
available. (The hairpins ``go along for the ride'' with slurs and ties.) First, there are the
original \ital{font-based} versions. These are constructed with traditional
\TeX\ font characters that were created with \MF\ and stored in \TeX\ font files.
Second, it is now possible
to generate these shapes directly with postscript, dispensing altogether with
the font characters\footnote{Please do not be confused by the availability of
postscript versions of the font-based slur fonts (along with all other \musixtex\ fonts).
Once installed in a \TeX\ system,
their function and use are 100\% transparently identical with bitmapped versions
of the slur fonts. On the other
hand, postscript slurs are functionally distinct from font-based slurs, and only
 share some of the same syntax.}. We shall first describe the font-based versions,
then Type K postscript slurs, which are one of two available postscript slur
options\footnote{An
alternate approach to postscript slurs, called {\it Type M} after its developer
Hiroaki {\sc Morimoto}, is available from the
\href{http://icking-music-archive.org/software/indexmt6.html}
{\underline{Icking Music Archive}}.}. If you plan to use Type K postscript slurs,
you may skip directly to section~\ref{PostscriptSlurs}.

 \subsection{Font-based slurs}

Font-based slurs and ties provided by \musixtex\ can be divided into two categories:
\label{simpleslur}
\begin{itemize}\setlength{\itemsep}{0ex}
   \item Those where the complete slur symbol is composed of a single
   character from one of the slur fonts, and
   \item those where the slur symbol is composed of three distinct
   characters, to form the beginning, middle and end of the slur.
\end{itemize}

The former are called \ital{simple slurs} and the latter,
\ital{compound slurs}. In many cases the distinction between the two is
invisible to the user, in that many of the macros described below will
automatically select between the two types. However, there are other macros
that allow simple slurs to be forced.

The next few subsections describe the usual method of slur coding, where the
choice between simple or compound slurs is made automatically.
In this case, slurs are initiated and terminated by separate macros,
similar to beams.

 \subsubsection{Font-based slur initiation}
A slur must be initiated \ital{before}
the spacing note on which the slur begins, and terminated \ital{before} the
note on which it ends. 

The basic slur initiation macro is
\keyindex{isluru}\enpee, which initiates an upper slur, with reference
number $n$, beginning on a note at pitch $p$. The starting point of the slur is
centered above a virtual quarter note head at pitch
$p$\footnote{The slur will start in the same place regardless of whether there is
{\it actually} a note at pitch $p$.}.
Similarly, \keyindex{islurd}\enpee\ initiates a lower slur.
These slurs are terminated by coding
\keyindex{tslur}\enpee\ where $n$ is the reference number and $p$ is the
termination pitch. 

As with beams, the reference number $n$ by default can take values from 0 to 5, or
up to 8 or 11 respectively if \ttxem{musixadd.tex} or \ttxem{musixmad.tex} is included.
You can also specify the maximum number 
directly with \keyindex{setmaxslurs}\verb|{|$m$\verb|}| where 
$7<m\leq 100$%
\footnote{This may require e-\TeX.}; 
the reference number $n$ will be in the range between 0 and $(m-1)$.
\label{musixmad_setmaxslurs}

To illustrate with an elementary example, the following passage

\begin{music}\nostartrule
\startextract
\NOtes\islurd0g\qu g\tslur0{'c}\qu c\en
\Notes\isluru0{'e}\ibl0e{-2}\qb0{edc}\tslur0b\tqb0b\en
\bar
\NOtes\islurd0{'a}\qu a\tslur0{`f}\qu f\en
\NOTes\hu g\en
\zendextract
\end{music}

\noindent was coded as:
\begin{quote}\begin{verbatim}
\NOtes\islurd0g\qu g\tslur0{'c}\qu c\en
\Notes\isluru0{'e}\ibl0e{-2}\qb0{edc}\tslur0b\tqb0b\en
\bar
\NOtes\islurd0{'a}\qu a\tslur0{`f}\qu f\en
\NOTes\hu g\en
\end{verbatim}\end{quote}

Other macros are provided to change the starting and ending point of the slur
in relation to the initial and final notes. Thus, \keyindex{issluru}\enpee\
initiates a ``short'' upper slur suitable for linking notes involved in chords.
The
starting point is shifted to the right, and is vertically aligned with the
center of a virtual quarter note head at pitch $p$. If a lower short slur
is wanted, one should use \keyindex{isslurd}\enpee.

Sometimes, busy scores call for slurs which are vertically aligned with the
ends of note stems rather than note heads. These ``beam'' slurs---so called
because the slur is written at usual beam height---are provided by the
macros \keyindex{ibsluru}\enpee\ and \keyindex{ibslurd}\enpee. These macros
initiate slurs raised or lowered by the current stem height to accommodate
stems or beams above or below.
 %\check

 \subsubsection{Font-based slur termination}
Font-based slurs that are not forced to be simple must be terminated by an explicit
command right before the last note under the slur. There are termination
commands analogous to each of the initiation commands alread presented. They
are summarized in the following table:
\begin{center}
  \begin{tabular}{ll}
    Initiation                             &   Termination \\
    \hline
    \keyindex{isluru},  \keyindex{islurd}  & \keyindex{tslur}  \\
    \keyindex{issluru}, \keyindex{isslurd} & \keyindex{tsslur} \\
    \keyindex{ibsluru}                     & \keyindex{tbsluru}\\
    \keyindex{ibslurd}                     & \keyindex{tbslurd}\\
    \hline
  \end{tabular} \end{center}

\noindent All of these command have two parameters, $n$ and $p$.

These specific termination macros are not restricted to being used with their
initiation counterpart. A slur
started in one sense can be terminated in another. For example, a slur beginning
as a ``beam'' slur may be terminated as a normal slur. This would be achieved
using the macro pair \verb|\ibslur...\tslur|.

 \subsubsection{Font-based ties} Font-based ties will have the same shapes as
ordinary font-based slurs of the same length between notes of equal pitch,
but there are two important distinctions:
(1) There cannot be any pitch difference between start and end, and (2) the
positions of both the beginning and end of a tie relative to the note heads
are slightly different from those of an ordinary slur\footnote{Editor's note: In
fact, it appears that the default positioning of the ends of ties is exactly
the same as that of short slurs.}.
Upper ties are initiated by \keyindex{itieu}\enpee, which starts an upper tie of
reference number $n$ at pitch $p$.
Lower ties are initiated by \keyindex{itied}\enpee, which starts an lower tie of
reference number $n$ at pitch $p$. The starting position of the tie is the same
as \verb|\issluru| and
\verb|\isslurd| respectively. The tie is terminated by coding
\keyindex{ttie}\onen. Note that no pitch parameter is required.

There are also \ital{short ties}, which bear the same relation to ordinary
ties as short slurs to ordinary slurs. They are intended to be used between chords.
They are initiated with
\keyindex{itenu}\enpee~or \keyindex{itenl}\enpee\footnote{Editor's note: It is
not clear why this command uses ``{\tt l}'' when all other similar ones use
``{\tt d}''.}, and terminated with \keyindex{tten}\onen.

The following example illustrates the differences in positioning of the various
slur and tie options:

\begin{music}\nostartrule
\startextract
\NOTes\islurd0g\qu g\tslur0g\qu g\isslurd0g\qu g\tsslur0g\qu g%
\ibsluru0g\qu g\tbsluru0g\qu g\itied0g\qu g\ttie0\qu g%
\itenl0g\qu g\tten0\qu g\en
\zendextract
\end{music}
\noindent It was coded as
\begin{quote}\begin{verbatim}
\NOTes\islurd0g\qu g\tslur0g\qu g\isslurd0g\qu g\tsslur0g\qu g%
\ibsluru0g\qu g\tbsluru0g\qu g\itied0g\qu g\ttie0\qu g%
\itenl0g\qu g\tten0\qu g\en
\end{verbatim}\end{quote}

\noindent Here are some more general examples of font-based slurs and ties
discussed so far:

\begin{music}\nostartrule
\startextract
\NOTes\isluru0g\hl g\tslur0h\hl h\en
\NOTes\islurd0c\issluru1g\zh{ce}\hu g\tslur0d\tsslur1h\zh{df}\hu h\en
\NOTes\ibsluru0g\islurd1g\hu g\tubslur0h\hu h\en
\NOTes\itieu0k\hl k\ttie0\tdbslur1f\hl k\en
\zendextract
\end{music}
\noindent This was coded as:
\begin{quote}\begin{verbatim}
\NOTes\isluru0g\hl g\tslur0h\hl h\en
\NOTes\islurd0c\issluru1g\zh{ce}\hu g\tslur0d\tsslur1h\zh{df}\hu h\en
\NOTes\ibsluru0g\islurd1g\hu g\tubslur0h\hu h\en
\NOTes\itieu0k\hl k\ttie0\tdbslur1f\hl k\en
\end{verbatim}\end{quote}

\subsubsection{Dotted slurs}

Any font-based slur may be made dotted by specifying \keyindex{dotted}
just before it is initiated:

\begin{music}\nostartrule
\startextract
\NOtes\dotted\islurd0g\qu g\tslur0{'c}\qu c\en
\Notes\dotted\isluru0{'e}\ibl0e{-2}\qb0{edc}\tslur0b\tqb0b\en\bar
\NOtes\dotted\slur{'a}{`f}d1\qu{'a`f}\en
\NOTes\hu g\en
\zendextract
\end{music}

\noindent This was coded as:
\begin{quote}\begin{verbatim}
\NOtes\dotted\islurd0g\qu g\tslur0{'c}\qu c\en
\Notes\dotted\isluru0{'e}\ibl0e{-2}\qb0{edc}\tslur0b\tqb0b\en\bar
\NOtes\dotted\slur{'a}{`f}d1\qu{'a`f}\en
\NOTes\hu g\en
\end{verbatim}\end{quote}

 \subsubsection{Modifying font-based slur properties}
Several macros are provided to modify the shape of slurs already initiated.
These macros must be coded right before the slur \ital{termination}. Invoking
any of the macros described in this section will force the slur to be compound.

By default, the midpoint of a font-based slur is three \verb|\internote|s
above or below a line between its ends. This can be changed using the macro
\keyindex{midslur}~$h$ where $h$ is the revised vertical displacement. For
example, \verb|\midslur6| coded right before \verb|\tslur| causes an upper slur to
rise to a maximum height of \verb|6\internote| above the starting position.
For example,

\begin{music}\nostartrule
\startextract
\NOtes\multnoteskip8\isluru0g\ql g\en
\notes\tslur0g\ql g\en
\zendextract
\startextract
\NOtes\multnoteskip8\isluru0g\ql g\en
\notes\midslur7\tslur0g\ql g\en
\zendextract
\startextract
\NOtes\multnoteskip8\isluru2g\ql g\en
\notes\midslur{11}\tslur2g\ql g\en
\zendextract
\end{music}
\noindent was coded as
\begin{quote}\begin{verbatim}
\NOtes\multnoteskip8\isluru0g\ql g\en
\notes\tslur0g\ql g\en
\NOtes\multnoteskip8\isluru0g\ql g\en
\notes\midslur7\tslur0g\ql g\en
\NOtes\multnoteskip8\isluru2g\ql g\en
\notes\midslur{11}\tslur2g\ql g\en
\end{verbatim}\end{quote}

The macro \keyindex{curve}$hij$ allows more precise control over the shape
of a slur or tie. The first parameter $h$
is the vertical deviation and it works exactly like the sole parameter of
\verb|\midslur| described above. The second and
third parameters $i$ and $j$ set the initial and final gradient respectively.
They are defined as the horizontal distance required to attain maximum
vertical deviation. Thus smaller numbers for $i$
and $j$ lead to more extreme gradients.
The default setting is \verb|\curve344|. Hence, coding \verb|\curve322|
doubles the initial and final gradient relative to the default.
As with \verb|\midslur|, \verb|\curve| must be coded {\em immediately} before
the slur termination. The example below illustrates the use of \verb|\curve|.

\begin{music}\nostartrule
\startextract
\NOtes\multnoteskip8\itieu0g\ql g\en
\notes\ttie0\ql g\en
\zendextract
\startextract
\NOtes\multnoteskip8\itieu1g\ql g\en
\notes\curve 322\ttie1\ql g\en
\zendextract
\startextract
\NOtes\multnoteskip8\itieu2g\ql g\en
\notes\curve 111\ttie2\ql g\en
\zendextract
\end{music}
\noindent This was coded as
\begin{quote}\begin{verbatim}
\NOtes\multnoteskip8\itieu0g\ql g\en
\notes\ttie0\ql g\en
\zendextract
\startextract
\NOtes\multnoteskip8\itieu1g\ql g\en
\notes\curve 322\ttie1\ql g\en
\zendextract
\startextract
\NOtes\multnoteskip8\itieu2g\ql g\en
\notes\curve 111\ttie2\ql g\en
\end{verbatim}\end{quote}

Two macros are provided to control the behaviour of slurs which extend across
line breaks. Normally, the part of the slur before the line break is
treated as a tie.  This can be changed using \keyindex{breakslur}\enpee,
which sets the termination height of
the broken slur at the line break to pitch $p$, for slur number $n$.

After the line break, the slur is normally resumed at the initial pitch
reference, i.e., the one coded in \verb|\islur|. To change this, the macro
\keyindex{Liftslur}\verb|{|$n$\verb|}{|$h$\verb|}| may be used.
Here $n$ is again the slur reference
number and $h$ is the change in height relative to the initialization height.
This macro is normally used following line breaks, in which case it is best
coded using the \verb|\atnextline| macro. For example, coding
\verb|\def\atnextline{\Liftslur06}| raises the continuation of slur zero
by \verb|6\internote| relative to its initialization height.

The following example illustrates the use of the macros for broken slurs:

\begin{minipage}[t]{75mm}
Default, without adjustments:\\
\begin{music}
\hsize=50mm
\generalmeter{\meterfrac24}%
\parindent 0pt\staffbotmarg0pt%
%
%\interstaff{14\internote}%
%
\startpiece\bigaccid
\Notes\ibsluru1b\qu b\qu g\en\bar
\Notes\qu{'c!}\tslur1f\qu f\en
\mulooseness1\Endpiece
\end{music}
\begin{verbatim}
\Notes\ibsluru1b\qu b\qu g\en\bar
\Notes\qu{'c!}\tslur1f\qu f\en
\end{verbatim}
\end{minipage}%
\begin{minipage}[t]{75mm}
With~\verb|\Liftslur|~and~\verb|\breakslur|~:\\
\begin{music}
\hsize=50mm
\generalmeter{\meterfrac24}%
\parindent 0pt\staffbotmarg0pt%
%
%\interstaff{14\internote}%
%
\startpiece\bigaccid
\def\atnextline{\Liftslur17}%
\Notes\ibsluru1b\qu b\qu g\breakslur1g\en\bar
\Notes\qu{'c!}\tslur1f\qu f\en
\mulooseness1\Endpiece
\end{music}
\begin{verbatim}
\def\atnextline{\Liftslur17}%
\Notes\ibsluru1b\qu b\qu g%
\breakslur1g\en\bar
\Notes\qu{'c!}\tslur1f\qu f\en
\end{verbatim}
\end{minipage}\medskip

Occasionally in keyboard works one needs to begin a slur in one
staff but end it in another. This can be done using the macro
\keyindex{invertslur}\onen\ which is best described by
reference to the example
shown below.

% Finding the final pitch is try and error, sorry, no way out.

\begin{music}
\setstaffs1{2}
\setclef1\bass\interstaff{10.5}
\startextract
\NOtes\multnoteskip5\isluru0a\ql a\en
\notes\invertslur0\curve311\tslur0j|\qu d\en
\NOtes\multnoteskip{10}\isluru0a\ql a\en
\notes\invertslur0\curve333\tslur0j|\qu d\en
\zendextract
\end{music}
\noindent This was coded as
\begin{quote}\begin{verbatim}
\NOtes\multnoteskip5\isluru0a\ql a\en
\notes\invertslur0\curve311\tslur0g|\qu d\en
\NOtes\multnoteskip{10}\isluru0a\ql a\en
\notes\invertslur0\curve333\tslur0g|\qu d\en
\end{verbatim}\end{quote}

\noindent Slur inversion as just described takes effect where the slope is zero; therefore it
only works with ascending slurs that were started with \verb|\isluru|, and with
descending slurs started with \verb|\islurd|. Otherwise no horizontal place
can be found and the result is erratic.

A different approach removes this restriction. The idea is to stop the slur at a
the desired inversion point and restart it in the other sense at the
same place. The commands to do this are as follows:
 \begin{itemize}\setlength{\itemsep}{0ex}
 \item \keyindex{Tslurbreak}\enpee~stops slur number $n$ \ital{exactly} at pitch $p$,
not above or below the virtual note head.
 \item \keyindex{Islurubreak}\enpee~restarts an upper slur at the same position,
not above a virtual note head.
 \item \keyindex{Islurdbreak}\enpee~restarts a lower slur at the same position,
not below a virtual note head.
 \end{itemize}

\noindent The vertical position may have to be adjusted to minimize any discontinuity
in the slope. For example, the following pattern

 \begin{music}\nostartrule
\setclef1\treble
\startextract
\NOTes\multnoteskip 3\isluru0a\ql a\en
\NOTes\multnoteskip 3\Tslurbreak0d\Islurdbreak0d\sk\en
\Notes\tslur0h\qu h\en
\NOTes\multnoteskip 3\islurd0a\ql a\en
\NOTes\multnoteskip 3\Tslurbreak0d\Islurubreak0d\sk\en
\Notes\tslur0h\qu h\en
\zendextract
\end{music}
\noindent was coded as
\begin{quote}\begin{verbatim}
\begin{music}\nostartrule
\NOTes\multnoteskip 3\isluru0a\ql a\en
\NOTes\multnoteskip 3\Tslurbreak0d\Islurdbreak0d\sk\en
\Notes\tslur0h\qu h\en
\NOTes\multnoteskip 3\islurd0a\ql a\en
\NOTes\multnoteskip 3\Tslurbreak0d\Islurubreak0d\sk\en
\Notes\tslur0h\qu h\en
\end{music}
\end{verbatim}\end{quote}

\ital{Simple} slurs and ties have advantages in some cases: (1) They will
always have the best possible shape, and (2) if \verb|\noteskip| doesn't
change from start to finish, they are easier to code. But they have drawbacks
as well: (1) They are limited in length to 68pt
for slurs and 220pt for
ties, (2) their maximum vertical extent is \verb|8\internote|, and (3) they
may not extend across a line break.
Despite all these limitations, simple slurs are extremely useful in many
applications where the slurs are short and contained within a bar.

Simple slurs must be coded {\em before} the note on which the slur begins.
The primary macro is
\keyindex{slur}\verb|{|$p_1$\verb|}{|$p_2$\verb|}|$sl$
where $p_1$ and $p_2$ are respectively the initial and final pitches,
$s$ is the sense (either ``{\tt u}'' or ``{\tt d}''), and $l$ is the length,
in {\tt noteskip}s.
Thus, thirds slured in pairs can be coded

 %\check

\begin{quote}\begin{verbatim}
\NOtes\slur ced1\qu{ce}\en
\NOtes\slur dfd1\qu{df}\en
\NOtes\slur egd1\qu{eg}\en
\NOtes\slur{'e}cu1\ql{ec}\en
\NOtes\slur{'d}bu1\ql{db}\en
\NOtes\slur{'c}au1\ql{ca}\en
\end{verbatim}\end{quote}
which yields

\begin{music}\nostartrule
\startextract
\NOtes\slur ced1\qu{ce}\en
\NOtes\slur dfd1\qu{df}\en
\NOtes\slur egd1\qu{eg}\en
\NOtes\slur{'e}cu1\ql{ec}\en
\NOtes\slur{'d}bu1\ql{db}\en
\NOtes\slur{'c}au1\ql{ca}\en
\zendextract
\end{music}
%%

There are similar commands to force simple versions of the other
variants of font-based slurs and ties. Simple ties may be set using
\keyindex{tie}\verb|{|$p$\verb|}|$sl$ (only one pitch is needed).
Simple short slurs and short ties can be forced with the macros
\keyindex{sslur}\verb|{|$p_1$\verb|}{|$p_2$\verb|}|$sl$ and
\keyindex{stie}\verb|{|$p$\verb|}|$sl$ respectively. Finally,
simple beam slurs can be forced with
\keyindex{bslur}$s$\verb|{|$p_1$\verb|}{|$p_2$\verb|}|$l$.

 \subsubsection{Limitations of font-based slurs}
  The change in altitude between slur initiation and slur termination is limited to
16\verb|\Internote|. Thus unexpected vertical gaps can appear, as seen in

 \begin{music}\nostartrule
\instrumentnumber{1}
\generalmeter{\meterfrac34}
\startextract\NOTes\multnoteskip3\isluru0c\ql c\tslur0j\ql j\enotes
  \bar\NOTes\multnoteskip3\isluru0c\ql c\tslur0n\ql n\enotes\zendextract

\startextract\NOTes\multnoteskip3\isluru0c\ql c\tslur0s\ql s\enotes
  \bar\NOTes\multnoteskip3\isluru0c\ql c\tslur0z\ql z\enotes\zendextract
 \end{music}
 \noindent whose coding was
 \begin{quote}\begin{verbatim}
\NOTes\multnoteskip3\isluru0c\ql c\tslur0j\ql j\enotes
  \bar\NOTes\multnoteskip3\isluru0c\ql c\tslur0n\ql n\enotes
\NOTes\multnoteskip3\isluru0c\ql c\tslur0s\ql s\enotes
  \bar\NOTes\multnoteskip3\isluru0c\ql c\tslur0z\ql z\enotes
 \end{music}
 \end{verbatim}\end{quote}

Furthermore, if the slope becomes too steep, even worse results can occur, such as

 \begin{music}\nostartrule
\instrumentnumber{1}
\generalmeter{\meterfrac34}
\startextract\NOTes\isluru0c\ql c\enotes\notes\tslur0j\ql j\enotes
  \bar\NOTes\isluru0c\ql c\enotes\notes\tslur0n\ql n\enotes
 \NOTes\isluru0c\ql c\enotes\notes\tslur0s\ql s\enotes
 \bar\NOTes\isluru0c\ql c\enotes\notes\tslur0z\ql z\enotes
 \zendextract
 \end{music}

Another limitation of font-based slurs crops up when one attempts to
generate bitmapped versions for high-resolution printers: the characters
for long ties can exceed \MF's maximum capacity.

 \subsection{Type K postscript slurs}\label{PostscriptSlurs}
All of the aforementioned limitations of font-based slurs can be circumvented
by using Type~K postscript slurs\footnote{``K'' stands for Stanislav Kneifl,
the developer of the Type K postscript slur package.}. As well as slurs, the package includes
ties and crescendos (see section \ref{sec:crescnd}).
Its use is very similar to font-based slurs, and in fact identical if only
the elementary slur and tie initiation and termination macros are used.

In order to use Type K postscript slurs, you must first place \texttt{musixps.tex}
anywhere \TeX\ can find it. You must also place \texttt{psslurs.pro}
somewhere that \texttt{dvips} can find it.

The \index{mxsk font}\texttt{mxsk} font is required for ``half ties,'' which
are special symbols that are used by default for the second portion of a tie
that crosses a line break. If you like
this treatment you must install the font in your \TeX\ system. However, perfectly
acceptable line-breaking ties will appear if you invoke \keyindex{nohalfties},
and then you will not have to install this font.

Once the software mentioned in the prior two paragraphs is emplaced and the
\TeX\ filename database is refreshed, the Type K package can be
invoked by including the command \verb|\input musixps|
near the beginning of your source file.
The resulting dvi file should then be converted into postscript using \textbf{dvips}.
If desired, a PDF file can then be generated with \textbf{ps2pdf}, \textbf{ghostscript},
\textbf{Adobe Acrobat} (see chapter \ref{installation} for more information on this).

Two minor inconveniences with Type K postscript slurs are that (1) they won't appear
in any standard dvi previewer, and (2) they won't appear in PDF files generated with
\textbf{pdftex}. The former limitation can be circumvented by using a postscript
viewer such as \textbf{GSview}. The latter simply requires that you create an intermediate
postscript file with \textbf{dvips}, then make the PDF with any of the software
mentioned above.

\subsubsection{Initiating and terminating type K postscript slurs}

Basic usage of type K slurs is the same as for font-based slurs. To initiate one,
use for example \verb|\isluru0g| to start an upper slur with ID 0 above a virtual note
at pitch level {\tt g}. To terminate one, use a command like
\verb|\tslur0i| which terminates the slur with ID 0 on a
virtual note at pitch level {\tt i}.
Both types of commands are non-spacing and must precede the first or last
note under the slur.

You can shift the starting or ending point slightly to the left
or right by substituting one of the commands \keyindex{ilsluru},
\keyindex{ilslurd}, \keyindex{irsluru},
\keyindex{irslurd}, \keyindex{trslur} or \keyindex{tlslur}.

You can control the shape of Type K slurs with variants of
the termination command. To make the slur a bit flatter than default use
\keyindex{tfslur}0f; a bit higher, \keyindex{thslur}0f; higher still,
\keyindex{tHslur}0f;  or
even higher, \keyindex{tHHslur}0f.  These commands have an effect like
\keyindex{midslur} does for font-based slurs.

All combinations of the shifting and curvature variants are allowed,
e.g.\ \verb|\trHHslur|.

The following examples demonstrate how much better the type K slurs perform
in the extreme situations of the prior two typeset examples. The coding is
exactly the same as above except that \verb|\input musixps| has been added:
\begin{quote}\begin{verbatim}
\begin{music}\nostartrule
\input musixps
\startextract\NOTes\multnoteskip3\isluru0c\ql c\tslur0s\ql s\enotes
....
\end{verbatim}\end{quote}

%\begin{center}
%\includegraphics[scale=1]{./mxdexamples/pslurvgap.eps}
%\end{center}

\begin{music}\nostartrule
\input musixps
\instrumentnumber{1}
\generalmeter{\meterfrac34}
\startextract\NOTes\multnoteskip3\isluru0c\ql c\tslur0j\ql j\enotes
  \bar\NOTes\multnoteskip3\isluru0c\ql c\tslur0n\ql n\enotes\zendextract

\startextract\NOTes\multnoteskip3\isluru0c\ql c\tslur0s\ql s\enotes
  \bar\NOTes\multnoteskip3\isluru0c\ql c\tslur0z\ql z\enotes\zendextract
% \end{music}

%\begin{music}\nostartrule
%\input smallmusixpsx
%\instrumentnumber{1}
%\generalmeter{\meterfrac34}
\startextract
\NOTes\isluru0c\ql c\enotes\notes\tslur0j\ql j\enotes\bar
\NOTes\isluru0c\ql c\enotes\notes\tslur0n\ql n\enotes
\NOTes\isluru0c\ql c\enotes\notes\tslur0s\ql s\enotes\bar
\NOTes\isluru0c\ql c\enotes\notes\tslur0z\ql z\enotes
\zendextract
 \end{music}


For maximal control over type K slurs, you can use one of the commands
\keyindex{iSlur}~$npvh$ and
\keyindex{tSlur}~$npvhca$, where the characters in
$npvhca$ respectively stand for ID number, height, vertical offset, horizontal offset,
curvature, and angularity.
All offsets are in \verb|internote|, and the slur direction is determined
by the sign of the vertical offset. See the comments in \verb|musixps.tex|
for precise definitions of the other parameters. Examples of permissible forms for
these commands are
\verb|iSlur0c11| and \verb|tSlur0{!d}11{.2}0|.

The next example shows how you can use \verb|iSlur| in difficult
circumstances:

\begin{music}\nostartrule
\input musixps
\nobarnumbers
\generalsignature1%
\generalmeter{\meterfrac{2}{4}}%
\nostartrule
\startextract
\NOtes\zchar{-8}{\Bslash irslur...\Bslash tlslur}\irslurd0{'b}\zhl b\sk\zq{!e}\qu{'c}\en\bar
\NOtes\zh{'d}\zhu{`f}\off{7.2pt}\tlslur0{'b}\ql b\ql b\en\bar
%
\NOtes\zchar{-8}{\Bslash irslur...\Bslash tlfslur}\irslurd0{'b}\zhl b\sk\zq{!e}\qu{'c}\en\bar
\NOtes\zh{'d}\zhu{`f}\off{7.2pt}\tlfslur0{'b}\ql b\ql b\en\bar
%
\NOtes\zchar{-8}{\Bslash iSlur...\Bslash tSlur}\iSlur0{'b}{-.5}3\zhl b\sk\zq{!e}\qu{'c}\en\bar
\NOtes\zh{'d}\zhu{`f}\off{7.2pt}\tSlur0{'b}{.5}{-1}{.3}0\ql b\ql b\en
\zendextract
 \end{music}
\noindent with this coding:

\begin{quote}\begin{verbatim}
\NOtes\irslurd0{'b}\zhl b\sk\zq{!e}\qu{'c}\en\bar
\NOtes\zh{'d}\zhu{`f}\off{7.2pt}\tlslur0{'b}\ql b\ql b\en\bar
%
\NOtes\irslurd0{'b}\zhl b\sk\zq{!e}\qu{'c}\en\bar
\NOtes\zh{'d}\zhu{`f}\off{7.2pt}\tlfslur0{'b}\ql b\ql b\en\bar
%
\NOtes\iSlur0{'b}{-.5}3\zhl b\sk\zq{!e}\qu{'c}\en\bar
\NOtes\zh{'d}\zhu{`f}\off{7.2pt}\tSlur0{'b}{.5}{-1}{.3}0\ql b\ql b\en
\end{verbatim}\end{quote}


The ID number for a slur, tie or crescendo should
normally range from 0 to 9. If it is bigger
than nine but less than 15,
the object can cross a line break but not a page break. If bigger than
14 but less than $2^{31}$, it can't be broken at all,
        nor can a slur termination be positioned at a beam with
        e.g.~\verb|\tbsluru{17684}{16}|; however \verb|\ibsluru{152867}{16}| is OK.

It's also OK to have opened simultaneously a slur, tie and crescendo all
with the same ID, or a slur, tie and decrescendo, but not a crescendo and
decrescendo.

\def\emen{{\tt\char123}$n${\tt\char125\char123}$m${\tt\char125}}

\subsubsection{Type K postscript beam slurs}
Type K beam slurs are defined differently than the font-based ones:
They require as parameters both a slur ID number $n$ and a beam ID number $m$,
but that's all. The commands are
\keyindex{iBsluru}\emen, \keyindex{iBslurd}\emen, and
\keyindex{tBslur}\emen.
They must be placed \ital{after} the beam initiation or termination command.
Type K slurs may start on one beam and end on another.
For example,

\begin{quote}\begin{verbatim}
\Notes\ibu0i0\iBsluru00\qb0{eh}\tbu0\qb0i\ibu0j0\qb0{jl}\tbu0%
 \slurtext{6}\tBslur00\qb0e\en
\end{verbatim}\end{quote}

\noindent produces
%\begin{center}
%\includegraphics[scale=1]{./mxdexamples/psbeamslur.eps}
%\end{center}

\begin{music}\nostartrule\input musixps\startextract
\Notes\ibu0i0\iBsluru00\qb0{eh}\tbu0\qb0i\ibu0j0\qb0{jl}\tbu0%
 \slurtext{6}\tBslur00\qb0e\en
\zendextract\end{music}

The above example also illustrates the use of the macro
\keyindex{slurtext}. It has just one parameter---some text to be
printed---and it centers it just above or below the midpoint of
the next slur that is closed.
This works only for non-breaking slurs; if the slur is broken,
the text
disappears\footnote{If you insist on viewing files with a dvi viewer despite the
fact that Type K slurs will not be visible, you may also find that
figures emplaced with \keyindex{slurtext} will appear at the end of the
slur rather than the middle.}.
%The placing of the slur text is done with a very dirty postscript
%hack, so I am not really sure that everything you want to typeset
%will be placed at the correct position (if you are interested, see
%the end of psslurs.pro for details). If you find something that won't
%work, let me know.

% DAS: why do we need this ???
%\paragraph{General coding for postscript slurs and ties of type K.}
%This can be done by:
%\begin{quote}\begin{verbatim}
%\i[h.shift]slur[u|d]{slur ID}{note height}
%\t[h.shift][slur height]slur{slur ID}{note height}
%\iBslur[u|d]{slur ID}{beam ID}
%\tB[slur.height]slur{slur ID}{beam ID}
%\end{verbatim}\end{quote}
%\noindent where h.shift can be 'l', 'r' or nothing
%and slur.height can be 'f', nothing, 'h', 'H' or 'HH'
%
%Example: \keyindex{tlfslur} means 'terminate left flat slur'.
%
%
%There are also simple slurs with same invocation and parameters as the
%original ones.

\subsubsection{Type K postscript ties}
All of the foregoing Type K slur commands 
except the shape-changing ones
have counterparts for ties. Simply
replace ``\verb|slur|'' with ``\verb|tie|'', and for terminations omit
the pitch parameter. Type K ties not only are positioned differently by
default, but they also have different shapes than slurs. If you want to change
the shape of a tie, redefine \verb|\pstiehgt| from its default of 0.7. 

\subsubsection{Dotted type K slurs and ties.}%please check the capital D(otted)
A slur or tie can be made dotted simply by entering \keyindex{dotted} anywhere before
the beginning of the slur or tie. Only the first slur or tie following this
command will be affected. On the other hand, if you enter \keyindex{Dotted}, then ALL
slurs and ties from this point forward will be dotted until you say
\keyindex{Solid}. Furthermore, inside \verb|\Dotted...\Solid| you can make any individual
slur or tie solid saying \verb|\solid| before its beginning.

\subsubsection{Avoiding collisions of Type K slurs and ties with staff lines.}
In postscript it is possible to do some computations which would be very hard
to implement directly in TeX. Type K slurs can use this facility to check whether
the curve of a slur or tie is anywhere nearly tangent to any staff line, and if
so, to adjust
the altitude of the curve to avoid the collision. By default this feature is turned
on.
You can disable it either
globally (\keyindex{Nosluradjust}, \keyindex{Notieadjust}) or locally
(\keyindex{nosluradjust}, \keyindex{notieadjust}), and you can also turn it
back on globally (\keyindex{Sluradjust},
\keyindex{Tieadjust}) or locally (\keyindex{sluradjust}, \keyindex{tieadjust}).
Here ``locally''
means that the command will only affect the next slur or tie to be opened.


\subsubsection{Line breaking slurs and ties}
Tyle K slurs and ties (and crescendos) going across line breaks are handled
automatically. In fact they can go over more lines than two (this is
true also for ties, though it would be somewhat strange).

There is a switch \verb|\ifslopebrkslurs| that controls the default
height of the end point
of the first segment of all broken slurs. By default the height will be
the same as the beginning. To have it raised by \verb|3\internote|, simply
issue the command \keyindex{slopebrkslurstrue}. To revert to the default,
use \keyindex{slopebrkslursfalse}.

To locally override the default height of the end of the first segment,
use the command \keyindex{breakslur}\enpee,
which sets the height for slur number $n$ to pitch $p$, just like with
font-based slurs.

You can raise or lower the starting point of the second segment of a broken slur
with the command \keyindex{liftslur}\verb|{|$n$\verb|}{|$h$\verb|}|, with
parameters slur ID and relative offset in
\verb|\internote|s measured from the slur beginning. Its effect is
the same as \keyindex{Liftslur} for font-based slurs, except it is not
necessary to code
it within \verb|\atnextstaff{}|, just anywhere inside the slur.

As already mentioned, anything with ID$<$10 is broken fully automatically, but
you should be careful about slurs, ties and crescendos with 10$\le$ID$<$15.
These cannot cross page breaks, although they can cross line breaks.

If the second segment of a broken tie is less than 15pt long, then by
default it will have a special shape which begins horizontally. These shapes are
called \ital{half ties} and are contained in the font \index{mxsk font}\verb|mxsk|.
Of course if they are to be used, the font files must be integrated into the
\TeX\ installation. Their use can be turned off with \keyindex{nohalfties}
and back on with \keyindex{halfties}.

%\paragraph{Backwards compatibility.}
%There are several ``aliases'' which allow to use the old, bitmapped slur
%commands for PS slurs without any change. There are however a few differences:
%the \keyindex{invertslur} is not implemented yet and
%the \keyindex{curve} and \keyindex{midslur} macros have no effect.

%\paragraph{Memory requirements.}
\subsubsection{A few final technical details}
Each \verb|\i...| and \verb|\t...|
produces a \verb|\special| command, which must be stored in TeX's main memory.
Therefore, if too many slurs occur in one page, some memory problems could
occur. The only solutions are to use Big\TeX\ (if you aren't already), or to use
font-based slurs.

Type K slurs need the postscript header file
\index{psslurs.pro}\texttt{psslurs.pro} to be included in the
output postscript file. This is made to happen by the \TeX\ command
\verb|\special{header=psslurs.pro}|, which is automatically
included when you \verb|\input musixps|. So normally this is not
of concern. However if you wish to extract a subset of pages from
the master \verb|dvi| file using \index{dvidvi}\texttt{dvidvi}, then
you have three options: (1)~include the first page in the subset,
(2)~manually issue the
\verb|\special| command in the \TeX\ source for the first page of the subset, or
(3)~use the option \texttt{-h psslurs.pro} when you run \texttt{dvips}.


\section{Bar Lines}
 \subsection{Single, double, and invisible bar lines}\label{doublebars}
The usual macro to typeset a single bar line is \keyindex{bar}.
There is a possibility of
confusion with a command in \TeX's math mode that has exactly the same name.
However, there will generally be no problem, because inside
\verb|\startpiece...\endpiece|,
\verb|\bar| will take the musical meaning, while outside, it will have the
mathematical one.
If for some reason you need the math \verb|\bar| inside, you can use
\verb|\endcatcodesmusic...\bar...\catcodesmusic|.

To typeset a double bar line with two thin rules, use \keyindex{doublebar}. You could
also issue \keyindex{setdoublebar} to cause the next \verb|\bar| (or
\keyindex{stoppiece}, \keyindex{alaligne}, or \keyindex{alapage}) to be replaced
by a double bar.

There is no specific command to print a thin-thick double bar line, but
\keyindex{setdoubleBAR} will cause one in the same cases where
\verb|\setdoublebar| would cause a thin-thin double bar line.

To make the next bar line invisible, use \keyindex{setemptybar} before
\verb|\bar|.

You can suppress the beginning vertical rule with saying \keyindex{nostartrule}
and restore the default with \keyindex{startrule} after that.

\subsection{Simple discoutinuous bar lines}
Normally, bars (as well as double bars, final bars and repeat bars) are
drawn continuously from the bottom of the lowest staff of the
lowest instrument to the top of the highest staff of the uppermost
instrument. However, if desired, they can be made discontinuous between
instruments by saying \keyindex{sepbarrules}. An example of this is given in
%avrb
%{\tt ANGESCAO} (or {\tt ANGESCAM}) example; it has also been used in the
%avre
\texttt{avemaria.tex} in section \ref{avemaria}\label{avemaria2}.

Continuous bar lines can be restored with \keyindex{stdbarrules}. In
the extension library are some more types of bar rules, mainly for very old
music, see section \ref{otherbars}.

% DAS. Andre, are there some other kinds of bars in an addon?

\subsection{Elementary asynchronous bar lines}

 Situations may arise where the bar lines in different instruments are not
aligned with one another.  To set this up, first say \verb|\sepbarrules|.
Then use a combination of the following five commands:

\begin{itemize}\setlength{\itemsep}{0ex}
\item\keyindex{hidebarrule}\onen~hides the bar rule for instrument $n$, until
this is changed by \verb|\showbarrule|\onen.
\item\keyindex{showbarrule}\onen~stops hiding the bar rule for instrument $n$,
until this is changed by \verb|\hidebarrule|\onen.
\item\keyindex{Hidebarrule}\onen~hides the bar rule for instrument $n$, only
for the next bar.
\item\keyindex{Showbarrule}\onen~shows the bar rule for instrument
$n$ once only,
and then resets it \verb|Hidebarrule|.
%
% DAS ???
%  and then resets it to \verb|hidebarrule|.
%
\item\keyindex{showallbarrules} resets all defined instruments to
\verb|\showbarrule|\onen. This command is automatically inserted with double
bars, final bars and repeats.
\end{itemize}

Thus, this example

\begin{music}
\instrumentnumber3
\setmeter3{{\meterfrac{3}{4}}}
\setmeter2{{\meterfrac{2}{4}}}
\setmeter1{{\meterfrac{3}{8}}}
\nobarnumbers
\sepbarrules

\startextract
\NOtes\pt f\qa f&\qa f&\qa f\en
\hidebarrule2\hidebarrule3\bar
\NOtes\multnoteskip{.333}\Tqbu fff&\qa f&\qa f\en
\showbarrule2\bar
\NOtes\pt f\qa f&\qa f&\qa f\en
\hidebarrule2\showbarrule3\bar
\NOtes\multnoteskip{.333}\Tqbu fff&\qa f&\qa f\en
\showbarrule2\hidebarrule3\bar
\NOtes\pt f\qa f&\qa f&\qa f\en
\hidebarrule2\bar
\NOtes\multnoteskip{.333}\Tqbu fff&\qa f&\qa f\en
\setdoublebar
\bar\hidebarrule3
\NOtes\pt f\qa f&\qa f&\qa f\en
\Hidebarrule2\bar
\NOtes\multnoteskip{.333}\Tqbu fff&\qa f&\qa f\en
\bar
\NOtes\pt f\qa f&\qa f&\qa f\en
\message{Showbarrule3 coming}%
\Hidebarrule2\Showbarrule3\bar
\NOtes\multnoteskip{.333}\Tqbu fff&\qa f&\qa f\en
\bar
\NOtes\pt f\qa f&\qa f&\qa f\en
\Hidebarrule2\bar
\NOtes\multnoteskip{.333}\Tqbu fff&\qa f&\qa f\en
\setrightrepeat
\endextract
\end{music}

\noindent was obtained with the following coding:
\begin{quote}\begin{verbatim}
\instrumentnumber3
\setmeter3{{\meterfrac{3}{4}}}
\setmeter2{{\meterfrac{2}{4}}}
\setmeter1{{\meterfrac{3}{8}}}
\nobarnumbers
\sepbarrules

\startextract
\NOtes\pt f\qa f&\qa f&\qa f\en
\hidebarrule2\hidebarrule3\bar
\NOtes\multnoteskip{.333}\Tqbu fff&\qa f&\qa f\en
\showbarrule2\bar
\NOtes\pt f\qa f&\qa f&\qa f\en
\hidebarrule2\showbarrule3\bar
\NOtes\multnoteskip{.333}\Tqbu fff&\qa f&\qa f\en
\showbarrule2\hidebarrule3\bar
\NOtes\pt f\qa f&\qa f&\qa f\en
\hidebarrule2\bar
\NOtes\multnoteskip{.333}\Tqbu fff&\qa f&\qa f\en
\setdoublebar
\bar\hidebarrule3
\NOtes\pt f\qa f&\qa f&\qa f\en
\Hidebarrule2\bar
\NOtes\multnoteskip{.333}\Tqbu fff&\qa f&\qa f\en
\bar
\NOtes\pt f\qa f&\qa f&\qa f\en
\message{Showbarrule3 coming}%
\Hidebarrule2\Showbarrule3\bar
\NOtes\multnoteskip{.333}\Tqbu fff&\qa f&\qa f\en
\bar
\NOtes\pt f\qa f&\qa f&\qa f\en
\Hidebarrule2\bar
\NOtes\multnoteskip{.333}\Tqbu fff&\qa f&\qa f\en
\setrightrepeat
\zendextract
\end{verbatim}\end{quote}

\subsection{Dotted, dashed, and more general asynchronous and discontinuous bar lines}\label{musixdbr}

The extension
package \href{http://icking-music-archive.org/software/musixtex/add-ons/musixdbr.tex}
{\underline{\ttxem{musixdbr.tex}}} by Rainer {\sc Dunker} provides commands for
dashed, dotted, and arbitrarily discontinuous bar lines. It supports
individual bar line settings for each instrument, multi-staff instruments,
different sizes of staves, and even different numbers of lines per staff,

To use the package, you must \verb|\input musixdbr| after \verb|musixtex|, and
execute the macro \keyindex{indivbarrules} which activates individual bar line
processing. Then the following commands are available:

\begin{itemize}\setlength{\itemsep}{0ex}

\item  \keyindex{sepbarrule}\onen~separates bar lines of instrument $n$ from those of instrument $n-1$

\item \keyindex{condashbarrule}\onen~connects bar lines of instrument $n$ to those of instrument $n-1$
   with a dashed line

\item \keyindex{condotbarrule}\onen~connects bar lines of instrument $n$ to those of instrument $n-1$
   with a dotted line

\item \keyindex{conbarrule}\onen~connects bar lines of instrument $n$ to those of instrument $n-1$

\item \keyindex{hidebarrule}\onen~hides bar lines of instrument $n$

\item \keyindex{showdashbarrule}\onen~dashes bar lines of instrument $n$

\item \keyindex{showdotbarrule}\onen~dots bar lines of instrument $n$

\item \keyindex{showbarrule}\onen~shows bar lines of instrument $n$

\item \keyindex{sepmultibarrule}\onen~separates bar lines within multistaff instrument $n$

\item \keyindex{condashmultibarrule}\onen~dashes bar lines between staves of multistaff instrument $n$

\item \keyindex{condotmultibarrule}\onen~dots bar lines between staves of multistaff instrument $n$

\item \keyindex{conmultibarrule}\onen~ shows bar lines between staves of multistaff instrument $n$

\item \keyindex{allbarrules}[\ital{any of the above commands, without numerical parameter}] sets bar
line style for all instruments together.

\end{itemize}

Dashing and dotting style may be changed by redefining the macros
\verb|\barlinedash| or \verb|\barlinedots| respectively (see original definitions in \verb|musixdbr.tex|).

Here is an example of the use of these macros:

\begin{music}
\input musixdbr

\instrumentnumber4 \setstaffs23 \setstaffs32 \setlines14\setsize2\tinyvalue
\indivbarrules
\parindent0pt\startextract
%\startpiece
%\scale{2}
  % normal barlines
  \bar
  % separate instrument 2 from 1
  \sepbarrule2
  \notes\en\bar
  % barlines on staves
  \allbarrules\sepbarrule
  \allbarrules\sepmultibarrule
  \allbarrules\showbarrule
  \NOTes\en\bar
  % barlines between staves
  \allbarrules\conbarrule
  \allbarrules\conmultibarrule
  \allbarrules\hidebarrule
  \NOTes\en\bar
  % dashed barlines on staves
  \allbarrules\sepbarrule
  \allbarrules\sepmultibarrule
  \allbarrules\showdashbarrule
  \NOTes\en\bar
  % dashed barlines between staves
  \allbarrules\condashbarrule
  \allbarrules\condashmultibarrule
  \allbarrules\hidebarrule
  \NOTes\en\bar
  % dotted barlines on staves
  \allbarrules\sepbarrule
  \allbarrules\sepmultibarrule
  \allbarrules\showdotbarrule
  \NOTes\en\bar
  % dotted barlines between staves
  \allbarrules\condotbarrule
  \allbarrules\condotmultibarrule
  \allbarrules\hidebarrule
  \NOTes\en\bar
  % a wild mixture of all
  \showdotbarrule1\hidebarrule2\showdashbarrule3\showbarrule4%
  \condashbarrule2\conbarrule3\condotbarrule4%
  \condashmultibarrule2\sepmultibarrule3%
  \NOTes\en\bar
  % conventional ending
  \allbarrules\showbarrule
  \allbarrules\conbarrule
  \allbarrules\conmultibarrule
  \NOTes\en\setdoubleBAR\endextract
\end{music}

This is the code:

\begin{quote}\begin{verbatim}
\input musixdbr
\instrumentnumber4\setstaffs23\setstaffs32\setlines14\setsize2\tinyvalue
\indivbarrules\parindent0pt\startextract
  % normal barlines
  \bar
  % separate instrument 2 from 1
  \sepbarrule2
  \notes\en\bar
  % barlines on staves
  \allbarrules\sepbarrule
  \allbarrules\sepmultibarrule
  \allbarrules\showbarrule
  \NOTes\en\bar
  % barlines between staves
  \allbarrules\conbarrule
  \allbarrules\conmultibarrule
  \allbarrules\hidebarrule
  \NOTes\en\bar
  % dashed barlines on staves
  \allbarrules\sepbarrule
  \allbarrules\sepmultibarrule
  \allbarrules\showdashbarrule
  \NOTes\en\bar
  % dashed barlines between staves
  \allbarrules\condashbarrule
  \allbarrules\condashmultibarrule
  \allbarrules\hidebarrule
  \NOTes\en\bar
  % dotted barlines on staves
  \allbarrules\sepbarrule
  \allbarrules\sepmultibarrule
  \allbarrules\showdotbarrule
  \NOTes\en\bar
  % dotted barlines between staves
  \allbarrules\condotbarrule
  \allbarrules\condotmultibarrule
  \allbarrules\hidebarrule
  \NOTes\en\bar
  % a wild mixture of all
  \showdotbarrule1\hidebarrule2\showdashbarrule3\showbarrule4%
  \condashbarrule2\conbarrule3\condotbarrule4%
  \condashmultibarrule2\sepmultibarrule3%
  \NOTes\en\bar
  % conventional ending
  \allbarrules\showbarrule
  \allbarrules\conbarrule
  \allbarrules\conmultibarrule
  \NOTes\en\setdoubleBAR\zendextract
\end{verbatim}\end{quote}

\section{Bar numbering}
The current bar number is stored in a count register call \keyindex{barno}.
When \verb|\startpiece| is encountered, \verb|\barno| is set equal to
another count register called \verb|\startbarno|, whose default value is one.
Therefore, if you want the first bar to have a number $n$ different from 1, you
may either say \verb|\startbarno=|$n$ before \verb|\startpiece|, or say
\verb|\barno=|$n$ afterwards, but before the first bar line. You may also alter the bar
number at any time, either by explicitly resetting \verb|\barno|, or by
incrementing it with a command like \verb|\advance\barno-1|.

\musixtex\ supports two distinct modes for printing bar numbers. In \ital{periodic}
bar numbering, the bar number is
placed above the top staff with a user-selectable frequency. In
\ital{system} bar numbering, the
number will appear at the beginning of each system.

\subsection{Periodic bar numbering}
In a normal piece, periodic bar number printing is turned on by default, with a frequency
of one. In an extract, the default is to not print bar numbers.
To turn off bar numbering say \keyindex{nobarnumbers}. To reinstate periodic
bar numbering, or to initiate it in an extract, say \keyindex{barnumbers}.
To change to a
different frequency $n$, say \verb|\def|\keyindex{freqbarno}\verb|{|$n$\verb|}|.

The appearance and positioning of the bar number is controlled by the
token \keyindex{writethebarno}, which by default is defined as\\
\verb|\def\writethebarno{\fontbarno\the\barno\kernm\qn@width}|
where the font is defined as \verb|\def\fontbarno{\it}|. You can change
either of these as desired, for example

  \medskip
\begin{music}\barnumbers
\parindent0pt\startpiece
\Notes\Dqbu gh\Dqbl jh\en
\notes\Dqbbu fg\Dqbbl hk\en\bar
\Notes\Tqbu ghi\Tqbl mmj\en
\def\fontbarno{\bf}%
\notes\Tqbbu fgj\Tqbbl njh\en\bar
\Notes\Qqbu ghjh\Qqbl jifh\en\bar
\notes\Qqbbu fgge\Qqbbl jhgi\en\endpiece
\end{music}
\noindent which was coded as
\begin{quote}\begin{verbatim}\barnumbers
\Notes\Dqbu gh\Dqbl jh\en
\notes\Dqbbu fg\Dqbbl hk\en\bar
\Notes\Tqbu ghi\Tqbl mmj\en
\def\fontbarno{\bf}%
\notes\Tqbbu fgj\Tqbbl njh\en\bar
\Notes\Qqbu ghjh\Qqbl jifh\en\bar
\notes\Qqbbu fgge\Qqbbl jhgi\en
\end{verbatim}\end{quote}

\subsection{System bar numbering}
To have a bar number printed just above the beginning of each system, use
\keyindex{systemnumbers}. The distance above the staff is controlled
by \verb|\raisebarno|, which by default is \verb|4\internote| (to fit
above a treble clef). This can be redefined with the command\\
\verb|\def|\keyindex{raisebarno}\verb|{|\ital{any \TeX~dimension}\verb|}|.
The horizontal position similarly is defined by \keyindex{shiftbarno}
which by default is \verb|0pt|.

The number normally is enclosed in a
box. If you don't like that, you may redefine the macro \verb|\writebarno|
which by default is defined as\\
\verb|\def|\keyindex{writebarno}\verb|{\boxit{\eightbf\the\barno\barnoadd}}|.\\
This uses the utility \musixtex\ macro \keyindex{boxit} which will enclose
any text string in a box.

Here are some possible alternate formats for system bar numbers:

\medskip
\begin{music}\nostartrule
\def\fontbarno{\it}%
\let\extractline\hbox
\startbarno=36
\hbox to \hsize{%
\hss
  \raise20pt\hbox{(a) }%
  \systemnumbers\startextract
    \Notes\wh g\en
  \zendextract
\hss
  \def\writebarno{\tenrm\the\barno\barnoadd}%
  \def\raisebarno{2\internote}%
  \def\shiftbarno{2.5\Interligne}%
  \raise20pt\hbox{(b) }%
  \systemnumbers\startextract
    \Notes\wh g\en
  \zendextract
\hss
  \def\writebarno{\llap{\tenbf\the\barno\barnoadd}}%
  \def\raisebarno{2\internote}%
  \def\shiftbarno{1.3\Interligne}%
  \raise20pt\hbox{(c) }%
  \systemnumbers\startextract
    \Notes\wh g\en
  \zendextract
\hss}
\end{music}
\noindent This was coded as
\begin{itemize}\setlength{\itemsep}{0ex}
 \item[(a)] (default)
 \item[(b)]
  \begin{verbatim}
\def\writebarno{\tenrm\the\barno\barnoadd}%
\def\raisebarno{2\internote}%
\def\shiftbarno{2.5\Interligne}%
  \end{verbatim}
 \item[(c)]
  \begin{verbatim}
\def\writebarno{\llap{\tenbf\the\barno\barnoadd}}%
\def\raisebarno{2\internote}%
\def\shiftbarno{1.3\Interligne}%
  \end{verbatim}
\end{itemize}

If the previous line does not stop with a bar rule, then the next
printed system bar number will immediately be followed by the contents
of the token \keyindex{writezbarno}, whose
default setting is the lower case character `\verb|a|'. You might want to
change this to `\verb|+|', in which case you should say
\verb|\def\writezbarno{+}|.

%DAS: Who cares ???
%\noindent Besides, you can suppress the messages of bar numbers on
%\verb|stdout| (normally screen) with \keyindex{nobarmessage}. In the same way,
%you can suppress the messages abour new lines (new systems) with
%\keyindex{nolinemessages}.
 %\check

\section{Managing the layout of your score}
\subsection{Line and page breaking}\label{linebreak}
If every bar ends with \verb|\bar| and no other line- or page-breaking commands
are used, then the external program \verb|musixflx| will decide where to insert
line and page breaks. Line breaks will only come at bar lines. The total
number of lines will depend on the initial value of \verb|\elemskip|, which by
default is 6 pt in \verb|\normalmusicsize|.

You can force a line or page break by replacing \keyindex{bar} with
\keyindex{alaligne} or \keyindex{alapage} respectively. On the other hand,
to forbid line-breaking at a particular bar, replace \verb|\bar|
with \keyindex{xbar}. To force a line or page break where there is
no bar line, use \keyindex{zalaligne} or \keyindex{zalapage}. To mark any
mid-bar location as an optional line-breaking point, use \keyindex{zbar}.

The heavy final double bar of a piece is one of the consequences of
\keyindex{Endpiece} or \keyindex{Stoppiece}. If you just want to terminate
the text with a simple bar, say \keyindex{stoppiece} or \keyindex{endpiece}.
To terminate it with no bar line at all, code \keyindex{zstoppiece}.

Once you have stopped the score by any of these means, you may want to restart
it, and there are several ways to do so. If you don't need to change the key
signature, meter, or clef,
you can use \keyindex{contpiece} for no indentation, or \keyindex{Contpiece}
to indent by \keyindex{parindent}. If you do want to change some score
attribute that takes up space, for example
with \keyindex{generalsignature} after \verb|\stoppiece|, then to restart you
must use \keyindex{startpiece}. However, if you don't want \verb|\barno| reset
to 1, you must save its new starting value to \verb|\startbarno|. You may also
wish to modify instrument names or \verb|\parindent| before restarting.

Recall that thin-thin or thin-thick double bars or blank bar lines can be
inserted without stopping by using the commands described in section
\ref{doublebars}. Those commands can also be used before \verb|\stoppiece|,
\verb|\alaligne|, or \verb|\alapage|

\subsection{Page layout}
Blank space above and below each staff is controlled by the dimension
registers \keyindex{stafftopmarg} and \keyindex{staffbotmarg}. For more
info see section \ref{LayoutParameters}.

The macro \keyindex{raggedbottom} will remove all vertical glue and
compact everything toward the top of page.
In contrast, the macro \keyindex{normalbottom} will restore default
behavior, in which vertical space between the systems is distributed
so that the first staff
on the page is all the way at the top and the last staff all the way at
the bottom.

The macro \keyindex{musicparskip} will allow the existing space between
systems to increase by up to \verb|5\Interligne|.

The following values of page layout parameters will allow the maximum material
to fit on each page, provided the printer allows the margins that are implied.

\begin{center}
\begin{tabular}{|l|l|l|}\hline
\multicolumn{1}{|c|}{A4}&\multicolumn{1}{|c|}{letter}&\multicolumn{1}{|c|}{A4 or letter}\\\hline
\multicolumn{3}{|c|}{\tt\Bslash parindent= 0pt}\\\hline
\multicolumn{3}{|c|}{\tt\Bslash hoffset= -15.4mm}\\\hline
\multicolumn{3}{|c|}{\tt\Bslash voffset= -10mm}\\\hline
\verb+\hsize= 190mm+&\verb+\hsize= 196mm+&\verb+\hsize= 190mm+\\\hline
\verb+\vsize= 260mm+&\verb+\vsize= 247mm+&\verb+\vsize= 247mm+\\\hline
\end{tabular}
\end{center}
%avre
\zkeyindex{parindent}\zkeyindex{hoffset}\zkeyindex{voffset}
\zkeyindex{hsize}\zkeyindex{vsize}


\subsection{Page numbering, headers and footers}\index{page
number}\index{footnote}

There are no special page numbering facilities in \musixtex; you must rely on
macros from plain \TeX. There is a count register in \TeX\ called
\verb|\pageno|. It is always initialized to \verb|1| and incremented by one
at every page break. By saying \keyindex{pageno}$=n$, it can be reset to any
value at any time, and will continue to be incremented from the new value.

By default, \TeX\ will place a page number on
every page, centered at the bottom. Unfortunately, this is not the preferred
location according to any standard practice. To suppress this default
behavior, say \keyindex{nopagenumbers}.

One way to initiate page numbering in a more acceptable location is to take
advantage of the facts that (a) \TeX\ prints the contents of the control sequences
\keyindex{headline} and \keyindex{footline} at the top and bottom
respectively of every page, and (b) the value of \verb|\pageno| can be printed by
saying \keyindex{folio}. Therefore, for example, the following sequence of
commands, issued anywhere before the end of the first page, will cause page
numbers and any desired text to be printed at the top of every page,
alternating between placement of the number at the left and right margins, and
alternating between the two different text strings (the capitalized text in
the example):

\begin{quote}\begin{verbatim}
\nopagenumbers
\headline={\ifodd\pageno\rightheadline\else\leftheadline\fi}%
\def\rightheadline{\tenrm\hfil RIGHT RUNNING HEAD\hfil\folio}%
\def\leftheadline{\tenrm\folio\hfil LEFT RUNNING HEAD\hfil}%
\voffset=2\baselineskip
\end{verbatim}\end{quote}

\subsection{Controlling the total number of systems and pages}\index{page and line
layout (global)}

Once all the notes are entered into a \musixtex\ score, it would be convenient
if there were a simple sequence of commands to
cause a specified number of systems to be uniformly distributed over a
specified number of pages. Unfortunately that's not directly
possible\footnote{It \textit{is} possible in \textbf{PMX}.}.
Rather, some trial and error will usually be required to achieve the desired final
layout. To this end, some combination of the following strategies may be used:

 \begin{enumerate}\setlength{\itemsep}{0ex}
  \item Assign a value to the count register \keyindex{linegoal} representing
the total number of systems. The count register \keyindex{mulooseness} must be 0 if using
\verb|\linegoal|.
  \item Explicitly force line and page breaking as desired, using
the macros \verb|\alaligne|, \verb|\alapage|, \verb|\zalaligne|
or \verb|\zalapage|.
  \item Adjust both \keyindex{mulooseness} and the initial value of
\keyindex{elemskip}: increasing \verb|\mulooseness| from its default of 0 will
increase the total number of systems, while changing the initial value of
\verb|\elemskip| (use \verb|\showthe\elemskip| to find its default value) may change the
total number of systems accordingly.
 \end{enumerate}
 %\check

\section{Changing clefs, key signatures, and meters}

\subsection{Introduction}\label{contextintro}

To define clefs, key signatures, or meters at the start of a piece,
or to change one or more of these attributes\footnote{In this section,
\ital{attribute} will refer to any clef, key signature, or meter.}~anywhere
else in a score,
\musixtex\ requires two steps. The first step is to \ital{set} the new values of
the attributes.
Most of the commands for this have the form \verb|\set|...~. They will be
described in the following subsections.

But this alone will
not cause anything to be changed or printed. The second step is to activate the
change. This is done by issuing one of the
following commands (outside \verb|\notes|$\dots$\verb|\en|):
\verb|\startpiece|, \verb|\startextract|, \verb|\contpiece|,
\verb|\Contpiece|,
\verb|\alaligne|, \verb|\alapage|, \verb|\zalaligne|, \verb|\zalapage|,
\verb|\changecontext|,
\verb|\Changecontext|, \verb|\zchangecontext|, \verb|\changesignature|,
\verb|\changeclefs|, or \verb|\zchangeclefs|. Most of these perform
other functions as well, and some may be used even when no attributes
change. Features unrelated to changing attributes are detailed elsewhere.
The first 11 will activate all pending new attributes.
If more than one type is activated by a single command in this manner,
then regardless of the order
they were set, they will always appear in the following order: clef, key signature,
meter.
The last three commands in the above list obviously activate only
the specific type of attribute referred to in the name of the command.

The macros \verb|\changecontext|, \verb|\Changecontext|, \verb|\zchangecontext|
will respectively insert a single, double, or invisible bar
line before printing the attributes.

\subsection{Key Signatures}

We've already seen in section \ref{whatspecify} how to set key
signatures for all instruments with \keyindex{generalsignature}, or for
specific instruments with \keyindex{setsign}. As just noted, these commands only
prepare for the insertion of the signatures into the score; it is
really \verb|\startpiece| that puts them in place at the beginning of the
score.

The commands \verb|\generalsignature| and \verb|\setsign| also serve to set
new key signature(s) anywhere in score. The change can then be activated with one of
the 11 general commands listed above, or with \keyindex{changesignature} if
in the middle of a bar.
While neither \verb|\changesignature| nor \verb|\zchangecontext| prints a bar
line, the differences are that the latter increments the bar number counter and
inserts a horizontal space of \verb|\afterruleskip| after the new signature(s).
All of these options will repost signatures that have not been changed.

Normally, changing a signature from flats to sharps or vice-versa, or
reducing the number of sharps or flats, will produce the appropriate set of
naturals to indicate what has been suppressed. This standard feature can be
temporarily inhibited by the command \keyindex{ignorenats} issued right before
the change-activating command.

Here is an example showing various possibilities for changing key signatures.
Note the comments between the code lines.

\begin{quote}\begin{verbatim}
\instrumentnumber2\setstaffs22%
\setclef1{\bass}\generalsignature2%
\startextract
\Notes\qu K&\qu d|\qu e\en
% Signature change in a single instrument with two staves.
% Naturals appear by default, indicating cancelled sharps.
\setsign20\changesignature
\Notes\qu J&\qu d|\qu e\en
% When changing signature in the middle of a bar and no naturals
% are posted, the new signature can be confused with a simple accidental.
\setsign11\ignorenats\changesignature
\Notes\qu M&\qu d|\qu e\en
% New signatures after a double bar line
\generalsignature{-2}\Changecontext%
\Notes\qu K&\qu d|\qu e\en%
\Notes\qu K&\qu d|\qu e\en%
% New signatures after an invisible bar line. Note the
% difference in spacing compared with beat 3 of the prior measure
\generalsignature{1}\zchangecontext%
\Notes\qu K&\qu d|\qu e\en%
\end{verbatim}\end{quote}

\begin{music}
\instrumentnumber2\setstaffs22%
\setclef1{\bass}\generalsignature2%
\startextract
\Notes\qu K&\qu d|\qu e\en
% Signature changing in a single instrument with 2 staves
% Naturals are allowed to shown there are no more sharps
\setsign20\changesignature
\Notes\qu J&\qu d|\qu e\en
% Signature changing in a single staff without naturals
% if there is no bar line, the signatures are confusing:
% it is not clear if they are for a single note or not
\setsign11\ignorenats\changesignature
\Notes\qu M&\qu d|\qu e\en
% New signatures after a double bar line
\generalsignature{-2}\Changecontext%
\Notes\qu K&\qu d|\qu e\en%
\Notes\qu K&\qu d|\qu e\en%
% New signatures after an invisible bar line
% see the difference in space comparing beat 2
\generalsignature{1}\zchangecontext%
\Notes\qu K&\qu d|\qu e\en%
\endextract
\end{music}


\subsection{Clefs}\label{treblelowoct}

Macros that define clefs have already been discussed in section
\ref{whatspecify}. By way of review, here are all of the possible clefs
(applied to the lowest staff):
%avrb
\newcommand{\musicintextsign}[1]{\musicintext#1{\notes\en}}

\begin{center}\vskip-1ex\footnotesize%\small
\begin{tabular}{||c|c|c|c|c||}\hline\hline
&&&&\\
\verb+\setclef1\treble+&&&\verb+\setclef1\alto+&\\
\verb+\setclef10+&\verb+\setclef11+&\verb+\setclef12+
&\verb+\setclef13+&\verb+\setclef14+\\[-2ex]%\hline
\musicintextsign{\treble}&\musicintextsign1&\musicintextsign2
&\musicintextsign{\alto}&\musicintextsign4\\[3ex]\hline
%\end{tabular}
&&&&\\
%\begin{tabular}{ccccc}
&\verb+\setclef1\bass+&&\verb+\setclefsymbol1\empty+\footnotemark&\\
\verb+\setclef15+&\verb+\setclef16+&\verb+\setclef17+
&\verb+\setclef18+&\verb+\setclef19+\\[-2ex]%\hline
\musicintextsign5&\musicintextsign{\bass}&\musicintextsign7
&\musicintextsign8&\musicintextsign9\\[3ex]\hline\hline
%This one uses \setclefsymbol1\empty :
%&\musicintextnoclef{\notes\qu h\en}&\musicintextsign9\\[3ex]\hline\hline
\end{tabular}
\footnotetext{Details of the macro {\tt\Bslash setclefsymbol} will be discussed later}
\end{center}
%avre

Just as with key signatures, these commands only \ital{prepare} for clef changes.
To \ital{activate} them, any of the first 11 commands listed in section
\ref{contextintro} could be used. However, one should keep in mind that
according to modern conventions, a clef change at a bar line is posted
before the bar line, while for example \verb|\changecontext| would post it
after the bar line. In part for this reason, we have the special command
\keyindex{changeclefs}. It can be used anywhere outside
\verb+\notes...\enotes+ to activate a clef change and insert an amount of
horizontal space to accommodate the new clef symbol(s), without printing a
bar line. Sometimes no added
space is required, in which case \keyindex{zchangeclefs} should be used.

Here are some examples of clef changes:

\begin{quote}\begin{verbatim}
\instrumentnumber2\setstaffs22%
\setclef1{\bass}\generalsignature2%
\startextract
% Change in one staff only, with added space
\setclef1\treble\changeclefs%
\Notes\qu k&\qu e|\cu{.d}\ccu{e}\en%
% Combined with signature change, also no extra space needed
% twice the same clef in staff 2 - with the help of a blank clef
\setclef28\zchangeclefs\setclef2\treble%
\setclef1\bass\zchangeclefs\setsign1{-2}\setsign2{-2}%
\ignorenats\changesignature%
\Notes\qu K&\cu{de}|\qu e\en%
% clef change before barline
\setclef1\treble\zchangeclefs\bar%
\Notes\qu k&\cu{de}|\qu e\en%
% clef change after barline
\setclef1\bass\bar\changeclefs%
\Notes\qu K&\cu{de}|\qu e\en%
% clef change after barline with changecontext
\setclef1\treble\changecontext%
\Notes\cu k&\cu d|\qu e\en%
% twice the same clef
\setclef18\zchangeclefs\setclef1\treble\changeclefs%
\Notes\cu k&\cu e |\en%
\zendextract
\end{verbatim}\end{quote}
\begin{music}
\instrumentnumber2\setstaffs22%
\setclef1{\bass}\generalsignature2%
\startextract
% Change in one staff only, with added space
\setclef1\treble\changeclefs%
\Notes\qu k&\qu e|\cu{.d}\ccu{e}\en%
% Combined with signature change, also no extra space needed
% twice the same clef in staff 2 - with the help of a blank clef
\setclef28\zchangeclefs\setclef2\treble%
\setclef1\bass\zchangeclefs\setsign1{-2}\setsign2{-2}%
\ignorenats\changesignature%
\Notes\qu K&\cu{de}|\qu e\en%
% clef change before barline
\setclef1\treble\zchangeclefs\bar%
\Notes\qu k&\cu{de}|\qu e\en%
% clef change after barline
\setclef1\bass\bar\changeclefs%
\Notes\qu K&\cu{de}|\qu e\en%
% clef change after barline with changecontext
\setclef1\treble\changecontext%
\Notes\cu k&\cu d|\qu e\en%
% twice the same clef
\setclef18\zchangeclefs\setclef1\treble\changeclefs%
\Notes\cu k&\cu e |\en%
\endextract
\end{music}

\noindent Of course the examples in the last two bars are contrary
to accepted practice.

Clef changes initiated with the \verb|\setclef| command have several features
in common. When activated after the beginning of the piece, the printed symbol
is smaller than the normal one used at the beginning of the piece. Also,
\musixtex\ automatically adjusts vertical positions of noteheads consistent
with the new clef.

There is an additional group of macros for setting new clefs which does not share
either of these features. In other words, they will always print full sized symbols, and they
won't change the vertical positions of noteheads from what they would have been
before the new symbol was printed. We could call this process ``clef symbol substitution'',
because all it does is print a different symbol (or no symbol at all) in place of
the underlying clef which was set in the normal way.

You'll need to use clef symbol substitution if you want to have a so-called
\index{octave clefs}\label{octclef}octave treble clef or octave bass
clef, i.e., one containing a numeral 8 above or
below the normal symbol. The syntax for setting upper
octaviation
is \keyindex{setbassclefsymbol}\onen\keyindex{bassoct}\\
or \keyindex{settrebleclefsymbol}\onen\keyindex{trebleoct}; for lower octaviation
it is\\ \keyindex{setbassclefsymbol}\onen\keyindex{basslowoct} or
\keyindex{settrebleclefsymbol}\onen\keyindex{treblelowoct}. Because these sequences act to
\ital{replace} normal bass or treble clefs with a different symbol, they
require that the normal clefs be set first. For example

\noindent\begin{minipage}{80mm}
\begin{music}\nostartrule
 \parindent 19mm
 \instrumentnumber{4}
 \generalmeter{\empty}
 \setclef1\bass \setclef2\bass \setclef3\treble \setclef4\treble
 \setbassclefsymbol1\basslowoct
 \setbassclefsymbol2\bassoct
 \settrebleclefsymbol3\treblelowoct
 \settrebleclefsymbol4\trebleoct
\startextract
\Notes\qu{`abcdefghi}&\qu{`abcdefghi}&\qu{abcdefghi}&\qu{abcdefghi}&\enotes
\zendextract
\end{music}
\end{minipage}
\begin{minipage}{50mm}
\begin{verbatim}
\parindent 19mm
\instrumentnumber{4}
\generalmeter{\empty}
\setclef1\bass \setclef2\bass
\setclef3\treble \setclef4\treble
\setbassclefsymbol1\basslowoct
\setbassclefsymbol2\bassoct
\settrebleclefsymbol3\treblelowoct
\settrebleclefsymbol4\trebleoct
\startextract
\Notes\qu{`abcdefghi}&\qu{`abcdefghi}%
&\qu{abcdefghi}&\qu{abcdefghi}&\enotes
\zendextract
\end{verbatim}
\end{minipage}

\index{clefs (empty)}
Another application of clef symbol substitution is to cause no clef to be
printed, as for example might be desired in percussion music, This
can be accomplished with \keyindex{setclefsymbol}\onen\verb|\empty|, which
once activated would replace \ital{all} clef symbols in the first (lowest)
staff of instrument $n$ with blanks.

Normal symbols for those clefs that have been substituted can be restored by
\keyindex{resetclefsymbols}.

Four other small clef symbols are available: \verb|\smalltrebleoct|, 
\verb|\smalltreblelowoct|, \verb|\smallbassoct|, and \verb|\smallbasslowoct|. 
They look just like the corresponding normal-sized symbols, and are useful by
clef symbol substitution for clef changes after the beginning of a piece, as
demonstrated in the following example.

The various clef symbol substitution commands can only be used to substitute for
\texttt{treble}, \texttt{alto}, or \texttt{bass} clefs.

In the following example, (1) is two normal clef changes. At (2) the clef is first changed
back to treble and then the \verb|\treblelowoct| symbol is substituted by using
\verb|\settrebleclefsymbol|.
When changing the clef away from treble and then back as at (3), the substitution
symbol is still in force. At (4), \verb|\resetclefsymbols| cancels the symbol substitution.
If using \verb|\setclefsymbol| all available clefs are changed to the same symbol, as you can see
in the three clefs after (5) in comparison with (2). These also illustrate the use of the small
octave clef symbol. Obviously the second clef after (5) is
nonsense; \verb|\resetclefsymbols| puts matters in order at (6) and (7).

\begin{music}
\instrumentnumber1\setclef1\bass
\startpiece
\notes\zchar{-5}{1}\qu H\en\setclef1\treble\changeclefs
\notes\qu i\en\setclef1\bass\changeclefs\bar
\notes\zchar{-5}{2}\qu J\en
\setclef1\treble\settrebleclefsymbol1\treblelowoct\changeclefs
\notes\qu i\en\setclef1\bass\changeclefs\bar
\notes\zchar{-5}{3}\qu I\en\setclef1\treble\changeclefs
\notes\qu i\en\setclef1\bass\changeclefs\bar
\notes\zchar{-5}{4}\qu J\en\resetclefsymbols\setclef1\treble\changeclefs
\notes\qu i\en\setclef1\bass\changeclefs\doublebar
\notes\zchar{-5}{5}\qu J\en
\setclef1\treble\setclefsymbol1\smalltreblelowoct\changeclefs
\notes\qu i\en\setclef1\bass\changeclefs\bar
\notes\qu I\en\setclef1\treble\changeclefs
\notes\zchar{-5}{6}\qu i\en\resetclefsymbols\setclef1\bass\changeclefs\bar
\notes\zchar{-5}{7}\qu J\en\setclef1\treble\changeclefs
\notes\qu i\en
\endpiece
\end{music}

\noindent This is the code:

\begin{quote}\begin{verbatim}
\begin{music}\nostartrule
\instrumentnumber1\setclef1\bass
\startpiece
\notes\zchar{-5}{1}\qu H\en\setclef1\treble\changeclefs
\notes\qu i\en\setclef1\bass\changeclefs\bar
\notes\zchar{-5}{2}\qu J\en
\setclef1\treble\settrebleclefsymbol1\treblelowoct\changeclefs
\notes\qu i\en\setclef1\bass\changeclefs\bar
\notes\zchar{-5}{3}\qu I\en\setclef1\treble\changeclefs
\notes\qu i\en\setclef1\bass\changeclefs\bar
\notes\zchar{-5}{4}\qu J\en\resetclefsymbols\setclef1\treble\changeclefs
\notes\qu i\en\setclef1\bass\changeclefs\doublebar
\notes\zchar{-5}{5}\qu J\en
\setclef1\treble\setclefsymbol1\smalltreblelowoct\changeclefs
\notes\qu i\en\setclef1\bass\changeclefs\bar
\notes\qu I\en\setclef1\treble\changeclefs
\notes\zchar{-5}{6}\qu i\en\resetclefsymbols\setclef1\bass\changeclefs\bar
\notes\zchar{-5}{7}\qu J\en\setclef1\treble\changeclefs
\notes\qu i\en
\endpiece
\end{verbatim}\end{quote}

\subsection{Meter changes}

As mentioned in section \ref{generalmeter}, a common \itxem{meter} for all
staves can be specified by \keyindex{generalmeter}\verb|{|$m$\verb|}|,
where $m$ denotes the meter. On the other hand, meter changes in specific
staves are implemented with
\keyindex{setmeter}\onen\verb|{{|$m1$\verb|}{|$m2$\verb|}{|$m3$\verb|}{|$m4$\verb|}}|,
where $n$ is the number of the instrument, $m1$ specifies the meter
of the first (lowest) staff, $m2$ the second staff, and so forth. (Only enter
as many $m$'s as necessary.)

Since meter changes are meaningful only across bars, there is no special command to
activate a new meter; rather, they are activated with the general commands
\keyindex{changecontext}, etc., listed in section \ref{contextintro}.

The next example shows a few methods to get a meter change, in all staves or
in a single staff.

\begin{quote}\begin{verbatim}
\instrumentnumber2\setstaffs22%
\generalmeter{\meterfrac{4}{4}\meterfrac{2}{4}\meterfrac{1}{4}}%
\setclef1{\bass}\generalsignature2%
\startextract
\setmeter1{{\meterfrac{2}{4}}}%
\setmeter2{{\lower2pt\hbox{\meterfrac{\Bigtype 2}{}}}%
{\meterfrac{3}{4}}}\changecontext
\Notes\qu K&\cu{de}|\qu e\en
% bar 11
% Meters, clefs, and key signatures.
% All 3 clefs after bar (probably bad form) if no changeclefs
\setmeter1{{\meterfrac{2}{8}}}%
\setmeter2{{\meterfrac{3}{6}}{\meterfrac{3}{8}}}%
\setsign2{-1}%
% How to force showing the bass clef?
\setclef1\bass\setclef2{23}%
\Changecontext
\Notes\qu K&\cu{de}|\qu e\en
% bar 12
% Meters, clefs, and key signatures, with clef before the bar.
% Maybe not best form if signatures are involved
\setmeter1{{\meterfrac{2}{4}}}%
\setmeter2{{\meterfrac{3}{8}}{\meterfrac{3}{6}}}%
\setsign2{-1}%
\setclef1\treble\zchangeclefs\changecontext
\Notes\qu k&\cu{de}|\qu e\en
\end{verbatim}\end{quote}
\begin{music}
\instrumentnumber2\setstaffs22%
\generalmeter{\meterfrac{4}{4}\meterfrac{2}{4}\meterfrac{1}{4}}%
\setclef1{\bass}\generalsignature2%
\startextract
\setmeter1{{\meterfrac{2}{4}}}%
% How big the '2' must be?
\setmeter2{{\lower2pt\hbox{\meterfrac{\Bigtype 2}{}}}%
{\meterfrac{3}{4}}}\changecontext
\Notes\qu K&\cu{de}|\qu e\en
% bar 11
% Meter Clefs and Key Signatures
% all 3 clefs after bar if no changeclefs
\setmeter1{{\meterfrac{2}{8}}}%
\setmeter2{{\meterfrac{3}{6}}{\meterfrac{3}{8}}}%
\setsign2{-1}%
% How to force showing the bass clef?
\setclef1\bass\setclef2{23}%
\Changecontext
\Notes\qu K&\cu{de}|\qu e\en
% bar 12
% Meter Clefs Keys Signatures all 3 with clef before the bar
% probably not if signatures are involved
\setmeter1{{\meterfrac{2}{4}}}%
\setmeter2{{\meterfrac{3}{8}}{\meterfrac{3}{6}}}%
\setsign2{-1}%
\setclef1\treble\zchangeclefs\changecontext
\Notes\qu k&\cu{de}|\qu e\en
\endextract
\end{music}

\section{Repeats}

To replace a bar line with a left, right, or left-right repeat, use one of the
commands \keyindex{leftrepeat}, \keyindex{rightrepeat} or
\keyindex{leftrightrepeat} in place of \verb|\bar|. If a \verb|\leftrepeat|
happens to come at the end of a system, it will automatically be moved to the
start of the next system. If a \verb|\leftrightrepeat| happens to come at the
end if a system, \musixtex\ will automatically post a right repeat at the end
of the system and a left repeat at the beginning of the next.

For example,

\begin{music}\nostartrule
\startextract
\NOTes\ha g\enotes
\leftrepeat
\NOTes\ha h\enotes
\leftrightrepeat
\NOTes\ha i\enotes
\rightrepeat
\NOTEs\wh j\enotes
\zendextract
\end{music}
 \noindent has been coded as:
\begin{quote}\begin{verbatim}
\NOTes\ha g\enotes
\leftrepeat
\NOTes\ha h\enotes
\leftrightrepeat
\NOTes\ha i\enotes
\rightrepeat
\NOTEs\wh j\enotes
\end{verbatim}\end{quote}

To insert a right repeat at a forced line break or at the end of a piece, use
\verb|\setrightrepeat| \textit{before}\ \verb|\alaligne| or \verb|\endpiece|.
In contrast, to insert a left repeat at a forced line break or at the
beginning of a piece, simply use \verb|\leftrepeat| immediately \textit{after}
\verb|\startpiece| or \verb|\alaligne|. To insert a left-right repeat at a
forced line break, use \verb|\setrightrepeat\alaligne\leftrepeat|.

In fact it is possible to use \keyindex{setleftrepeat},
\keyindex{setrightrepeat} or \keyindex{setleftrightrepeat} before any
\keyindex{bar}, \keyindex{stoppiece} or \keyindex{changecontext}. But
be aware that while \verb|\setleftrepeat| behaves properly if the
bar is at the end of a system, \verb|\setleftrightrepeat| does not, placing
the symbol only at the end of the system.

 \subsection{First and second endings (Voltas)}\index{volta}

All volta commands must be entered right before the bar line command
(or repeat, etc.) where they are to take effect. There are three
commands that suffice to set all voltas. To start one, use
\keyindex{Setvolta}\verb|{|\textit{text}\verb|}|; to terminate it
with or without a vertical line, use \keyindex{endvolta} or
\keyindex{endvoltabox} respectively. The text by default will be followed
by a period. There are also various alternate
commands (e.g., \verb|\setendvoltabox| is equivalent to \verb|endvoltabox|).
Some such alternate forms are used in the following example, but the first three
mentioned above are all that are required:

\medskip \begin{music}
 \parindent0pt
 \startpiece
 \addspace\afterruleskip
 \NOTEs\wh a\en\bar
 \NOTEs\wh b\en\setvoltabox{1.-3}\bar
 \NOTEs\wh c\en\setvolta4\setendvolta\rightrepeat
 \NOTEs\wh d\en\doublebar
 \NOTEs\wh e\en\bar
 \NOTEs\wh f\en\leftrepeat
 \NOTEs\wh g\en\bar
 \NOTEs\wh h\en\Setvolta1\bar
 \NOTEs\wh i\en\bar
 \NOTEs\wh j\en\Setvolta2\setendvoltabox\rightrepeat
 \NOTEs\wh i\en\bar
 \NOTEs\wh h\en\setendvoltabox
 \Endpiece
 \end{music}
 \noindent This was coded as
 \begin{quote}\begin{verbatim}
 \startpiece \addspace\afterruleskip
 \NOTEs\wh a\en\bar
 \NOTEs\wh b\en\setvoltabox{1.-3}\bar
 \NOTEs\wh c\en\setvolta4\setendvolta\rightrepeat
 \NOTEs\wh d\en\doublebar
 \NOTEs\wh e\en\bar
 \NOTEs\wh f\en\leftrepeat
 \NOTEs\wh g\en\bar
 \NOTEs\wh h\en\Setvolta1\bar
 \NOTEs\wh i\en\bar
 \NOTEs\wh j\en\Setvolta2\setendvoltabox\rightrepeat
 \NOTEs\wh i\en\bar
 \NOTEs\wh h\en\setendvoltabox
 \Endpiece
\end{verbatim}\end{quote}
 \zkeyindex{leftrepeat}\zkeyindex{rightrepeat}

If the volta only spans one measure and ends without a vertical segment,
it can be specified simply by saying
\keyindex{setvolta}\verb|{|\textit{text}\verb|}| before the bar line command
that starts it, and it will automatically terminate at the
second bar line command:

\medskip \begin{music}
 \parindent0pt \startpiece \addspace\afterruleskip
 \NOTEs\wh a\en\bar
 \NOTEs\wh b\en\setvolta{1.-3}\bar \NOTEs\wh c\en\setvolta4\rightrepeat
 \NOTEs\wh d\en\bar
 \NOTEs\wh e\en\Endpiece
\end{music}
\noindent which was coded as:
 \begin{quote}\begin{verbatim}
 \parindent0pt \startpiece \addspace\afterruleskip
 \NOTEs\wh a\en\bar
 \NOTEs\wh b\en\setvolta{1.-3}\bar \NOTEs\wh c\en\setvolta4\rightrepeat
 \NOTEs\wh d\en\bar
 \NOTEs\wh e\en\Endpiece
\end{verbatim}\end{quote}

The height above the top staff line of the horizontal line in a volta symbol
is determined by the token \keyindex{raisevolta} which is 4\verb|\internote| by
default. You can change this to any desired dimension.

The period after the text can be
removed by saying \verb|\def|\keyindex{voltadot}\verb|{}|
and restored by \verb|\def|\keyindex{voltadot}\verb|{.}|\ .

 \subsection{Special symbols for repeating long sections}

Four special symbols and corresponding macros are available, namely
\keyindex{coda}~$p$, \keyindex{Coda}~$p$,
and \keyindex{segno}~$p$, where $p$ specifies the pitch; and \keyindex{Segno}
with no argument. Their behavior is illustrated in this example:

\begin{music}
\startextract
\NOtes\segno n\enotes
\bar
\NOtes\coda n\enotes
\NOtes\Segno\enotes
\bar
\NOtes\Coda n\enotes
\endextract
\end{music}
\noindent which has been coded:
\begin{quote}\begin{verbatim}
\NOtes\segno m\enotes\bar
\NOtes\coda m\enotes
\NOtes\Segno\enotes\bar
\NOtes\Coda m\enotes
\end{verbatim}\end{quote}

\subsection{Repeating a single bar}
 The special symbol for a single bar repeat is generated by \keyindex{duevolte}, as shown
in the following example:

\begin{music}
\generalmeter\meterC
\setclef1\bass\setstaffs1{2}
\startextract
\NOtes\sk\sk\pause|\qa{cegj}\en
\bar\NOtes\qa{cdef}|\sk\hsk\duevolte\en
\endextract
\end{music}
\noindent whose coding is:
\begin{quote}\begin{verbatim}
\generalmeter\meterC
\setclef1\bass\setstaffs1{2}
\startextract
\NOtes\sk\sk\pause|\qa{cegj}\en
\bar\NOtes\qa{cdef}|\sk\hsk\duevolte\en
\endextract
\end{verbatim}\end{quote}

\noindent This is often used with \verb|\centerbar| to more easily center the
symbol between bar lines, as shown in the example in section \ref{barcentered}.

\section{Font selection and text placement}

\subsection{Predefined text fonts}
While any \TeX\ font can be used by \musixtex, there are certain styles and
sizes that can be selected using shortcut commands.
For ordinary text the shortcuts cover fonts of eight different
sizes and three styles. The sizes in points are 8, 9, 10, 12, 14, 17, 20, and 25;
the styles are from the standard Computer Modern family: Roman, bold and italic. The
four smallest sizes are each available in all three styles, while the larger
sizes, which are intended for titles, are available only in bold style.
The size selection macros from smallest to biggest are \keyindex{smalltype},
\keyindex{Smalltype},
\keyindex{normtype}, \keyindex{medtype}, \keyindex{bigtype},
\keyindex{Bigtype} ,
\keyindex{BIgtype} and \keyindex{BIGtype}. Following size selection, the
style may be selected or changed using \keyindex{rm} (Roman), \keyindex{bf}
(bold) or \keyindex{it} (italic). If no style is explicitly selected, Roman style
will be used for the sizes \verb|\medtype| or smaller. For the larger sizes,
only bold style is provided and no style selection is required. Thus, for example,
eight point italic is selected with \verb|\smalltype\it|, while
twelve point Roman is selected using \verb|\medtype\rm| or simply
\verb|\medtype|. To change between styles while maintaining the same size,
code \verb|\rm|, \verb|\it| or \verb|\bf| as in Plain \TeX.
When \musixtex\ is started, the default font for ordinary text is
ten point Roman, equivalent to \verb|\normtype\rm|.

Another group of fonts, in bold extended italic style, is predefined in point sizes
10, 12, 14, and 17 for dynamic markings. The appropriate font for the current
staff size may be selected simply by using \keyindex{ppff} as a
font specification. Macros \verb|\smalldyn|, \verb|\normdyn|, or \verb|\meddyn| may be
used to redefine \verb|\ppff| to represent one of the smallest three.

All predefined fonts are summarized in the following table. The second column gives an
explicit control sequence that can alternatively be used locally as a font specification.

\begin{center}
  \begin{tabular}{lll}
    \hline
    Size and style  &  Font specification & Example \\
    \hline
    \verb|\smalltype|    & \verb|\eightrm| & {\smalltype    small Roman}  \\
    \verb|\smalltype\bf| & \verb|\eightbf| & {\smalltype\bf small bold}   \\
    \verb|\smalltype\it| & \verb|\eightit| & {\smalltype\it small italic} \\
    \verb|\Smalltype|    & \verb|\ninerm| & {\Smalltype    Small Roman}  \\
    \verb|\Smalltype\bf| & \verb|\ninebf| & {\Smalltype\bf Small bold}   \\
    \verb|\Smalltype\it| & \verb|\nineit| & {\Smalltype\it Small italic} \\
    \verb|\normtype|     & \verb|\tenrm| & {\normtype     normal Roman} \\
    \verb|\normtype\bf|  & \verb|\tenbf| & {\normtype\bf  normal bold}  \\
    \verb|\normtype\it|  & \verb|\tenit| & {\normtype\it  normal italic}\\
    \verb|\medtype|      & \verb|\twelverm| & {\medtype      medium Roman} \\
    \verb|\medtype\bf|   & \verb|\twelvebf| & {\medtype\bf   medium bold}  \\
    \verb|\medtype\it|   & \verb|\twelveit| & {\medtype\it   medium italic}\\
    \verb|\bigtype|      & \verb|\bigfont| & {\bigtype      big bold}     \\
    \verb|\Bigtype|      & \verb|\Bigfont| & {\Bigtype      Big bold}     \\
    \verb|\BIgtype|      & \verb|\BIgfont| & {\BIgtype      BIg bold}     \\
    \verb|\BIGtype|      & \verb|\BIGfont| & {\BIGtype      BIG bold}     \\
    \verb|\smalldyn|  & \verb|\ppffsixteen| & {\ppffsixteen  pp ff diminuendo}\\
    \verb|\normdyn|   & \verb|\ppfftwenty| & {\ppfftwenty   pp ff crescendo}\\
    \verb|\meddyn|   & \verb|\ppfftwentyfour| & {\ppfftwentyfour   pp ff crescendo}\\[.4ex]
    ~                & \verb|\ppfftwentynine| & {\ppfftwentynine   pp ff diminuendo}\\[.4ex]
    \hline
  \end{tabular}
\end{center}

\subsection{User-defined text fonts}

Since \musixtex\ is a superset of \TeX, you are free to use the standard \TeX\
machinery for defining and using any special font you desire. You must first
of course ensure that (a) all the necessary font files (e.g., \verb|bla10.tfm|,
\verb|bla10.pfb|, or equivalents) are installed in the right places in your system, (b) all
configuration files (e.g., \verb|config.ps| or equivalent) have been updated, and
(c) the \TeX\ system has been ``rehashed''. Then you can use the font just as
in any \TeX\ document, e.g. by coding \verb|\font blafont=bla10| and then
\verb|\zchar{10}{\blafont Text in user-defined font}|.

You might also wish to replace once and for all the typefaces invoked by the
commands described in the previous section. Again, before doing this, you must
follow steps (a-c) of the previous paragraph for all fonts in questions.
\label{musixtmr}
You can use the standard bitmapped fonts which are converted to
postscript by e.g.\ \verb|dvips|, but you also may replace them
by native postscript fonts.

As an example, you can replace the standard \verb|musixtex| fonts by
the Times series of fonts as follows:

\begin{verbatim}
% 8pt roman, bold, and italic
\font\eightrm=ptmr7t at 8pt
\font\eightbf=ptmb7t at 8pt
\font\eightit=ptmri7t at 8pt
% 9pt
\font\ninerm=ptmr7t at 9pt
\font\ninebf=ptmb7t at 9pt
\font\nineit=ptmri7t at 9pt
% 10pt
\font\tenrm=ptmr7t
\font\tenbf=ptmb7t
\font\tenit=ptmri7t
% 12pt
\font\twelverm=ptmr7t scaled \magstep 1
\font\twelvebf=ptmb7t scaled \magstep 1
\font\twelveit=ptmri7t scaled \magstep 1
% Large fonts for titles : normal shaped Times-Roman fonts are applied
\font\bigfont=ptmr7t scaled \magstep2 % 14pt
\font\Bigfont=ptmr7t scaled \magstep3 % 17pt
\font\BIgfont=ptmr7t scaled \magstep4 % 20pt
\font\BIGfont=ptmr7t scaled \magstep5 % 25pt
%
\normtype
\end{verbatim}
The above definitions are made available in a file
\verb|musixtmr.tex|\footnote{by Hiroaki {\sc Morimoto}}, to be used as
\begin{quote}\begin{verbatim}
\input musixtex
\input musixps
\input musixtmr
...
\end{verbatim}\end{quote}

\font\tnormtype=ptmr7t
\font\tnormtypebf=ptmb7t
\font\tnormtypeit=ptmri7t
\font\tbigtype=ptmr7t scaled \magstep2
\font\tBIgtype=ptmr7t scaled \magstep4

\noindent Here's a comparison of some of the Computer Modern Roman and Times Roman fonts:
\begin{center}
  \begin{tabular}{ll}
    \hline
    {\normtype     normal (10pt) cm Roman} &  {\tnormtype  normal Times Roman} \\
    {\normtype\bf  normal cm bold}         &  {\tnormtypebf  normal Times bold}  \\
    {\normtype\it  normal cm italic}       &  {\tnormtypeit  normal Times italic}\\
    {\bigtype      cm big}                 &  {\tbigtype      Times big}     \\
    {\BIgtype      cm BIg }                &  {\tBIgtype      Times BIg bold}     \\
    \hline
  \end{tabular}
\end{center}

\subsection{Text placement}\label{textplacement}
Special macros are provided to allow precise placement of any \TeX\ text, vertically
relative to the staff, and horizontally relative to any note in
the staff.

The macros in the first group will vertically position the text with the
baseline at any specified pitch or staff line. They must be used
inside \verb|\notes...\enotes|. They will not insert any additional
horizontal space. They have the forms
\keyindex{zcharnote}\pitchp\verb|{|\ital{text}\verb|}|, \keyindex{lcharnote}\pitchp\verb|{|\ital{text}\verb|}|,
and \keyindex{ccharnote}\pitchp\verb|{|\ital{text}\verb|}|,
where \ital{p} is the pitch. With the first one, text will spill to the right
from the current insertion point, with the second it will spill to the left, and with the
third it will be centered horizontally.

There are similar macros \keyindex{zchar}\pitchp\verb|{|\ital{text}\verb|}|,
\keyindex{lchar}\pitchp\verb|{|\ital{text}\verb|}|, and \keyindex{cchar}\pitchp\verb|{|\ital{text}\verb|}|,
which differ from the previous three in that the pitch \ital{must} be given with
a number (representing the number of staff positions up from the lowest line),
and that the number need not be an integer.

To vertically position any text midway between two consecutive staves, use
\keyindex{zmidstaff}\verb|{|{\it text}\verb|}|, \keyindex{lmidstaff}\verb|{|{\it text}\verb|}|,
or \keyindex{cmidstaff}\verb|{|{\it text}\verb|}| at the appropriate point in the lower staff.

The macros \keyindex{uptext}\verb|{|\ital{text}\verb|}| and
\keyindex{Uptext}\verb|{|\ital{text}\verb|}| are simply shorthands for
\verb|\zchar{10}{|\ital{text}\verb|}| and \verb|\zchar{14}{|\ital{text}\verb|}| respectively.

The text items handled by all of the above macros can include any appropriate
string of \TeX\
control sequences, including font definitions, \verb|\hbox|'es, etc.

Material posted with any of the macros described in this section will not
create any additional horizontal or vertical space within the current system,
and will overwrite anything in the current system that gets in the way. It is
the typesetter's responsibility to ensure there is adequate white space
within the current system to accommodate any text emplaced with
these macros. On the other hand, if text is emplaced far above or below a
system, \musixtex\ will usually insert additional vertical space if needed.

\subsection{Rehearsal marks}

Rehearsal marks are usually
boxed or circled uppercase letters or digits. They can be defined using the macros
\keyindex{boxit}\verb|{|\ital{text}\verb|}| or
\keyindex{circleit}\verb|{|\ital{text}\verb|}|. For boxed text,
the margin between the text and box is controlled by the dimension
register \keyindex{boxitsep}, which can be reset to any \TeX\ dimension if the
default value of \verb|3pt| is unsatisfactory. To emplace the mark, use \verb|\Uptext| or
any of the other macros defined in the previous section.


\section{Miscellaneous other notations}
%avre
 \subsection{Metronomic indications}
%Metronomic indication deserves a special macro. The mention:
%
%\medskip
%\centerline{\def\nbinstruments{0}\metron{\hup}{60}}
%
%\smallskip\noindent
%is coded by \keyindex{metron}\verb|{\hup}{60}| (normally embedded in
%\keyindex{Uptext} which is in turn embedded within \verb|\notes...\enotes|).
%
% On the other hand, music writers sometimes want to specify that the duration
%of a previous note is equal to a distinct furthernote. Thus
%
% \smallskip
%\centerline{\def\nbinstruments{0}\metronequiv{\qup}{\qu}}
%
%\smallskip\noindent
%is coded by \keyindex{metronequiv}\verb|{\qup}{\qu}|.

By way of example, the notations~~~{\def\nbinstruments{0}
\metron{\hup}{60}}~~~and~~~{\def\nbinstruments{0}\metronequiv{\qup}{\qu}}~~~are
respectively coded as
\keyindex{metron}\verb|{\hup}{60}| and \keyindex{metronequiv}\verb|{\qup}{\qu}|,
which are normally emplaced using \keyindex{Uptext}.

Also, you can set a smaller metronomic indication like~~~{\def\nbinstruments{0}
\metron{\smallnotesize\lqu}{ca. 72}}~~~ or ~~~{\def\nbinstruments{0}
\metron{\tinynotesize\lqu}{ca. 72}}~~~ by writing 
\verb|\metron{\smallnotesize\lqu}{ca. 72}| or \verb|\metron{\tinynotesize\lqu}{ca. 72}|.


 \subsection{Accents}

You may use\footnote{Note from the editor: the reason the names of some of
these macros don't seem to be constructed to suggest the terms used in the
descriptions is that whoever originally defined the macros had in mind terms
that did not agree with normal English usage.}
\begin{itemize}\setlength{\itemsep}{0ex}
 \item \keyindex{upz}\pitchp~(upper \itxem{staccato}) to put a dot above a note
head at pitch $p$,
 \item \keyindex{lpz}\pitchp~(lower \ital{staccato}) to put a dot below a note
head at pitch $p$,
 \item \keyindex{usf}\pitchp~(upper \itxem{sforzando}) to put a $>$ accent above
a note head at pitch $p$,
 \item \keyindex{lsf}\pitchp~(lower \itxem{sforzando}) to put a $>$ accent below
a note head at pitch $p$,
 \item \keyindex{ust}\pitchp~(upper \itxem{tenuto}) to put a
hyphen above a note head at pitch $p$,
 \item \keyindex{lst}\pitchp~(lower \ital{tenuto}) to put a
hyphen below a note head at pitch $p$,
 \item \keyindex{uppz}\pitchp~(upper \itxem{staccatissimo}) to put a solid vertical
wedge above a note head at pitch $p$,
 \item \keyindex{lppz}\pitchp~(lower \ital{staccatissimo}) to put an inverted
solid vertical wedge below a note head at pitch $p$,
 \item \keyindex{usfz}\pitchp~(upper \itxem{forzato}) to put a
``dunce cap'' above a note head at pitch $p$,
 \item \keyindex{lsfz}\pitchp~(lower \ital{forzato}) to put an inverted
``dunce cap'' below a note head at pitch $p$,
 \item \keyindex{upzst}\pitchp~(upper \itxem{staccato/tenuto}) to put a
dot and a hyphen above a note head at pitch $p$,
 \item \keyindex{lpzst}\pitchp~(lower \ital{staccato/tenuto}) to put a
dot and a hyphen below a note head at pitch $p$,
 \item \keyindex{flageolett}\pitchp~to put a
small circle above a note head at pitch $p$.
\end{itemize}
These marks are horizontally centered relative to solid note heads. To
compensate for the fact that whole notes are wider, you should use
\keyindex{wholeshift}\verb|{|\ital{Any nonspacing macro}\verb|}| to center accents
and other items (e.g.~\verb|\Fermataup|) above a whole note.

There are also variants of the most common accents\footnote{Thanks to Klaus
{\sc Bechert}'s corrections.} which will be automatically positioned
above or below a beam. They are spelled like the corresponding normal
accent, but preceded with the letter \verb|b|, and their argument, instead of
the pitch, is the beam reference number . Thus

\medskip
\begin{music}\nostartrule
\startextract
\Notes\ibu0f3\busfz0\qb0f\bupz0\qb0g\bust0\qb0h%
  \buppz0\qb0i\busf0\qb0j\butext0\tqh0k\en
\Notes\Ibl0lg5\blsfz0\qb0l\blpz0\qb0k\blst0\qb0j%
  \blppz0\qb0i\blsf0\qb0h\bltext0\tqb0g\en
\zendextract
\end{music}
\noindent was coded as
\begin{quote}\begin{verbatim} \startextract
\Notes\ibu0f3\busfz0\qb0f\bupz0\qb0g\bust0\qb0h%
  \buppz0\qb0i\busf0\qb0j\butext0\tqh0k\en
\Notes\Ibl0lg5\blsfz0\qb0l\blpz0\qb0k\blst0\qb0j%
  \blppz0\qb0i\blsf0\qb0h\bltext0\tqb0g\en
\zendextract
\end{verbatim}\end{quote}
\zkeyindex{busfz}
\zkeyindex{bupz}
\zkeyindex{bust}
\zkeyindex{buppz}
\zkeyindex{busf}
\zkeyindex{butext}
\zkeyindex{blsfz}
\zkeyindex{blpz}
\zkeyindex{blst}
\zkeyindex{blppz}
\zkeyindex{blsf}
\zkeyindex{bltext}

The macros \verb|\bltext| and \verb|\butext| are detailed in the next section,
where the mystery of why they produce the number 3 is resolved.


\subsection{Numbers and brackets for xtuplets}\label{tuplet}

The following table lists all the special macros that place a number indicating
an xtuplet. Some also place a bracket above or below the notes, and are
intended for use with unbeamed notes.
In the table, $p$ is a pitch, $k$ is a number, $n$ is a beam number,
$w$ is a bracket width in \verb|\internote|s, and $s$ is the bracket slope as
a multiple of 1.125 degrees. The macro \verb|\txt| contains a default number
and its font, which will be emplaced by the first and third through sixth macros,
and is initially defined as \verb|\def\txt{\eightit 3}|. The macro
\verb|\tuplettxt| serves the same role for the last two macros. The first four
are to be used with beamed xtuplets. As indicated in the
last column, the last four produce a sloping bracket and are to be used with
unbeamed xtuplets. The last two require the extension file
\verb|tuplet.tex| to be input right after \verb|\input musixtex|.


\begin{center}
  \begin{tabular}{lclcc}
    \hline
    ~ & Number & ~ & Needs & ~ \\
    Macro and arguments & printed & Where invoked & \verb|tuplet.tex|? & Bracket \\
    \hline
    \verb|\triolet|\pitchp & \verb|\txt| & before beam & no & none \\
    \verb|\xtuplet|\itbrace{k}\pitchp & $k$ & before beam & no & none \\
    \verb|\butext|\itbrace{n} & \verb|\txt| & before note at number & no & none \\
    \verb|\bltext|\itbrace{n} & \verb|\txt| & before note at number & no & none \\
    \verb|\uptrio|\pitchp\itbrace{w}\itbrace{s} & \verb|\txt| & before first note & no & solid \\
    \verb|\downtrio|\pitchp\itbrace{w}\itbrace{s} & \verb|\txt| & before first note & no &  solid \\
    \verb|\uptuplet|\pitchp\itbrace{w}\itbrace{s} & \verb|\tuplettxt| & before first note & yes & with gap \\
    \verb|\downtuplet|\pitchp\itbrace{w}\itbrace{s} & \verb|\tuplettxt| & before first note & yes & with gap \\
    \hline
  \end{tabular}
\end{center}
\medskip

Here are some examples of the first six macros in the table:\\
\vskip-3pt
\begin{music}
\parindent0mm
\generalmeter{\meterfrac24}
\startpiece
\addspace\afterruleskip
\notesp\xtuplet6n\isluru0l\ibl0l0\qb0{lllll}\tslur0l\tqb0l\en\bar
\notesp\triolet n\isluru0l\Ibl0ln2\qb0{lm}\tslur0n\tqb0n\en
\notesp\ibslurd0k\Ibl0km2\qb0k\bltext0\qb0l\tdbslur0m\tqb0m\en\bar
\Notesp\triolet o\isluru0l\ql{lm}\tslur0n\ql n\en\bar
\Notesp\uptrio o16\ql l\en\notesp\cl n\en
%avrb
\Notesp\downtrio O16\qu e\en\notesp\cu g\en
%avre
\endpiece
whose coding is
\begin{quote}\begin{verbatim}
\notesp\xtuplet6n\isluru0l\ibl0l0\qb0{lllll}\tslur0l\tqb0l\en\bar
\notesp\triolet n\isluru0l\Ibl0ln2\qb0{lm}\tslur0n\tqb0n\en
\notesp\ibslurd0k\Ibl0km2\qb0k\bltext0\qb0l\tdbslur0m\tqb0m\en\bar
\Notesp\triolet o\isluru0l\ql{lm}\tslur0n\ql n\en\bar
\Notesp\uptrio o16\ql l\en\notesp\cl n\en
\Notesp\downtrio O16\qu e\en\notesp\cu g\en
\end{verbatim}\end{quote}
\end{music}
\zkeyindex{xtuplet}\zkeyindex{triolet}

\medskip
Next are examples using the macros from \verb|\tuplet.tex|. Comparison of the
coding with the printed result reveals that (a) a redefinition of
\verb|\tuplettxt| inside a notes group only applies inside that group and
leaves the default definition intact (bar 2),
(b) the default definition is
\verb|\def\tuplettxt{\ppffsixteen3}|, a 10-pt bold extended italic numeral 3
which may be larger and heavier than desired (first set, bar 3) (c) to get the
number properly centered in the gap, you must include some extra spaces after
the number in the definition of \verb|\tuplettxt|, and (d) as you can see in
the last bar, if the span becomes to small, the macros still won't provide
enough room for the number in the gap; in this case it would be better to
use to the macros without gaps.

%   Examples:
%
%       Draw a triplet bracket over the notes, starting at pitch "p",
%       1.25\noteskip wide, with a slope up of 3-3/8 degrees:
%
%           \uptuplet p{1.25}3
%
%       Draw a sextuplet bracket under the notes, starting at pitch "a",
%       2.5\noteskip wide, with a slope down of 4-1/2 degrees:
%
%           \def\tuplettxt{\smallppff 6\/\/}
%           \downtuplet{a}{2.5}{-4}

\begin{music}
\input tuplet
\parindent0mm
\generalmeter{\meterfrac24}
%\startpiece
\startextract
\addspace\afterruleskip
\notesp\triolet o\Ibl0ln2\qb0{lm}\tqb0n\en
\notesp\Ibl0km2\qb0k\bltext0\qb0l\tqb0m\en\bar
\notesp\def\tuplettxt{\eightit 5\/\/}\uptuplet o{4.1}2\cl{jklmn}\en\bar
\Notesp\uptuplet o16\ql l\en\notesp\cl n\en
\def\tuplettxt{\eightit 3\/\/}
\Notesp\downtuplet O16\qu e\en\notesp\cu g\en\bar
\notesp\uptuplet o16\ccl l\en\notesp\cccl n\en
\notesp\downtuplet O16\ccu e\en\notesp\cccu g\en
%\endpiece
\endextract
\end{music}

\noindent with coding

\begin{quote}\begin{verbatim}
\input tuplet
\parindent0mm\generalmeter{\meterfrac24}
\startextract\addspace\afterruleskip
\notesp\triolet o\Ibl0ln2\qb0{lm}\tqb0n\en
\notesp\Ibl0km2\qb0k\bltext0\qb0l\tqb0m\en\bar
\notesp\def\tuplettxt{\eightit 5\/\/}\uptuplet o{4.1}2\cl{jklmn}\en\bar
\Notesp\uptuplet o16\ql l\en\notesp\cl n\en
\def\tuplettxt{\eightit 3\/\/}
\Notesp\downtuplet O16\qu e\en\notesp\cu g\en\bar
\notesp\uptuplet o16\ccl l\en\notesp\cccl n\en
\notesp\downtuplet O16\ccu e\en\notesp\cccu g\en
\endextract
\end{verbatim}\end{quote}

 \subsection{Ornaments}

 \subsubsection{Arpeggios}
\ital{Arpeggios} (i.e.~~~\arpeggio{-2}3) can be coded with the macro
\keyindex{arpeggio}\pitchp\itbrace{m}
where $p$ is the pitch of
the base of the arpeggio symbol and $m$ is its height in units of
\verb|\interligne|, the distance from one staff line to the next. It should be
issued before the affected chord. It is
automatically positioned to the left of the chord, but inserts no spacing.
Its variant \keyindex{larpeggio} sets the arpeggio symbol roughly
one note head width to the left of the default position, and is intended to
avoid collision with single accidentals on chord notes.

 \subsubsection{Arbitrary length trills}
 There are two styles of arbitrary length trills\index{trills}, each with two
variants.
For a trill with preassigned length, use \keyindex{trille}\pitchp\itbrace{l}
for \hbox to 1cm{\noteskip1cm\trille11\hss} or
\keyindex{Trille}\pitchp\itbrace{l} for~~\hbox to
2cm{\noteskip1cm\Trille12\hss}, where $p$ is the pitch and $l$ the length in
current \verb|\noteskips|.

To let \musixtex\ compute the length of the trill, or if it extends across a
system break, you can use \keyindex{Itrille}$n$\pitchp\ to start a plain
trill, where $n$ is a trill reference number between 0 and 5; then
\keyindex{Ttrille}$n$ to terminate it. To get the {\it tr} at the
beginning, use \keyindex{ITrille}$n$\pitchp\ to start the trill and
\verb|\Ttrille|$n$ to close it.

As with other elements, you can specify the maximum number 
directly with \keyindex{setmaxtrills}\verb|{|$m$\verb|}| where $7<m\leq 100$%
\footnote{This may require e-\TeX.}; 
the reference number $n$ will be in the range between 0 and $(m-1)$.
\label{musixmad_setmaxtrills}


For example:

\medskip
 \begin{music}
\instrumentnumber{1}
\setstaffs12
\setclef1{6000}
%
\startextract
\notes\qu{CDEFGH}|\hu k\sk\ITrille 1p\itenu1l\wh l\enotes
\bar
\notes\Itrille 2A\itenu2E\whp E|\tten1\hlp l\sk\Ttrille 1\qu {mno}\enotes
\bar
\Notes\tten2\itenu2E\whp E|\ql{nmlkji}\enotes
\bar
\Notes\tten2\whp E\sk\sk\sk\sk\Ttrille2|\qu{hgfedc}\enotes
\endextract
\end{music}
whose coding is
\begin{quote}\begin{verbatim}
\begin{music}\nostartrule
\instrumentnumber{1}
\setstaffs12
\setclef1{6000}
%
\startextract
\notes\qu{CDEFGH}|\hu k\sk\ITrille 1p\itenu1l\wh l\enotes
\bar
\notes\Itrille 2A\itenu1e\whp E|\tten1\hlp l\sk\Ttrille 1\qu {mno}\enotes
\bar
\Notes\tten1\itenu1E\whp E|\ql{nmlkji}\Toctfin1\enotes
\bar
\Notes\tten1\whp E\sk\sk\sk\sk\Ttrille2|\qu{hgfedc}\enotes
\endextract
\end{verbatim}\end{quote}

 \subsubsection{Piano pedal commands}
The macro \keyindex{PED} inserts a piano pedal command below the staff;
\keyindex{DEP}, a pedal release. Alternate symbols, the first of which
occupies less space, are invoked with \keyindex{sPED} and \keyindex{sDEP}.
For example,

\begin{music}
\setclef1\bass
\setstaffs1{2}
\startextract
\NOtes\PED\wh J|\qu h\enotes
\NOtes|\qu g\enotes
\NOtes|\hu k\enotes
\Notes\DEP\enotes
\bar
\NOtes\sPED\wh J|\qu h\enotes
\NOtes|\qu g\enotes
\NOtes|\hu k\enotes
\Notes\sDEP\enotes
\endextract
\end{music}
 \noindent was coded as \begin{quote}\begin{verbatim}
 \NOtes\PED\wh J|\qu h\enotes
 \NOtes|\qu g\enotes
 \NOtes|\hu k\enotes
 \Notes\DEP\enotes \bar
 \NOtes\sPED\wh J|\qu h\enotes
 \NOtes|\qu g\enotes
 \NOtes|\hu k\enotes
 \Notes\sDEP\enotes
 \end{verbatim}\end{quote}

 The vertical position of
\keyindex{PED}, \keyindex{sPED}, \keyindex{DEP} and \keyindex{sDEP} can be
globally changed by redefining its elevation, which has the default
definition \verb|\def|\keyindex{raiseped}\verb|{-5}|.
To locally change the vertical position of a pedal symbol, use one of the more
fundamental
macros \keyindex{Ped}, \keyindex{sPed}, \keyindex{Dep} and \keyindex{sDep}
in combination with \keyindex{zchar} or \keyindex{zcharnote}.
Since the \hbox to .7cm{ \Ped\hss} symbol is rather wide, it might collide
with adjacent bass notes. To shift it horizontally, you could use \verb|\loff{\PED}|.

 \subsubsection{Other ornaments}\index{ornaments}

The argument $p$ in the following macros for ordinary ornaments is the
pitch at which the ornament itself appears. They are all nonspacing macros.
You may use
\begin{itemize}\setlength{\itemsep}{0ex}
 \item \keyindex{mordent}\pitchp\ for \hbox to .75em{\mordent 0\hss},
 \item \keyindex{Mordent}\pitchp\ for \hbox to 1.5em{\kern 0.6em\Mordent 0\hss},
 \item \keyindex{shake}\pitchp\ for \hbox to .75em{\shake 0\hss},
 \item \keyindex{Shake}\pitchp\ for \hbox to 1.5em{\kern 0.6em\Shake 0\hss},
 \item \keyindex{Shakel}\pitchp\ for \hbox to 1.5em{\kern 0.6em\Shakel 0\hss},
 \item \keyindex{Shakesw}\pitchp\ for \hbox to 1.5em{\kern 0.6em\Shakesw 0\hss},
 \item \keyindex{Shakene}\pitchp\ for \hbox to 1.5em{\kern 0.6em\Shakene 0\hss},
 \item \keyindex{Shakenw}\pitchp\ for \hbox to 1.5em{\kern 0.6em\Shakenw 0\hss},
 \item \keyindex{turn}\pitchp\ for \hbox to 1.5em{\kern 0.6em\turn 0\hss},
 \item \keyindex{backturn}\pitchp\ for \hbox to 1.5em{\kern 0.6em\backturn 0\hss}.
 \end{itemize}

In the following macros for fermatas, the argument $p$ is the pitch of the
notehead on which the fermata rests, assuming no additional vertical
adjustments are needed for stems or intervening staff lines. They are all
nonspacing macros. You may use
\begin{itemize}\setlength{\itemsep}{0ex}
 \item \keyindex{fermataup}\pitchp\ for
\raisebox{0ex}[0ex][0ex]{\notesintext{\notes\fermataup1\en}},
\end{itemize}
\begin{itemize}\setlength{\itemsep}{3ex}
 \item \keyindex{fermatadown}\pitchp\ for
\raisebox{0ex}[0ex][0ex]{\notesintext{\notes\fermatadown2\en}},
 \item \keyindex{Fermataup}\pitchp\ for
\raisebox{-1ex}[-1ex][0ex]{\musicintextnoclef{\notes\Fermataup 7\zwh{'c}\en}}~,
centered over a whole note,
 \item \keyindex{Fermatadown}\pitchp\ for
\raisebox{-1ex}[0ex][0ex]{\musicintextnoclef{\notes\Fermatadown1\zwh{'c}\en}}~,
centered under a whole note.
 \end{itemize}
\bigskip

A \itxem{breath} mark
\raisebox{-5ex}[0ex][0ex]{\notesintext{\notes\zbreath\en}}
can be put just above the staff with \keyindex{zbreath}. This is a nonspacing
macro. On the other hand, \keyindex{cbreath} will cause a space of one
\verb|\noteskip| and place the comma midway through the space.

The \keyindex{caesura} command inserts a slash \verb|0.5\noteskip| before
the place it is entered, while adding no space:

\begin{music}\nostartrule
\startextract
\NOTes\zhu j\hl{^e}\caesura\en
\Notes\zcu j\cl e\en
\zendextract
\end{music}

\subsection{Alphabetic dynamic marks}
 Conventional dynamic symbols
\pppp, \ppp, \pp,
\p,
\mezzopiano,
\mf,
\f,
\fp,
\sF,
\ff,
\fff\ and
\ffff\ can be posted using the macros
\keyindex{pppp},
\keyindex{ppp},
\keyindex{pp},
\keyindex{p},
\keyindex{mp},
\keyindex{mf},
\keyindex{f},
\keyindex{fp},
\keyindex{sF},
\keyindex{ff},
\keyindex{fff},
\keyindex{ffff} respectively as the second argument of \verb|\zcharnote| or
\verb|\ccharnote|.

\subsection{Hairpins (crescendos and decrescendos)}\label{sec:crescnd}

The syntax and properties of hairpins will differ depending on whether or not you
have input \verb|musixps.tex|, which activates not only type K postscript slurs and
ties but hairpins as well. The font-based variety are always horizontal,
are limited in length, and cannot span system breaks. Using the postscript variety
removes all these restrictions, but they will not be visible in DVI previewers.

\subsubsection{Font-based hairpins}

There are two categories of font-based hairpins. The first type requires only
one command, \keyindex{crescendo}\verb|{|$\ell$\verb|}| or
\keyindex{decrescendo}\verb|{|$\ell$\verb|}|, where $\ell$ is any \TeX{}
dimension, either a fixed one---for example in points---or a scalable one
expressed either explicitly or implicitly as some number of
\verb|\noteskip|s. These should be used as arguments to
\verb|\zcharnote|, \verb|\zchar|, \verb|\uptext|, \verb|\zmidstaff|, etc., to
post them at the desired altitude. The longest such symbol is $\simeq$ 68 mm.

The second type of font-based hairpin requires two commands, one to start it
and another to end it. The starting macro is \keyindex{icresc}. It has no
arguments. Only one invocation suffices to start any number and combination of
crescendos and diminuendos. The ending macros are \keyindex{tcresc} or
\keyindex{tdecresc}. They should be used as arguments of \verb|\zcharnote| etc,
which will set the altitude. For example,

\begin{music}
\parindent0pt
\generalmeter{\meterfrac{12}8}
\setstaffs1{2}
\startextract
\Notes\cmidstaff\ppp|\ca c\en
\Notes\icresc|\ca{defgh'abcde}\en
\Notes\zmidstaff{\loff\tcresc}\cmidstaff\fff|\ca{'f}\en
\endextract
\end{music}
\noindent which was coded as
\begin{quote}\begin{verbatim}
\Notes\cmidstaff\ppp|\ca c\en
\Notes\icresc|\ca{defgh'abcde}\en
\Notes\zmidstaff{\loff\tcresc}\cmidstaff\fff|\ca{'f}\en
\end{verbatim}\end{quote}
\medskip\noindent while
\begin{music}
\parindent0pt
\generalmeter{\meterfrac{12}8}
\setstaffs1{2}
\startextract
\Notes\cmidstaff\ppp|\ca c\en
\Notes\icresc|\ca{defgh'abcde}\en
\Notes\zcharnote N{\tcresc}\cmidstaff\fff|\zcharnote q{\tcresc}\ca{'f}\en
\endextract
\end{music}
\noindent was coded as
\begin{quote}\begin{verbatim}
\Notes\cmidstaff\ppp|\ca c\en
\Notes\icresc|\ca{defgh'abcde}\en
\Notes\zcharnote N{\tcresc}\cmidstaff\fff|\zcharnote q{\tcresc}\ca{'f}\en
\end{verbatim}\end{quote}

\subsubsection{Postscript hairpins}
As already noted, these require that you have input \verb|musixps.tex|.
Once having done that, you should not try to use font-based hairpins
because the syntax is incompatible.

Again,
there are two different types. The first type is normally initiated with either
\keyindex{icresc}\onen~or \keyindex{idecresc}\onen, and terminated with
\keyindex{tcresc}\onen, where $n$ is a hairpin index,
which is virtually unlimited but certain restrictions apply if it exceeds 14.
The altitude is set by the value of \keyindex{setcrescheight}, which by default
is -5 and which must be expressed numerically. Note that
\keyindex{tcresc}~is the same as \keyindex{tdecresc}.

You can shift the starting or ending point horizontally by replacing the
foregoing macros with
\keyindex{ilcresc},    \keyindex{ildecresc},
\keyindex{ircresc},    \keyindex{irdecresc},
\keyindex{tlcresc},    \keyindex{tldecresc},
\keyindex{trcresc},    \keyindex{trdecresc},
for example to make space for an alphabetic dynamic mark.

The second form of postscript hairpin macros allows individual and arbitrary
specification of the altitude and horizontal offset. The syntax is
\keyindex{Icresc}\itbrace{n}\itbrace{h}\itbrace{s}, where $h$ is the
altitude---which must be numerical---and $s$ is the horizontal offset
in \verb|\internote|. Similar syntax obtains for \keyindex{Idecresc}\
and \keyindex{Tcresc}.

These hairpins may span several lines. If one of them spans three systems
then the height of the middle section can be adjusted with
\keyindex{liftcresc}\itbrace{n}\itbrace{h}. The height of the first and last
parts of a broken crescendo will be defined by the height parameter in
\keyindex{Icresc}~or \keyindex{Tcresc}.

There are numerous other nuances and shorthand macros that are described
in the comments in \verb|musixps.tex|.

As an example of a postscript hairpin,
\begin{music}\nostartrule
%\input smallmusixpsx
\input musixps
\generalmeter{\meterfrac{12}8}
\startextract
\Notes\ccharnote{-8}{\ppp}\Icresc0{-7}6\ca{bdegh'bde}\en
\Notes\Tcresc0{-4}{-2}\zcharnote{-5}{\fff}\ca{'f}\en
\zendextract
\end{music}
\noindent  was coded as
\begin{quote}\begin{verbatim}
\input musixps
\generalmeter{\meterfrac{12}8}
\startextract
\Notes\ccharnote{-8}{\ppp}\Icresc0{-7}6\ca{bdegh'bde}\en
\Notes\Tcresc0{-4}{-2}\zcharnote{-5}{\fff}\ca{'f}\en
\zendextract
\end{verbatim}\end{quote}

 \subsection{Length of note stems}

The default length of note stems is the distance of one octave, i.e.
\verb|7\internote| or 4.66 \keyindex{interbeam}. The default may be changed
with the macro \keyindex{stemlength}\itbrace{b}\ where $b$ is the length in
\verb|\interbeam|s.

The command \keyindex{stemcut} causes stems that extend outside the staff
to be shortened depending on the pitch of the notes. It is the default
behavior. To suppress this adjustment, issue the command
\keyindex{nostemcut}.

Normally, down stems never end above the middle line of the staff and up stems
never below that line. The command \keyindex{stdstemfalse} will inhibit this
adjustment, but only for the next stem generated.  There is no command to globally
suppress this behavior.

 \subsection{Brackets, parentheses, and oblique lines}\label{brapar}

Several varieties of brackets, parentheses and oblique lines are provided for
use within a score.
 \begin{itemize}\setlength{\itemsep}{0ex}
 \item \keyindex{lpar}\verb|{|$p$\verb|}| and
\keyindex{rpar}\verb|{|$p$\verb|}| yield left and right
parentheses at pitch $p$. They could be used to enclose notes or to build
\ital{cautionary} accidentals, although the latter are more easily
obtained with predefined macros (see \ref{cautionary}).

%avrb
For example,

\begin{tabular}{ll}
\raisebox{-1.5ex}[0ex][0ex]{\musicintextnoclefn{\notes\bsk\lpar{g}\rpar{g}\hu{g}\sk%
\loffset{1.5}{\lpar{g}}\loffset{1.5}{\rpar{g}}\loffset{.4}{\sh g}\hu{g}\en}}
&\begin{tabular}{ll}
&\verb+\notes\lpar{g}\rpar{g}\hu{g}\sk%+\\
is coded as&\verb+\loffset{1.5}{\lpar{g}}\loffset{1.5}{\rpar{g}}%+\\
&\verb+\loffset{.4}{\sh g}\hu{g}\en}+
\end{tabular}
\end{tabular}
%avre

 \item \keyindex{bracket}\verb|{|$p$\verb|}{|$n$\verb|}| posts a
square bracket to the left of a chord, vertically spanning $n$
\verb|internote|s.
 \item \keyindex{doublethumb}\verb|{|$p$\verb|}| indicates a bracket as
above spanning 2\verb|\internote|s.
 \item \keyindex{ovbkt}\verb|{|$p$\verb|}{|$n$\verb|}{|$s$\verb|}| and
\keyindex{unbkt}\verb|{|$p$\verb|}{|$n$\verb|}{|$s$\verb|}|
draw a sloped bracket starting at the current position at pitch $p$,
with horizontal extent $n$ \verb|noteskip|s and slope $s$ in multiples of
1-1/8 degree.
 \item \keyindex{uptrio}\verb|{|$p$\verb|}{|$n$\verb|}{|$s$\verb|}| and
\keyindex{downtrio}\verb|{|$p$\verb|}{|$n$\verb|}{|$s$\verb|}| are like
\verb|\ovbkt| but with freely definable \keyindex{txt} centered inside.

 \item \keyindex{varline}\itbrace{h}\itbrace{\ell}\itbrace{s}\ builds an
oblique line starting at the current horizontal position. It must be used
inside a zero box, such as for example as the second argument of
\verb|\zcharnote|. $h$ is the height of the starting point, $\ell$ is the
length, and $s$ is the slope. It is used in the definitions of some of
the foregoing macros, and could be used, for example, to construct various
obscure baroque ornaments made up of diagonal lines.

\end{itemize}

For example,

\begin{music}
\setstaffs1{2}
\setclef1\bass
\startextract
\NOtes\bracket C8\zq C\qu J\en
\NOtes|\doublethumb g\rq h\qu g\en
\NOtes\lpar c\rpar c\qu c\en
\NOtes\unbkt C15\qu {FH}|\ovbkt n14\ql{kl}\en
\NOtes\downtrio C16\qu {FH}|\uptrio o14\ql{lm}\en
\endextract
\end{music}
\noindent is coded as
\begin{quote}\begin{verbatim}
\begin{music}
\setstaffs1{2}
\setclef1{\bass}
\startextract
\NOtes\bracket C8\zq C\qu J\en
\NOtes|\doublethumb g\rq h\qu g\en
\NOtes\lpar c\rpar c\qu c\en
\NOtes\unbkt C15\qu {FH}|\ovbkt n14\ql{kl}\en
\NOtes\downtrio C16\qu {FH}|\uptrio o14\ql{lm}\en
\zendextract
\end{music}
\end{verbatim}\end{quote}

 \subsection{Forcing activity at the beginning of systems}
A macro named \keyindex{everystaff} is executed each time a new system
begins. It is normally void, but it can be defined (simply by \verb|\def|%
\keyindex{everystaff}\verb|{...}|) to cause \musixtex\ to post anything
reasonable at the beginning of each system. For it to affect the first system
in a score, it must be defined \ital{before}
\verb|\startpiece|.

If a macro named \keyindex{atnextline} is defined at any point in a score,
it will be executed just once, viz., at the next computed or forced system
break. More
precisely, it is executed after the break and before the next system begins.
Thus it is suitable for redefining layout parameters.

 \index{octave treble clef}In some scores, tenor parts are not coded using the
\ital{bass} clef, but using rather the \ital{octave treble clef}, which is
subscripted by a numeral {\tt8}.
This clef is supported by the clef substitution command
\keyindex{settrebleclefsymbol}\onen\keyindex{treblelowoct}, as already
explained in section \ref{treblelowoct}. However, if for some reason you
aren't happy with the height of the ``8'', it can be posted on selected
staves at the beginning of every system using \keyindex{everystaff} and
\keyindex{zcharnote} as follows:

\begin{music}
\instrumentnumber{4}
\setclef1\bass
\def\everystaff{%
  \znotes&\zchar{-6}{\eightrm\kern-2\Interligne8}%
  &\zchar{-6}{\eightrm\kern-2\Interligne8}\en}%
\startextract
\NOTes\ha{HIJK}&\ha{efgh}&\ha{hijk}&\ha{hmlk}\en
\zendextract
\end{music}
\noindent The coding is
\begin{quote}\begin{verbatim}
\instrumentnumber{4}
\setclef1\bass
\def\everystaff{%
  \znotes&\zchar{-6}{\eightrm \kern -2\Interligne 8}%
  &\zchar{-6}{\eightrm \kern -2\Interligne 8}\en}%
\startextract
\NOTes\ha{HIJK}&\ha{efgh}&\ha{hijk}&\ha{hmlk}\en
\endextract
\end{verbatim}\end{quote}

\section{Smaller notes in normal-sized staves}
Here we describe how to reduce the size of note symbols without changing the
size of the staff itself. Changing overall staff size will be treated in
section \ref{staffspacing}.

\subsection{Arbitrary sequences of notes}
Written-out ornaments and \itxem{cadenzas} are usually typeset with
smaller notes and spacing than normal. The smaller \ital{size} can be
initiated anywhere inside a \verb|\notes| group by stating
\keyindex{smallnotesize} or \keyindex{tinynotesize}. Normal note size is then
restored by \keyindex{normalnotesize} or simply by terminating the
\verb|notes| group and starting another. Smaller \ital{spacing} must also be
explicitly indicated, usually by redefining \verb|\noteskip| in some way.

As an example, this excerpt, from the beginning
of the Aria of the ``Creation'' by Joseph {\sc Haydn})\index{Haydn, J.@{\sc
Haydn, J.}},

\begin{music}
\instrumentnumber{2}
\generalmeter{\meterfrac44}
\def\qbl#1#2#3{\ibl{#1}{#2}{#3}\qb{#1}{#2}}
\setstaffs2{2}
\setclef1\bass
\setclef2\bass
\startbarno0
\startextract
\NOtes\qp&\zmidstaff{\bf II}\qp|\qu g\en
% mesure 1
\bar
\Notes\itieu2J\wh J&\zw N\ibl0c0\qb0e|\qu j\en
\notes&\qbl0c0|\multnoteskip\tinyvalue\tinynotesize
  \Ibbu1ki2\qb1{kj}\tqh1i\hqsk\en
\Notes&\qb0e\tbl0\qb0c|\qu j\en
\Notes&\ibl0c0\qb0{ece}\tbl0\qb0c|\ql l\sk\ql j\en
% mesure 2
\bar
\Notes\ttie2\wh J&\ql J\sk\ql L|\zqupp g\qbl1e0%
  \zq c\qb1e\zq c\qb1e\zq c\tbl1\zqb1e\en
\notes&|\sk\ccu h\en
\Notes&\ql N\sk\ibl0L{-4}\qbp0L|\ibl1e0\zq c\zqb1e\cu g%
  \zq c\zqb1e\raise\Interligne\ds\zqu g\qb1g\en
\notes&\sk\tbbl0\tbl0\qb0J|\tbl1\zq c\qb1e\en
\endextract
\end{music}
\noindent can be coded as
\begin{quote}\begin{verbatim}
\instrumentnumber{2}
\generalmeter{\meterfrac44}
\setstaffs2{2}
\setclef2{\bass}
\setclef1{\bass}
\startbarno=0
\startextract
\NOtes\qp&\zmidstaff{\bf II}\qp|\qu g\en
% mesure 1
\bar
\Notes\itieu2J\wh J&\zw N\ibl0c0\qb0e|\qu j\en
\notes&\ibl0c0\qb0c|\multnoteskip\tinyvalue\tinynotesize
  \Ibbu1ki2\qb1{kj}\tqh1i\en
\Notes&\qb0e\tbl0\qb0c|\qu j\en
\Notes&\ibl0c0\qb0{ece}\tbl0\qb0c|\ql l\sk\ql j\en
% mesure 2
\bar\Notes\ttie2\wh J&\ql J\sk\ql L|\zqupp g\qbl1e0%
  \zq c\qb1e\zq c\qb1e\zq c\tbl1\zqb1e\en
\notes&|\sk\ccu h\en
\Notes&\ql N\sk\ibl0L{-4}\qbp0L|\ibl1e0\zq c\zqb1e\cu g%
  \zq c\zqb1e\raise\Interligne\ds\zqu g\qb1g\en
\notes&\sk\tbbl0\tbl0\qb0J|\tbl1\zq c\qb1e\en
\endextract
\end{verbatim}\end{quote}

 \subsection{Grace notes}
Grace notes are a special case of small and tiny notes, namely single-stemmed
eighth notes with a diagonal slash through the flag. To enable this, there
are the macros \keyindex{grcu}\pitchp\ and \keyindex{grcl}\pitchp, which by
themselves would
produce normal-sized eighth notes with a slash. They should be used along
with the note size reduction macros and spacing reduction macros just
discussed. In addition, chordal grace notes can be built as in the following
example:

\begin{music}\nostartrule
\startextract
\NOTes\hu h\enotes
\notes\multnoteskip\smallvalue\smallnotesize\grcu j\enotes
\NOTes\hu i\enotes
\bar
\notes\multnoteskip\tinyvalue\tinynotesize\zq h\grcl j\enotes
\NOTEs\wh i\enotes
\zendextract
\end{music}
\noindent which was coded as
\begin{quote}\begin{verbatim}
\startextract
\NOTes\hu h\enotes
\notes\multnoteskip\smallvalue\smallnotesize\grcu j\enotes
\NOTes\hu i\enotes
\bar
\notes\multnoteskip\tinyvalue\tinynotesize\zq h\grcl j\enotes
\NOTEs\wh i\enotes
\zendextract
\end{verbatim}\end{quote}

 \subsection[Ossia]{Ossia\texorpdfstring{\protect\footnote{(Italian O sia) Or else}}{}}
This clever example had been provided by Olivier Vogel:\label{ossia}

%\begin{center}
%\includegraphics[scale=1]{./mxdexamples/ossiavogel.eps}
%\end{center}

\begin{music}
%\startextract
%\hsize70mm
\let\extractline\hbox
\hbox to \hsize{\hss
\def\xnum#1#2#3{\off{#1\elemskip}\zcharnote{#2}{\smalltype\it #3}%
\off{-#1\elemskip}}%
\newbox\ornamentbox
\setbox\ornamentbox=\hbox to 0pt{\kern-4pt\vbox{\hsize=2.6cm%
\makeatletter
\nostartrule\smallmusicsize\setsize1{\smallvalue}\setclefsymbol1\empty\global\clef@skip0pt%
\makeatother
%\smallmusicsize\setsize1{\smallvalue}\setclefsymbol1\empty%
%\startpiece
%\addspace{2pt}%
\startextract
\addspace{-2pt}%
\let\notest\notes\def\notes{\vnotes2.1\elemskip}%
\notes\ibbbl2{'c}0\qb2b\qb2c\qb2d\tbbbl2\qb2c\en%
\notes\xnum{1.15}{'e}3\qb2d\qb2c\tbl2\qb2d\en%
\notes\ibl2{'c}1\usf e\qb2c\en%
\notes\tbl2\qb2{'d}\en
\let\notes\notest%
%\zstoppiece%
\endextract
}\hss}
%\setbox\ornamentbox=\hbox to 0pt{Hello}
%
%
%\normalmusicsize\nopagenumbers
%\def\nbinstruments{1}%
\setstaffs12\setclef1{60}%
\generalsignature{-2}\generalmeter{\meterfrac{3}{4}}%
%\parindent 0pt%
%\stafftopmarg0pt\staffbotmarg5\Interligne\interstaff{10}\relax
%\startpiece\addspace\afterruleskip%
\startextract\addspace\afterruleskip%
\NOtes\ibl1{'G}{-1}\qb1G\sk\bigna F\tbl1\qb1F|%
\ibbl2{'b}0\qb2b\qb2a\qb2b\tbl2\qb2c\en%
\NOtes\hl{'E}\bsk\raise6\internote\ds\ibu3{G}1\bigsh F%
\qb3F\qb3G\tbu3\qb3{'A}|\zcharnote{10}{\copy\ornamentbox}\qlp{'c}\sk\sk%
\cl d\en%
%\endpiece
\endextract
\hss}
%\vfill\eject\endmuflex
\end{music}


The code is:
\begin{quote}\begin{verbatim}
\hsize70mm%
\def\xnum#1#2#3{\off{#1\elemskip}\zcharnote{#2}{\smalltype\it #3}%
\off{-#1\elemskip}}%
\newbox\ornamentbox
\setbox\ornamentbox=\hbox to 0pt{\kern-4pt\vbox{\hsize=2.6cm%
\nostartrule\smallmusicsize\setsize1{\smallvalue}\setclefsymbol1\empty%
\startpiece\addspace{2pt}%
\notes\ibbbl2{'c}0\qb2b\qb2c\qb2d\tbbbl2\qb2c\en%
\notes\xnum{1.15}{'e}3\qb2d\qb2c\tbl2\qb2d\en%
\notes\ibl2{'c}1\usf e\qb2c\en%
\notes\tbl2\qb2{'d}\en%
\zstoppiece%
}\hss}
\normalmusicsize\nopagenumbers
\def\nbinstruments{1}%
\setstaffs12\setclef1{60}%
\generalsignature{-2}\generalmeter{\meterfrac{3}{4}}%
\parindent 0pt%
\stafftopmarg0pt\staffbotmarg5\Interligne\interstaff{10}\relax
\startpiece\addspace\afterruleskip%
\notes\ibl1{'G}{-1}\qb1G\sk\bigna F\tbl1\qb1F|%
\ibbl2{'b}0\qb2b\qb2a\qb2b\tbl2\qb2c\en%
\notes\hl{'E}\bsk\raise6\internote\ds\ibu3{G}1\bigsh F%
\qb3F\qb3G\tbu3\qb3{'A}|\zcharnote{10}{\copy\ornamentbox}\qlp{'c}\sk\sk%
\cl d\en%
\end{verbatim}\end{quote}

%
% Why is this here?? DAS 4/30/06
%
% \subsection{Other note shapes}\label{othernotes}
% The classical note heads
%given above --- namely \raise.5ex\hbox{\musixchar7}~~, \raise.5ex
%\hbox{\musixchar8}~~ and \raise.5ex\hbox{\musixchar9}~~~
%--- can be replaced with less classical note heads, for example to code
%special \itxem{violin harmonic notes} or \itxem{percussion music}. See an
%example in \ref{abnormalscores}.
%
% At present time, alternate available note heads can be found in the
%extension library, see \ref{diam}, \ref{perc}, \ref{gregnotes} and \ref{litu}.

\section{Staff size} \index{staff size}\label{staffspacing}
In contrast with the prior section, here we describe how to change the sizes
of everything...staff, notes, and all other symbols. In
section~\ref{whatspecify} we saw how to set the size for all instruments at
the start of a score. Any one of the same macros---\verb|\normalmusicsize|,
\verb|\smallmusicsize|, \verb|\largemusicsize|, or
\verb|\Largemusicsize|---can be used to change the size of all instruments
midway through a score, but in this case it must come between
\verb|\stoppiece| and \verb|\startpiece|.

Once the overall staff size is set, you can alter the size of any desired
instrument with the macro \keyindex{setsize}\itbrace{n}\itbrace{s}, where
$n$ is the instrument number and $s$ is a factor by which the size is to be
changed from the prevailing overall size. There are five predefined macros
that should be used for the size factor $s$. Their names and respective values
are \keyindex{normalvalue}~(1.0), \keyindex{smallvalue}~(0.80),
\keyindex{tinyvalue}~(0.64), \keyindex{largevalue}~(1.2), and
\keyindex{Largevalue}~(1.44). \musixtex\ should not crash if you use an
explicit number different from any of these, but the result may be
less than satisfactory.

Once again, if used at the beginning of a piece, the \verb|\setsize| macro
must precede \verb|\startpiece| (not \verb|\contpiece|), and if used after
the beginning, must be preceded by \verb|\stoppiece|.

As an example, we give two bars of the \ital{Ave Maria} by Charles {\sc
Gounod}\index{Gounod, C.@{\sc Gounod, C.}}, based on the first prelude of
J. S. Bach's \ital{Well Tempered Clavier}, as transcribed for
organ, violin and voice by Markus {\sc Veittes}:\label{avemaria}

\begin{music}
\def\oct{\advance\transpose by 7}
\def\liftqs#1{\raise#1\Interligne\qs}
\parindent0pt
\sepbarrules
\instrumentnumber{3}
\generalmeter{\meterC}
\setinterinstrument2{3\Interligne}
\setsize3\tinyvalue
\setsize2\tinyvalue
\setclef1\bass
\setstaffs1{2}
\startpiece\addspace\afterruleskip
%Takt 9
\notes\zhl c\liftqs6\qupp e|\ds&\oct
  \itieu5h\hl h&\tx ~~~gra---*\itied4h\hu h\enotes
\notes|\ibbl0j3\qb0h\tqb0l\enotes
\notes|\ibbl1k0\qb1{ohl}\tqb1o\enotes
\notes\zhl c\liftqs6\qupp e|\ds&\oct
  \ttie5\ibl4c0\qb4h&\ttie4\ibu5g{-3}\qb5h\enotes
\notes|\ibbl0j3\qb0h\tqb0l&\oct\qb4a&\tx ---*\tqh5a\enotes
\notes|\ibbl1k0\qb1o\qb1h&\oct\qb4b&\tx ~~ti~-*\cu b\enotes
\notes|\qb1l\tqb1o&\oct\tqb4c&\tx a*\cu c\enotes
\bar
%Takt 10
\notes\zhl c\liftqs6\qupp d|\ds&\oct
  \qlp d&\tx ~~~ple---*\ibsluru4e\qup d\enotes
\notes|\ibbu1g3\bigaccid\qb1{^f}\tqh1h\enotes
\notes|\ibbu2i0\qb2k\qb2f\enotes
\notes|\qb2h\tqh2k&\oct\cl e&\tubslur4f\cu e\enotes
\notes\zhl c\liftqs6\qupp d|\ds&\oct\ql d&\tx na,*\qu d\enotes
\notes|\ibbu1g3\qb1f\tqh1h\enotes
\notes|\ibbu2i0\qb2{kfh}\tqh2k&\qp&\qp\enotes
\endpiece
\end{music}
 This example was coded as:
\begin{quote}\begin{verbatim}
\def\oct{\advance\transpose by 7}
\def\liftqs#1{\raise#1\Interligne\qs}
\parindent0pt
\sepbarrules
\instrumentnumber{3}
\generalmeter{\meterC}
\setinterinstrument2{3\Interligne}
\setsize3\tinyvalue
\setsize2\tinyvalue
\setclef1\bass
\setstaffs1{2}
\startpiece\addspace\afterruleskip
%Takt 9
\notes\zhl c\liftqs6\qupp e|\ds&\oct
  \itieu5h\hl h&\tx ~~~gra---*\itied4h\hu h\enotes
\notes|\ibbl0j3\qb0h\tqb0l\enotes
\notes|\ibbl1k0\qb1{ohl}\tqb1o\enotes
\notes\zhl c\liftqs6\qupp e|\ds&\oct
  \ttie5\ibl4c0\qb4h&\ttie4\ibu5g{-3}\qb5h\enotes
\notes|\ibbl0j3\qb0h\tqb0l&\oct\qb4a&\tx ---*\tqh5a\enotes
\notes|\ibbl1k0\qb1o\qb1h&\oct\qb4b&\tx ~~ti~-*\cu b\enotes
\notes|\qb1l\tqb1o&\oct\tqb4c&\tx a*\cu c\enotes
\bar
%Takt 10
\notes\zhl c\liftqs6\qupp d|\ds&\oct
  \qlp d&\tx ~~~ple---*\ibsluru4e\qup d\enotes
\notes|\ibbu1g3\bigaccid\qb1{^f}\tqh1h\enotes
\notes|\ibbu2i0\qb2k\qb2f\enotes
\notes|\qb2h\tqh2k&\oct\cl e&\curve222\tubslur4f\cu e\enotes
\notes\zhl c\liftqs6\qupp d|\ds&\oct\ql d&\tx na,*\qu d\enotes
\notes|\ibbu1g3\qb1f\tqh1h\enotes
\notes|\ibbu2i0\qb2{kfh}\tqh2k&\qp&\qp\enotes
\endpiece
\end{verbatim}\end{quote}

\section{Layout parameters}\label{LayoutParameters}
 Most layout parameters are set by \musixtex\ to reasonable default values.
However, some projects will require altering one or more of them. In this
section we discuss the most important parameters and how to change them.

 \subsection{List of layout parameters}
 In the following, the indication ``\nochange'' does not mean that this
parameter cannot be changed at all, but that it should not be modified
directly, e.g.\ by saying something like \verb|\Interligne=14pt|. In other
words, changes in these parameters must be accomplished only by more
comprehensive macros which not only revise them but at the same time perform
other necessary related changes. Even though you cannot \ital{change} these,
you may \ital{refer} to them in your coding if that proves useful.

\begin{description}\setlength{\itemsep}{0ex}
 \item[\keyindex{Interligne} :]vertical distance between the bottoms of
consecutive staff lines of the current instrument, taking no account of a
possible alteration by \keyindex{setsize} \nochange.
 \item[\keyindex{internote} :]vertical spacing between notes one scale
step apart in the current instrument, taking account of a possible alteration
by \keyindex{setsize} \nochange.
 \item[\keyindex{Internote} :]vertical spacing between notes one scale
step apart in any instrument whose \keyindex{setsize} has the default
value \keyindex{normalvalue} (1.0), equal to \verb|0.5\Interligne| \nochange
\item[\keyindex{staffbotmarg} :]margin below the first (lowest) staff of the
first (lowest) instrument. Changes are recognized at the next system.
Default is \verb|3\Interligne|.
 \item[\keyindex{stafftopmarg} :]margin above the last (uppermost) staff of
last (uppermost) instrument. Changes are recognized at the next system.
Default is \verb|3\Interligne|.
 \item[\keyindex{interbeam} :]vertical distance between beams. \nochange.
\item[\keyindex{interstaff} :]a very important macro with a single numerical
argument representing the factor that multiplies \verb|2\internote| to give the
distance between the bottom of one staff and the bottom of the next one. In
fact the macro redefines the parameter \verb|\interfacteur|.
\item[\keyindex{interportee} :]distance between the bottom of one staff and
the bottom of the next one. It is always reset to
\verb|2|\keyindex{interfacteur}\verb|\internote| at the next system.
Therefore, trying to change \verb|\interportee| will have no effect. Change
\verb|\interstaff| instead. Further, note that \verb|\interstaff| applies to all the
instruments, but each distinct instrument may have a different
\verb|\internote| (see \ref{staffspacing}).

\item[\keyindex{interinstrument} :]additional vertical distance between
two consecutive instruments. This means that the distance between the lowest
line of the previous instrument and the lowest line of the top staff of the current
instrument is \keyindex{interportee+\Bslash interinstrument}. The
default value of \verb|\interinstrument| is zero, but sometimes you
may want additional space between distinct instruments. This is a general
dimension register. As usual in \TeX, it can be set using a command such as
\verb|\interinstrument=10pt| or \verb|\interinstrument=6\internote|. Its
value can be overridden for the space above any particular instrument
with the macro \keyindex{setinterinstrument}\itbrace{n}\itbrace{s}, where
$n$ is the instrument and $s$ is the replacement value of the space to
be added. The \verb|\setinterinstrument| macro may be useful in some vocal
scores to provide vertical space for lyrics. Note that
after you have used \keyindex{setinterinstrument},
you cannot reset the distances for that instrument with
\keyindex{interinstrument}; you must subsequently use
\keyindex{setinterinstrument} for that purpose.

 \item[\keyindex{systemheight} :]distance from the bottom of the
lowest staff to the top of the highest one. This is the length of any
vertical lines such as repeats that span the full height of a system.
\nochange.
\end{description}

 In addition, when handling notes of a given staff of a given instrument, the
following dimensions are available (note these are not true registers, but
\ital{equivalenced symbols} through a \verb|\def|):

\begin{description}\setlength{\itemsep}{0ex}
 \item[\keyindex{altplancher} :]altitude of the lowest line of the lowest
instrument \nochange.
 \item[\keyindex{altitude} :]altitude of the lowest line of the lowest
staff of the current instrument \nochange.
 \item[\keyindex{altportee} :]altitude of the lowest line of the current
staff \nochange.

 %??? Next perhaps will change to steps of \verb|\internote|.

 \item[\keyindex{stemfactor} :]parameter defining the length of stems on
half, quarter, and beamed eighth notes, in units of \verb|\interbeam|.
Normally a stem has the length of one octave,
i.e. 3.5\keyindex{Interligne}. However, this is not correct for small or tiny
note sizes. Therefore, stem length is defined as a multiple of the
dimension \keyindex{interbeam}, which is chosen because it is automatically
redefined as a
different multiple of \keyindex{Interligne} whenever note size is changed. For
example, with \verb|\normalmusicsize| when \verb|\setsize| is
\verb|\normalvalue|, \verb|\interbeam| is 0.75 \keyindex{Interligne}. This
legislates a default value for \keyindex{stemfactor} of 4.66 (=3.5/0.75).
To change stem length, it is easiest to use e.g.
\keyindex{stemlength}\verb|{3.5}|, which simply redefines \verb|\stemfactor|.
Subsequently, \verb|\stemfactor| will not automatically be reset to the default,
but keep in mind that if it is changed
inside a notes group, the change will only be effective within that notes group.
\end{description}

 \subsection{A convenient macro for changing layout parameters in mid-score}
Of the parameters just described that can be changed, many should only be
changed between the end of one system and the beginning of the next. The
command sequence \verb|\def\atnextline{|\ital{any control sequence}\verb|}|
may be useful for this purpose. It will cause \ital{any control sequence} to
be inserted right before the next new line is begun, provided the line break
is not initiated by a \verb|\startpiece|. Thus this will work with
automatically generated line breaks, with those forced by \verb|\alaligne|, and
with those forced by explicit use of \verb|\endpiece| or \verb|\stoppiece|
followed by \verb|\contpiece|. (Note, however, that in the latter case it
would be just as convenient to enter the parameter changes explicitly as well.)
The control sequence will only be executed once,
after which \verb|\atnextline| is redefined as \verb|\empty|.

 \subsection{Changing the number of lines per staff}\label{stafflinenumber}
Naturally, the default number of lines per staff is five. But you may want a
different number in some or all staves, for example for gregorian music,
percussion music, guitar tablature, or early baroque keyboard music. To do so,
use the command \keyindex{setlines}\itbrace{n}\itbrace{m}~where $n$ is the
instrument number and $m$ is the number of lines.

 \subsection{Resetting normal layout parameters}
The general size can only be changed with one of the commands
\keyindex{smallmusicsize},
\keyindex{normalmusicsize}, \keyindex{largemusicsize}, or
\keyindex{Largemusicsize}. Beyond that, the command
\keyindex{resetlayout} will reset the following key layout parameters
to their default values: \keyindex{staffbotmarg} (3\verb|\Interligne|),
\keyindex{stafftopmarg} (3\verb|\Interligne|), \keyindex{interstaff} (9),
 number of lines per staff for all instruments (5); and will reset all clef
symbols to standard clef symbols.

% The whole subsection on excerpts should go with "MusiXTeX and LaTeX", currently 2.25
% \subsection{Typesetting one-line excerpts rather than large scores}\label{excerpts}

% All the old lyrics stuff should be deleted, with a reference to some older version of
% musixdoc.

\section{Lyrics}\index{lyrics}\label{lyrics}
\musixtex{} itself doesn't manage lyrics very well. You should use
\verb+musixlyr+ instead, a \musixtex\ extension package for lyrics handling by Rainer Dunker.
The \TeX~source and
\href{http://icking-music-archive.org/software/musixtex/add-ons/mxlyrdoc.pdf}
{\underline{documentation}}~are included in the \musixtex~distribution.

But first we recall briefly on the older methods, who are still usefull when
only a small number of words are involved.

\subsection{Native lyrics method: placing single words}
\subsubsection{Native \musixtex\ commands for lyrics}
 \begin{enumerate}\setlength{\itemsep}{0ex}
 \item An obvious solution consists in using the commandc
\keyindex{zcharnote} (expanded to the right),
\keyindex{lcharnote} (expanded to the left),
\keyindex{ccharnote} (centered), to post the text at any position (computed in
\verb|\internote|s) with respect to the lower line of the current staff.
The pitch should be usually negative to have the text below the staff.

Example:\quad
\raisebox{0ex}[4ex][3ex]{\musicintextnoclefn{\notes\sk\zcharnote{N}{Word}\wh g\sk\en}}
\quad is coded by \quad
\verb|\zcharnote{N}{Word}\wh g|~.

\item The vertical position can also be given with a number in the commands
\keyindex{zchar} (expanded to the right),
\keyindex{lchar} (expanded to the left),
\keyindex{cchar} (centered). The number is internally multiplied by \verb|\internote|~.

Example:\quad
\raisebox{0ex}[4ex][3ex]{\musicintextnoclefn{\notes\sk\cchar{-4}{Word}\wh g\sk\en}}
\quad is coded by \quad
\verb|\cchar{-5}{Word}\wh g|~.

\item Of easier use are the commands \keyindex{zsong} (right of the note),
\keyindex{lsong} (left) and \keyindex{csong} (centered) which post the lyrics
at the lower staff line \ital{minus} the previous
\keyindex{interinstrument}~$n$ or the \keyindex{staffbotmarg} quantity. These
commands only have one argument, namely the lyrics text:
%\begin{center}
\keyindex{zsong}\verb|{|\ital{text}\verb|}|\quad
\keyindex{lsong}\verb|{|\ital{text}\verb|}|\quad
\keyindex{csong}\verb|{|\ital{text}\verb|}|
%\end{center}
 Depending on the values of the inter-instrument spacings and margins, the
resulting vertical position might be inappropriate. Then it can be changed for
any specific $n$-th instrument until further change using
\begin{quote}
\keyindex{setsongraise}~$n$\verb|{|\ital{any \TeX-dimension}\verb|}|
\end{quote}
As an example, the following French song\\
 \begin{music}
 \generalsignature{1}
 \startextract
 \geometricskipscale
 \NOtes\zsong{Au }\qu g\en
 \NOtes\zsong{clair }\qu g\en
 \NOtes\zsong{de }\qu g\en
 \NOtes\zsong{la }\qu h\en
 \bar
 \NOTes\zsong{lu- }\hu i\en
 \NOTes\zsong{ne, }\hu h\en
 \bar
 \NOtes\zsong{mon }\qu g\en
 \NOtes\zsong{a- }\qu i\en
 \NOtes\zsong{mi }\qu h\en
 \NOtes\zsong{Pier- }\qu h\en
 \bar
 \NOTes\zsong{rot, }\wh g\sk\en
 \endextract
 \end{music}

\noindent was coded as:
\begin{quote}\begin{verbatim}
 \generalsignature{1}
 \startextract
 \geometricskipscale
 \NOtes\zsong{Au }\qu g\zsong{clair }\qu g\en
 \NOtes\zsong{de }\qu g\zsong{la }\qu h\en\bar
 \NOTes\zsong{lu- }\hu i\zsong{ne, }\hu h\en\bar
 \NOtes\zsong{mon }\qu g\zsong{a- }\qu i\en
 \NOtes\zsong{mi }\qu h\zsong{Pier- }\qu h\en\bar
 \NOTes\zsong{rot, }\wh g\sk\en
 \zendextract
\end{verbatim}\end{quote}
\end{enumerate}

\subsubsection{Adapting note spacing for lyrics}

The command \keyindex{hardlyrics}\verb|{longword}| provides a spacing that is equal to the length
of the text argument \verb|{longword}|. In the same time the argument \verb|{longword}|
is saved in \keyindex{thelyrics}

As an example \quad
\raisebox{0ex}[5ex][4ex]%
{\musicintextnoclefn{\staffbotmarg2\Interligne%
\hardlyrics{clair}\notes\hsong{\thelyrics}\wh g\en\notes\wh{gg}\en}}
\quad is coded by:
\begin{tabular}{l}
\verb|\hardlyrics{clair}%|\\
\verb|\notes\hsong{\thelyrics}\wh g\en|\\
\verb|\notes\wh{gg}\en|\\
\end{tabular}

All notes with long lyrics need such a treatment. The commands only carry out on
\verb|\notes| (not on \verb|\Notes|, \verb|\NOtes|...).

If you want to go back to the normal placing on an easy way, you simply can replace
'\verb|\hardlyrics|' by '\verb|\softlyrics|'.

 A complete score is given in example {\tt glorias.tex}\label{glorias} and in
{\tt gloriab.tex}, the latter exhibiting not only the song tune but also the
organ accompaniment.

Alternate versions of \verb|\hsong| are \keyindex{dhsong} which has a fixed
length of \verb|2\noteskip| and \keyindex{thsong} whose fixed length is
\verb|3\noteskip|. These are useful when the text is set below (or above) a
collective coding of two or three notes.

\subsection{Musixlyr}\label{musixlyr}

Lyrics are best handled by the \texttt{musixlyr}\index{musixlyr} package by Rainer Dunker.
The package can be used by inserting a file in your source code:
\begin{quote}\begin{verbatim}
\input musixtex
\input musixlyr
...
\end{verbatim}\end{quote}

The manual, the input file and a few examples can be downloaded as
compact musixlyr packet
\href{http://icking-music-archive.org/software/musixtex/add-ons/musixlyr21c.zip}
{\underline{for Windows}} or \href{http://icking-music-archive.org/software/musixtex/add-ons/musixlyr21c.tgz}
{\underline{for \unix}}. Look at the manual for a detailed description. Here you can find an overview
of the commands and an example of use.

\def\keyexample#1{\keyindex{#1}}

\noindent\begin{tabbing}
\verb|\NOtes|\keyexample{assignlyricshere}\verb|{alto}\qa c\en|\quad\= assigning without staff number\kill
Example\> Explanation
\end{tabbing}
\vspace{-1ex}
\hrule
\vspace{-2ex}
\begin{tabbing}
\verb|\NOtes|\keyexample{assignlyricshere}\verb|{alto}\qa c\en|\quad\= assigning without staff number\kill
\keyexample{setlyrics}\verb|{sopr}{the ly_-ric words_}|\> defining the lyrics text\\
\keyexample{copylyrics}\verb|{sopr}{alto}|\> alto has same lyrics as soprano\\
\keyexample{appendlyrics}\verb|{alto}{more words}|\> alto lyrics is longer\\
\keyexample{assignlyrics}\verb|2{sopr,alto}|\>soprano and alto lyrics at staff 2\\
\keyexample{assignlyricsmulti}\verb|{1}{2}{alto}|\>assign alto lyrics to staff 2 of instrument 1\\
\verb|\NOtes|\keyexample{assignlyricshere}\verb|{alto}\qa c\en|\>assigning without staff number\\
\keyexample{auxlyr}\verb|\assignlyrics{2}{sopr}|\> assign soprano above staff 2\\[.8ex]
\keyexample{lyrrule}\verb|\qu c|...\keyexample{lyrruleend}\verb|\qu c|\>make a melisma by hand\\
\keyexample{beginmel}\verb|\qu c|...\keyexample{endmel}\verb|\qu c|\>melisma, same as word extension underline\\[.8ex]
\keyexample{lyr}\verb|\qu c|\>force a syllable from lyrics text at this note or rest\\
\keyexample{lyric}\verb|{word}\qu c|\>insert syllable 'word' at this note\\
\verb|\loffset{2}{|\keyexample{lyric*}\verb|{1.}}\qu c|\>combine '1.' with regular syllable\\
\keyexample{lyrich}\verb|{syl}\qu c|\>same as \verb|\lyric|, but with hyphenation\\
\keyexample{lyrich*}\verb|{}\qu c|\>same as \verb|\lyric*|, but with hyphenation\\
\keyexample{lyricsoff}...\keyexample{lyricson}\>stop lyrics, then start again\\
\keyexample{nolyr}\verb|\qu c|\>no syllable at this note\\[.8ex]
\keyexample{llabel}\verb|{labelname}name|\>labelling a ``go to'' target in text\\
\keyexample{golyr}\verb|{labelname}\qu c|\>perform a jump, in music code\\[.8ex]
\keyexample{lyrpt}\verb|,\qu c|\>add a comma to the syllable under this note\\
\keyexample{lyrnop}\verb|\qu c|\>remove last character in syllable\\
\keyexample{lclyr}\verb|\qu c|\>make first character lower case\\
\keyexample{llyr}\verb|\qu c|\>left justified syllable\\
\keyexample{leftlyrtrue}\verb|\qu c|...\keyexample{leftlyrfalse}\verb|\qu c|\>start and stop left justification as the default\\
\keyexample{lyroffset}\verb|{-4}\qu c|\>shift syllable 1 notehead to the left\\[.8ex]
\keyexample{minlyrspace}\verb|{3pt}\qu c|\>define minimum space between the words\\
\keyexample{forcelyrhyphenstrue}\verb|\qu c|\>always use a hyphen from now on\\
\keyexample{forcelyrhyphensfalse}\verb|\qu c|\>remove hyphen and make one word if necessary \\
\keyexample{showlyrshifttrue}\verb|\qu c|\>show the lyric shift\\[.8ex]
\keyexample{lyrraise}\verb|{1}{a 2\Interligne}|\>raise lyrics below staff 1 by \verb|2\Interligne|\\
\keyexample{lyrraisemulti}\verb|{1}{2}{a 2\Interligne}|\>raise alto lyrics above staff 2 of instrument 1\\
\keyexample{lyrraisehere}\verb|{b 2\Interligne}\qu c|\>raise lyrics below this staff by \verb|2\Interligne|\\[.8ex]
\keyexample{minlyrrulelength}\verb|{2mm}|\>melismas shorter than 2mm are not shown \\
\keyexample{minmulthyphens}\verb|{15mm}|\>distance between hyphens in 'hyphen melisma'\\
\verb|\def|\keyexample{lyrhyphenchar}\verb|{-}|\>chose a hyphen character\\
\verb|\setlyrics{|\keyexample{lyrlayout}\verb|{\it}..}|\>apply italics to all lyrics lines\\
\keyexample{verses}\verb|{,\beginmel}\qu c|\>initiate melisma at second verse\\[.8ex]
\verb|\small|\keyexample{setlyrstrut}\>adapt the vertical distance between lyrics lines\\
\keyexample{lyrstrutbox}\verb|{10pt}|\> (re)define the distance between the lyrics lines\\[.8ex]
\keyexample{lyrmodealter}\verb|2|\>attach lyrics of staff 2 to the upper voice\\
\keyexample{lyrmodealtermulti}\verb|{1}{2}|\>attach lyrics of instr.\ 1 staff 2 to the upper voice\\
\keyexample{lyrmodealterhere}\verb|\qu c|\>attach lyrics of this staff to the upper voice\\
\keyexample{lyrmodenormal}\verb|2|\>restore the default behaviour\\
\keyexample{lyrmodenormalmulti}\verb|{1}{2}|\>restore the default behaviour at staff 2 of instr.\ 1\\
\keyexample{lyrmodenormalhere}\verb|\qu c|\>restore the default behaviour of this staff\\[.8ex]
\keyexample{lyrlink}\>linking 2 words with a '$_{_\smile}$'\\
\keyexample{lowlyrlink}\>same as \verb|\lyrlink| but a little bit lower\\[.8ex]
\keyexample{resetlyrics}\>set word pointer to the first word in all lyrics lines\\[.8ex]
\keyexample{enableauxlyrics}\>don't use this anymore\\
\keyexample{setsongraise}\verb|{1}{2\Interligne}|\>same as \verb|{\lyrraise}{1}{b 2\Interligne}|\\
\keyexample{auxsetsongraise}\verb|{1}{2\Interligne}|\>same as \verb|{\lyrraisemulti}{1}{b 2\Interligne}|\\
\keyexample{oldlyrlinestart}\>don't let the lyrics extent to the left margin\\
\end{tabbing}

As a further illustration of the use of the commands, have a look at the following
example\footnote{The example is taken from the musixlyr manual.}:

\medskip\medskip
%\oneversespace
\begin{music}
\input musixlyr
 \resetlyrics \small
\lyrmodealter0
 \setlyrics{soprano}{bring her die Gans,} \auxlyr{\assignlyrics1{soprano}}
 \copylyrics{soprano}{alto}                       \assignlyrics1{alto}

 \generalsignature{-2}
 \advance\stafftopmarg1\Interligne
 \advance\staffbotmarg2\Interligne

 \startextract\addspace\afterruleskip
  \NOtes\zqu g\ql e\en
  \bar
  \Notes\zqu i\beginmel\ibslurd0f\ibl0f{-1}\qb0{fe}\en
  \Notes\auxlyr\beginmel\ibsluru1i\zqup i\qb0d\en
  \Notes\tqb0e\en
  \Notes\endmel\tbslurd0e\ql c\en
  \notes\ibbu0h{-1}\qb0h\tqh0g\en
  \NOtes\auxlyr\endmel\tbsluru1h\zqu h\ql c\en
  \bar
  \NOTEs\auxlyr\lyr\zwh i%       The lyrics of the whole notes
    \lyr\wh b\en     %       must be given manually.
 \endextract
 \lyrmodenormal0
\end{music}
\medskip

%\oneversespace

which was coded as:
\begin{quote}
\begin{verbatim}
% define lyrics above the staff
\setlyrics{soprano}{bring her die Gans,}
% lyric beneath the staff are the same
\copylyrics{soprano}{alto}
% assign alto lyrics below staff 1 on the notes with stem down
\assignlyrics1{alto}
% assign soprano lyrics above staff 1 on the notes with stem up
\auxlyr{\assignlyrics1{soprano}}
% attach both lyrics to the upper voice
\lyrmodealter0
\generalsignature{-2}
% make place for the lyrics
\advance\stafftopmarg1\Interligne\advance\staffbotmarg2\Interligne
\startextract\addspace\afterruleskip
\NOtes\zqu g\ql e\en\bar
% start melisma in lower lyrics
\Notes\zqu i\beginmel\ibslurd0f\ibl0f{-1}\qb0{fe}\en
% start melisma in upper lyrics
\Notes\auxlyr\beginmel\ibsluru1i\zqup i\qb0d\tqb0e\en
% end melisma in lower lyrics
\Notes\endmel\tbslurd0e\ql c\en
\notes\ibbu0h{-1}\qb0h\tqh0g\en
% start melisma in upper lyrics
\NOtes\auxlyr\endmel\tbsluru1h\zqu h\ql c\en\bar
% The lyrics of the whole notes (without stem) must be given manually.
\NOTEs\auxlyr\lyr\zwh i\lyr\wh b\en
\endextract
\lyrmodenormal0
\end{verbatim}\end{quote}

  \subsection{Getting enough vertical space for lyrics}
  Since songs are
usually equivalent to a one-staff instrument (possibly with several voices)
the recommended solution consists in adjusting the distance between
instruments using either \keyindex{interinstrument}\verb|=|\ital{any
\TeX-dimension} to give more place below all instruments or using
\keyindex{setinterinstrument} to make more space above. Note that {\Bslash
setinterinstrument} defines spacing above and not below an instrument. Since
lyrics are usually set below the staff, the first argument of a
\verb|\setinterinstrument| should be the song instrument number \ital{minus
one}.

 In the case of a single staff tune, or if the song instrument is the lowest
one, then additional place can be provided using \keyindex{staffbotmarg}.

\subsection{Fine tuning the placement of the lyrics}
 When not using \verb|\hardlyrics|, on short notes, sometimes
 the lyrics are shifted away from the notes or they collide with other words.
 This are a few approaches to get around this:
\begin{enumerate}\setlength{\itemsep}{0ex}
\item Making more music lines for the notes to go further apart. This could be done with
\verb|\mulooseness|.
\item Stretch a bar with short notes in it by i.e.\ replacing \verb|\notes| by \verb|\NOTes|.
\item Insert space between the notes by using \verb|\sk|, \verb|\hsk|, \verb|\qsk|...
\item Stretch a bar with short notes in it by using the command \keyindex{scale}:
\begin{verbatim}
\scale{1.6}\notes..\en\scale{1}%
\end{verbatim}
This method can be used in \textbf{PMX} but only with care, because it changes horizontal spacing
in a way that \textbf{PMX} will not be aware of. It will not move bars to the next line, but
will shorten the other bars on the line.

As an example, the lyrics of this music line are better placed by using \verb|\scale| in
the first bar and moving to the left the word 'mon'. Note that the hyphen is removed
when there is no place for it:

 \begin{music}
 \input musixlyr
 \resetlyrics
 \setlyrics{v1}{Au clair de la lu-ne, mon a-mi Pier-rot,}%
 \assignlyrics1{}\assignlyrics1{v1}%
 \staffbotmarg2\Interligne\generalsignature{1}%
 \startextract
 \geometricskipscale
 \notes\qu{gggh}\en\bar
 \Notes\hu{ih}\en\bar
 \notes\qu{gihh}\en\bar
 \Notes\wh g\en
 \endextract
% \end{music}
%%%%% combined for saving registers %%%%%
% \begin{music}\nostartrule
% \input musixlyr
 \resetlyrics
 \setlyrics{v1}{Au clair de la lu-ne, \kernm1exmon a-mi Pier-rot,}%
 \assignlyrics1{}\assignlyrics1{v1}%
 \staffbotmarg2\Interligne\generalsignature{1}%
 \startextract \geometricskipscale
 \scale{1.4}\notes\qu{gggh}\en\bar\scale{1}%
 \Notes\hu{ih}\en\bar
 \notes\qu{gihh}\en\bar
 \Notes\wh g\en
 \endextract
 \end{music}

\medskip
The code of the second music line is (\verb|\assignlyrics1{}| is only needed because
lyrics are assigned before this in this manual):
\begin{quote}\begin{verbatim}
 \begin{music}
 \input musixlyr
 \resetlyrics
 \setlyrics{v1}{Au clair de la lu-ne, \kernm1exmon a-mi Pier-rot,}%
 \assignlyrics1{}\assignlyrics1{v1}%
 \staffbotmarg2\Interligne\generalsignature{1}%
 \startextract \geometricskipscale
 \scale{1.4}\notes\qu{gggh}\en\bar\scale{1}%
 \Notes\hu{ih}\en\bar
 \notes\qu{gihh}\en\bar
 \Notes\wh g\en
 \endextract
 \end{music}
\end{verbatim}\end{quote}

\item Moving a word in any direction

\noindent\begin{tabular}{ll}
\multicolumn{2}{l}{\Bslash setlyrics\{alto\}\{\Bslash kernm3ex1.$\sim\sim$firstsyllable...\}}\\
& left moving for numbering verses\\
\verb|\setlyrics{alto}{...\kern1exword...}| &right moving a single word\\
\verb|\setlyrics{alto}{...\lower2pt\hbox{word}...}|& lowering a single word\\
\verb|\setlyrics{alto}{...\raise2pt\hbox{word}...}|& raising a single word\\
\multicolumn{2}{l}{\Bslash def\Bslash strut\{\Bslash vbox to 2\Bslash Interligne\{\}\}%
\Bslash setlyrics\{alto\}\{\Bslash lyrlayout\{\Bslash strut\}...\}} \\
& controlling distance between verses\\
\verb|\lyrlayout{\vphantom{Mp(\lowlyrlink}|& minimum distance between verses\\
\multicolumn{2}{l}{\Bslash setbox\Bslash lyrstrutbox=\Bslash hbox\{\Bslash vphantom\{yM\Bslash lyrlink\}\}}\\
& redefining default lyrstrut
\end{tabular}

\item Placing of accents can be made easier as shows this example:
\begin{center}
\begin{minipage}{50mm}
\begin{music}\nostartrule
\input musixlyr
\catcode`\�\active \def�{\"a}
\catcode`\�\active \def�{\"o}
\catcode`\�\active \let�\aa
\setlyrics1{� � �}
\assignlyrics1{}\assignlyrics11
\startextract
 \setsongraise1{1ex}\NOtes\qa{ggg}\en
\zendextract
\end{music}
\end{minipage}%
\begin{minipage}{80mm}
\verb|\catcode`\|\"a\verb|\active \def|\"a\verb|{\"a}|\\
\verb|\catcode`\|\"o\verb|\active \def|\"o\verb|{\"o}|\\
\verb|\catcode`\|\aa\verb|\active \let|\aa\verb|\aa|\\
\verb|\setlyrics1{|\aa\ \"a\ \"o\verb|} \assignlyrics1{}|\\
\verb|\assignlyrics11|\\
\verb|\startextract|\\
\verb|\NOtes\qa{ggg}\en|\\
\verb|\zendextract|\\
\end{minipage}
\end{center}
\item Using an 8-bit encoded characterset\label{8bit}

This can be achieved
by putting
\verb+\input plainenc\relax\inputencoding{cp850}+
at the beginning of the file.

Here you must adapt the character
set to your region.
To do this, if necessary, replace \verb+cp850+ by
\verb+cp1250+ (Czech, Croation region),
\verb+cp1251+ (Russian, Bulgarian region),
\verb+cp1252+ (most other European regions),
\verb+cp1253+ (Greek region)
or check your \TeX distribution for the right file.

This is the coding for the previous example by using an 8-bit characterset:

\begin{center}
\begin{minipage}{25mm}
\begin{music}\nostartrule
% this manual is in latex, the 8bit is taken by
% \usepackage[ansinew]{inputenc}
%
%\input plainenc\relax\inputencoding{cp850}
\input musixtex
\input musixlyr
\setlyrics1{� � �}\assignlyrics1{}\assignlyrics11%
\startextract%
\znotes\zcharnote{12}{Po�me na�f}\en%
\Notes\qa{ggg}\en%
\zendextract%
\end{music}
\end{minipage}
~~~~~~
\begin{minipage}{100mm}% program to run apart from musixdoc
\verb+\input plainenc\relax\inputencoding{cp850}+\\
\verb+\input musixtex+\\
\verb+\input musixlyr+\\
\verb+\startpiece%+\\
\verb+\znotes\zcharnote{16}{Po�me na�f}\en%+\\
\verb+\setlyrics1{� � �}\assignlyrics11%+\\
\verb+\Notes\qa{ggg}\en%+\\
\verb+\Endpiece\vfill\eject \bye+
\end{minipage}
\end{center}

\end{enumerate}

\section{Embedding musical excerpts in text documents}

Here we discuss the options for including music in text documents. The first
decision is whether or not to use \LaTeX\footnote{We'll assume a user wanting
to embed a musical excerpt in a \LaTeX\ document is already familiar with the
fundamentals of \LaTeX. For more information about it, see for example the manual
\ital{\LaTeX: A Document Preparation System} by Leslie {\sc Lamport}}.
Because \LaTeX\ so effectively simplifies
production of text-based \TeX\ documents, most users take that path, and most of the
descriptions given here will assume that's the case\footnote{Please do not be confused;
while \LaTeX\ is recommended for text-based documents containing musical
exceprts, its use is definitely discouraged for ordinary self-contained musical scores
of any sort.}.

If for some reason you choose not to use \LaTeX\ to emplace musical excerpts,
the basic approach is simply to set off the musical
parts between \verb|\startpiece| or \verb|\contpiece| and \verb|\stoppiece| or
\verb|\endpiece|, or between \verb|\startextract| and \verb|\zendextract|. But
some details in what follows will also apply with no \LaTeX.

There are two basic approaches to embedding musical excerpts in \LaTeX\
documents. The first method is to directly include the \musixtex\ code in
the \LaTeX\ source file. That will be the subject of the next subsection. The
other is to create an EPS (encapsulated Postscript) file containing only the
excerpt, and then ``paste'' it into the \LaTeX\ file. That will be covered in
subsection \ref{embedeps}.

The advantages of using the direct method are that all of the source code for
all excerpts can be kept in the same file as the text, and that there is no
limit on the length of the excerpt.
The advantage of the EPS method is that you don't have to burden the \LaTeX\
source with any of the \musixtex\ paraphernalia. That in turn permits use of
primitive versions of the \TeX\ compiler that may not have the capacity
to handle the direct method (due to the number of registers consumed by \LaTeX\
and \musixtex). The disadvantages are that you must create and keep
track of a separate \TeX\ and EPS file for every excerpt, and that the excerpt
must not span any page breaks. On balance, the
direct method is probably to be preferred.

\subsection{Directly embedding excerpts in \LaTeX\ documents}\label{excerpts}

To use the direct method, you should include ``\verb|musixtex|'' as one of the
arguments of the \keyindex{documentstyle} command. This will cause the file
\ttxem{musixtex.sty} to be loaded, so naturally you must make that file
available in a place where \TeX\ can find it. That file simply inputs two
other files, \verb|musixtex.tex| and \ttxem{musixltx.tex}, which again must
obviously be available to \TeX.

Now you are in position to directly embed an excerpt by inserting code at the
appropriate place in the source file. The most common type of excerpt is
one that occupies less than a full line and is to be horizontally centered. In
that case, the extract should begin with the command \keyindex{begin\LBR music\RBR},
followed by any preliminary commands. Then, instead of \verb|\startpiece|,
use \keyindex{startextract}. Now comes the normal \musixtex\ coding. Finally,
end the extract with \keyindex{endextract} instead of \verb|\endpiece|
or \verb|\stoppiece|, followed by \keyindex{end\LBR music\RBR}.

To terminate an extract without any bar line, use \keyindex{zendextract}
instead of \verb|\endextract|.

To create a left-justified excerpt, use the sequence
\keyindex{let\Bslash extractline\Bslash leftline}.

If several extracts are to be placed on the same line, you can
redefine \keyindex{extractline} as demonstrated in the following
example\footnote{The macro {\tt\Bslash extractline} is defined once and for all in
{\tt musixtex.tex} as {\tt\Bslash centerline}. You might think that the suggested
coding would permanently redefine {\tt\Bslash extractline}, thereby upsetting the
normal function of {\tt\Bslash startextract ... \Bslash endextract} for subsequent use.
But it doesn't, because any actions within
\keyindex{begin\LBR music\RBR}...\keyindex{end\LBR music\RBR} are local, not
global.}:

% DAS temp (where is the reference???)
%An example of this is given at \ref{extractline} on page \pageref{extractline}.
% end DAS temp
\begin{quote}\begin{verbatim}
\begin{music}\nostartrule
\let\extractline\hbox
\hbox to \hsize{%
\hss\startextract ... \zendextract\hss%
\hss\startextract ... \zendextract\hss}
\end{music}
\end{verbatim}\end{quote}

An even shorter type of extract is one that is embedded \ital{within} a line of text.
To insert \musixtex\ symbols within a line of text, you could begin by defining
\keyindex{notesintext} as follows\footnote{provided by Rainer {\sc Dunker}}:

\begin{quote}\begin{verbatim}\begin{music}\nostartrule
\makeatletter
\def\notesintext#1{%
  {\let\extractline\relax
   \setlines10\smallmusicsize \nobarnumbers \nostartrule
   \staffbotmarg0pt \setclefsymbol1\empty \global\clef@skip0pt
   \startextract\addspace{-\afterruleskip}#1\zendextract}}
\makeatother
\end{music}\end{verbatim}\end{quote}

Then, for example, the code

\begin{verbatim}
Use \raisebox{0ex}[0ex][0ex]{\notesintext{\notes\rql1\qu2\en}}
not \raisebox{0ex}[0ex][0ex]{\notesintext{\notes\ql2\lqu1\en}}
\end{verbatim}

\noindent would produce: ``Use \raisebox{0ex}[0ex][0ex]{\notesintext{\notes\rql1\qu2\en}} not
\raisebox{0ex}[0ex][0ex]{\notesintext{\notes\ql2\lqu1\en}}~''.

\noindent The \verb|\raisebox| voids the vertical space that is introduced by the notes.

Finally, you may want to insert an extract containing more than one line
of music. Once having mastered \musixtex, this is the easiest of all: Between
\verb|\begin{music}\nostartrule| and \verb|\end{music}|, use \ital{exactly} the same
coding you would to make an ordinary score.

The best way to learn how to apply these methods is to study
\verb|musixdoc.tex|, the source file for this document\footnote{Do note,
however, that {\tt musixdoc.tex} includes {\tt musixdoc} but not {\tt musixtex}
as an argument of {\tt\Bslash documentstyle}. The former performs the functions
of the latter as well as numerous tasks peculiar to this particular document.}.

%DAS I don't think this section is worth including.

% \subsection{Wide music in \LaTeX}\label{musixblx}
% Another difficulty appears with \LaTeX: internal \LaTeX\ macros handle the
%page size in a way which is not supposed to be changed within a given document.
%This means that text horizontal and vertical sizes are somewhat frozen so that
%one can hardly insert pieces of music of page size different from the size
%specified by the \LaTeX{} \itxem{style}.
%Although a \ttxem{musixblx.tex} has been provided, which makes the
%\ital{environment} \verb|bigmusic| available.
%\zkeyindex{begin\LBR bigmusic\RBR}
%The main drawback is an unpredictable behaviour of top and bottom
%printouts, especially page numberings.
%
% If the whole of a document has wide pages, it can be handled with the
%\ttxem{a4wide} style option, or any derivate of it.

\subsection{Embedding musical excerpts as encapsulated postscript files}
\label{embedeps}
To use this method of including excerpts, you first must create a separate
\musixtex\ input file for each excerpt. Process each such file with
\TeX\ and \verb|musixflx| to generate a \verb|.dvi| file. Generate
a postscript file from each \verb|.dvi| using \verb|dvips|. Then
convert each postscript file to an \verb|.eps| file. One way to do that is
with \textbf{ghostscript} and---if you are using Windows---\textbf{GSview}.
In general this is possible only for single-page postScript files.

To set up your \LaTeX\ document for including \verb|.eps| files, you must post
the command \verb|\usepackage[dvips]{graphicx}| in the preamble of the
document. Now, you may include each \texttt{.eps} file at the appropriate place
in the {\LaTeX} document with a command
like \keyindex{includegraphics}\verb|{sample.eps}|.

\subsection{Issues concerning \texorpdfstring{{\Bslash catcodes}}{catcodes}}
\label{catcodeprobs}

\musixtex\ uses the following symbols differently from plain \TeX: \verb|>|,
 \verb|<|, \verb&|&, \verb|&|, \verb|!|, \verb|*|, \verb|.|, and \verb|:|\ .\\
The symbols are given their special meanings by executing the macro
\keyindex{catcodesmusic}, and are restored to their plain \TeX\ meanings with
\keyindex{endcatcodesmusic}. When setting either a self-contained score or
a musical extract, you normally need not worry
about this at all, because \verb|\startpiece| or \verb|\startextract| executes
\verb|\catcodesmusic| and \verb|\endpiece| or \verb|\endextract| executes
\verb|\endcatcodesmusic|. But there are some special situations where you
might need to use these catcode-modifying macros explicitly. One is if you
were to define a personalized macro outside
\verb|\startpiece ... \endpiece|, but which incorporated any of the
symbols with their \musixtex\ meanings. Another would be if you wished to have
access to facilities enabled by alternate style files such as
{\tt\ixem{french.sty}} which change \keyindex{catcode}s themselves. In
such cases, provided you have input \verb|musixtex.tex|, you can always
invoke \keyindex{catcodesmusic} to set the \keyindex{catcode}s at their
\musixtex\ values, and \keyindex{endcatcodesmusic} to restore them to their
prior values.

\section{Extension Library}
All following files are invoked by saying \keyindex{input} \ital{filename}\ .
Most of them are fully compatible with \musixtex\ in that they do not redefine
any existing macros but rather provide additional functionality. In future
versions of \musixtex\ we may very well incorporate many of them directly into
\verb|musixtex.tex|, but for now we leave them separate.

 \subsection{curly}
Allows you to group more than 1 instrument with curly
brackets (see \ref{curlybrackets}).

 \subsection{musixadd}\ixtt{musixadd.tex}
Increases the number of instruments, slurs and beams from six to nine.


 \subsection{musixbm}
This file does nothing; it is retained only for compatibility with 
\musixtex{} T.110 or earlier. 
Since version T.111, \verb|musixtex.tex| itself contains all the functions 
of the older \texttt{musixbm}, namely 
commands for 128th notes, either with flags or with beams:
\keyindex{ibbbbbu},
\keyindex{ibbbbbl},
\keyindex{nbbbbbu},
\keyindex{nbbbbbl},
\keyindex{tbbbbbu},
\keyindex{tbbbbbl},
\keyindex{Ibbbbbu},
\keyindex{Ibbbbbl},
\keyindex{cccccu},
\keyindex{cccccl},
\keyindex{ccccca},
\keyindex{zcccccu} and
\keyindex{zcccccl}.


 \subsection{musixbbm}\label{musixbbm}
 Provides 256th notes, but only for use within beams, via the commands
\keyindex{ibbbbbbu},
\keyindex{ibbbbbbl},
\keyindex{nbbbbbbu},
\keyindex{nbbbbbbl},
\keyindex{tbbbbbbu},
\keyindex{tbbbbbbl},
\keyindex{Ibbbbbbu} and
\keyindex{Ibbbbbbl}.

By default \texttt{musixbbm} provides six 256th beams with 
reference number 0 to 5.  
You can specify a larger maximum number 
directly with \keyindex{setmaxcclvibeams}\verb|{|$m$\verb|}| within the
range\footnote{This may require e-\TeX.} $7\leq m\leq 100$.

\subsection{musixcho}\label{song}
 Enables certain macros intended for choral music\footnote{Remember
that we now recommend using \texttt{musixlyr} for any except the
simplest lyrics. The extension \texttt{musixcho} is only for those
diehards who choose to ignore this advice}. Provides the following commands:
\keyindex{biglbrace}, \keyindex{bigrbrace}, \keyindex{braceheight},
\keyindex{Dtx} and \keyindex{Drtx} for two-line text, \keyindex{Ttx} and
\keyindex{Trtx} for three-line text, \keyindex{Qtx} and \keyindex{Qrtx} for
four-line text.  To eliminate zigzagging lyrics lines, all multiple line texts
are automatically vertically justified with the macro \keyindex{ChroirStrut},
defined as \verb|\vphantom{\^Wgjpqy}|.

The macros \keyindex{tx}\verb|{|$text$\verb|}|,
\keyindex{rtx}\verb|{|$text$\verb|}| cause song text to be left-justified
on the insertion point rather then centered.
 \keyindex{hf}\verb|{|$m$\verb|}| sets a text
continuation rule of length $m$ \verb|\noteskip|.

Consult the source file {\tt musixdoc.tex} to see the coding of the following
example:

\begin{music}
\tenrm
\parindent11.5mm
\braceheight5.4\Interligne\relax
%\rightline{Arr.: H.~W.~Eichholz}
\instrumentnumber2
\generalsignature{-2}
\generalmeter{\meterfrac22}
\setclef1\bass
\relativeaccid
\setinterinstrument1{11\Interligne}
\songtop2
\songbottom1
\setname2{\vbox{\hsize\parindent\centerline{Sopran}\centerline{Alt}}}
\setname1{\vbox{\hsize\parindent\centerline{Tenor}\centerline{Bass}}}
\sepbarrules\nobarnumbers
\beforeruleskip-2pt
\startpiece
%% bar 1
\znotes&\rlap{\kernm2em\Qrtx 1.|2.|3.|4.*}\en
\NOTes\sslur ILd1\sslur bNu1\zhl I\hu b%
  &\Qtx\ixhf{Oh}|No|No|There'll*\issluru0f\sslur dad1\zhup f\hl d\en
\Notes\zhl L\hu N&\Qtx ~|more|more|~be*\hl{^c}\en
\NOtes&\tsslur0g\qu g\en
%%% bar 2
\bar
\NOtes\zql I\qu b&\Qrtx\thf freedom,|weepin',|moanin',|singin',*\zqu i\ql d\en
\NOTesp\lpt I\zhl I\hup b&\zhup i\hlp d\en
%%% bar 3
\bar
\nspace
\NOTes\sslur JMd1\sslur bau1\zhl J\hu b%
  &\Qtx\ixhf{oh}|no|no|there'll*\issluru0j\sslur edd1\zhup j\hl{^e}\en
\Notes\zhl M\hu a&\Qtx~|more|more|~be*\hl{=e}\en
\NOtes&\tsslur0k\qu k\en
%%% bar 4
\bar
\NOtes\zql I\qu b&\Qrtx\thf freedom,|weepin',|moanin',|singin',*\zqu j\ql d\en
\NOTesp\lpt I\zhl I\hup b&\zhup i\hlp d\en
%%% bar 5
\bar
\NOTesp\sslur IJd1\sslur bau1\lpt I\zhl I\hup b%
  &\Qtx\ixhf{oh}|no|no|~there'll*\isslurd0f\issluru1j\zhup k\hlp f\en
\NOtes\zql J\qu a&\Qtx~|more|more|~be*\tsslur1f\zqu f\roff{\tsslur0e\ql e}\en
%%% bar 6
\bar
\NOtes\zql K\qu b&\Qrtx\thf freedom,|weepin',|moanin',|singin',*\zqu i\ql f\en
\NOtes\zql J\qu b&\zqu i\ql e\en
\NOtes\zql I\qu b&\rlap{\kernm2\Internote\bigrbrace}\rtx~~~over*\zqu i\ql d\en
\NOtes\zql G\qu b&\zqu k\ql f\en
%%% bar 7
\bar
\nspace
\NOTes\zhl J\hu b\caesura&\tx ~me,*\zhu j\hl{^e}\caesura\en
\Notes\zcl J\cu b&\rtx ~over*\zcu j\cl e\en
\NOtesp\zqlp J\qup b&\zqup k\lpt e\ql e\en
\endpiece
\end{music}

\subsection{musixcpt}

Empowers \musixtex\ to run files created under
Musi\textbf{c}TeX, the predecessor of \musixtex, such as some of the examples
provided by Dani\"el {\sc Taupin}. It is not needed for any files created
under \musixtex, and is included mainly for historical completeness.

\subsection{musixdat}
Enables the command \keyindex{today}, which sets the current date in one of
several possible languages. The language is selected by an optional
 preparatory command \verb|\date...|.
The default is \keyindex{dateUSenglish}, but this can changed, either at the
end of \ttxem{musixdat.tex} for a permanant change, or right before issuing
\verb|\today|. Available choices and sample results are summarized below:
\smallskip
\begin{quote}\begin{tabular}{ll}\hline
\verb|\dateUSenglish|&\dateUSenglish\today\\
\verb|\dateaustrian|&\dateaustrian\today\\
\verb|\dateenglish|&\dateenglish\today\\
\verb|\datefrench|&\datefrench\today\\
\verb|\dategerman|&\dategerman\today\\\hline
\end{tabular}\end{quote}

\subsection{musixdbr}
Enables dashed and dotted bar lines (see \ref{musixdbr}).

\subsection{musixdia}\label{diam}
Enables notes with diamond-shaped heads as follows:
\begin{itemize}\setlength{\itemsep}{0ex}
 \item Solid note heads (\raise.5ex\hbox to .6em{\musixchar37}) are obtained
using the macros
 \keyindex{yqu}, \keyindex{yqup},
\keyindex{yqupp}, \keyindex{yql}, \keyindex{yqlp}, \keyindex{yqlpp},
\keyindex{yzq}, \keyindex{yzqp}, \keyindex{yzqpp}, \keyindex{yqb},
\keyindex{ycu}, \keyindex{yccu}, \keyindex{ycccu}, \keyindex{yccccu},
\keyindex{ycl}, \keyindex{yccl}, \keyindex{ycccl}, \keyindex{yccccl},
\keyindex{ycup}, \keyindex{ycupp}, \keyindex{yclp}, \keyindex{yclpp}.
(Think of d{\it y}\kern.5pt amond). A solid diamond with no stem is obtained
with \keyindex{ynq} (spacing) or \keyindex{yznq} (non-spacing).
 \item Open note heads (\raise.5ex\hbox to .6em{\musixchar38}) are obtained
using the macros
\keyindex{dqu}, \keyindex{dqup},
\keyindex{dqupp}, \keyindex{dql}, \keyindex{dqlp}, \keyindex{dqlpp},
\keyindex{dzq}, \keyindex{dzqp}, \keyindex{dzqpp}, \keyindex{dqb},
\keyindex{dcu}, \keyindex{dccu}, \keyindex{dcccu}, \keyindex{dccccu},
\keyindex{dcl}, \keyindex{dccl}, \keyindex{dcccl}, \keyindex{dccccl},
\keyindex{dcup}, \keyindex{dcupp}, \keyindex{dclp}, \keyindex{dclpp}.
(Think of {\it d}\kern.5pt iamond).
An open diamond with no stem is obtained
with \keyindex{dnq} (spacing) or \keyindex{dznq} (non-spacing).
 \end{itemize}

One use of these note heads is for a string part with \itxem{harmonic notes}.
%(see \ref{othernotes})
% DAS removed description which was misplaced in the samll=notes section 2.18 4/30/06
For example,

\begin{music}
\parindent0pt
\generalsignature{-2}
\generalmeter\allabreve
\startextract
\NOTes\dzq o\zh d\hu h\enotes
\Notes\ibu0k0\zq g\yqb0k\qb0j\zq e\yqb0i\tbu0\qb0j\enotes
\bar
\NOTes\dzq g\hu k\enotes
\NOTes\hpause\enotes
\bar
\NOTes\dzq o\zh d\hl h\enotes
\Notes\ibl0e0\zq g\yqb0k\qb0j\zq e\yqb0i\tbl0\qb0j\enotes
\bar
\NOTes\dzq g\hu k\enotes
\NOTes\hpause\enotes
\endextract
\end{music}
\noindent was coded as follows:
\begin{quote}\begin{verbatim}
\generalsignature{-2}
\generalmeter\allabreve
\startextract
\NOTes\dzq o\zh d\hu h\enotes
\Notes\ibu0k0\zq g\yqb0k\qb0j\zq e\yqb0i\tbu0\qb0j\enotes
\bar
\NOTes\dzq g\hu k\enotes
\NOTes\hpause\enotes
\bar
\NOTes\dzq o\zh d\hl h\enotes
\Notes\ibl0e0\zq g\yqb0k\qb0j\zq e\yqb0i\tbl0\qb0j\enotes
\bar
\NOTes\dzq g\hu k\enotes
\NOTes\hpause\enotes
\endextract
\end{verbatim}\end{quote}

Another use is for percussion parts. In fact the file \verb|musixdia.tex|
is automatically loaded if you input {\tt musixper.tex} (see \ref{perc}).

 \subsection{musixeng}
 This package is provided for music typesetters who are allergic to the default
rest names, which are
taken from French, German or Italian. It does not provide new features, only
new command names:
\medskip
 \begin{quote}\begin{tabular}{ll}\hline
\ital{original}&\ital{alternate}\\\hline
\keyindex{PAUSe}&\keyindex{Qwr}\\
\keyindex{PAuse}&\keyindex{Dwr}\\
\keyindex{liftpause}&\keyindex{liftwr}\\
\keyindex{pausep}&\keyindex{wrp}\\
\keyindex{pause}&\keyindex{wr}\\
\keyindex{lifthpause}&\keyindex{lifthr}\\
\keyindex{hpausep}&\keyindex{hrp}\\
\keyindex{hpause}&\keyindex{hr}\\
\keyindex{qp}&\keyindex{qr}\\
\keyindex{ds}&\keyindex{er}\\
\keyindex{qs}&\keyindex{eer}\\
\keyindex{hs}&\keyindex{eeer}\\
\keyindex{qqs}&\keyindex{eeeer}\\\hline
\end{tabular}\end{quote}

\subsection{musixext}
Defines the following two specific macros:

\keyindex{slide}\itbrace{p}\itbrace{x}\itbrace{s}~, which provides a glissando
starting at pitch {\it p} and extending for {\it x} \verb|\internote|s with
slope {\it s} (ranging from -8 to 8).

\keyindex{raggedstoppiece}~, which inhibits right-justification of the last
line of a score.

 \subsection{musixf{}l{}l}
 \input musixfll
 Enables modification of \ixem{ledger lines}. Ledger lines normally exceed the
width of a note head by 25 percent in each direction. If the space between
notes is insufficient, the ledger lines of
consecutive notes may meet, creating visual ambiguities. Therefore,
\musixtex{} shortens the
ledger lines if notes are set so close together that the ledger lines may
meet. But because \musixtex{} does not know whether consecutive notes need
ledger lines, this automatic shortening may be superfluous. The extension
file \ttxem{musixfll.tex} allows this feature to be switched off and on.
Upon inputting \verb|musixfll.tex|, the automatic shortening of ledger lines
is switched off. From then on, it may be switched on again using
\keyindex{autoledgerlines} and switched off again using
\keyindex{longledgerlines}. Both macros have global effect.

 The following example shows that narrowly set scales look better with
\keyindex{autoledgerlines} (the default behavior), while single notes
requiring ledger lines look better with \keyindex{longledgerlines}.
\medskip
\begin{music}
\startextract
\notes\multnoteskip{0.7}\Uptext{autoledgerlines}\autoledgerlines
    \ibbbu0b0\qb0{cba`gfg'a}\tbu0\qb0b%
    \ibbbl0{''b}0\qb0{abcdedc}\tbl0\qb0b%
    \ibbbu0{``b}0\qb0{dad}\tbu0\qb0a%
    \ibbbl0{''b}0\qb0{`g'c`g}\tbl0\qb0{'c}\en
    \bar
\notes\multnoteskip{0.7}\Uptext{longledgerlines}\longledgerlines
    \ibbbu0b0\qb0{cba`gfg'a}\tbu0\qb0b%
    \ibbbl0{''b}0\qb0{abcdedc}\tbl0\qb0b%
    \ibbbu0{``b}0\qb0{dad}\tbu0\qb0a%
    \ibbbl0{''b}0\qb0{`g'c`g}\tbl0\qb0{'c}\en
\endextract
\autoledgerlines
\end{music}

 \subsection{musixgre}\label{gregnotes}\index{gregorian chant}

% \subsection{Gregorian chant: staffs and clefs}
Gregorian chant is often coded using four line staffs
(see
%sections \ref{gregorian} and
section \ref{stafflinenumber}) and using special notes called
\itxem{neumes} (which are described later in this section). It also requires
special clefs. One way to substitute them for the modern ones is for example
with commands like

\keyindex{setaltoclefsymbol}\verb|3\gregorianCclef|

\noindent or

\keyindex{setbassclefsymbol}\verb|3\gregorianFclef| ,

\noindent which will cause instrument number 3 to display the selected gregorian
clef. The standard clefs can be restored for every instrument with
\keyindex{resetclefsymbols}. Note that when using this method you must
specify whether to substitute for the bass or alto clef
symbol (there is no treble clef in gregorian
chant). The reason is that \musixtex\ selects and
raises the F and C clefs differently, according to the arguments of the
\keyindex{setclef} command. Therefore, if one had substituted any F clef
symbol while saying \verb|\setclef1{1000}|, then an F clef would duly appear
on the staff, but it would be set at the position of an alto clef, thus
seriously misleading the musician.

Another method of clef substitution employs \keyindex{setclefsymbol} (see
section  \ref{treblelowoct}), which substitutes the clef given by
the second argument \ital{for all clef symbols} of the instrument given by the first,
regardless of the actual musical meaning of the new clef symbol. This method is
generally appropriate only if you want to change the clef symbol(s) of
an instrument for the whole of the score.

 As an example, the same gregorian scale has been written with a gregorian C
clef on all four lines of the staff:

 \begin{music}\nostartrule
 \parindent 19mm
 \instrumentnumber{4}
 \setname1{1st line} \setname2{2nd line} \setname3{3rd line} \setname4{4th line}
 \setlines1{4}\setlines2{4}\setlines3{4}\setlines4{4}
 \sepbarrules
 \generalmeter{\empty}
 \setclef1{1000} \setclef2{2000} \setclef3{3000} \setclef4{4000}
 \setaltoclefsymbol1\gregorianCclef
 \setaltoclefsymbol2\gregorianCclef
 \setaltoclefsymbol3\gregorianCclef
 \setaltoclefsymbol4\gregorianCclef
\startextract
\Notes\squ{abcdefghi}&\squ{abcdefghi}&\squ{abcdefghi}&\squ{abcdefghi}&\enotes
\zendextract
\end{music}

 The coding was:
 \begin{verbatim}
 \instrumentnumber{4}
 \setname1{1st line} \setname2{2nd line} \setname3{3rd line} \setname4{4th line}
 \setlines1{4}\setlines2{4}\setlines3{4}\setlines4{4}
 \sepbarrules
 \generalmeter{\empty}
 \setclef1{1000} \setclef2{2000} \setclef3{3000} \setclef4{4000}
 \setaltoclefsymbol1\gregorianCclef
 \setaltoclefsymbol2\gregorianCclef
 \setaltoclefsymbol3\gregorianCclef
 \setaltoclefsymbol4\gregorianCclef
\startextract
\Notes\squ{abcdefghi}&\squ{abcdefghi}&\squ{abcdefghi}&\squ{abcdefghi}&\enotes
\zendextract
 \end{verbatim}

All of the special gregorian symbols available in \musixtex\ are described
in the following subsections.

\subsubsection{Clefs}

\begin{itemize}\setlength{\itemsep}{0ex}
 \item Gregorian C clef: \raise 2.5pt\hbox to 1cm{\gregorianCclef\hfil}~=
 \keyindex{gregorianCclef}, normally activated for instrument $n$ with the
command
 \keyindex{setaltoclefsymbol}\itbrace{n}\keyindex{gregorianCclef}
\item Gregorian F clef:  \raise 2.5pt\hbox to 1cm{\gregorianFclef\hfil}~=
\keyindex{gregorianFclef}, normally activated with the command
\keyindex{setbassclefsymbol}\itbrace{n}\keyindex{gregorianFclef}
 \end{itemize}

 \subsubsection{Elementary symbols}


 \begin{itemize}\setlength{\itemsep}{0ex}

 \item Diamond shaped \itxem{punctum} (This has a different shape compared to the
percussion diamond): \raise 2.5pt\hbox{\xgregchar1}~ =
\keyindex{diapunc}\pitchp\ .
 \item Square \itxem{punctum}: \raise 2.5pt\hbox{\xgregchar5}~ =
\keyindex{squ}\pitchp\ or \keyindex{punctum}\pitchp\ .
 \item Left stemmed \itxem{virga} (not in the 1905 gregorian standard): \raise
2.5pt\hbox{\xgregchar6}~ = \keyindex{lsqu}\pitchp\ .
 \item Right stemmed \itxem{virga}: \raise 2.5pt\hbox{\xgregchar7}~ =
\keyindex{rsqu}\pitchp\ or \keyindex{virga}\pitchp\ .
 \item \ital{Apostropha}\index{apostropha}: \raise 2.5pt\hbox{\xgregchar3}~ =
\keyindex{apostropha}\pitchp\ .
 \item \ital{Oriscus}\index{oriscus}: \raise 2.5pt\hbox{\xgregchar10}~ =
\keyindex{oriscus}\pitchp\ .

 \item \ital{Quilisma}\index{quilisma}: \raise 2.5pt\hbox{\xgregchar125}~ =
\keyindex{quilisma}\pitchp\ .
 \item \ital{Punctum auctum} (up)\index{punctum auctum}: \raise
2.5pt\hbox{\xgregchar9}~ =
\keyindex{punctumauctup}\pitchp\ .
 \item \ital{Punctum auctum} (down)\index{punctum auctum}: \raise
2.5pt\hbox{\xgregchar8}~ =
\keyindex{punctumauctdown}\pitchp\ .
 \item Diamond shaped \ital{punctum auctum} (down)\index{punctum auctum}:
\raise 2.5pt\hbox{\xgregchar2}~ = \keyindex{diapunctumauctdown}\pitchp\ .
 \item \ital{Punctum deminutum}\index{punctum deminutum}: \raise
2.5pt\hbox{\xgregchar4}~ = \keyindex{punctumdeminutum}\pitchp\ .
 \item \ital{Apostropha aucta}\index{apostropha aucta}: \raise
2.5pt\hbox{\xgregchar11}~ = \keyindex{apostropha aucta}\pitchp\ .

\end{itemize}
 All non-\ital{liquescens} symbols have non-spacing variants, namely
\keyindex{zdiapunc}, \keyindex{zsqu}, \keyindex{zlsqu}, \keyindex{zrsqu},
\keyindex{zapostropha} and \keyindex{zoriscus}.

\subsubsection{Plain complex neumes}
Other \itxem{neumes} can be obtained by combining two or more of these
symbols. Since \itxem{neumes} have a special note head width, an additional
shifting macro is provided, namely \keyindex{groff}. It is similar to
\verb|\roff|, but the offset is smaller. For use with comples neumes, another
shifting macro is provided, namely \keyindex{dgroff}, which causes an
offset twice the offset of \verb|\groff|.

 Since most of these symbols depend on relative pitches of their components,
we cannot provide all possible compact combinations as single symbols. The ones
that are available in \verb|musixgre| are described below. In the following,
$p_1$, $p_2$, $p_3$, and $p_4$ represent pitches specified as usual. Please refer to
the source file \verb|musixtex.tex| if you wish to see the coding of those
examples for which it is not quoted here.

\def\twop{\itbrace{p_1}\itbrace{p_2}}
\def\threep{\twop\itbrace{p_3}}
\def\fourp{\threep\itbrace{p_4}}

\begin{description}\setlength{\itemsep}{0ex}
 \item[\keyindex{bivirga}\twop], for example:

 \begin{music}\nostartrule
 \elemskip 10pt
 \setsize1{\Largevalue}
 \instrumentnumber 1
 \setstaffs 1 1
 \setlines 1 4
 \setclef 1{3000}
 \setaltoclefsymbol 1 \gregorianCclef
 \startextract
 \notes \bivirga ab\enotes
 \notes \bivirga cc\enotes
 \zendextract
 \end{music}

  This example was coded as:
  \begin{quote}\begin{verbatim}
 \instrumentnumber 1
 \setstaffs 1 1
 \setlines 1 4
 \setclef 1{3000}
 \setaltoclefsymbol 1 \gregorianCclef
 \startextract
 \notes \bivirga ab\enotes
 \notes \bivirga cc\enotes
 \zendextract
  \end{verbatim}\end{quote}

 \item[\keyindex{trivirga}\threep], for example:

 \begin{music}\nostartrule
 \elemskip 10pt
 \setsize1{\Largevalue}
 \instrumentnumber 1
 \setstaffs 1 1
 \setlines 1 4
 \setclef 1{3000}
 \setaltoclefsymbol 1 \gregorianCclef
 \startextract
 \Notes \trivirga abc\enotes
 \Notes \trivirga cca\enotes
 \zendextract
 \end{music}

 \item[\keyindex{bistropha}\twop], for example:

 \begin{music}\nostartrule
 \elemskip 10pt
 \setsize1{\Largevalue}
 \instrumentnumber 1
 \setstaffs 1 1
 \setlines 1 4
 \setclef 1{3000}
 \setaltoclefsymbol 1 \gregorianCclef
 \startextract
 \notes \bistropha ab\enotes
 \notes \bistropha cc\enotes
 \zendextract
 \end{music}

 \item[\keyindex{tristropha}\threep], for example:

 \begin{music}\nostartrule
 \elemskip 10pt
 \setsize1{\Largevalue}
 \instrumentnumber 1
 \setstaffs 1 1
 \setlines 1 4
 \setclef 1{3000}
 \setaltoclefsymbol 1 \gregorianCclef
 \startextract
 \Notes \tristropha abc\enotes
 \Notes \tristropha cca\enotes
 \zendextract
 \end{music}

  \item[\keyindex{clivis}\twop], for example:

 \begin{music}\nostartrule
 \elemskip 10pt
 \setsize1{\Largevalue}
 \instrumentnumber 1
 \setstaffs 1 1
 \setlines 1 4
 \setclef 1{3000}
 \setaltoclefsymbol 1 \gregorianCclef
 \startextract
 \notes \clivis ba\enotes
 \notes \clivis ca\enotes
 \zendextract
 \end{music}

 \item[\keyindex{lclivis}\twop], for example:

 \begin{music}\nostartrule
 \setsize1{\Largevalue}
 \elemskip 10pt
 \instrumentnumber 1
 \setstaffs 1 1
 \setlines 1 4
 \setclef 1{3000}
 \setaltoclefsymbol 1 \gregorianCclef
 \startextract
 \notes \lclivis ba\enotes
 \notes \lclivis ca\enotes
 \zendextract
 \end{music}

 \item[\keyindex{podatus}\twop], for example:

 \begin{music}\nostartrule
 \elemskip 10pt
 \instrumentnumber 1
 \setsize1{\Largevalue}
 \setstaffs 1 1
 \setlines 1 4
 \setclef 1{3000}
 \setaltoclefsymbol 1 \gregorianCclef
 \startextract
 \notes \podatus ab\enotes
 \notes \podatus ac\enotes
 \notes \podatus cf\enotes
 \zendextract
 \end{music}

 \item[\keyindex{podatusinitiodebilis}\twop], for example:

 \begin{music}\nostartrule
 \elemskip 10pt
 \instrumentnumber 1
 \setsize1{\Largevalue}
 \setstaffs 1 1
 \setlines 1 4
 \setclef 1{3000}
 \setaltoclefsymbol 1 \gregorianCclef
 \startextract
 \notes \podatusinitiodebilis ab\enotes
 \notes \podatusinitiodebilis ac\enotes
 \notes \podatusinitiodebilis cf\enotes
 \zendextract
 \end{music}

 \item[\keyindex{lpodatus}\twop], for example:

 \begin{music}\nostartrule
 \elemskip 10pt
 \setsize1{\Largevalue}
  \instrumentnumber 1
 \setstaffs 1 1
 \setlines 1 4
 \setclef 1{3000}
 \setaltoclefsymbol 1 \gregorianCclef
 \startextract
 \notes \lpodatus ab\enotes
 \notes \lpodatus ce\enotes
 \zendextract
 \end{music}

 \item[\keyindex{pesquassus}\twop], for example:

 \begin{music}\nostartrule
 \elemskip 10pt
 \instrumentnumber 1
 \setsize1{\Largevalue}
 \setstaffs 1 1
 \setlines 1 4
 \setclef 1{3000}
 \setaltoclefsymbol 1 \gregorianCclef
 \startextract
 \notes \pesquassus ab\enotes
 \notes \pesquassus ae\enotes
 \zendextract
 \end{music}

 \item[\keyindex{quilismapes}\twop], for example:

 \begin{music}\nostartrule
 \elemskip 10pt
 \instrumentnumber 1
 \setsize1{\Largevalue}
 \setstaffs 1 1
 \setlines 1 4
 \setclef 1{3000}
 \setaltoclefsymbol 1 \gregorianCclef
 \startextract
 \notes \quilismapes ab\enotes
 \notes \quilismapes ae\enotes
 \zendextract
 \end{music}

  \item[\keyindex{torculus}\threep], for example:

 \begin{music}\nostartrule
 \elemskip 10pt
 \instrumentnumber 1
 \setsize1{\Largevalue}
 \setstaffs 1 1
 \setlines 1 4
 \setclef 1{3000}
 \setaltoclefsymbol 1 \gregorianCclef
 \startextract
 \notes \torculus aba\enotes
 \notes \torculus cfd\enotes
 \notes \torculus afc\enotes
 \zendextract
 \end{music}

  \item[\keyindex{torculusinitiodebilis}\threep], for example:

 \begin{music}\nostartrule
 \elemskip 10pt
 \instrumentnumber 1
 \setsize1{\Largevalue}
 \setstaffs 1 1
 \setlines 1 4
 \setclef 1{3000}
 \setaltoclefsymbol 1 \gregorianCclef
 \startextract
 \notes \torculusinitiodebilis aba\enotes
 \notes \torculusinitiodebilis cfd\enotes
 \notes \torculusinitiodebilis afc\enotes
 \zendextract
 \end{music}

 \item[\keyindex{Porrectus}\threep], for example:

 \begin{music}\nostartrule
 \elemskip 10pt
 \instrumentnumber 1
 \setsize1{\Largevalue}
 \setstaffs 1 1
 \setlines 1 4
 \setclef 1{3000}
 \setaltoclefsymbol 1 \gregorianCclef
 \startextract
 \notes \Porrectus bab\enotes
 \notes \Porrectus bac\enotes
 \notes \Porrectus bNd\enotes
 \notes \Porrectus bMe\enotes
 \notes \Porrectus bLe\enotes
 \zendextract
 \end{music}
\noindent coded:
 \begin{quote}\begin{verbatim}
 \notes \Porrectus bab\enotes
 \notes \Porrectus bac\enotes
 \notes \Porrectus bNd\enotes
 \notes \Porrectus bMe\enotes
 \notes \Porrectus bLe\enotes
 \end{verbatim}\end{quote}

 \verb|\Porrectus| exists in four different shapes, depending on the
difference between first and second argument. The constraint is that
 $$ p_1-4 \leq p_2 \leq p_1-1 $$ otherwise a diagnostic occurs. Note also that
\keyindex{bporrectus} provides the first curved part of the \verb|porrectus|
command, if you should need it. It has two arguments, the starting pitch and the lower
pitch.

 \item[\keyindex{Porrectusflexus}\fourp], for example:

 \begin{music}\nostartrule
 \elemskip 10pt
 \instrumentnumber 1
 \setsize1{\Largevalue}
 \setstaffs 1 1
 \setlines 1 4
 \setclef 1{3000}
 \setaltoclefsymbol 1 \gregorianCclef
 \startextract
 \notes \Porrectusflexus  bacN\enotes
 \notes \Porrectusflexus  bNdb\enotes
 \notes \Porrectusflexus  bMeb\enotes
 \notes \Porrectusflexus  bLea\enotes
 \zendextract
 \end{music}
\noindent coded:
 \begin{quote}\begin{verbatim}
 \notes \Porrectusflexus  bacN\enotes
 \notes \Porrectusflexus  bNdb\enotes
 \notes \Porrectusflexus  bMeb\enotes
 \notes \Porrectusflexus  bLea\enotes
 \end{verbatim}\end{quote}


 \item[\keyindex{climacus}\threep], for example:

 \begin{music}\nostartrule
 \elemskip 10pt
 \instrumentnumber 1
 \setsize1{\Largevalue}
 \setstaffs 1 1
 \setlines 1 4
 \setclef 1{3000}
 \setaltoclefsymbol 1 \gregorianCclef
 \startextract
 \Notes \climacus cbN\enotes
 \Notes \climacus cba\enotes
 \Notes \climacus dbN\enotes
 \zendextract
 \end{music}

 \item[\keyindex{climacusresupinus}\fourp], for example:

 \begin{music}\nostartrule
 \elemskip 10pt
 \instrumentnumber 1
 \setsize1{\Largevalue}
 \setstaffs 1 1
 \setlines 1 4
 \setclef 1{3000}
 \setaltoclefsymbol 1 \gregorianCclef
 \startextract
 \Notes \climacusresupinus cbNa\enotes
 \Notes \climacusresupinus cbab\enotes
 \Notes \climacusresupinus dbNb\enotes
 \zendextract
 \end{music}

 \item[\keyindex{lclimacus}\threep], for example:

 \begin{music}\nostartrule
 \elemskip 10pt
  \setsize1{\Largevalue}
\instrumentnumber 1
 \setstaffs 1 1
 \setlines 1 4
 \setclef 1{3000}
 \setaltoclefsymbol 1 \gregorianCclef
 \startextract
 \notes \lclimacus cbN\enotes
 \notes \lclimacus cfd\enotes
 \notes \lclimacus afc\enotes
 \zendextract
 \end{music}

 \item[\keyindex{scandicus}\threep], for example:

 \begin{music}\nostartrule
 \elemskip 10pt
 \instrumentnumber 1
 \setsize1{\Largevalue}
 \setstaffs 1 1
 \setlines 1 4
 \setclef 1{3000}
 \setaltoclefsymbol 1 \gregorianCclef
 \startextract
 \notes \scandicus abe\enotes
 \notes \scandicus ceg\enotes
 \zendextract
 \end{music}

 \item[\keyindex{salicus}\threep], for example:

 \begin{music}\nostartrule
 \elemskip 10pt
 \instrumentnumber 1
 \setsize1{\Largevalue}
 \setstaffs 1 1
 \setlines 1 4
 \setclef 1{3000}
 \setaltoclefsymbol 1 \gregorianCclef
 \startextract
 \Notes \salicus abe\enotes
 \Notes \salicus ceg\enotes
 \zendextract
 \end{music}

 \item[\keyindex{salicusflexus}\fourp], for example:

 \begin{music}\nostartrule
 \elemskip 10pt
 \instrumentnumber 1
 \setsize1{\Largevalue}
 \setstaffs 1 1
 \setlines 1 4
 \setclef 1{3000}
 \setaltoclefsymbol 1 \gregorianCclef
 \startextract
 \Notes \salicusflexus abec\enotes
 \Notes \salicusflexus cegd\enotes
 \zendextract
 \end{music}

 \item[\keyindex{trigonus}\threep],
%DAS ????????????
%for example\footnote{The second example is in principle irrelevant,
%but it shows the possibilities, in case of.}:
for example:

 \begin{music}\nostartrule
 \elemskip 10pt
 \instrumentnumber 1
 \setsize1{\Largevalue}
 \setstaffs 1 1
 \setlines 1 4
 \setclef 1{3000}
 \setaltoclefsymbol 1 \gregorianCclef
 \startextract
 \Notes \trigonus aaN\enotes
 \Notes \trigonus cef\enotes
 \zendextract
 \end{music}

\end{description}

\subsubsection{Liquescens complex neumes}\index{liquescens neumes}
\begin{description}\setlength{\itemsep}{0ex}
  \item[\keyindex{clivisauctup}\twop], for example:

 \begin{music}\nostartrule
 \elemskip 10pt
 \setsize1{\Largevalue}
 \instrumentnumber 1
 \setstaffs 1 1
 \setlines 1 4
 \setclef 1{3000}
 \setaltoclefsymbol 1 \gregorianCclef
 \startextract
 \notes \clivisauctup ba\enotes
 \notes \clivisauctup ca\enotes
 \zendextract
 \end{music}
  \item[\keyindex{clivisauctdown}\twop], for example:

 \begin{music}\nostartrule
 \elemskip 10pt
 \setsize1{\Largevalue}
 \instrumentnumber 1
 \setstaffs 1 1
 \setlines 1 4
 \setclef 1{3000}
 \setaltoclefsymbol 1 \gregorianCclef
 \startextract
 \notes \clivisauctdown ba\enotes
 \notes \clivisauctdown ca\enotes
 \zendextract
 \end{music}
 \item[\keyindex{podatusauctup}\twop], for example:

 \begin{music}\nostartrule
 \elemskip 10pt
 \instrumentnumber 1
 \setsize1{\Largevalue}
 \setstaffs 1 1
 \setlines 1 4
 \setclef 1{3000}
 \setaltoclefsymbol 1 \gregorianCclef
 \startextract
 \notes \podatusauctup ab\enotes
 \notes \podatusauctup ac\enotes
 \notes \podatusauctup cf\enotes
 \zendextract
 \end{music}
 \item[\keyindex{podatusauctdown}\twop], for example:

 \begin{music}\nostartrule
 \elemskip 10pt
 \instrumentnumber 1
 \setsize1{\Largevalue}
 \setstaffs 1 1
 \setlines 1 4
 \setclef 1{3000}
 \setaltoclefsymbol 1 \gregorianCclef
 \startextract
 \notes \podatusauctdown ab\enotes
 \notes \podatusauctdown ac\enotes
 \notes \podatusauctdown cf\enotes
 \zendextract
 \end{music}

 \item[\keyindex{pesquassusauctdown}\twop], for example:

 \begin{music}\nostartrule
 \elemskip 10pt
 \instrumentnumber 1
 \setsize1{\Largevalue}
 \setstaffs 1 1
 \setlines 1 4
 \setclef 1{3000}
 \setaltoclefsymbol 1 \gregorianCclef
 \startextract
 \notes \pesquassusauctdown ab\enotes
 \notes \pesquassusauctdown ae\enotes
 \zendextract
 \end{music}

 \item[\keyindex{quilismapesauctdown}\twop], for example:

 \begin{music}\nostartrule
 \elemskip 10pt
 \instrumentnumber 1
 \setsize1{\Largevalue}
 \setstaffs 1 1
 \setlines 1 4
 \setclef 1{3000}
 \setaltoclefsymbol 1 \gregorianCclef
 \startextract
 \notes \quilismapesauctdown ab\enotes
 \notes \quilismapesauctdown ae\enotes
 \zendextract
 \end{music}

  \item[\keyindex{torculusauctdown}\threep], for example:

 \begin{music}\nostartrule
 \elemskip 10pt
 \instrumentnumber 1
 \setsize1{\Largevalue}
 \setstaffs 1 1
 \setlines 1 4
 \setclef 1{3000}
 \setaltoclefsymbol 1 \gregorianCclef
 \startextract
 \notes \torculusauctdown aba\enotes
 \notes \torculusauctdown cfd\enotes
 \notes \torculusauctdown afc\enotes
 \zendextract
 \end{music}

 \item[\keyindex{Porrectusauctdown}\threep], for example:

 \begin{music}\nostartrule
 \elemskip 10pt
 \instrumentnumber 1
 \setsize1{\Largevalue}
 \setstaffs 1 1
 \setlines 1 4
 \setclef 1{3000}
 \setaltoclefsymbol 1 \gregorianCclef
 \startextract
 \notes \Porrectusauctdown bac\enotes
 \notes \Porrectusauctdown bNd\enotes
 \notes \Porrectusauctdown bMe\enotes
 \notes \Porrectusauctdown bLe\enotes
 \zendextract
 \end{music}

 \item[\keyindex{climacusauctdown}\threep], for example:

 \begin{music}\nostartrule
 \elemskip 10pt
 \instrumentnumber 1
 \setsize1{\Largevalue}
 \setstaffs 1 1
 \setlines 1 4
 \setclef 1{3000}
 \setaltoclefsymbol 1 \gregorianCclef
 \startextract
 \Notes \climacusauctdown cbN\enotes
 \Notes \climacusauctdown caM\enotes
 \Notes \climacusauctdown aNM\enotes
 \zendextract
 \end{music}

 \item[\keyindex{scandicusauctdown}\threep], for example:

 \begin{music}\nostartrule
 \elemskip 10pt
 \instrumentnumber 1
 \setsize1{\Largevalue}
 \setstaffs 1 1
 \setlines 1 4
 \setclef 1{3000}
 \setaltoclefsymbol 1 \gregorianCclef
 \startextract
 \notes \scandicusauctdown abe\enotes
 \notes \scandicusauctdown ceg\enotes
 \zendextract
 \end{music}

 \item[\keyindex{salicusauctdown}\threep], for example:

 \begin{music}\nostartrule
 \elemskip 10pt
 \instrumentnumber 1
 \setsize1{\Largevalue}
 \setstaffs 1 1
 \setlines 1 4
 \setclef 1{3000}
 \setaltoclefsymbol 1 \gregorianCclef
 \startextract
 \Notes \salicusauctdown abe\enotes
 \Notes \salicusauctdown ceg\enotes
 \zendextract
 \end{music}

  \item[\keyindex{clivisdeminut}\twop], for example:

 \begin{music}\nostartrule
 \elemskip 10pt
 \setsize1{\Largevalue}
 \instrumentnumber 1
 \setstaffs 1 1
 \setlines 1 4
 \setclef 1{3000}
 \setaltoclefsymbol 1 \gregorianCclef
 \startextract
 \notes \clivisdeminut ba\enotes
 \notes \clivisdeminut ca\enotes
 \zendextract
 \end{music}

 \item[\keyindex{podatusdeminut}\twop], for example:

 \begin{music}\nostartrule
 \elemskip 10pt
 \instrumentnumber 1
 \setsize1{\Largevalue}
 \setstaffs 1 1
 \setlines 1 4
 \setclef 1{3000}
 \setaltoclefsymbol 1 \gregorianCclef
 \startextract
 \notes \podatusdeminut ab\enotes
 \notes \podatusdeminut ac\enotes
 \notes \podatusdeminut cf\enotes
 \zendextract
 \end{music}

  \item[\keyindex{torculusdeminut}\threep], for example:

 \begin{music}\nostartrule
 \elemskip 10pt
 \instrumentnumber 1
 \setsize1{\Largevalue}
 \setstaffs 1 1
 \setlines 1 4
 \setclef 1{3000}
 \setaltoclefsymbol 1 \gregorianCclef
 \startextract
 \notes \torculusdeminut aba\enotes
 \notes \torculusdeminut cfd\enotes
 \notes \torculusdeminut afc\enotes
 \zendextract
 \end{music}

  \item[\keyindex{torculusdebilis}\threep], for example:

 \begin{music}\nostartrule
 \elemskip 10pt
 \instrumentnumber 1
 \setsize1{\Largevalue}
 \setstaffs 1 1
 \setlines 1 4
 \setclef 1{3000}
 \setaltoclefsymbol 1 \gregorianCclef
 \startextract
 \notes \torculusdebilis aba\enotes
 \notes \torculusdebilis cfd\enotes
 \notes \torculusdebilis afc\enotes
 \zendextract
 \end{music}

 \item[\keyindex{Porrectusdeminut}\threep], for example:

 \begin{music}\nostartrule
 \elemskip 10pt
 \instrumentnumber 1
 \setsize1{\Largevalue}
 \setstaffs 1 1
 \setlines 1 4
 \setclef 1{3000}
 \setaltoclefsymbol 1 \gregorianCclef
 \startextract
 \notes \Porrectusdeminut bac\enotes
 \notes \Porrectusdeminut bNd\enotes
 \notes \Porrectusdeminut bMe\enotes
 \notes \Porrectusdeminut bLe\enotes
 \zendextract
 \end{music}

 \item[\keyindex{climacusdeminut}\threep], for example:

 \begin{music}\nostartrule
 \elemskip 10pt
 \instrumentnumber 1
 \setsize1{\Largevalue}
 \setstaffs 1 1
 \setlines 1 4
 \setclef 1{3000}
 \setaltoclefsymbol 1 \gregorianCclef
 \startextract
 \Notes \climacusdeminut cbN\enotes
 \Notes \climacusdeminut caM\enotes
 \Notes \climacusdeminut aML\enotes
 \zendextract
 \end{music}

 \item[\keyindex{scandicusdeminut}\threep], for example:

 \begin{music}\nostartrule
 \elemskip 10pt
 \instrumentnumber 1
 \setsize1{\Largevalue}
 \setstaffs 1 1
 \setlines 1 4
 \setclef 1{3000}
 \setaltoclefsymbol 1 \gregorianCclef
 \startextract
 \notes \scandicusdeminut abe\enotes
 \notes \scandicusdeminut ceg\enotes
 \zendextract
 \end{music}
\end{description}

\subsection{musixgui}\ixtt{musixgui.tex} Provides macros for
typesetting modern style \itxem{guitar tablatures}. For example:

\begin{music}
\hsize130mm
\tenrm
\parindent0pt
\generalmeter{\meterfrac34}
\generalsignature1
\startbarno0
\def\txh{-6.5}
\def\tx#1*{\zchar\txh{\lrlap{\kern3\Internote#1}}}
\def\rtx#1*{\zchar\txh{\kern-3\Internote#1}}
\stafftopmarg10\Interligne
\raiseguitar{20}
\nostartrule
\startpiece
\addspace{.5\afterruleskip}%
\NOtes\tx We*\qa d\en
\bar
\NOtes\guitar G{}o-----\gbarre3\gdot25\gdot35\gdot44\tx wish*\qa g\en
\Notes\tx you*\ca g\en
\Notes\tx a*\ca h\en
\Notes\zchar\txh{merry}\ca g\en
\Notes\ca f\en
\bar
\NOtes\guitar C5o-----\gbarre4\gdot26\gdot36\gdot45\rtx christmas,*\qa e\en
\NOtes\qa e\en
\NOtes\guitar {e/H}5o-----\gbarre3\gdot35\gdot45\gdot54\tx we*\qa e\en
\bar
\NOtes\guitar {A$\!^7$}5o-----\gbarre1\gdot23\gdot42\tx wish*\qa h\en
\Notes\tx you*\ca h\en
\Notes\tx a*\ca i\en
\Notes\zchar\txh{merry}\ca h\en
\Notes\ca g\en
\bar
\NOtes\guitar D{}xxo---\gdot42\gdot53\gdot62\rtx christmas,*\qa f\en
\NOtes\qa d\en
\zbar
\NOtes\guitar{D/c}{}xo----\gdot23\gdot42\gdot53\gdot62\tx we*\qa d\en
\bar
\NOtes\guitar{B$^7$}{}xo----\gdot22\gdot31\gdot42\gdot62\tx wish*\qa i\en
\Notes\tx you*\ca i\en
\Notes\tx a*\ca j\en
\Notes\tx ~mer-*\ca i\en
\Notes\tx ry*\ca h\en
\bar
\NOtes\guitar e{}xxo---\gdot32\tx ~christ-*\qa g\en
\NOtes\tx mas*\qa e\en
\Notes\guitar {G/d}{}xxo---\gdot63\tx and*\ca d\en
\Notes\tx a*\ca d\en
\bar
\NOtes\guitar{C$^6$}{}xo----\gdot23\gdot32\gdot42\gdot51\tx ~~hap~-*\qa e\en
\NOtes\tx py*\qa h\en
\NOtes\guitar{D$^7$}{}xo---x\gdot25\gdot34\gdot45\gdot45\gdot53\tx new*\qa f\en
\bar
\NOTes\guitar G{}o-----\gbarre3\gdot25\gdot35\gdot44\tx ~year.*\ha g\en
\setdoublebar\endpiece
\end{music}

\medskip
The macro \keyindex{guitar} sets the grid, chord name, barre type,
and on-off indicators for the strings. For example, the first
chord in above example was coded as

\verb|\guitar G{}o-----\gbarre3\gdot25\gdot35\gdot44|

\noindent where the first argument is the text to be placed above the grid, the
second is empty (relative barre), and the next six characters indicate if the string
is played or not with either \verb|x|, \verb|o| or \verb|-|. The dots are set with
\keyindex{gdot}~$sb$ where the $s$ is the string and $b$ is the barre. The
rule is set with \keyindex{gbarre}~$b$ where $b$ indicates the position of the barre.

The whole symbol may be vertically shifted with
\keyindex{raiseguitar}\onen, where $n$ is a number
in units of \keyindex{internote}. When using guitar tablatures, it might be
useful to reserve additional
space above the chord by advancing \keyindex{stafftopmarg} to
something like \verb|stafftopmarg=10\Interligne|.

For frequently used chords, it might be useful to define your own
macros, e.g.

\verb|\def\Dmajor{\guitar D{}x-----\gdot42\gdot53\gdot62}%|

 \subsection{musixlit}\label{litu}\label{otherbars}
 Provides a notation style intermediate between gregorian and baroque, for
example

\makeatletter
\catcodesmusic
%\def\vnotes#1\elemskip{\noteskip#1\@l@mskip \multnoteskip\scal@noteskip
%  \not@s}

%\def\not@s{\def|{\nextstaff}\def&{\nextinstrument}\normaltranspose\transpose
%  \check@nopen\notes@open\@ne
%  \kern\n@skip\advance\x@skip\n@skip \locx@skip\x@skip
%  \n@skip\noteskip \noinstrum@nt\z@ \begininstrument}

%\def\en{\@ndstaff\notes@open\z@
%  \ifx\@ne\V@sw \widthtyp@\z@\t@rmskip \let\V@sw\empty \fi}

\def\double#1{\roffset{1.2}{\advancefalse#1}#1}
\makeatother

 % Don't know if this example correct ??? But looks nice...
\begin{music}
\parindent0pt
\instrumentnumber{2}
\interstaff{11}
\generalsignature2
\setclefsymbol2\oldGclef
\setstaffs1{2}
\setclef1\bass
\setinterinstrument1{-\Interligne}
\startpiece
\shortbarrules
\addspace\afterruleskip
\hardlyrics{Il nous a sign\'es de son
  }\notes\zw d\wh K|\zw f\wh h&\rtx\thelyrics*\Hpause h1\en
\qspace\qspace
\NOTes\zhl N\hu d|\zhl g\hu i&\tx sang*\double{\cnql i}\en
\bar
\hardlyrics{Et nous avons \'e-
  }\notes\zw d\wh K|\zw f\wh h&\rtx\thelyrics~-*\Hpause h1\en
\qspace\qspace
\NOTes\zhl M\hu c|\zhl f\hu h&\rtx t\'e*\double{\cnql i}\en
\NOtes\zql L\qu e|\zql b\qu g&\tx ~pro-*\cnqu g\en
\NOtes\zql b\qu d|\zql d\qu f&\tx ~t\'e-*\cnqu f\en
\NOTes\zhl a\hu c|\zhl e\hu h&\tx g\'es.*\double{\cnqu h}\en
\bar
\NOtes\zql M\qu d|\zql d\qu h&\tx ~~Al~-*\cnqu h\en
\NOtes\zql K\qu a|\zql f\qu k&\tx ~~le~-*\cnql k\en
\NOTes\zhl H\hu a|\zhl e\hu j&\tx ~~lu~-*\double{\cnql j}\en
\bar
\NOTEs\zhl K\hu a|\zhl f\hu k&\tx ~~ia !*\cnhl k\en
\sepbarrules
\endpiece

\startpiece
\interbarrules
\addspace\afterruleskip
\hardlyrics{Il nous a sign\'es de son
  }\notes\zw d\wh K|\zw f\wh h&\rtx \thelyrics*\Hlonga h1\en
\qspace\qspace
\NOTes\zhl N\hu d|\zhl g\hu i&\tx sang*\chl i\en
\bar
\hardlyrics{Et nous avons \'e
  }\notes\zw d\wh K|\zw f\wh h&\rtx \thelyrics~-*\Hlonga h1\en
\qspace\qspace
\NOTes\zhl M\hu c|\zhl f\hu h&\rtx t\'e*\chl i\en
\NOtes\zql L\qu e|\zql b\qu g&\tx ~pro-*\cqu g\en
\NOtes\zql b\qu d|\zql d\qu f&\tx ~t\'e-*\cqu f\en
\NOTes\zhl a\hu c|\zhl e\hu h&\tx g\'es.*\chu h\en
\bar
\NOtes\zql M\qu d|\zql d\qu h&\tx ~~Al~-*\cqu h\en
\NOtes\zql K\qu a|\zql f\qu k&\tx ~~le~-*\cql k\en
\NOTes\zhl H\hu a|\zhl e\hu j&\tx ~~lu~-*\chl j\en
\bar
\NOTEs\zbreve K\breve a|\zbreve f\breve k&\tx ~~ia !*\zbreve k\en
\sepbarrules
\endpiece
\end{music}

 This package provides:
 \begin{itemize}\setlength{\itemsep}{0ex}
 \item\keyindex{oldGclef} which replaces the ordinary G clef with an old one,
using (for instrument 2 as an example):
 \verb|\settrebleclefsymbol2\oldGclef|

 \item\keyindex{cqu} $p$ provides a square headed quarter note with stem up at
pitch $p$.

 \item\keyindex{cql} $p$ provides a square headed quarter note with stem down at
pitch $p$.

 \item\keyindex{chu} $p$ provides a square headed half note with stem up at
pitch $p$.

 \item\keyindex{chl} $p$ provides a square headed half note with stem down at
pitch $p$.

 \item\keyindex{cnqu} $p$ and \keyindex{cnql} $p$ provide a stemless square
headed
quarter note at pitch $p$.

 \item\keyindex{cnhu} $p$ and \keyindex{cnhl} $p$ provide a stemless square
headed half note at pitch $p$.

 \item\keyindex{Hpause} $p$ $n$ provides an arbitrary length pause at pitch
$p$ and of length $n$ \keyindex{noteskip}. However, in the first of the above
example, this feature has been used to denote an arbitrary length note rather
than a rest!

 \item\keyindex{Hlonga} $p$ $n$ provides an arbitrary length note at pitch
$p$ and of length $n$ \keyindex{noteskip}.
This feature has been used to denote an arbitrary length note in the second of
the above examples.

 \item\keyindex{shortbarrules} has been used to provide bar rules shorter than
the staff vertical width.

 \item\keyindex{interbarrules} has been used to provide bars between the
staffs, rather that over them. This is an arbitrary question of taste...
 \end{itemize}

 \subsection{musixlyr} \ixtt{musixlyr.tex}
 Enables the recommended method for adding lyrics to a score (see \ref{musixlyr}).

 \subsection{musixmad} \ixtt{musixmad.tex} Increases the number of
instruments, slurs and beams up to twelve. When using this extension, it is
not necessary to explicitly input \verb|musixadd.tex|.

If you need greater numbers of these elements, see sections 
\ref{musixmad_setmaxgroups}, 
\ref{musixmad_setmaxinstruments_ccxviiibeams}, 
\ref{musixmad_setmaxoctlines}, 
\ref{musixmad_setmaxslurs}, and
\ref{musixmad_setmaxtrills}.


\subsection{musixper}\label{perc}

Provides special symbols intended for percussion parts. Included are a
\ital{drum clef}---comprising two vertical parallel lines---and notes with
various specially shaped heads. The note symbols that are available are as
follows:

 \begin{itemize}\setlength{\itemsep}{0ex}
 \item The \raise.5ex\hbox{\musixchar113}~{}~symbol which is obtained using the
\verb|\qu|, \verb|\qb|, \verb|\cu|, etc. macros preceeded with a
``\verb|dc|'' (think of {\it d}iagonal {\it c}ross).
Available are
\keyindex{dcqu},
\keyindex{dcql},
\keyindex{dcqb},
\keyindex{dczq},
\keyindex{dccu},
\keyindex{dcccu},
\keyindex{dccl} and
\keyindex{dcccl}.

 \item The \raise.5ex\hbox{\musixchar112}~{}~symbol which is obtained using the
\verb|\qu|, \verb|\qb|, \verb|\cu|, etc. macros preceeded with a
``\verb|dh|''
(think of {\it d}iagonal cross {\it h}alf open).
Available are
\keyindex{dhqu},
\keyindex{dhql},
\keyindex{dhqb},
\keyindex{dhzq},
\keyindex{dhcu},
\keyindex{dhccu},
\keyindex{dhcl} and
\keyindex{dhccl}.

 \item The \raise.5ex\hbox{\musixchar111}~{}~symbol which is obtained using the
\verb|\qu|, \verb|\qb|, \verb|\cu|, etc. macros preceeded with a
``\verb|do|''
(think of {\it d}iagonal cross {\it o}pen).
Available are
\keyindex{doqu},
\keyindex{doql},
\keyindex{doqb},
\keyindex{dozq},
\keyindex{docu},
\keyindex{doccu},
\keyindex{docl} and
\keyindex{doccl}.

 \item The \raise.5ex\hbox{\musixchar114}~{}~symbol which is obtained using the
\verb|\qu|, \verb|\qb|, \verb|\cu|, etc. macros preceeded by
``\verb|x|'' (e.g.\ for spoken text of songs).
Available are
\keyindex{xqu},
\keyindex{xql},
\keyindex{xqb},
\keyindex{xzq},
\keyindex{xcu},
\keyindex{xccu},
\keyindex{xcl} and
\keyindex{xccl}.

 \item The \raise.5ex\hbox{\musixchar115}~{}~symbol which is obtained using the
\verb|\qu|, \verb|\qb|, \verb|\cu|, etc. macros preceeded by
``\verb|ox|'' .
Available are
\keyindex{oxqu},
\keyindex{oxql},
\keyindex{oxqb},
\keyindex{oxzq},
\keyindex{oxcu},
\keyindex{oxccu},
\keyindex{oxcl} and
\keyindex{oxccl}.

 \item The \raise.5ex\hbox{\musixchar118}~{}~symbol which is obtained using the
\verb|\qu|, \verb|\qb|, \verb|\cu|, etc. macros preceeded by
``\verb|ro|'' (think of {\it r}h{\it o}mbus).
Available are
\keyindex{roqu},
\keyindex{roql},
\keyindex{roqb},
\keyindex{rozq},
\keyindex{rocu},
\keyindex{roccu},
\keyindex{rocl} and
\keyindex{roccl}.

 \item The \raise.5ex\hbox{\musixchar116}~{}~symbol which is obtained using the
\verb|\qu|, \verb|\qb|, \verb|\cu|, etc. macros preceeded by
``\verb|tg|'' (think of {\it t}rian{\it g}le).
Available are
\keyindex{tgqu},
\keyindex{tgql},
\keyindex{tgqb},
\keyindex{tgzq},
\keyindex{tgcu},
\keyindex{tgccu},
\keyindex{tgcl} and
\keyindex{tgccl}.

 \item The \raise.5ex\hbox to 1em{\musixchar117\hfil}~{}~symbol which is obtained
using the
\verb|\qu|, \verb|\qb|, \verb|\cu|, etc.~macros preceeded by
``\verb|k|'' .
Available are
\keyindex{kqu},
\keyindex{kql},
\keyindex{kqb},
\keyindex{kzq},
\keyindex{kcu},
\keyindex{kccu},
\keyindex{kcl} and
\keyindex{kccl}.
 \end{itemize}

The diamond shaped noteheads described in section \ref{diam} are also
available, because \verb|musixper.tex| inputs \verb|musixdia.tex|.

If any of the foregoing notes need to be dotted, you must use the explicit
dotting macros \verb|\pt|, \verb|\ppt|, or \verb|\pppt| as described in
section \ref{dots}.

Since the usage of these note symbols is
not standardized, it would be wise to include in the score a explanation
of which symbol corresponds to which specific percussion instrument.

A special \itxem{drum clef}---comprising two heavy vertical bars---can
be made to replace the normal clef for the $n$-th intrument by saying
\keyindex{setclefsymbol}\onen\keyindex{drumclef} .
To cause this to appear at the right vertical position, the instrument should
previously have been assigned a treble clef (or not explicitly assigned any
clef, thereby giving it a treble clef by default).

Percussion music might be written on a staff with either one or five lines.
If there are several different percussions instruments it may be useful to
use a five-line staff with a drum clef, and differentiate the instruments
by the type of the note heads and the apparent
pitch of the note on the staff. Here is an example of the
latter\footnote{provided by Agusti {\sc Mart\'in Domingo}}:

\medskip
\begin{music}
\generalmeter{\meterfrac44}
\setclefsymbol1\drumclef
\parindent0pt\startpiece
\leftrepeat
\Notes\zql f\rlap\qp\ibu0m0\xqb0{nn}\enotes
\Notes\kzq d\zql f\zq j\xqb0n\tbu0\xqb0n\enotes
\Notes\zql f\rlap\qp\ibu0m0\xqb0{nn}\enotes
\Notes\kzq d\zql f\zq j\xqb0n\tbu0\xqb0n\enotes
\bar
\Notes\zql f\rlap\qp\ibu0m0\kqb0{nn}\enotes
\Notes\xzq d\zql f\zq j\kqb0n\tbu0\kqb0n\enotes
\Notes\zql f\rlap\qp\ibu0m0\kqb0{nn}\enotes
\Notes\xzq d\zql f\zq j\kqb0n\tbu0\kqb0n\enotes
\bar
\Notes\zql f\rlap\qp\ibu0m0\oxqb0{nn}\enotes
\Notes\oxzq d\zql f\zq j\kqb0n\tbu0\oxqb0n\enotes
\Notes\zql f\rlap\qp\ibu0m0\oxqb0{nn}\enotes
\Notes\oxzq d\zql f\zq j\kqb0n\tbu0\oxqb0n\enotes
\setrightrepeat\endpiece
\end{music}
\noindent Its coding is
 \begin{quote}\begin{verbatim}
\begin{music}
\instrumentnumber{1}
\generalmeter{\meterfrac44}
\setclefsymbol1\drumclef
\parindent0pt\startpiece
\leftrepeat
\Notes\zql f\rlap\qp\ibu0m0\xqb0{nn}\enotes
\Notes\kzq d\zql f\zq j\xqb0n\tbu0\xqb0n\enotes
\Notes\zql f\rlap\qp\ibu0m0\xqb0{nn}\enotes
\Notes\kzq d\zql f\zq j\xqb0n\tbu0\xqb0n\enotes
\bar
\Notes\zql f\rlap\qp\ibu0m0\kqb0{nn}\enotes
\Notes\xzq d\zql f\zq j\kqb0n\tbu0\kqb0n\enotes
\Notes\zql f\rlap\qp\ibu0m0\kqb0{nn}\enotes
\Notes\xzq d\zql f\zq j\kqb0n\tbu0\kqb0n\enotes
\bar
\Notes\zql f\rlap\qp\ibu0m0\oxqb0{nn}\enotes
\Notes\oxzq d\zql f\zq j\kqb0n\tbu0\oxqb0n\enotes
\Notes\zql f\rlap\qp\ibu0m0\oxqb0{nn}\enotes
\Notes\oxzq d\zql f\zq j\kqb0n\tbu0\oxqb0n\enotes
\setrightrepeat\endpiece
\end{music}
 \end{verbatim}\end{quote}

Here is an example of a single-line percussion staff using
diamond-shaped note heads:

\begin{music}
\parindent 19mm
\instrumentnumber{3}
\setname1{keyboard} \setname2{drum} \setname3{monks}
\setlines2{1}
\setlines3{4}
\setinterinstrument1{-2\Interligne}% less vertical space above
\setinterinstrument2{-2\Interligne}% and below the percussion
\sepbarrules
\setsign1{-1} % one flat at keyboard
\generalmeter{\meterfrac24}
\setmeter3\empty
\setclef3\alto
\setclef1\bass
\setstaffs12 % 2 staffs at keyboard
\setclefsymbol3\gregorianCclef % gregorian C clef at instrument 3
\setclefsymbol2\drumclef       % cancel G clef at instrument 2
\startextract
\Notes\hu F|\zh c\hu h&\dnq4&\squ{acd}\enotes\bar
\NOtes\qu I|\zq N\qu d&\qp&\diapunc f\enotes
\NOtes\qu J|\zq a\qu e&\ynq4&\diapunc e\enotes\bar
\notes\hu G|\zh b\hu d&\dnq4&\zsqu d\rsqu g\squ{hgh}\enotes
\endextract
\end{music}
\noindent which is coded as follows:

\begin{verbatim}
\parindent 19mm
\instrumentnumber{3}
\setname1{keyboard} \setname2{drum} \setname3{monks}
\setlines2{1}
\setlines3{4}
\setinterinstrument1{-2\Interligne}% less vertical space above
\setinterinstrument2{-2\Interligne}% and below the percussion
\sepbarrules
\setsign1{-1} % one flat at keyboard
\generalmeter{\meterfrac24}
\setmeter3{\empty}
\setclef3{\alto}
\setclef1{\bass}
\setstaffs1{2} % 2 staffs at keyboard
\setclefsymbol3{\gregorianCclef} % gregorian C clef at instrument 3
\setclefsymbol2{\drumclef}       % cancel G clef at instrument 2
\startextract
\end{verbatim}

% (DAS) Sorry, guys, I couldn't figure out how to get the |'s to work in this
% verbatim.
%
\vskip-11pt
\def\Vert{{\tt\char'174}}
\noindent\verb|\Notes\hu F|\Vert\verb|\zh c\hu h&\dnq4&\squ{acd}\enotes\bar|\\
\verb|\NOtes\qu I|\Vert\verb|\zq N\qu d&\qp&\diapunc f\enotes|\\
\verb|\NOtes\qu J|\Vert\verb|\zq a\qu e&\ynq4&\diapunc f\enotes\bar|\\
\verb|\notes\hu G|\Vert\verb|\zh b\hu d&\dnq4&\zsqu d\rsqu g\squ{hgh}\enotes|\\
\verb|\endextract|

 \subsection{musixpoi}
 Adds definitions of less common singly and doubly dotted notes.
Available are
\keyindex{ccup},
\keyindex{zccup},
\keyindex{cclp},
\keyindex{zcclp},
\keyindex{ccupp},
\keyindex{zccupp},
\keyindex{cclpp},
\keyindex{zcclpp},
\keyindex{cccup},
\keyindex{zcccup},
\keyindex{ccclp},
\keyindex{zccclp},
\keyindex{cccupp},
\keyindex{zcccupp},
\keyindex{ccclpp},
\keyindex{zccclpp},
\keyindex{ccccup},
\keyindex{zccccup},
\keyindex{cccclp},
\keyindex{zcccclp},
\keyindex{ccccupp},
\keyindex{zccccupp},
\keyindex{cccclpp} and
\keyindex{zcccclpp}.

 %\check
 \subsection{musixps}\label{musixps}\index{musixps@{\tt musixps.tex}}
 Activates type K postscript slurs, ties,  and hairpins (see \ref{PostscriptSlurs}).

 \subsection{musixstr}\label{musixstr}\index{musixstr@{\tt musixstr.tex}}
 Provides bowing and other symbols for \itxem{string instruments}\footnote{provided
by Werner {\sc Icking}}. The symbol can be posted at the desired position using
 \verb|\zcharnote|\pitchp\verb|{|$command$\verb|}|. The available symbols and
their meanings are as follows:

{\input musixstr
\begin{quote}\begin{description}\setlength{\itemsep}{0ex}

 \item[\hbox to 1em{\AB}~: \keyindex{AB} or  \keyindex{downbow}] down-bow

 \item[\hbox to 1em{\AUF}~: \keyindex{AUF} or \keyindex{upbow}] up-bow

 \item[\hbox to 1em{\SP}~: \keyindex{SP}] at the top of bow

 \item[\hbox to 1em{\FR}~: \keyindex{FR}] at the nut of bow

 \item[\GB\ or \Gb~: \keyindex{GB} or  \keyindex{Gb}] whole bow

 \item[\UH\ or \Uh~: \keyindex{UH} or  \keyindex{Uh}] lower half of bow

 \item[\OH\ or \Oh~: \keyindex{OH} or  \keyindex{Oh}] upper half of bow

 \item[\MI\ or \Mi~: \keyindex{MI} or \keyindex{Mi}] middle of bow

 \item[\UD\ or \Ud~: \keyindex{UD} or  \keyindex{Ud}] lower third of bow

 \item[\OD\ or \Od~: \keyindex{OD} or  \keyindex{Od}] upper third of bow

 \item[\Pizz~: \keyindex{Pizz}] left hand pizzicato or trill

 \end{description}\end{quote}
 }
 %\check
 \subsection{musixsty}\index{musixsty@{\tt musixsty.tex}}

 Provides certain text-handling facilities for titles, footnotes, and other
items not related to lyrics. It should not be used with \LaTeX. It includes
 \begin{itemize}\setlength{\itemsep}{0ex}
 \item definitions for commonly used fonts such as \verb|\tenrm|,
\verb|\eightrm|, etc.;
 \item definitions of \keyindex{hsize}, \keyindex{vsize},
\keyindex{hoffset}, \keyindex{voffset} suitable for A4 paper; those using
other sizes may wish to modify it once and for all;
 \item a set of text size commands:

 \begin{description}\setlength{\itemsep}{0ex}
  \item[\keyindex{eightpoint}] which sets the usual \keyindex{rm},
\keyindex{bf}, \keyindex{sl}, \keyindex{it} commands to 8 point font size;
  \item[\keyindex{tenpoint}] which sets the usual \keyindex{rm},
\keyindex{bf}, \keyindex{sl}, \keyindex{it} commands to 10 point font size;
  \item[\keyindex{twlpoint}] to get 12 point font size;
  \item[\keyindex{frtpoint}] to get 14.4 point font size;
  \item[\keyindex{svtpoint}] to get 17.28 point font size;
  \item[\keyindex{twtypoint}] to get 20.74 point font size;
  \item[\keyindex{twfvpoint}] to get 24.88 point font size;
 \end{description}
 \item commands for creating titles:
  \begin{itemize}\setlength{\itemsep}{0ex} \item \keyindex{author} or
\keyindex{fullauthor} to be put at the right of the first page, below the
title of the piece; the calling sequence is, for example

  \verb|     \author{Daniel TAUPIN\\organiste \`a Gif-sur-Yvette}|

  \noindent where the \verb|\\| causes the author's name to be displayed on
two lines;

  \item \keyindex{shortauthor} to be put at the bottom of each page;
  \item \keyindex{fulltitle} which is the main title of the piece;
  \item \keyindex{subtitle} is displayed below the main title of the piece;
  \item \keyindex{shorttitle} or \keyindex{title}
  which is the title repeated at the bottom of each page;
  \item \keyindex{othermention} which is displayed on the left of the page,
vertically aligned with author's name. It may contain \verb|\\| to display it on
several lines;
  \item \keyindex{maketitle}  which displays all the previous stuff;
  \end{itemize}

 \item  commands for making \itxem{footnotes}:
  \begin{itemize}\setlength{\itemsep}{0ex}
   \item The normal Plain-\TeX\ \keyindex{footnote} command, which has two
arguments---not just one as in \LaTeX\protect\index{LATEX@\LaTeX}---namely
the label of the footnote, which can be any sequence of characters, and
the text of the footnote. This command does not work inside
boxes, so it cannot be issued within music;

 \item The \keyindex{Footnote} command, which counts the footnotes and uses a
number as the label of the footnote (equivalent to \LaTeX's \verb|\footnote|
command). The same restriction as with \verb|\footnote| applies concerning
its use within the music coding;

 \item The \keyindex{vfootnote} command, taken from the Plain-\TeX, which
places a footnote at the bottom of the current page, but does not put
the footnote label at the place the command is entered in the main text. This
also may not be used within music, but if a footnote is needed whose reference
lies inside the music, it can be entered in two steps:
 \begin{enumerate}
  \item manually insert the reference inside the music, using e.g. \verb|zcharnote|;
  \item post the footnote itself with \verb|\vfootnote| outside the music,
either before \keyindex{startpiece} or between \keyindex{stoppiece} and
\keyindex{contpiece} or equivalent commands.
 \end{enumerate}

  \end{itemize}

 \end{itemize}

 \subsection{musixtmr}\index{musixtmr@{\tt musixtmr.tex}}
 Replaces the standard \texttt{musixtex} fonts by the Times series fonts,
 as default (see \ref{musixtmr}).

 \subsection{musixtri}\index{musixtri@{\tt musixtri.tex}}
Provides triply dotted note symbols.
Available are:
\keyindex{lpppt},
\keyindex{whppp},
\keyindex{zwppp},
\keyindex{huppp},
\keyindex{hlppp},
\keyindex{zhppp},
\keyindex{zhuppp},
\keyindex{zhlppp},
\keyindex{quppp},
\keyindex{qlppp},
\keyindex{zquppp},
\keyindex{zqlppp},
\keyindex{zqppp},
\keyindex{cuppp},
\keyindex{zcuppp},
\keyindex{clppp},
\keyindex{zclppp},
\keyindex{qbppp} and
\keyindex{zqbppp}.

 \subsection{tuplet}\index{tuplet@{\tt tuplet.tex}}
 Causes the figure in xtuplets to be printed within a small gap in the bracket
(see \ref{tuplet}).

% DAS Discussed elsewhere
%\section{The \TeX-music Mailing List}
%
%If you decide to learn to use \musixtex, you are likely to wish that there were
%someone to answer your questions. That is exactly what you'll find on the \TeX-music
%mailing list: experts from all over the world with lots of experience with
%\musixtex{} and related software, who are glad to answer questions and help solve
%problems. You can find out more about the list and sign up at
%\href{http://icking-music-archive.org/mailman/listinfo/tex-music}
%{\underline{\tt http://icking-music-archive.org/mailman/listinfo/tex-music}}.


\chapter{Acquiring, Installing, and Using \musixtex}\label{installation}

This chapter will assume that \TeX\
is installed. Most \unix\ or Linux systems come with \TeX\ already installed.
If using MS Windows, you'll need to have manually installed some version of
\TeX. MiK\TeX\ is the most common variant by far and therefore will be assumed. If
you need to install MiK\TeX, you can get the software from
\href{http://www.miktex.org/}{\underline{\texttt{http://www.miktex.org}}}
or from CTAN. The Werner Icking Archive contains excellent
\href{http://icking-music-archive.org/software/musixtex/musixwinstall.pdf}%
{\underline{instructions for installing MiK\TeX}}%
\footnote{Thanks to Eva {\sc Jaksch}}.

\section{Where to get \musixtex: the Werner Icking Archive}\label{getstuff}

As already mentioned, the home base for all matters related to \musixtex\ is
the Werner Icking Music Archive, at
\href{http://icking-music-archive.org}{\underline{http://icking-music-archive.org}}.
The most up-to-date versions of \musixtex\ and friends are located in the
\href{http://icking-music-archive.org/software/indexmt6.html}{\underline{software}}
section of the archive. The stable distribution of \musixtex\ will be contained in
\href{http://icking-music-archive.org/software/musixtex/musixtex.zip}
{\underline{\verb|musixtex.zip|}}. This will serve any operating system, but 
has some Windows-specific installation instructions and batch scripts that simplify 
installing and running the software.
There is a somewhat differently organized distribution on the CTAN servers at\\ 
\href{http://www.ctan.org/tex-archive/macros/musixtex/}
{\underline{http://www.ctan.org/tex-archive/macros/musixtex/}}
with FTP link\\
\href{ftp://ftp.ctan.org/tex-archive/macros/musixtex/}
{\underline{ftp://ftp.ctan.org/tex-archive/macros/musixtex/}}.\\
This one is "TDS compliant," which means that it adheres to a recently announced
standard directory structure for \TeX\ systems, and does not contain any 
Windows batch scripts. 

\subsection{End of line char in text files}
The files in the Werner Icking Archive have been contributed by such a wide
range of authors that there is no common standard for the line endings in text files.
But the key text files serve as input to programs such as \TeX\ and \MF\
which are insensitive to this. If you use a PC and run into problems using any of
these text files, you might be able to solve them with the program 
\verb|flip.exe| which comes with the \musixtex\ distribution. Run it from a 
command window by typing \verb|flip -m| $<$filename$>$, and it will 
convert line endings from the \unix to MS-DOS convention.

\subsection{Pre-installed \musixtex}
Many versions of \TeX\ include \musixtex\ as a package. The included version is likely
to be out of date and probably should not be used. When an archived version is
downloaded and properly installed, it will supersede the pre-installed version. 

\section{Installing \musixtex}
The most useful \musixtex\ uses Postscript and produces PDF files. 
The installation is very system dependent and therefore beyond the scope of this manual.
You may refer to the installation manual or to the HOWTO files in the archive for either
\href{http://icking-music-archive/software/musixtex/musixwinstall.pdf}
{\underline{Windows}}\ or
\href{http://icking-music-archive/software/musixtex/musixtex-for-unix.html}
{\underline{\unix}}.
Whatever course you choose, there are really just a few
fundamental tasks you must accomplish:
\begin{itemize}\setlength{\itemsep}{0ex}
  \item \musixtex\ comprises some \TeX\ source files (\verb|.tex|, \verb|.sty|),
font files, and the executable \verb|musixflx|. The \TeX\ sources and font
files must be placed in locations where \TeX\ can find them;
  \item \TeX\ must be advised of the locations of those files;
  \item If you are installing any postscript fonts, you'll have to add a map
file somewhere in the \TeX\ file structure, and locate and modify a configuration
file such as \verb|config.ps| to record the addition of the map file.
  \item \verb|musixflx| must be placed in a folder that is in your search path,
so it can be executed from your working directory;
  \item As stated before, to compile a score, you must run in sequence \verb|tex|, \verb|musixflx|,
and \verb|tex| again. This will produce a \verb|.dvi| file;
  \item To view the result, convert the file to
postscript with \textbf{dvips} and then view it with a postscript viewer such
as \textbf{GSview} and \textbf{ghostscript}, or convert the \verb|.dvi|
directly to a PDF file with \textbf{pdftex} and then view it with any PDF viewer
such as \textbf{Adobe Reader}. You also can use a DVI viewer, but if you are using type K postscript slurs, note that DVI viewers will not display the slurs.
\end{itemize}


\section{Running \musixtex}

In most systems the commands run in a shell. Unix people are familiar with it.
Also in MS Windows systems, there are a large number of editor/shell programs. Although not recommended, you always can use the DOS
\verb|cmd| shell (\verb|cmd.exe|).


In the examples, a file \verb|mymp|
(\textbf{my} \textbf{m}aster\textbf{p}iece) is used, for demonstrating
the batch commands.



\subsection{Batch commands for making DVI and postscript (MS Windows)}

This commands are available in the \musixtex\ distribution. They produce
a DVI file that can be displayed by a viewer
(normally a part of the \TeX\ installation), or/and a Postscript file (normally with GSview).

After running \TeX\ on a music file, \verb|musixflx| must run. The program \verb|musixflx| calculates the number of bars on a system (music line), in order to have a nice layout. Then, \TeX\ must run again.

When the paths to the binaries and the batch files are set properly, then the following commands can be used as is.

\subsubsection{Using \TeX\ }
The \ixem{musixtex.bat} command fulfills the sequence \texttt{\TeX\ $\Rightarrow$ musixflx $\Rightarrow$ \TeX}~.

\begin{quote}\begin{verbatim}
rem musixtex.bat
if not exist %1.tex goto end
if exist %1.mx1 del %1.mx1
if exist %1.mx2 del %1.mx2
tex %1
if errorlevel 2 goto end
musixflx %1
if errorlevel 1 goto end
tex %1
:end
\end{verbatim}\end{quote}
The next command (\verb|mupsgs.bat|) processes \verb|mymp.tex| and makes a Postscript file  and dispays the file.
\begin{quote}\begin{verbatim}
rem mupsgs.bat
call musixtex.bat %1
DVIPS.EXE -e0 -t a4 -P pdf -G0 %1.dvi
gsview32.exe %1.ps
\end{verbatim}\end{quote}
The \verb|dvips| output looks a little bit nicer using the options
\verb|-e0| (character drift value~=~0) and \verb|-G0| (shift low chars
to higher position).
\verb|-t a4| is for a4 paper (\verb|-t letter| for letter format).
If the output is to be converted in Adobe Acrobat pdf format, the option \verb|-P pdf|
(load \verb|config.pdf|) is needed.

If you want to make a PDF output, you may add the line (after \verb+dvips+):

\noindent{\small
\verb+gswin32c -dBATCH -dNOPAUSE -sDEVICE=pdfwrite -r600 -sOutputFile=%1.pdf -c save pop -f %1.ps+
}

\subsubsection{Using \LaTeX\ }
Replace ``\verb+tex+'' by ``\verb+latex+'' in the file \verb+musixtex.bat+~.


\subsection{Running \musixtex\ without a shell}
You also may run the commands without a shell.
To forward the argument in the batch file,
use an other batch file which stays in the same directory as your
masterpiece \verb|mymp.tex| and has the same name: \verb|mymp.bat|~.
\verb|mymp.bat| consists of 1 line: \verb|musixtex.bat %~n0|~.

\noindent Thus \verb|mymp.bat| passes his own name (\verb|mymp|) via \verb|%~n0|
to the \verb|musixtex.bat| command.

Here are four examples of batch files, dependent of the method you are using.
\endcatcodesmusic
%\begin{quote}\begin{verbatim}
\begin{center}
\begin{tabular}{|l|l|l|}
\multicolumn{3}{c}{Choose only one batch file}\\[.4ex]\hline
Batch file for&making DVI&making postscript\\\hline
name &\verb+rem mymp.bat+&\verb+rem mymp.bat+\\
contents&\verb+musixtex.bat %~n0+&\verb+mupsgs.bat %~n0+\\\hline
\end{tabular}
%\end{verbatim}\end{quote}
\end{center}

\subsection{Running \musixtex\ without giving a file name}
If you want to process i.e.\ all \texttt{.tex} files in a directory, then you could make a batch file
\texttt{texall.bat} with the following commands:

\begin{quote}\begin{verbatim}
rem texall
for /F %%I in ('dir /B *.tex') do call mupsgs.bat %%~nI
\end{verbatim}\end{quote}

Of course, this command also works if there is only one file in a directory, which means you
may use \texttt{texall.bat} for a single file, without giving an argument.


\chapter{\musixtex\ examples}

The file \verb|musixdoc.tex|, the source for this manual, contains many useful
examples. In the manual, many examples are accompanied by a display of the code
that produced them, while for a few only an image of the extract is included and you'll
have to look in \verb|musixdoc.tex| to see the coding.

Other useful examples cannot be embedded in the source, either because they are
meant to be in \TeX, not \LaTeX, or because they are simply to large. For
these the source files also are provided separately.

When compiling or viewing any of the examples, you should keep in mind that
most DVI previewers and laser printers have
their origin one inch below and one inch to the right of the upper right corner
of the paper, while the musical examples have their upper left
corner just one centimeter to the right and below the top left corner of the page.
Therefore, special parameters may have to be given to the DVI transcription
programs unless special \keyindex{hoffset} and \keyindex{voffset} \TeX\
commands have been included within the \TeX\ source.

\section{Small examples}

\begin{itemize}\setlength{\itemsep}{0ex}

\item{\tt ossiaexa.tex}~: This is a stand-alone example of the use of ossia,
provided by Olivier Vogel (section \ref{ossia} on page \pageref{ossia}).

\item{\tt 8bitchar.tex}~: Using 8bit characters. Here, the European
character set is used. See section \ref{8bit} on page \pageref{8bit} for other sets.

\end{itemize}

\section{Full examples}
All of these examples and many others are available in
\href{http://icking-music-archive.org/software/musixtex/musixexa.zip}
{\underline{\texttt{musixexa.zip}}} or
\href{http://icking-music-archive.org/software/musixtex/musixexa-T112.tar.gz}
{\underline{\texttt{musixexa-T112.tar.gz}}}. Some of them require the
presence of \verb|musixcpt.tex| which makes examples created in
Music\TeX\ compatible with \musixtex. Here we only mention a few of
special interest.

\subsection{Examples mentioned in the manual}

\begin{itemize}\setlength{\itemsep}{0ex}
 \item{\tt avemaria.tex}~: the ``M\'editation'' (alias ``Ave Maria'') by
Charles {\sc Gounod} for organ and violin or voice.\index{Gounod, C.@{\sc Gounod, C.}}
To run this five-page example you'll also need \texttt{avemariax.tex}.
It demonstrates the use of separated bar rules (section \ref{avemaria2})
and the use of staves of different sizes (section \ref{avemaria}).
Also, an additional instrument is created for lyrics. This was a common
practice before the \texttt{musixlyr} package was created by Rainer Dunker.

 \item{\tt glorias.tex}~: a local melody for the French version of
\ital{Gloria in excelsis Deo}, a three-page piece demonstrating the use of the hardlyrics
commands (section \ref{glorias}). {\tt gloriab.tex} is the same piece, but with organ accompaniment.

\end{itemize}

\subsection{Other examples, provided by the authors of \musixtex }

\begin{itemize}\setlength{\itemsep}{0ex}

 \item{\tt traeumer.tex}~: the famous ``Tr\"aumerei'' by
Robert {\sc Schumann}\index{Schumann, R.@{\sc Schumann, R.}} for piano, in genuine
\musixtex\ but with some
additions to perform ascending bitmapped \itxem{crescendos}.
There are also S-shaped slurs between 2 staves.

 \item{\tt parnasum.tex}~: the first page of ``Doctor gradus ad
Parnassum'' by Claude {\sc Debussy}\index{Debussy,
C.@{\sc Debussy, C.}} for piano.
It contains a rather complex example of a new command \verb+\Special+
to create staff-jumping doubly beamed notes.
\end{itemize}

\subsection{Additional documentation}

\begin{itemize}\setlength{\itemsep}{0ex}

 \item{\tt sottieng}~:  Notation mistakes, provided by Jean-Pierre Coulon.

\end{itemize}

\section{Compiling musixdoc.tex}
\subsection{Command sequence}

Before compiling or recompiling \verb|musixdoc.tex|, you should remove all the auxilliary 
files {\tt musixdoc.[mx1|\allowbreak mx2|\allowbreak aux|\allowbreak toc|\allowbreak ind|\allowbreak idx|\allowbreak ilg|\allowbreak out]} 
if they are present.

Here is an example of a command sequence:
\begin{quote}\begin{verbatim}
latex musixdoc
musixflx musixdoc
latex musixdoc
makeindex -l musixdoc
latex musixdoc
latex musixdoc
dvips -e0 -t a4 -P pdf -G0 musixdoc
\end{verbatim}\end{quote}

The initial three steps \verb|latex|\allowbreak$\to$\allowbreak\verb|musixflx|\allowbreak$\to$\allowbreak\verb|latex| build up the basic appearance of the document including many musical examples.
\verb|makeindex| produces the database for the index.  
After that, \verb|latex| must be run at least twice to complete cross referencing.  
Finally, \verb|dvips| converts the \verb|.dvi| file into \verb|.ps| (making PostScript type K slurs
visible).
To produce \verb|musixdoc.pdf|, open \verb|musixdoc.ps| in \texttt{GSview}. Go to
\verb+File|Convert+, select \verb|pdfwrite| at 600 dpi resolution, and click
\verb|OK|.

If you use an older version of \LaTeX\ which doesn't automatically 
invoke $\varepsilon$\hbox{-}\nobreak\TeX, you will encounter the 
error ``{\tt !~No room for new \string\count}''. 
This is because \verb|musixdoc.tex| invokes \verb|musixtex.tex|, which together with \LaTeX\ 
requires more storage registers than available in \TeX.
You may be able get around this by using the command 
\verb|elatex| instead of \verb|latex|;
however, it is strongly recommended to upgrade your \TeX\ system to a more recent 
version in which \verb|latex| automatically invokes $\varepsilon$\hbox{-}\nobreak\TeX.   

\subsection{Using batch files}
There are
many other files that are required to compile this document but all of them
are included in \verb|musixtex.zip|, and will be placed in the proper locations
if you have followed the installation instructions for \musixtex.

The two following batch files reside in the same dir as \texttt{musixdoc.tex}.

This batch file makes a DVI, Postscript and PDF file, then displays the Postscript file.
The full path names are shown as a comment.
{\small
\begin{quote}\begin{verbatim}
rem filename: lapdfview
del %1.mx1
del %1.mx2
del %1.aux
del %1.toc
del %1.ind
del %1.idx
del %1.ilg
del %1.out
LATEX.EXE %1 %2.tex
if errorlevel 1 goto end
rem C:\musixtex\system\musixflx\win32\musixflx %1
musixflx %1
if errorlevel 1 goto end
LATEX.EXE %1 %2.tex
if errorlevel 1 goto end
makeindex -l  %1
if errorlevel 1 goto end
LATEX.EXE %1 %2.tex
if errorlevel 1 goto end
LATEX.EXE %1 %2.tex
rem C:\"Program Files\MiKTeX 2.7"\miktex\bin\DVIPS.EXE -e0 -t a4 -P pdf -G0 %1.dvi
DVIPS.EXE -e0 -t a4 -P pdf -G0 %1.dvi
rem C:\"program files"\gs\gs8.63\bin\gswin32c -dBATCH ^
rem -dNOPAUSE -sDEVICE=pdfwrite -r600 \
rem -sOutputFile=%1.pdf -c save pop -f %1.ps
gswin32c -dBATCH -dNOPAUSE -sDEVICE=pdfwrite -r600 ^
-sOutputFile=%1.pdf -c save pop -f %1.ps
pause
rem C:\"program files"\Ghostgum\gsview\gsview32.exe %1.ps
gsview32.exe %1.ps
rem pause
:end
\end{verbatim}\end{quote}
}

This batch file searches the folder for tex files and calls \texttt{lapdfview.bat}
{\small
\begin{quote}\begin{verbatim}
rem latexall.bat
rem proces all files with extension tex in this directory
for /F %%I in ('dir /B *.tex') do call lapdfview.bat %%~nI
pause
\end{verbatim}\end{quote}
}

Thus, to produce \texttt{musixdoc.pdf}, double click on \texttt{latexall.bat}.

\clearpage
{\parindent0pt
\hoffset-15.4mm
\voffset-42mm
\hsize190mm
\vsize285mm
\parskip0pt plus 2pt
% not fine, not clean, but works
% !!! don't look at this !!!
%>>>
\makeatletter
\textheight\vsize
\@colht\vsize
\@colroom\vsize
\makeatother
%<<<
\let\MYTEMP\addcontentsline
\def\addcontentsline#1#2#3{}
\chapter{}
\let\addcontentsline\MYTEMP
\addcontentsline{toc}{chapter}{%
  \protect\numberline\thechapter{Summary of denotations}}
\vglue-20mm
\centerline{\bigfont Pitches}
\medskip
\tentt
\def\bs{\tentt\char92}
\setclef1\bass
\afterruleskip2pt
\startpiece
%
% Pitches
%
\addspace\afterruleskip
\notes\wh{`ABCDEFG}\wh{!ABCDEFG}\wh{HIJKLMNabcde}\off{-26\noteskip}%
  \zchar{14}{\llap`A}\sk
  \zchar{14}{\llap`B}\sk
  \zchar{14}{\llap`C}\sk
  \zchar{14}{\llap`D}\sk
  \zchar{14}{\llap`E}\sk
  \zchar{14}{\llap`F}\sk
  \zchar{14}{\llap`G}\sk
  \zchar{14}A\sk
  \zchar{14}B\sk
  \zchar{14}C\sk
  \zchar{14}D\sk
  \zchar{14}E\sk
  \zchar{14}F\sk
  \zchar{14}G\sk
  \zchar{-5}{\llap'A}\zchar{14}H\sk
  \zchar{-5}{\llap'B}\zchar{14}I\sk
  \zchar{-5}{\llap'C}\zchar{14}J\sk
  \zchar{-5}{\llap'D}\zchar{14}K\sk
  \zchar{-5}{\llap'E}\zchar{14}L\sk
  \zchar{-5}{\llap'F}\zchar{14}M\sk
  \zchar{-5}{\llap'G}\zchar{14}N\sk
  \zchar{14}a\sk
  \zchar{14}b\sk
  \zchar{14}c\sk
  \zchar{14}d\sk
  \zchar{14}e\sk\en
\stoppiece\vskip-3\Interligne\setclef1\treble\contpiece
\notes\wh{abcdefg}\wh{hijklmno}\wh{pqrstuvwxyz}\off{-26\noteskip}%
  \zchar{-8}a\sk
  \zchar{-8}b\sk
  \zchar{-8}c\sk
  \zchar{-8}d\sk
  \zchar{-8}e\sk
  \zchar{-8}f\sk
  \zchar{-8}g\sk
  \zchar{-8}h\zchar{-4}{\llap'a}\sk
  \zchar{-8}i\zchar{-4}{\llap'b}\sk
  \zchar{-8}j\zchar{-4}{\llap'c}\sk
  \zchar{-8}k\zchar{-4}{\llap'd}\sk
  \zchar{-8}l\zchar{-4}{\llap'e}\sk
  \zchar{-8}m\zchar{-4}{\llap'f}\sk
  \zchar{-8}n\zchar{-4}{\llap'g}\sk
  \zchar{-8}o\zchar{-4}{\llap{'\kern-2pt'}a}\sk
  \zchar{-8}p\zchar{-4}{\llap{'\kern-2pt'}b}\sk
  \zchar{-8}q\zchar{-4}{\llap{'\kern-2pt'}c}\sk
  \zchar{-8}r\zchar{-4}{\llap{'\kern-2pt'}d}\sk
  \zchar{-8}s\zchar{-4}{\llap{'\kern-2pt'}e}\sk
  \zchar{-8}t\zchar{-4}{\llap{'\kern-2pt'}f}\sk
  \zchar{-8}u\zchar{-4}{\llap{'\kern-2pt'}g}\sk
  \zchar{-8}v\zchar{-4}{\llap{'\kern-2pt'\kern-2pt'}a}\sk
  \zchar{-8}w\zchar{-4}{\llap{'\kern-2pt'\kern-2pt'}b}\sk
  \zchar{-8}x\zchar{-4}{\llap{'\kern-2pt'\kern-2pt'}c}\sk
  \zchar{-8}y\zchar{-4}{\llap{'\kern-2pt'\kern-2pt'}d}\sk
  \zchar{-8}z\zchar{-4}{\llap{'\kern-2pt'\kern-2pt'}e}\sk\en
\stoppiece
\medskip
\centerline{\bigfont Notes, Accidentals, Accents, Clefs and Rests}
\medskip
%
% Notes& Accidentals
%
\contpiece
\Notes
  \zchar{-4}{Accidentals:}%
  \zchar9{\bs maxima}\maxima i\relax
  \zchar{12}{\bs longa}\longa i%
  \zchar9{\bs breve}\breve i%
  \zchar{12}{\bs wq}\wq i%
  \zchar{12}{\bs wqq}\wqq i%
  \zchar{12}{\bs wh}\wh i%
  \zchar{12}{\bs hu}\zchar{-4}>\hu{>f}%
  \zchar{12}{\bs hl}\loff{\zchar{-4}{\bs cdsh}}\cdsh l\hl l%
  \zchar{12}{\bs qu}\zchar{-4}\^\qu{^f}%
  \zchar{12}{\bs ql}\loff{\zchar{-4}{\bs csh}}\csh l\ql l%
  \zchar{12}{\bs cu}\zchar{-4}=\cu{=f}%
  \zchar{12}{\bs cl}\loff{\zchar{-4}{\bs cna}}\cna l\cl l%
  \zchar{12}{\bs ccu}\zchar{-4}\_\ccu{_f}%
  \zchar{12}{\bs ccl}\loff{\zchar{-4}{\bs cfl}}\cfl l\ccl l%
  \zchar{12}{\bs cccu}\zchar{-4}<\cccu{<f}%
  \zchar{12}{\bs cccl}\loff{\zchar{-4}{\bs cdfl}}\cdfl l\cccl l%
\multnoteskip{1.2}%
  \zchar{12}{\bs ccccu}\ccccu f%
  \zchar{12}{\bs ccccl}\ccccl l%
\multnoteskip{.8}%
  \zchar{12}{\bs grcu}\grcu f%
  \zchar{12}{\bs grcl}\grcl l\en
\stoppiece
%
% various note heads
%
\contpiece
\Notes
\multnoteskip{1.1}%
  \zchar{10}{\bs dqu$^{123}$}%
    \zchar{-3}{\eightrm 1 musixdia.tex~~2 musixper.tex~~3 musixgre.tex%
                        ~~4 musixlit.tex~~5 musixext.tex}\dqu g%
  \zchar{10}{\bs yqu$^{123}$}\yqu g%
\multnoteskip{.9}%
  \zchar{10}{\bs dcqu$^2$}\dcqu g%
  \zchar{10}{\bs dhqu$^2$}\dhqu g%
  \zchar{10}{\bs doqu$^2$}\doqu g%
  \zchar{10}{\bs xqu$^2$}\xqu g%
  \zchar{10}{\bs oxqu$^2$}\oxqu g%
  \zchar{10}{\bs roqu$^2$}\roqu g%
  \zchar{10}{\bs tgqu$^2$}\tgqu f%
  \zchar{10}{\bs kqu$^2$}\kqu f%
  \zchar{10}{\bs squ$^3$}\squ g%
  \zchar{10}{\bs lsqu$^3$}\lsqu j%
  \zchar{10}{\bs rsqu$^3$}\rsqu k%
  \zchar{10}{\bs cqu$^4$}\cqu g%
  \zchar{10}{\bs cql$^4$}\cql k%
  \zchar{10}{\bs chu$^4$}\chu g%
  \zchar{10}{\bs chl$^4$}\chl k\en
%
% Pointed Notes and Accents
%
\stoppiece
\medskip
\contpiece
\Notes
  \zchar{-5}{\eightrm Accent on beam with prefix {\tt b} and beam reference
number instead of the pitch}%
  \zchar{13}{\bs lpz}\lpz f\qu f%
  \zchar{13}{\bs upz}\upz l\ql l%
  \zchar{13}{\bs lsf}\lsf f\qu f%
  \zchar{13}{\bs usf}\usf l\ql l%
  \zchar{13}{\bs lst}\lst f\qu f%
  \zchar{13}{\bs ust}\ust l\ql l%
  \zchar{13}{\bs lppz}\lppz f\qu f%
  \zchar{13}{\bs uppz}\uppz l\ql l%
  \zchar{13}{\bs lsfz}\lsfz f\qu f%
  \zchar{13}{\bs usfz}\usfz l\ql l%
  \zchar{13}{\bs lpzst}\lpzst f\qu f%
  \zchar{13}{\bs upzst}\upzst l\ql l%
  \zchar{16}{\bs downbow}\zchar9\downbow\ql l%
  \zchar{13}{\bs upbow}\zchar9\upbow\ql l%
  \zchar{16}{\bs flageolett}\flageolett l\ql l%
  \zchar{13}{\bs whp}\whp i%
  \zchar{13}{\bs qupp}\qupp h\en
\stoppiece
\medskip
%
% clefs
%
\setclefsymbol1\empty
%\setlines 11\relax
\nostartrule\contpiece
\NOtes
  \zchar{13}{\bs trebleclef}\zchar0\trebleclef\sk
  \zchar9{\bs altoclef}\zchar0\altoclef\sk
  \zchar{13}{\bs bassclef}\zchar0\bassclef\sk
  \zchar9{\bs smalltrebleclef}\zchar0\smalltrebleclef\sk
  \zchar{13}{\bs smallaltoclef}\zchar0\smallaltoclef\sk
  \zchar9{\bs smallbassclef}\zchar0\smallbassclef\sk
  \zchar{13}{\bs drumclef$^2$}\zchar0\drumclef\sk
  \zchar9{\bs gregorianCclef$^3$}\zchar0\gregorianCclef\sk
  \zchar{13}{\bs gregorianFclef$^3$}\zchar0\gregorianFclef\sk
  \zchar9{\bs oldGclef$^4$}\zchar0\oldGclef\sk\en
\zstoppiece\smallskip
%
% Rests
%
\setlines 15\relax
\resetclefsymbols\startrule\contpiece
\NOtes
\multnoteskip{.5}%
  \zchar{13}{\bs qqs}\qqs
  \zchar{13}{\bs hs}\hs
  \zchar{13}{\bs qs}\qs
  \zchar{13}{\bs ds}\ds
  \zchar{13}{\bs qp}\qp
\multnoteskip2%
  \zchar{13}{\bs hpause}\hpause
  \zchar{13}{\bs hpausep}\hpausep
  \zchar{9}{\bs lifthpause}\roff{\lifthpause5}\sk
  \zchar{13}{\bs pause}\pause
  \zchar{13}{\bs pausep}\pausep
  \zchar{9}{\bs liftpause}\roff{\liftpause4}\sk
  \zchar{13}{\bs PAuse}\PAuse
  \zchar{13}{\bs PAUSe}\PAUSe
  \zchar{13}{\bs Hpause$^4$}\Hpause i{.8}\sk\en
\stoppiece\vskip-2\Interligne
%
\centerline{\bigfont Other Symbols}
\medskip
%
% more Symbols
%
\contpiece
\NOtes
  \zchar{14}{\bs Trille}\zchar{-4}{\bs allabreve}\Trille n1%
    \zchar0\allabreve\sk
  \zchar{14}{\bs trille}\zchar{-4}{\bs meterC}\trille n1\zchar0\meterC\sk
  \zchar{14}{\bs shake}\zchar{-4}{\bs reverseC}\shake n\zchar0\reverseC\sk
  \zchar{14}{\bs Shake}\zchar{-4}{\bs reverseallabreve}\Shake n%
    \zchar0\reverseallabreve\sk
  \zchar{14}{\bs mordent}\mordent n\sk
  \zchar{14}{\bs Mordent}\Mordent n\zchar{-4}{\bs meterplus}%
    \zchar0{\meterfrac{3\meterplus2\meterplus3}8}\sk
\multnoteskip{.66}%
  \zchar{14}{\bs turn}\turn n\sk
\multnoteskip{1.25}%
  \zchar{14}{\bs backturn}\backturn n\zchar{-4}{\bs duevolte}\duevolte\sk
  \zchar{14}{\bs Shakel}\Shakel n\sk
  \zchar{14}{\bs Shakesw}\Shakesw n\zchar{-4}{\bs l[r]par}%
    \lpar f\rpar f{\stemlength4\qu f}%
  \zchar{14}{\bs Shakene}\Shakene n\sk
  \zchar{14}{\bs Shakenw}\Shakenw n\sk\en
\stoppiece
\medskip
%
% again
%
\contpiece
\NOtes\zchar9{\bs metron}\zchar{13}{\metron\qu{99}}\sk\en
\setvoltabox1\bar
\NOtes\loffset2{\zchar{17}{\bs setvoltabox}}\en
\setvolta2\setendvolta\rightrepeat
\NOtes\loff{\zchar{17}{\bs setvolta}}\en
\doublebar
\notes\zchar{17}{\kern-\afterruleskip\bs coda}\coda n\en\bar
\notes\zchar{17}{\kern-\afterruleskip\bs Coda}\Coda n\en
\NOtes
  \zchar{17}{\bs segno}\segno n\sk
  \zchar{17}{\kern-2\afterruleskip\bs Segno}\Segno\hsk
  \zchar{17}{\bs caesura}\sk\caesura
  \zchar{17}{\bs cbreath}\cbreath\sk
\multnoteskip{.66}%
  \zchar{17}{\bs PED}\PED\sk
  \zchar{17}{\bs sPED}\sPED\sk
  \zchar{17}{\bs DEP}\DEP\sk
  \zchar{17}{\bs sDEP}\sDEP\sk\en
\stoppiece
\medskip
%
% and again ...
%
\contpiece
\Notes
  \zchar{17}{\bs fermataup}\fermataup l%
    \zchar{-8}{\bs fermatadown}\fermatadown f{\stemlength3\ql i}%
  \zchar{13}{\bs Fermataup}\Fermataup l%
    \zchar{-4}{\kern8pt\bs Fermatadown}\Fermatadown f\wh i%
  \loff{\zchar{17}{\bs arpeggio d5}}\arpeggio d5\sk
\multnoteskip{.66}%
  \loff{\zchar{13}{\bs bracket}}\bracket e{10}\zq n\ql e\en
\notes\stemlength3%
  \zchar{17}{\bs uptrio}\uptrio{11}16\ql l\ql n\bsk\bsk
  \zchar{-8}{\bs downtrio}\downtrio{-5}17\qu c\qu e\en
\Notes
  \zchar{17}{\bs octfinup}\octfinup n{.8}%
    \zchar{-4}{\bs octfindown}\octfindown K{.8}\sk\hsk
  \zchar{17}{\bs slide$^5$}%
    \slide i58\slide i57\slide i56\slide i55\slide i54%
    \slide i53\slide i52\slide i51\slide i5{-1}%
    \slide i5{-2}\slide i5{-3}\slide i5{-4}\slide i5{-5}%
    \slide i5{-6}\slide i5{-7}\slide i5{-8}\sk\en
\leftrepeat
\Notes
  \loffset2{\zchar{-4}{\bs leftrepeat}}%
  \boxitsep=2pt\zchar{17}{\bs boxit A}\zchar9{\boxit A}\en
\leftrightrepeat
\Notes
  \loffset2{\zchar{-8}{\bs leftrightrepeat}}%
  \zchar{17}{\bs circleit B}\uptext{\circleit B}\en
\rightrepeat
\NOtes
  \loffset2{\zchar{-4}{\bs rightrepeat}}\en
\endpiece\advance\textheight-40mm\eject%
\endmuflex
}

%\makeatletter
%\renewenvironment{theindex}{\newpage
% \addcontentsline{toc}{chapter}{Index}%
% \pagestyle{plain}
% \let\item\@idxitem
% \parskip\z@ \@plus .3\p@\relax
% \begin{multicols}{2}[{\Huge\bfseries Index} ]
% \par\bigskip}%
%{\end{multicols}}%
%\makeatother
%\printindex

\makeatletter
\renewenvironment{theindex}{\newpage
 \setcounter{secnumdepth}{-1}
 \chapter[Index]{}\vspace*{-3cm}
% \addcontentsline{toc}{chapter}{Index}
  \pagestyle{plain}
 \let\item\@idxitem
 \parskip\z@ \@plus .3\p@\relax
 \begin{multicols}{3}[{\Huge\bfseries Index} ]
 \par\bigskip}%
{\end{multicols}}%
\makeatother
 \printindex

\end{document}
