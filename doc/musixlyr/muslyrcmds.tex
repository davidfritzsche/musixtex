%%
%% mxlyrdoc.tex:  Manual for the MusiXTeX lyrics package musixlyr
%%
%% Copyright (C) 1996-2003  Rainer Dunker
%%
%% This document is free software; you can redistribute it and/or modify
%% it under the terms of the GNU General Public License as published by
%% the Free Software Foundation; either version 2 of the License, or
%% any later version.
%%
%% This document is distributed in the hope that it will be useful,
%% but WITHOUT ANY WARRANTY; without even the implied warranty of
%% MERCHANTABILITY or FITNESS FOR A PARTICULAR PURPOSE.  See the
%% GNU General Public License for more details.
%%
%% You should have received a copy of the GNU General Public License
%% along with this document; if not, write to the Free Software
%% Foundation, Inc., 675 Mass Ave, Cambridge, MA 02139, USA.
%%
%% Author:
%%  Rainer Dunker
%%  Wachtelweg 31
%%  85591 Vaterstetten
%%  Germany
%%
%%  E-mail:  rainer.dunker@web.de
%%
\documentclass[twoside]{article}
\usepackage{musixtex,multicol,makeidx}

\input musixlyr

\textwidth210mm \advance\textwidth-2in               % These format settings
\textheight297mm \advance\textheight-2in \voffset-1in            % can be freely changed!
%\advance\textheight-\headheight \advance\textheight-\headsep
%\footskip0pt \topmargin0pt \evensidemargin0pt \oddsidemargin0pt

%\pagestyle{headings}
\parskip0.5ex plus 0.25ex minus 0.25ex
\parindent0pt
\sloppy

\nobarnumbers
\smallmusicsize

%
% command index
%
\makeatletter
\renewenvironment{theindex}
  {\let\item\@idxitem
   \begin{multicols}{2}
   \parskip0pt}
  {\end{multicols}}
\def\printcmd#1{{\tt\char92 #1}}
\def\ci#1{\printcmd{#1}\bci{#1}}
\def\bci#1{\index{#1@\printcmd{#1}}} % "blind" command index entry
\makeindex

\let\myqnwidth\qn@width
\makeatother

\def\musixlyr{{\tt musixlyr}}
\def\PMX{{\bf PMX}}
\def\var#1{\hbox{$\langle$#1$\rangle$}}
\def\oneversespace{\par\vspace*{2mm}}
\def\twoversesspace{\par\vspace*{7mm}}




\begin{document}


\centering{\LARGE\bf \musixlyr\ --- commands}\\[1ex]

\thispagestyle{empty}


%\def\keyexample#1{\keyindex{#1}, \Bslash#1}
\def\keyexample#1{#1}

\noindent\begin{tabbing}
\verb|\NOtes|\keyexample{assignlyricshere}\verb|{alto}\qa c\en|\quad\= assigning without staff number\kill
{\bf Example}\> {\bf Explanation}
\end{tabbing}
\vspace{-1ex}
\hrule
\vspace{-2ex}
\begin{tabbing}
\verb|\NOtes|\keyexample{assignlyricshere}\verb|{alto}\qa c\en|\quad\= assigning without staff number\kill
\keyexample{setlyrics}\verb|{sopr}{the ly_-ric words_}|\> defining the lyrics text\\
\keyexample{copylyrics}\verb|{sopr}{alto}|\> alto has same lyrics as soprano\\
\keyexample{appendlyrics}\verb|{alto}{more words}|\> alto lyrics is longer\\
\keyexample{assignlyrics}\verb|2{sopr,alto}|\>soprano and alto lyrics at staff 2\\
\keyexample{assignlyricsmulti}\verb|{1}{2}{alto}|\>assign alto lyrics to staff 2 of instrument 1\\
\verb|\NOtes|\keyexample{assignlyricshere}\verb|{alto}\qa c\en|\>assigning without staff number\\
\keyexample{auxlyr}\verb|\assignlyrics{2}{sopr}|\> assign soprano above staff 2\\[.8ex]
\keyexample{lyrrule}\verb|\qu c|...\keyexample{lyrruleend}\verb|\qu c|\>make a melisma by hand\\
\keyexample{beginmel}\verb|\qu c|...\keyexample{endmel}\verb|\qu c|\>melisma, same as word extension underline\\[.8ex]
\keyexample{lyr}\verb|\qu c|\>force a syllable from lyrics text at this note or rest\\
\keyexample{lyric}\verb|{word}\qu c|\>insert syllable 'word' at this note\\
\verb|\loffset{2}{|\keyexample{lyric*}\verb|{1.}}\qu c|\>combine '1.' with regular syllable\\
\keyexample{lyrich}\verb|{syl}\qu c|\>same as \verb|\lyric|, but with hyphenation\\
\keyexample{lyrich*}\verb|{}\qu c|\>same as \verb|\lyric*|, but with hyphenation\\
\keyexample{lyricsoff}...\keyexample{lyricson}\>stop lyrics, then start again\\
\keyexample{nolyr}\verb|\qu c|\>no syllable at this note\\[.8ex]
\keyexample{llabel}\verb|{labelname}name|\>labelling a ``go to'' target in text\\
\keyexample{golyr}\verb|{labelname}\qu c|\>perform a jump, in music code\\[.8ex]
\keyexample{lyrpt}\verb|,\qu c|\>add a comma to the syllable under this note\\
\keyexample{lyrnop}\verb|\qu c|\>remove last character in syllable\\
\keyexample{lclyr}\verb|\qu c|\>make first character lower case\\
\keyexample{llyr}\verb|\qu c|\>left justified syllable\\
\keyexample{leftlyrtrue}\verb|\qu c|...\keyexample{leftlyrfalse}\verb|\qu c|\>start and stop left justification as the default\\
\keyexample{lyroffset}\verb|{-4}\qu c|\>shift syllable 1 notehead to the left\\[.8ex]
\keyexample{minlyrspace}\verb|{3pt}\qu c|\>define minimum space between the words\\
\keyexample{forcelyrhyphenstrue}\verb|\qu c|\>always use a hyphen from now on\\
\keyexample{forcelyrhyphensfalse}\verb|\qu c|\>remove hyphen and make one word if necessary \\
\keyexample{showlyrshifttrue}\verb|\qu c|\>show the lyric shift\\[.8ex]
\keyexample{lyrraise}\verb|{1}{a 2\Interligne}|\>raise lyrics below staff 1 by \verb|2\Interligne|\\
\keyexample{lyrraisemulti}\verb|{1}{2}{a 2\Interligne}|\>raise alto lyrics above staff 2 of instrument 1\\
\keyexample{lyrraisehere}\verb|{b 2\Interligne}\qu c|\>raise lyrics below this staff by \verb|2\Interligne|\\[.8ex]
\keyexample{minlyrrulelength}\verb|{2mm}|\>melismas shorter than 2mm are not shown \\
\keyexample{minmulthyphens}\verb|{15mm}|\>distance between hyphens in 'hyphen melisma'\\
\verb|\def|\keyexample{lyrhyphenchar}\verb|{-}|\>chose a hyphen character\\
\verb|\setlyrics{|\keyexample{lyrlayout}\verb|{\it}..}|\>apply italics to all lyrics lines\\
\keyexample{verses}\verb|{,\beginmel}\qu c|\>initiate melisma at second verse\\[.8ex]
\verb|\small|\keyexample{setlyrstrut}\>adapt the vertical distance between lyrics lines\\
\keyexample{lyrstrutbox}\verb|{10pt}|\> (re)define the distance between the lyrics lines\\[.8ex]
\keyexample{lyrmodealter}\verb|2|\>attach lyrics of staff 2 to the upper voice\\
\keyexample{lyrmodealtermulti}\verb|{1}{2}|\>attach lyrics of instr.\ 1 staff 2 to the upper voice\\
\keyexample{lyrmodealterhere}\verb|\qu c|\>attach lyrics of this staff to the upper voice\\
\keyexample{lyrmodenormal}\verb|2|\>restore the default behaviour\\
\keyexample{lyrmodenormalmulti}\verb|{1}{2}|\>restore the default behaviour at staff 2 of instr.\ 1\\
\keyexample{lyrmodenormalhere}\verb|\qu c|\>restore the default behaviour of this staff\\[.8ex]
\keyexample{lyrlink}\>linking 2 words with a '$_{_\smile}$'\\
\keyexample{lowlyrlink}\>same as \verb|\lyrlink| but a little bit lower\\[.8ex]
\keyexample{resetlyrics}\>set word pointer to the first word in all lyrics lines\\[.8ex]
\keyexample{enableauxlyrics}\>don't use this anymore\\
\keyexample{setsongraise}\verb|{1}{2\Interligne}|\>same as \verb|{\lyrraise}{1}{b 2\Interligne}|\\
\keyexample{auxsetsongraise}\verb|{1}{2\Interligne}|\>same as \verb|{\lyrraisemulti}{1}{b 2\Interligne}|\\
\keyexample{oldlyrlinestart}\>don't let the lyrics extent to the left margin\\
\end{tabbing}
\end{document}
