\documentclass[11pt]{article}
\usepackage[textwidth=6.5in,textheight=8.5in]{geometry}
\usepackage[osf]{mathpazo}
\usepackage{textcomp}
\PassOptionsToPackage{urlcolor=black,colorlinks}{hyperref}
\RequirePackage{hyperref}
\usepackage{xcolor}
\newcommand{\myurl}[1]{\textcolor{blue}{\underline{\textcolor{black}{\url{#1}}}}}
\newcommand{\musixflxVersion}{0.83.3}
\begin{document}
\title{Installation of MusiXTeX}
\author{Bob Tennent\\
\small\url{rdt@cs.queensu.ca}}
\date{\today}
\maketitle 
\thispagestyle{empty}

\section{Introduction}
Before trying to install from CTAN, check whether your TeX distribution
provides packages for MusiXTeX; this will be easier than doing it yourself.
But if your TeX distribution
doesn't have MusiXTeX (or doesn't have the most recent version), this distribution
of MusiXTeX is very easy to install, though
you may need to read the material on 
installation of (La)TeX files in the 
TeX FAQ\footnote{%
\myurl{http://www.tex.ac.uk/cgi-bin/texfaq2html}},
particularly
the pages on 
which tree to use\footnote{%
\myurl{http://www.tex.ac.uk/cgi-bin/texfaq2html?label=what-TDS}}
and installing files\footnote{%
\myurl{http://www.tex.ac.uk/cgi-bin/texfaq2html?label=inst-wlcf}}.

\section{Installing \texttt{musixtex.tds.zip}}

In this distribution of MusiXTeX, most of the files to be installed 
(including macros, scripts, and documentation) are in a zipped TEXMF hierarchy
\begin{list}{}{}\item
\myurl{http://mirror.ctan.org/install/macros/musixtex.tds.zip}
\end{list}
at CTAN. Simply download and unzip this archive in the root folder/directory of whichever TEXMF tree
you decide is most appropriate, likely a ``local'' or ``personal'' one.
This should work with any TDS\footnote{%
\myurl{http://www.tex.ac.uk/cgi-bin/texfaq2html?label=tds}}
compliant TeX distribution, including MikTeX, TeXlive and teTeX.

After unzipping the archive, update the filename database as necessary,
for example, by executing \verb\texhash ~/texmf\ or 
clicking the button labelled ``Refresh FNDB" in the MikTeX settings program.

MusiXTeX requires a large set of specialized fonts. If necessary, install the 
\texttt{musixtex-fonts} package.

You should now be able to process files that use the MusiXTeX macros using
\verb\etex\ or \verb\pdfetex\;
documentation for MusiXTeX is installed under \verb\.../doc/generic/musixtex...\
in the TEXMF tree.  But the music will be ``squeezed'' to the left; to produce
proper spacing, go on.

\section{Installing \texttt{musixflx}}

The next step in the installation is to install
the one crucial file that can't be installed in a TEXMF tree: a \texttt{musixflx} executable, which calls
the script that implements
the second pass of the three-pass MusiXTeX typesetting system described in Section~1.3 of the
MusiXTeX manual (\verb\musixdoc.pdf\). In this MusiXTeX package, the 
calculations are done by a Lua script 
\begin{list}{}{}
\item \verb|.../scripts/musixtex/musixflx.lua| 
\end{list}
On a Unix-like system (with \texttt{luatex} installed), you can put a
symbolic link to \texttt{musixflx.lua} in any directory on the executable PATH as follows:
\begin{list}{}{}
\item \verb\ln -s <path to musixflx.lua> musixflx\
\end{list}
On Windows, you can \emph{either}
copy the batch file
\begin{list}{}{}
\item \verb|Windows\musixflx.bat| 
\end{list}
to a folder
on the executable PATH \emph{or} add the folder
\verb|Windows| to the executable PATH
as follows: 
in ``My Computer''
click on 
\begin{center}
View System Information\quad$\rightarrow$\quad Advanced\quad$\rightarrow$\quad Environment Variables
\end{center}
scroll
down to ``path'', select it, click edit, and add the path to \verb|Windows| after a semi-colon.
Documentation for \verb\musixflx\ is in the \verb\doc/generic/musixtex/scripts\ directory.

\section{Installing \texttt{musixtex}}

The Lua script \verb\.../scripts/musixtex/musixtex.lua\ 
is simply a convenient wrapper that, by default, 
runs the following processes in order (and then deletes intermediate files):
\begin{itemize}\topsep=0pt\itemsep=0pt
\item \verb\etex\  (1st pass)
\item \verb\musixflx\ (2nd pass)
\item \verb\etex\ (3rd pass)
\item \verb\dvips\ (to convert \verb\dvi\ output to Postscript)
\item \verb\ps2pdf\ (to convert \verb\ps\ output to Portable Document Format)
\end{itemize}%
To install, follow the instructions given in the preceding section; i.e., 
on Unix-like systems, 
install a symbolic link \verb\musixtex\; on Windows,
ensure that the corresponding batch file
is in a folder on the executable PATH. Documentation is in the
\verb\doc/generic/musixtex/scripts\ directory.

If the \verb\musixps.tex\ package is not used to produce slurs, it is possible to use \verb\pdfetex\
rather than \verb\etex\, in which case a PDF file will be produced directly,
without the use of \verb\dvips\ and \verb\ps2pdf\;  use the \verb\-p\ option in the call to \verb\musixtex\. 
The \verb\-d\ option will replace \verb\dvips\ and \verb\ps2pdf\ by \verb\dvipdfm\; again,
this option is usable only if Postscript slurs are \emph{not} used. Use the \verb\-s\ option
to stop processing after the third pass; i.e., at the \verb\dvi\ file.
Use the \verb\-l\ option for \verb\LaTeX\ processing and the \verb\-f\ option to restore the defaults for subsequent files.

The cross-platform TeXWorks editing environment can 
be configured
to use the relevant \verb\musixtex\ script as a typesetting process as follows:
\begin{center}
Edit\quad$\rightarrow$\quad Preferences\quad$\rightarrow$\quad Typesetting
\end{center}
then click on + to add a new processing tool as follows:
\begin{description}\itemsep=0pt
\item[Name:] \verb\musixtex\
\item[Program:] \verb\musixtex\
\item[Arguments:] \verb\$basename\
\end{description}
If necessary, add the path to the directory/folder containing the \verb\musixtex\
script to the list of Paths for TeX and related programs.

\section{Discussion}
Many users, especially beginners, will find it easier to use the PMX and
M-Tx pre-processors, which accept a simpler input language than MusiXTeX
itself. These pre-processor packages may be found at CTAN under \verb\support\.  Additional documentation, additional
add-on packages, and many examples of MusiXTeX typesetting may be found
at the Werner Icking Music Archive\footnote{%
\myurl{http://icking-music-archive.org}}.
Support for users of MusiXTeX and related software may be obtained via
the MusiXTeX mail list\footnote{%
\myurl{http://tug.org/mailman/listinfo/tex-music}}.
MusiXTeX may be freely copied, duplicated and used in conformance to the
GNU General Public License (Version~2, 1991, see included file \verb\gpl.txt\).

\end{document}
