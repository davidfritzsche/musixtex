\chapter{Beams}
\section{Starting a beam}

Each beam must be declared with a macro issued before the first spacing
note under the beam is coded. Two distinct kinds of macros are provided
for this. The first kind initiates a ``fixed-slope'' beam, with an arbitrary
slope and starting height chosen by the user, while the second kind, a
``semi-automatic'' beam,
\ital{computes}~the slope and, in addition, adjusts the starting height in some
cases.

\def\nps{{\tt\char123}$n${\tt\char125}\pitchp{\tt\char123}$s${\tt\char125}}

The basic form of the macros for starting fixed-slope beams is exemplified
by the one for a single upper beam, \keyindex{ibu}\nps. Here
$n$ is the reference number of the beam, $p$ the starting ``pitch'', and
$s$ the slope. 

The reference number is assigned by the user. 
It is needed because more than one beam may be open at 
a time, and it tells \musixtex\ to which beam subsequent beamed notes and 
other beam specification commands are assigned. 
By default, the reference number must be in the range [$0,5$], 
but the range for 8th to 128th beams will be expanded to [$0,8$] or [$0,11$] 
if \verb|musixadd| or \verb|musixmad| respectively has been \verb|\input|. 

Alternatively, you can specify the number of $8$th to $64$th beams 
directly\footnote{$8$th to $64$th beams are so basic that the maximum 
number of these beams is related to the maximum number of instruments 
by this command. 
Using $m>12$ may require e-\TeX.}
with \keyindex{setmaxinstruments}\verb|{|$m$\verb|}| 
within the range $7<m\leq 100$; the corresponding reference number may then be 
in the range from 0 to $(m-1)$. 
\label{musixmad_setmaxinstruments_ccxviiibeams}  
For $128$th beams, use \keyindex{setmaxcxxviiibeams}\verb|{|$m$\verb|}|.
For $256$th notes, which can only appear in beams, see Section~\ref{musixbbm}.

The ``pitch'' parameter $p$ is a pitch that is three
staff spaces \ital{below} the bottom of the heavy connecting bar (\ital{above}
the bar for a lower beam); in many (but not all)
cases it should be input as the actual pitch of the first note. The slope $s$
is an integer in the range [$-9,9$]. When multipled
by $5$\%~it gives the actual slope of the heavy bar. Typically a slope of $2$ or $3$
is okay for ascending scales, and $6$ to $9$ for ascending arpeggios.

The full set of fixed-slope beam initiation macros is as follows:
\begin{quote}\begin{description}\setlength{\itemsep}{0ex}

 \item[\keyindex{ibu}\nps~:]initiates an \ital{upper beam}.
 \item[\keyindex{ibl}\nps~:]initiates a \ital{lower beam}.
 \item[\keyindex{ibbu}\nps~:]initiates a \ital{double upper beam}.
 \item[\keyindex{ibbl}\nps~:]initiates a \ital{double lower beam}.
 \item[\keyindex{ibbbu}\nps~:]initiates a \ital{triple upper beam}.
 \item[\keyindex{ibbbl}\nps~:]initiates a \ital{triple lower beam}.
 \item[\keyindex{ibbbbu}\nps~:]initiates a \ital{quadruple upper beam}.
 \item[\keyindex{ibbbbl}\nps~:]initiates a \ital{quadruple lower beam}.
\end{description}\end{quote}

A semi-automatic beam is initiated with a command that has \ital{four}
parameters, the beam number, the first and last pitches, and the total
horizontal extent in
\verb|\noteskip|s, based on the value in effect at the start. For example,
if you input \verb|\Ibu2gj3|, \musixtex\ will understand that you want to
build an upper beam (beam number $2$) horizontally extending \verb|3\noteskip|,
the first note of which is a \verb|g| and the last note a \verb|j|.
Knowing these parameters it will choose the highest slope number that
corresponds to a slope not more than $(\hbox{\tt j}-\hbox{\tt
g})/(3\keyindex{noteskip})$. The nominal height of the heavy bar is offset the same as
for fixed-slope beams. However, if there is no sufficiently steep beam
slope available, then \musixtex\ will raise (or lower) the starting point.

 Eight such macros are available: \keyindex{Ibu}, \keyindex{Ibbu},
\keyindex{Ibbbu}, \keyindex{Ibbbbu}, \keyindex{Ibl}, \keyindex{Ibbl},
\keyindex{Ibbbl} and \keyindex{Ibbbbl}.

\section{Adding notes to a beam}

Spacing notes belonging to beams are coded with the macro
\keyindex{qb}\verb|{|$n$\verb|}|\pitchp~where $n$ is
the beam number and $p$ the pitch of the note. \musixtex\ adjusts the
length of the note stem to link to the beam.

Chord notes within a beam are entered before the main note with the
non-spacing macro \keyindex{zqb}\verb|{|$n$\verb|}|\pitchp. Again,
the stem length will be automatically adjusted as required.

There are also special macros for semi-automatic beams with
two, three, or four notes:
\keyindex{Dqbu}, \keyindex{Dqbl}, \keyindex{Dqbbu}, \keyindex{Dqbbl},
\keyindex{Tqbu}, \keyindex{Tqbl}, \keyindex{Tqbbu}, \keyindex{Tqbbl},
\keyindex{Qqbu}, \keyindex{Qqbl}, \keyindex{Qqbbu} and \keyindex{Qqbbl}.
  %\check
For example \verb|\Dqbu gh| is equivalent to \verb|Iqbu1gh\qb1 g\tbu1\qb1 h|,
except that the special macros don't require a beam number.
Their use is illustrated in the following example:

\medskip
\begin{music}\nostartrule
\parindent0pt\startpiece
\Notes\Dqbu gh\Dqbl jh\en
\notes\Dqbbu fg\Dqbbl hk\en\bar
\Notes\Tqbu ghi\Tqbl mmj\en
\notes\Tqbbu fgj\Tqbbl njh\en\bar
\Notes\Qqbu ghjh\Qqbl jifh\en
\notes\Qqbbu fgge\Qqbbl jhgi\en\setemptybar\endpiece
\end{music}
\noindent This was coded as\footnote{Editor's note: Most music
typesetting books recommend beam slopes that are \ital{less} than
the slope between the starting and ending note; these macros cannot
provide that.}:
\begin{quote}\begin{verbatim}
\Notes\Dqbu gh\Dqbl jh\en
\notes\Dqbbu fg\Dqbbl hk\en\bar
\Notes\Tqbu ghi\Tqbl mmj\en
\notes\Tqbbu fgj\Tqbbl njh\en\bar
\Notes\Qqbu ghjh\Qqbl jifh\en
\notes\Qqbbu fgge\Qqbbl jhgi\en
\end{verbatim}\end{quote}
 %\check

\section{Ending a beam}
 The termination of a given
beam must be declared \ital{before} coding the last spacing note
connected to that beam. The macros for doing that are
\keyindex{tbu}\verb|{|$n$\verb|}| for an upper beam and
\keyindex{tbl}\verb|{|$n$\verb|}| for a lower one. These work for
beams of any
multiplicity. So for example an upper triple beam with
32nd notes is initiated by
\verb|\ibbbu|\nps\ but terminated by \verb|\tbu{|$n$\verb|}|.

 Since beams usually finish with a \verb|\qb| for the last note, the
following shortcut macros have been provided:

\begin{quote}\begin{description}\setlength{\itemsep}{0ex}
 \item \keyindex{tqb}\enpee~is equivalent to \verb|\tbl{|$n$\verb|}\qb|\enpee
 \item \keyindex{tqh}\enpee~is equivalent to \verb|\tbu{|$n$\verb|}\qb|\enpee
 \item \keyindex{ztqb}\enpee~is equivalent to \verb|\tbl{|$n$\verb|}\zqb|\enpee~i.e., no spacing afterwards
 \item \keyindex{ztqh}\enpee~is equivalent to \verb|\tbu{|$n$\verb|}\zqb|\enpee~i.e., no spacing afterwards
\end{description}\end{quote}
The following
synonyms may be used:
\begin{quote}\begin{tabular}{lcl}
\keyindex{tql} & for & \keyindex{tqb}\\
\keyindex{tqu} & for & \keyindex{tqh}\\
\keyindex{ztql} & for & \keyindex{ztqb}\\
\keyindex{ztqu} & for & \keyindex{ztqh}\\
\end{tabular}\end{quote}


\section{Changing multiplicity after the beam starts}
Multiplicity (the number of heavy bars) can be increased at any position after
the beam starts. The commands are \keyindex{nbbu}\onen~which increases the multiplicity of
upper beam
number $n$ to two starting at the current position, \keyindex{nbbbu}\onen~to
increase it to three, and \keyindex{nbbbbu}\onen~to increase to four. The
commands \keyindex{nbbl}\onen\dots\keyindex{nbbbbl}\onen~do the same
for lower beams.
Thus, the
sequence

\begin{music}\nostartrule
\startextract
\Notes\ibu0h0\qb0e\nbbu0\qb0e\nbbbu0\qb0e\nbbbbu0\qb0e\tbu0\qb0e\en
\zendextract
\end{music}
\noindent has been coded as
\begin{quote}\begin{verbatim}
\Notes\ibu0h0\qb0e\nbbu0\qb0e\nbbbu0\qb0e\nbbbbu0\qb0e\tbu0\qb0e\en
\end{verbatim}\end{quote}

To decrease multiplicity to one, use \keyindex{tbbu}\onen~or
\keyindex{tbbl}\onen. To decrease to two or three use
\keyindex{tbbbu}\onen\dots\keyindex{tbbbbl}\onen. For example,

\begin{music}\nostartrule
\startextract
\Notes\ibbbu0h0\qb0e\tbbbu0\qb0e\tbbu0\qb0e\tbu0\qb0e\en
\zendextract
\end{music}
\noindent has been coded as

\begin{quote}\begin{verbatim}
\startextract
\Notes\ibbbu0h0\qb0e\tbbbu0\qb0e\tbbu0\qb0e\tbu0\qb0e\en
\zendextract
\end{verbatim}\end{quote}

Although at first it may seem counterintuitive,
the macros \keyindex{tbbu} and \keyindex{tbbl} and higher order counterparts
may also be invoked when the multiplicity is one. In this case
a second, third, or fourth heavy bar will be opened one note width \ital{before}
the current stem, and immediately closed \ital{at} the stem.
Thus the following sequences

\begin{music}\nostartrule
\let\extractline\hbox
\hbox to \hsize{%
\hss
  \startextract
  \Notes\ibu0e0\qbp0e\roff{\tbbu0\tqh0e}\en
  \zendextract
\hss\hss
  \startextract
  \Notes\ibu0e0\qbpp0e\roff{\tbbbu0\tbbu0\tqh0e}\en
  \zendextract
\hss}
\end{music}
\noindent are coded
\hspace*{\fill}\begin{minipage}{.4\textwidth}\begin{verbatim}
\Notes\ibu0e0\qbp0e%
  \roff{\tbbu0\tqh0e}\en
\end{verbatim}\end{minipage}\hfill
\begin{minipage}{.4\textwidth}\begin{verbatim}
\Notes\ibu0e0\qbpp0e%
  \roff{\tbbbu0\tbbu0\tqh0e}\en
\end{verbatim}\end{minipage}\hfill\\[4ex]
The following abbreviations have been provided:
\begin{quote}\begin{description}\setlength{\itemsep}{0ex}
 \item \keyindex{tqqb}\enpee~is equivalent to \verb|\tbbl{|$n$\verb|}\tqb|\enpee
 \item \keyindex{tqqh}\enpee~is equivalent to \verb|\tbbu{|$n$\verb|}\tqh|\enpee
 \item \keyindex{tqqqh}\enpee~is equivalent to \verb|\tbbbu{|$n$\verb|}\tqqh|\enpee
 \item \keyindex{tqqqb}\enpee~is equivalent to \verb|\tbbbl{|$n$\verb|}\tqqb|\enpee
 \item \keyindex{nqqb}\enpee~is equivalent to \verb|\tbbl{|$n$\verb|}\qb|\enpee
 \item \keyindex{nqqh}\enpee~is equivalent to \verb|\tbbu{|$n$\verb|}\qb|\enpee
 \item \keyindex{nqqqh}\enpee~is equivalent to \verb|\tbbbu{|$n$\verb|}\qb|\enpee
 \item \keyindex{nqqqb}\enpee~is equivalent to \verb|\tbbbl{|$n$\verb|}\qb|\enpee
\end{description}\end{quote}
and the following
synonyms may be used:
\begin{quote}\begin{tabular}{lcl}
\keyindex{tqql} & for & \keyindex{tqqb}\\
\keyindex{tqqu} & for & \keyindex{tqqh}\\
\keyindex{tqqql} & for & \keyindex{tqqqb}\\
\keyindex{tqqqu} & for & \keyindex{tqqqh}\\
\keyindex{nqql} & for & \keyindex{nqqb}\\
\keyindex{nqqu} & for & \keyindex{nqqh}\\
\keyindex{nqqql} & for & \keyindex{nqqqb}\\
\keyindex{nqqqu} & for & \keyindex{nqqqh}\\
\end{tabular}\end{quote}

\medskip

The symmetrical pattern is also possible. For example:

\begin{music}\nostartrule
\startextract
\Notes\ibbl0j0\roff{\tbbl0}\qb0j\tbl0\qbp0j\en
\zendextract
\end{music}
\noindent has been coded as:
\begin{quote}\begin{verbatim}
\Notes\ibbl0j0\roff{\tbbl0}\qb0j\tbl0\qbp0j\en
\end{verbatim}\end{quote}

The constructions in this section illustrate some general
properties of
beam initiation and termination commands: To mate properly with the
expected stems, the starting position of the heavy bar(s) (for initiation commands)
and the ending position (for terminations) will be at
different horizontal locations
depending on whether they are for upper or lower beams: The position for
upper beam commands is one note head width to the right of those for
lower beams. In fact this is the \ital{only} difference between upper
and lower termination commands. Both types will operate on whatever kind of
beam is open and has the same beam number.

Recognizing this principle, in the example just given it was necessary to
shift the double termination to the right by one note head width, using the
command \keyindex{roff}\verb|{|\dots\verb|}|, which does precisely that for
any \musixtex\ macro.

Here is another, slightly more complicated example which also uses
\verb|\roff|:

\begin{music}\nostartrule
\startextract
\notes\ibbbu0e0\roff{\tbbbu0}\qb0f\en
\notesp\tbbu0\qbp0f\en
\Notes\tbu0\qb0f\en
\notesp\ibbu0f0\roff{\tbbu0}\qbp0f\en
\Notes\qb0f\en
\notes\tbbbu0\tbbu0\tbu0\qb0f\en
\zendextract
\end{music}
\noindent has been coded as:
\begin{quote}\begin{verbatim}
\notes\ibbbu0e0\roff{\tbbbu0}\qb0f\en
\notesp\tbbu0\qbp0f\en
\Notes\tbu0\qb0f\en
\notesp\ibbu0f0\roff{\tbbu0}\qbp0f\en
\Notes\qb0f\en
\notes\tbbbu0\tbbu0\tbu0\qb0f\en
\end{verbatim}\end{quote}

\noindent Note that the first beam opening command used a pitch one step
below the note. This makes the stem shorter by one pitch unit, since it is always
the \ital{closest} heavy bar that is separated from the given pitch by
three staff spaces.

We close this section with an example showing how to
open a beam of one sense, increase multiplicity, then terminate with
opposite sense:

\begin{music}\nostartrule
\startextract
\Notes\ibl0p0\qb0p\nbbl0\qb0p\nbbbl0\qb0p\tbu0\qb0e\en
\zendextract
\end{music}
\noindent which has been coded as
\begin{quote}\begin{verbatim}
\Notes\ibl0p0\qb0p\nbbl0\qb0p\nbbbl0\qb0p\tbu0\qb0e\en
\end{verbatim}\end{quote}

One may save some typing by defining personalized
\TeX\ macros to
perform any oft-repeated sequence of commands. For example,
one could define a set of four sixteenths by the macro:

\verb|\def\qqh#1#2#3#4#5{\ibbl0#2#1\qb#2\qb#3\qb#4\tbl0\qb#5}|

\noindent where the first argument is the slope and the other four
arguments are the pitches of the four successive sixteenths.

 %\check
 \section{Shorthand beam notations for repeated or alternated notes}\index{repeated patterns}
Sometimes you may want to indicate repeated or alternated short notes with open note heads joined
by a beam. Here are examples of how to do that
using the \keyindex{hb} macro:

\begin{music}\nostartrule
\startextract
\Notes\ibbl0j0\hb0j\tbl0\hb0j\en
\Notes\ibbl0h4\hb0h\tbl0\hb0j\en
\zendextract
\end{music}
\noindent which has been coded as:
\begin{quote}\begin{verbatim}
\Notes\ibbl0j0\hb0j\tbl0\hb0j\en
\Notes\ibbl0h4\hb0h\tbl0\hb0j\en
\end{verbatim}\end{quote}\noindent
There are also dotted and double-dotted versions \keyindex{hbp} and
\keyindex{hbpp}.


A different look could be obtained as follows:

\begin{music}\nostartrule
\startextract
\notes\ha j\loffset{0.5}{\ibbl1n3}\qsk\tbl1\qsk\ha l\sk\en
\zendextract
\end{music}
\noindent which has been coded as:
\begin{quote}\begin{verbatim}
\notes\ha j\loffset{0.5}{\ibbl1n4}\qsk\tbl1\qsk\ha l\sk\en
\end{verbatim}\end{quote}
It is possible to combine this notation
with a conventional beam, as in

\begin{music}\nostartrule
\startextract
\notes\ibl0h3\qb0j\loffset1{\ibbbl1m3}\qsk\tbl1\qsk\tql0l\en
\zendextract
\end{music}
\noindent which has been coded as:
\begin{quote}\begin{verbatim}
\notes\ibl0h3\qb0j\loffset1{\ibbbl1m3}\qsk\tbl1\qsk\tql0l\en
\end{verbatim}\end{quote}\noindent
or, for whole notes, to dispense with stems:

\begin{music}\nostartrule
\startextract
\Notes\loff{\zw j}\ibbl0l4\sk\tbl0\wh l\en
\Notes\ibbu0e4\wh g\tbu0\roff{\wh i}\en
\zendextract
\end{music}
\noindent which has been coded as:
\begin{quote}\begin{verbatim}
\Notes\loff{\zw j}\ibbl0l4\sk\tbl0\wh l\en
\Notes\ibbu0e4\wh g\tbu0\roff{\wh i}\en
\end{verbatim}\end{quote}

Short ``beams'' centered on note stems (or directly over
or under whole notes)
indicate \emph{tremolos}. Commands that generate tremolo indications are described in Section~\ref{tremolos}.


% \begin{music}\nostartrule
%\startextract
%\Notes\ibl0h0\qb0{hhh}\tbl0\qb0h\bsk\bsk\bsk\bsk
%      \ibu0j0\qb0{jjj}\tbu0\qb0j\en
%\NOTes\loffset{0.5}{\ibl0j9}\roffset{0.5}{\tbl0}\zhl h%
%      \loffset{0.5}{\ibu0g9}\roffset{0.5}{\tbu0}\hu j\en\bar
%\notes\ibbl0i0\qb0{hhh}\tbl0\qb0h\bsk\bsk\bsk\bsk
%      \ibbu0i0\qb0{jjj}\tbu0\qb0j%
%      \ibbl0i0\qb0{hhh}\tbl0\qb0h\bsk\bsk\bsk\bsk
%      \ibbu0i0\qb0{jjj}\tbu0\qb0j\en
%\NOTes\loffset{0.5}{\ibbl0k9}\roffset{0.5}{\tbl0}\zhl h%
%      \loffset{0.5}{\ibbu0f9}\roffset{0.5}{\tbu0}\hu j\en
%\zendextract
% \end{music}
% \noindent whose coding (due to Werner {\sc Icking}) is
%
%\begin{quote}\begin{verbatim}
%\Notes\ibl0h0\qb0{hhh}\tbl0\qb0h\bsk\bsk\bsk\bsk
%      \ibu0j0\qb0{jjj}\tbu0\qb0j\en
%\NOTes\loffset{0.5}{\ibl0j9}\roffset{0.5}{\tbl0}\zhl h%
%      \loffset{0.5}{\ibu0g9}\roffset{0.5}{\tbu0}\hu j\en\bar
%\notes\ibbl0i0\qb0{hhh}\tbl0\qb0h\bsk\bsk\bsk\bsk
%      \ibbu0i0\qb0{jjj}\tbu0\qb0j%
%      \ibbl0i0\qb0{hhh}\tbl0\qb0h\bsk\bsk\bsk\bsk
%      \ibbu0i0\qb0{jjj}\tbu0\qb0j\en
%\NOTes\loffset{0.5}{\ibbl0k9}\roffset{0.5}{\tbl0}\zhl h%
%      \loffset{0.5}{\ibbu0f9}\roffset{0.5}{\tbu0}\hu j\en
%\end{verbatim}\end{quote}

 \section{Beams that cross line breaks}

Although careful typesetting can usually avoid it, occasionally
a beam may need to cross a line break. If so, it must be manually terminated at
the end of one line and continued in the next. This
can be done by shifting beam terminations and initiations
using \keyindex{roff} and/or
\keyindex{loff}, or by inserting a spacing command such as
\keyindex{hsk}. We give an example from {\sc Grieg}'s ``Hochzeit auf
Troldhaugen'':\index{Grieg, E.@{\sc Grieg, E.}}\medskip

\begin{music}
\parindent0pt
\def\rqs{\lower\Interligne\rlap\qs}
\def\snotes{\vnotes1\elemskip}
\setstaffs1{2}
\generalsignature{2}
\setclef1{\bass}
\interstaff{12}
\startpiece
%%% bar 1
\addspace\afterruleskip
\snotes|\tinynotesize\ibsluru0n\ibbu0m5\qb0{=m}\tqh0n\en
\qspace
\Notes\zchar{-7}\sPed\loffset{.3}{\fl E}\zq E\qu{_I}%
  |\zql g\ibu2l0\busf2\qb2{=m}\en
\Notes|\tubslur0o\qb2{_l}\en
\Notes\loffset{.3}{\fl L}\zq L\ibl0L0\qb0{_b}%
  |\ibl1h0\zqb1g\bupz2\qb2l\en
\Notes\zq L\tqb0b|\tbl1\zqb1g\bupz2\tqh2l\en
\NOtes\zq L\ql b|\zql g\qu l\en
\notes\zchar{-7}\sPed\zchar{-7}{\eightit ~~~sempre}%
  \zchar{14}{\pp\eightit~sempre}\zq I\ibbu1J0\qb1L|\qs\en
\notes\qs|\zq N\ibbu3d0\qb3{_d}\en
\notes\qb1E|\rqs\en
\notes\qs|\fl e\zq N\rq e\qb3d\en
%%% bar 2
\bar
\notes\loffset{.3}{\fl I}\zq I\qb1{_L}|\rqs\en
\qspace
\notes\qs|\lfl d\zq d\zq {=f}\qb3N\en
\notes\tqh1{_E}|\rqs\en
\notes\qs|\zq d\zq g\tqh3N\en
\notes\zq I\ibbu0J0\qb0L|\rqs\en
\notes\qs|\zq d\ibbu1d0\qb1N\en
\notes\qb0E|\rqs\en
\notes\qs|\fl e\rq e\zq d\qb1N\en
\notes\zq I\qb0L|\rqs\en
\notes\qs|\zq f\zq d\qb1N\en
\notes\tqh0E|\rqs\en
\notes\qs|\zq g\zq d\tqh1N\en
\notes\zq I\ibbu0J0\qb0L|\rqs\en
\notes\qs|\zq d\ibbu1d0\qb1N\en
\notes\qb0E|\rqs\en
\notes\rlap\qs\hsk\tbu0|\rq e\zq d\zqb1N\hsk\tbu1\en
\endpiece
\end{music}
\noindent The prolongation of the two upper beams at the end is
illustrated in the code fragment
\begin{quote}\begin{verbatim}
\notes\rlap{\qs}\hsk\tbu0|\rq e\zq d\zqb1N\hsk\tbu1\en
\end{verbatim}\end{quote}
 %\check

 \section{Beams with notes on several different staves}

Here's a simple example from {\sc Brahms}'s
Intermezzo Op.~118, provided by
Miguel {\sc Filgueiras}:\index{Brahms, J.@{\sc Brahms, J.}}\medskip

\begin{music}
\interstaff{12}
\setstaffs1{2}
\setclef1\bass
\generalmeter\allabreve
\startextract
\NOtes\qp\nextstaff\isluru0q\zq{q}\ql{j}\en
\bar
\nspace
\Notes\ibu0L2\qb0{CEJLcL}%
  \nextstaff\roff{\zw{l}}\pt{p}\zh{_p}\pt{i}\hl{_i}\en
\Notes\qb0J\itied1a\qb0a\nextstaff\tslur0o\zq{o}\ql{h}\en
\bar
\Notes\ttie1\zh{.L.a}\hl{.e}%
  \nextstaff\qb0{chj}\tbl0\qb0l\cl{q}\ds\en
\NOtes\qp\nextstaff\zq{q}\ql{j}\en
\endextract
\end{music}
 %\check
\noindent The coding is
\begin{verbatim}
\interstaff{13}
\instrumentnumber{1}
\setstaffs1{2}
\setclef1\bass
\generalmeter\allabreve
\startextract
\NOtes\qp\nextstaff\isluru0q\zq{q}\ql{j}\en
\bar
\nspace
\Notes\ibu0a1\qb0{CEJLcL}%
  \nextstaff\roff{\zw{l}}\pt{p}\zh{_p}\pt{i}\hl{_i}\en
\Notes\qb0J\itied1a\qb0a\nextstaff\tslur0o\zq{o}\ql{h}\en
\bar
\Notes\ttie1\zh{.L.a}\hl{.e}%
  \nextstaff\qb0{chj}\tbl0\qb0l\cl{q}\ds\en
\NOtes\qp\nextstaff\zq{q}\ql{j}\en
\endextract
\end{verbatim}

\noindent (This example also shows that there is no problem in extending
a beam across a bar line.)

The general features that enable this type of coding as well as the
more complex example to follow are
 \begin{itemize}\setlength{\itemsep}{0ex}
%  \item Commands like \Bslash{\tt ibu}, \Bslash{\tt ibl},
%\Bslash{\tt Ibu}, and \Bslash{\tt Ibl} \rm
%define beams whose initial vertical position and slope are fixed
%relative
%to the staff where they begin, but notes in other staves can still be
%connected to them using \Bslash{\tt qb}\onen.
%
%  \item The commands \keyindex{tbu}\onen\rm~or \keyindex{tbl}\onen\rm~terminate
%beam $n$ at the specified position, but \musixtex\ remembers the beam
%parameters until a new beam with the same number is defined.
%Therefore, even after beam $n$ has been ``finished'' by a \verb|\tbu| or
%\verb|\tbl| command, commands like \Bslash{\tt qb}\enpee\rm~will still
%generate notes connected to the phantom extension of this beam,
%\ital{provided they are issued in a different staff}. If the command
%\verb|\qb|\enpee\rm~were issued on the same staff as the beam after
% the beam had ended, an error would result.
%
  \item Commands like {\Bslash\texttt{ibu}}, {\Bslash\texttt{ibl}},
{\Bslash\texttt{Ibu}}, and {\Bslash\texttt{Ibl}}
define beams whose initial vertical position and slope are fixed
relative
to the staff where they begin, but notes in other staves can still be
connected to them using {\Bslash\texttt{qb}\onen}.

  \item The commands {\keyindex{tbu}\onen}  or {\keyindex{tbl}\onen}
terminate
beam $n$ at the specified position, but \musixtex\ remembers the beam
parameters until a new beam with the same number is defined.
Therefore, even after beam $n$ has been ``finished'' by a \verb|\tbu| or
\verb|\tbl| command, commands like {\Bslash{\tt qb}\enpee} will still
generate notes connected to the phantom extension of this beam,
\ital{provided they are issued in a different staff}. If the command
\verb|\qb|\enpee\ were issued on the same staff as the beam after
 the beam had ended, an error would result.





 \item If the beam is initiated on one staff,
notes in a lower staff can be connected to it, but only \ital{after} the beam
has been defined. This may require using the command \keyindex{prevstaff}
to go back one staff, as described in Section~\ref{movingtostaffs}.

 \end{itemize}

 Here is an example:

 \begin{music}
 \setstaffs13
 \setclef1{6000}
 \startextract
 \notesp
 \nextstaff\Ibbbu0Af7\prevstaff
 \qb0{AEH^JLa}\relax\nextstaff
 \qb0{******^c}\tqh0e\relax
 |\zq{h^jl}\ql o\en
 \notesp
 \nextstaff
 \Ibbbu0hG6\qb0{hec}\prevstaff
 \qb0{***aLJ}\tqh0H\relax\nextstaff
 |\zq{h^jl}\ql o\en \nspace
 \zendextract
 \end{music}

 \noindent which is coded as:

 \begin{quote}
 \begin{verbatim}
 \setstaffs13
 \setclef1{6000}
 \startextract
 \notesp
 \nextstaff\Ibbbu0Af7\prevstaff
 \qb0{AEH^JLa}\relax\nextstaff
 \qb0{******^c}\tqh0e\relax
 |\zq{h^jl}\ql o\en
 \notesp
 \nextstaff
 \Ibbbu0hG6\qb0{hec}\prevstaff
 \qb0{***aLJ}\tqh0H\relax\nextstaff
 |\zq{h^jl}\ql o\en \nspace
 \zendextract
 \end{verbatim}
 \end{quote}
In this example we see not only multiple uses of \keyindex{nextstaff} and
\keyindex{prevstaff}, but also the character \verb|*| to make virtual beam notes
(see Section~\ref{CollectiveCoding}).


 \section{Discontinuities in Long Beams}
By default, \texttt{musixtex} produces beams using special fonts.  Unfortunately, long beams may have
unsightly discontinuities (gaps or bumps). See Section~\ref{musixvbm}
for a solution to this problem.
