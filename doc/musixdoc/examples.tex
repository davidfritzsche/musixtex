\chapter{\musixtex\ Examples}

The file \verb|musixdoc.tex|, the source for this manual, contains many useful
examples. In the manual, many examples are accompanied by a display of the code
that produced them, while for a few only an image of the extract is included and you'll
have to look in the \verb|musixdoc| source files to see the coding.

Other useful examples cannot be embedded in the source, either because they are
meant to be in \TeX, not \LaTeX, or because they are simply too large. For
these the source files also are provided separately.

When compiling or viewing any of the examples, you should keep in mind that
most DVI previewers and laser printers have
their origin one inch below and one inch to the right of the upper right corner
of the paper, while the musical examples have their upper left
corner just one centimeter to the right and below the top left corner of the page.
Therefore, special parameters may have to be given to the DVI transcription
programs unless special \keyindex{hoffset} and \keyindex{voffset} \TeX\
commands have been included within the \TeX\ source.

\section{Small examples}

\begin{itemize}\setlength{\itemsep}{0ex}

\item{\tt ossiaexa.tex}~: This is a stand-alone example of the use of ossia,
provided by Olivier Vogel (Section~\ref{ossia} on page \pageref{ossia}).

\item{\tt 8bitchar.tex}~: Using 8bit characters. 

\end{itemize}

\section{Full examples}
The small examples above and the longer ones mentioned below are included in 
the documentation folder of the \musixtex\ package.
Some of them require 
\verb|musixcpt.tex| which makes examples created in
Music\TeX\ compatible with \musixtex. Here we mention a few of
special interest.

\subsection{Examples mentioned in the manual}

\begin{itemize}\setlength{\itemsep}{0ex}
 \item{\tt avemaria.tex}~: the ``M\'editation'' (alias ``Ave Maria'') by
Charles {\sc Gounod} for organ and violin or voice.\index{Gounod, C.@{\sc Gounod, C.}}
To run this five-page example you'll also need \texttt{avemariax.tex}.
It demonstrates the use of separated bar rules (Section~\ref{avemaria2})
and the use of staves of different sizes (Section~\ref{avemaria}).
Also, an additional instrument is created for lyrics. This was a common
practice before the \texttt{musixlyr} package was created by Rainer Dunker.

 \item{\tt glorias.tex}~: a local melody for the French version of
\ital{Gloria in excelsis Deo}, a three-page piece demonstrating the use of the hardlyrics
commands (Section~\ref{glorias}). {\tt gloriab.tex} is the same piece, but with organ accompaniment.

\end{itemize}

\subsection{Other examples, provided by the authors of \musixtex }

\begin{itemize}\setlength{\itemsep}{0ex}

 \item{\tt traeumer.tex}~: the famous ``Tr\"aumerei'' by
Robert {\sc Schumann}\index{Schumann, R.@{\sc Schumann, R.}} for piano, in genuine
\musixtex\ but with some
additions to perform ascending bitmapped \itxem{crescendos}.
There are also S-shaped slurs between 2 staves.

 \item{\tt parnasum.tex}~: the first page of ``Doctor gradus ad
Parnassum'' by Claude {\sc Debussy}\index{Debussy,
C.@{\sc Debussy, C.}} for piano.
It contains a rather complex example of a new command \verb+\Special+
to create staff-jumping doubly beamed notes.
\end{itemize}

\subsection{Additional documentation}

\begin{itemize}\setlength{\itemsep}{0ex}

 \item{\tt sottieng}~:  Notation mistakes, provided by Jean-Pierre Coulon.

\end{itemize}

\section{Compiling \texttt{musixdoc.tex}}

This manual is an excellent example of a primarily text document with embedded
musical excerpts. For this reason, it is a \LaTeX\ document and must be compiled
with \verb|latex| rather than \verb|etex|. Those wishing to combine text and 
musical excerpts should carefully study how it is done here.

Before compiling or recompiling \verb|musixdoc.tex|, you should remove all the auxiliary 
files {\tt musixdoc.[mx1\|\allowbreak mx2\|\allowbreak aux\|\allowbreak toc\|\allowbreak ind\|\allowbreak idx\|\allowbreak ilg\|\allowbreak out]} 
if they are present. Then the following command sequence will produce \verb|musixdoc.ps|:
\begin{quote}\begin{verbatim}
latex musixdoc
musixflx musixdoc
latex musixdoc
makeindex musixdoc
latex musixdoc
latex musixdoc
dvips -e0 musixdoc
\end{verbatim}\end{quote}
The initial three steps \verb|latex|\allowbreak$\to$\allowbreak\verb|musixflx|\allowbreak$\to$\allowbreak\verb|latex| build up the basic appearance of the document including musical examples.
The \verb|makeindex| step produces the database for the index.  
After that, \verb|latex| must be run at least twice to complete cross referencing.  
Finally, \verb|dvips| converts the \verb|.dvi| file into \verb|.ps|.
To produce \verb|musixdoc.pdf|, use \verb|ps2pdf| or the Distiller component of Adobe Acrobat, or open \verb|musixdoc.ps| in \texttt{GSview}, go to
{\tt File\|Convert}, select \verb|pdfwrite| at 600 dpi resolution, and click
\verb|OK|.  
Or if you have the \verb|musixtex| script, just run 
\begin{quote}
\verb|musixtex -l -x musixdoc|
\end{quote}
where the \verb|-l| option is to use \verb|latex| as the processing engine
and \verb|-x| is to create a new index database.

%If you use an older version of \LaTeX\ which doesn't automatically 
%invoke $\varepsilon$\hbox{-}\nobreak\TeX, you will encounter the 
%error ``{\tt !~No room for new \string\count}''. 
%This is because \verb|musixdoc.tex| invokes \verb|musixtex.tex|, which together with \LaTeX\ 
%requires more storage registers than available in \TeX.
%You may be able get around this by using the command 
%\verb|elatex| instead of \verb|latex|;
%however, it is strongly recommended to upgrade your \TeX\ system to a more recent 
%version in which \verb|latex| automatically invokes $\varepsilon$\hbox{-}\nobreak\TeX.   

