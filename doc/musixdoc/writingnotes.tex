 \chapter{Writing Notes}\label{autostemdirections}
There are two major kinds of note macros, those that include a space (of
length \keyindex{noteskip}) after
the printed symbol, and those that don't cause any space. A single-line melody would be
written using the first type. All notes of a chord except the last would
use the second.

Another distinction concerns stemmed notes. Some macros explicitly set the
stem direction with either ``\verb|u|'' or ``\verb|l|'' contained in the name of
the macro. On the other hand, an ``\verb|a|'' in the macro's name usually signifies
\ital{automatic} stem direction selection. In this case notes below the middle
staff line will get up stems, otherwise down.

Note pitches can be specified either by letters or numbers. If no transposition
or octaviation is in effect, letters ranging
from \verb|a| to \verb|z| represent notes starting with the A below
middle C. Upper case letters from \verb|A| to \verb|N| represent pitches
two octaves lower than their lower case counterparts. Any letter can be used in
any clef, but some users may
prefer to use the lower case letters in treble clef, and the upper case ones
in bass clef.

Alternatively, a one- or two-digit, positive or negative integer can always be used.
The number represents
the vertical position on the staff, with \verb|0| for the lowest line and
\verb|1| for the space right above, \ital{regardless of the clef}.
Unlike with letters, the
associated pitch will depend on the clef, and notes entered this way are
immune to transposition and octaviation.

Notes lower than \verb|A| and higher than \verb|z| can be entered, with
either numbers as just described, or with octaviation as will be explained in
Chapter~\ref{octaviation}.

 \section{Normal (unbeamed) spacing notes}\label{NormalNotes}
In the following, \verb|{|$p$\verb|}| signifies a pitch specification as
described above. However it
is understood that if the pitch is a single character, the brackets are
not necessary, provided that if it is a letter, a space separates the
macro from the letter.


\begin{quote}\begin{description}\setlength{\itemsep}{0ex}
 \item[\keyindex{breve}{\tt\char123}$p${\tt\char125} :]breve (\hbox to 8pt{\zbreve1\hss}) .
 \item[\keyindex{longa}{\tt\char123}$p${\tt\char125} :]longa (\hbox to8pt{\zlonga1\hss}) .
 \item[\keyindex{longaa}{\tt\char123}$p${\tt\char125} :]longa with automatic stem direction\footnote{Editor's note: Evidently there is no explicit up-stemmed longa} .
 \item[\keyindex{zmaxima}{\tt\char123}$p${\tt\char125} :]maxima(\hbox to16pt{\zmaxima1\hss}) .
 \item[\keyindex{wq}{\tt\char123}$p${\tt\char125} :]arbitrary duration note (\hbox to8pt{\zwq1\hss})
(also used as alternate representation of a \ital{breve}).
 \item[\keyindex{wqq}{\tt\char123}$p${\tt\char125} :]long arbitrary duration note
(\hbox to8pt{\zwqq1\hss}) (also used as alternate
representation of a \ital{longa}).
 \item[\keyindex{wh}{\tt\char123}$p${\tt\char125} :]whole note.
 \item[\keyindex{hu}{\tt\char123}$p${\tt\char125} :]half note with stem up.
 \item[\keyindex{hl}{\tt\char123}$p${\tt\char125} :]half note with stem down.
 \item[\keyindex{ha}{\tt\char123}$p${\tt\char125} :]half note with automatic stem direction
 \item[\keyindex{qu}{\tt\char123}$p${\tt\char125} :]quarter note with stem up.
 \item[\keyindex{ql}{\tt\char123}$p${\tt\char125} :]quarter note with stem down.
 \item[\keyindex{qa}{\tt\char123}$p${\tt\char125} :]quarter note with automatic stem direction.
 \item[\keyindex{cu}{\tt\char123}$p${\tt\char125} :]eighth note\footnote{The ``{\tt c}''
within this macro name stands for the equivalent British term ``crotchet''} with stem up.
 \item[\keyindex{cl}{\tt\char123}$p${\tt\char125} :]eighth note with stem down.
 \item[\keyindex{ca}{\tt\char123}$p${\tt\char125} :]eighth note with automatic stem direction.
 \item[\keyindex{ccu}{\tt\char123}$p${\tt\char125} :]sixteenth note with stem up.
 \item[\keyindex{ccl}{\tt\char123}$p${\tt\char125} :]sixteenth note with stem down.
 \item[\keyindex{cca}{\tt\char123}$p${\tt\char125} :]sixteenth note with automatic stem direction.
 \item[\keyindex{cccu}{\tt\char123}$p${\tt\char125} :]$32$nd note with stem up.
 \item[\keyindex{cccl}{\tt\char123}$p${\tt\char125} :]$32$nd note with stem down.
 \item[\keyindex{ccca}{\tt\char123}$p${\tt\char125} :]$32$nd note with automatic stem direction.
 \item[\keyindex{ccccu}{\tt\char123}$p${\tt\char125} :]$64$th note with stem up.
 \item[\keyindex{ccccl}{\tt\char123}$p${\tt\char125} :]$64$th note with stem down.
 \item[\keyindex{cccca}{\tt\char123}$p${\tt\char125} :]$64$th note with automatic stem direction.
 \item[\keyindex{cccccu}{\tt\char123}$p${\tt\char125} :]$128$th note with stem up.
 \item[\keyindex{cccccl}{\tt\char123}$p${\tt\char125} :]$128$th note with stem down.
 \item[\keyindex{ccccca}{\tt\char123}$p${\tt\char125} :]$128$th note with automatic stem direction.
\end{description}\end{quote}
 As an example, the sequence

 %\check
\begin{music}\nostartrule
\startextract
\Notes\cu c\cl j\en\bar
\Notes\ccu c\ccl j\en\bar
\Notes\cccu c\cccl j\en\bar
\Notes\ccccu c\ccccl j\en\bar
\Notes\cccccu c\cccccl j\en
\zendextract
\end{music}
 \noindent was coded as
\begin{quote}\begin{verbatim}
\Notes\cu c\cl j\en\bar
\Notes\ccu c\ccl j\en\bar
\Notes\cccu c\cccl j\en\bar
\Notes\ccccu c\ccccl j\en\bar
\Notes\cccccu c\cccccl j\en
\end{verbatim}\end{quote}



For \keyindex{breve} there is a possibility of
confusion with a command in \TeX's math mode that has exactly the same name.
However, there will generally be no problem, because inside
\verb|\startpiece...\endpiece|,
\verb|\breve| will take the musical meaning, while outside, it will have the
mathematical one.
If for some reason you need the math \verb|\breve| inside, you can use
\verb|\endcatcodesmusic...\breve...\catcodesmusic|.

 \section{Non-spacing note heads}
These macros are used to create chords. Any number of them can be entered
in sequence, followed by a spacing note. All of the note heads will be
joined to the spacing note and the stem length will automatically be adjusted
as needed.

\begin{quote}\begin{description}\setlength{\itemsep}{0ex}
 \item[\keyindex{zq}{\tt\char123}$p${\tt\char125} :]quarter (or shorter) note head.
 \item[\keyindex{zh}{\tt\char123}$p${\tt\char125} :]half note head.
\end{description}\end{quote}
%das ???
% \begin{remark} Notes of duration longer than whole notes are
%always non-spacing. This saves one useless definition, since these notes are
%always longer than other simultaneous ones. If needed they can be followed by
%\keyindex{sk} to force spacing.
%\end{remark}

 \section{Shifted non-spacing note heads}
These symbols are used mainly
in chords containing an interval of a \ital{second}. They provide note
heads shifted either to the left or right of the default position by
the width of one note head.

\begin{quote}\begin{description}\setlength{\itemsep}{0ex}
\item[\keyindex{rw}{\tt\char123}$p${\tt\char125} :]whole note head shifted right.
\item[\keyindex{lw}{\tt\char123}$p${\tt\char125} :]whole note head shifted left.
\item[\keyindex{rh}{\tt\char123}$p${\tt\char125} :]half note head shifted right\footnote{Some may not
have realized that half and whole note heads have different shapes}.
\item[\keyindex{lh}{\tt\char123}$p${\tt\char125} :]half note head shifted left.
\item[\keyindex{rq}{\tt\char123}$p${\tt\char125} :]quarter note head shifted right.
\item[\keyindex{lq}{\tt\char123}$p${\tt\char125} :]quarter note head shifted left.
\end{description}\end{quote}

 \section{Non-spacing notes}
These macros provide normal notes, with stems if applicable, but without any
following space.
\begin{quote}\begin{description}\setlength{\itemsep}{0ex}
 \item[\keyindex{zhu}\pitchp~:]half note with stem up but no spacing. It acts like
\verb|\hu| for chord building, i.e., it will join together any immediately
preceding non-spacing note heads.
 \item[\keyindex{zhl}\pitchp~:]half note with stem down but no spacing. It acts like
\keyindex{hl} for chord building.
 \item[\keyindex{zqu}\pitchp~:]quarter note with stem up but no spacing. It acts like
\verb|\qu| for chord building.
 \item[\keyindex{zql}\pitchp~:]quarter note with stem down but no spacing. It acts
like \verb|\ql| for chord building.
 \item[\keyindex{zcu}\pitchp, \keyindex{zccu}, \keyindex{zcccu}, 
 \keyindex{zccccu}, \keyindex{zcccccu}
:]eighth, ..., note with stem up but no spacing. They act like
\verb|\cu| for chord building.
 \item[\keyindex{zcl}\pitchp, \keyindex{zccl}, \keyindex{zcccl}, 
 \keyindex{zccccl}, \keyindex{zcccccl}
:]eighth, ..., note with stem down but no spacing. They act
like \verb|\cl| for chord building.
 \item[\keyindex{rhu}\pitchp, \keyindex{rhl}, \keyindex{rqu}, \keyindex{rql},
  \keyindex{rcu}, \keyindex{rcl} :] \verb|\rhu| acts like \verb|\zhu|,
  but the note is shifted one note width to the right; others analogous.
 \item[\keyindex{lhu}\pitchp, \keyindex{lhl}, \keyindex{lqu}, \keyindex{lql},
  \keyindex{lcu}, \keyindex{lcl} :]same
  as above, but the note is shifted one note width to the left.
 \item[\keyindex{zw}\pitchp~:]whole note with no following space.
 \item[\keyindex{zwq}\pitchp~:]arbitrary duration note
  (\hbox to8pt{\zwq1\hss}) with no following space.
 \item[\keyindex{zbreve}\pitchp~:]breve
  (\hbox to8pt{\zbreve1\hss}) with no following space.
 \item[\keyindex{zlonga}\pitchp~:]longa
  (\hbox to8pt{\zlonga1\hss}) with no following space.
 \item[\keyindex{zmaxima}\pitchp~:]maxima
  (\hbox to16pt{\zmaxima1\hss}) with no following space.
\end{description}\end{quote}
 %\check

\section{Spacing note heads}

Although not needed in normal music scores, these may be useful in
very special cases.

\begin{quote}\begin{description}\setlength{\itemsep}{0ex}
\item[\keyindex{nh}{\tt\char123}$p${\tt\char125} :]spacing half note head.
\item[\keyindex{nq}{\tt\char123}$p${\tt\char125} :]spacing quarter note head.
\end{description}\end{quote}

\noindent As an example, the sequence

\begin{music}\nostartrule
 \startextract
\notes\nq c\nq j\en\barre
\Notes\nh c\nh j\en\barre
\notes\nq {cdef}\en
\zendextract
\end{music}

\noindent was coded as

\begin{quote}\begin{verbatim}
\notes\nq c\nq j\en\barre
\Notes\nh c\nh j\en\barre
\notes\nq {cdef}\en
\end{verbatim}\end{quote}

Non spacing variants are also provided, namely
\keyindex{znh} and \keyindex{znq}.
% DAS: why???

 \section{Dotted notes}\label{dots}

By appending one or two \verb|p|'s (for ``pointed'') to the name, many of the
macros just introduced provide one or two dots after the notehead:
\keyindex{whp}\pitchp, \keyindex{whpp},
\keyindex{zwp}, \keyindex{zwpp},
\keyindex{hup}, \keyindex{hupp},
\keyindex{hlp}, \keyindex{hlpp},
\keyindex{zhp}, \keyindex{zhpp},
\keyindex{qup}, \keyindex{qupp},
\keyindex{qlp}, \keyindex{qlpp},
\keyindex{zqp}, \keyindex{zqpp},
\keyindex{cup}, \keyindex{cupp},
\keyindex{clp}, \keyindex{clpp},
\keyindex{qbp} and \keyindex{qbpp}.
Naturally, the ones that start with ``\verb|z|'' are used in chords.
The dot(s) will be raised if the note is on a line.

A more explicit way uses one of the macros \keyindex{pt}{\tt\char123}$p${\tt\char125},
\keyindex{ppt}, or \keyindex{pppt} right
before any note macro to place one to three dots after the normal note
head at pitch $p$. Again they will be raised if on a line. In fact this is the only
way to get a triple-dotted note.
For example a
quarter note with one dot could be coded \verb|\pt h\qu h|, with two dots
as \keyindex{ppt}\verb| h\qu h| and with three as
\keyindex{pppt}\verb| h\qu h|.

Yet another method for posting a dot is to insert a \ital{period} before
the letter representing the pitch. Thus \verb|\qu{.a}| is equivalent to
either \verb|\pt a\qu a| or \verb|\qup a| . This may be useful when
using \ital{collective coding}, which will be discussed in the next
section.

Non-spacing dotted notes can be produced using
\keyindex{zhup}, \keyindex{zhlp}, \keyindex{zqup}, \keyindex{zqlp},
\keyindex{zcup}, \keyindex{zclp}, \keyindex{zqbp},
and similarly with two \verb|p|'s for
double-dotted notes.

As a matter of style, if two voices share one staff, the dots in
the lower voice should be lowered if the note is on a line. For
this you can use \keyindex{lpt}{\tt\char123}$p${\tt\char125} and
\keyindex{lppt}{\tt\char123}$p${\tt\char125}.

\section{Sequences of equally spaced notes; collective coding}
\label{CollectiveCoding}

It isn't necessary to write a separate macro
sequence \verb|\notes...\en| for every individual column of notes.
Rather, a single such macro can contain all the notes in all staves
over an extended horizontal range, as long as all spacings are equal or
multiples of a unique value of \keyindex{noteskip}.
The notes in each staff could be entered one after another as normal
spacing notes as already described in Section~\ref{NormalNotes}. Then
each spacing note will cause the insertion point to advance horizontally by the
operative value of \verb|\noteskip| defined by the choice of
\verb|\notes|, \verb|\Notes|, \verb|\NOtes|, etc. Of course in such sequences
non-spacing chord notes can be entered right before their associated
spacing note. If you need to skip forward by one \verb|\noteskip|, for
example after a quarter note when there are two eighth notes in
another staff, you can use \keyindex{sk}.

If there are only spacing notes in such a sequence, a further
simplification is available, called \ital{collective coding}. For instance
\verb|\qu{cdefghij}| writes the C major scale in quarters with
up stems. Similarly \verb|\cl{abcdef^gh}| writes the
\ital{A-minor} scale in non-beamed eighths. (Here ``\verb|^|'' represents
a sharp). If necessary a void can be inserted in a collective coding sequence
by using~\verb|*|\index{*}. Not all note-generating macros can be
used to perform collective coding, but most of them can.
