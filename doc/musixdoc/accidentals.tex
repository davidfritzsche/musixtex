\chapter{Accidentals}
\vspace*{-3ex}
Accidentals can be introduced in two ways.
 The first way, using explicit macros, consists for
example in coding \keyindex{fl}\pitchp~to put a \ital{flat} at the
pitch $p$, presumably right before a note at the same pitch. This is a
non-spacing command and will automatically place the accidental an
appropriate distance to the left of the anticipated note head.
Naturals, sharps, double flats and double sharps are coded \keyindex{na}\pitchp,
\keyindex{sh}\pitchp, \keyindex{dfl}\pitchp~and \keyindex{dsh}\pitchp~respectively.

The alternate macros \keyindex{lfl}, \keyindex{lna}, \keyindex{lsh},
\keyindex{ldfl} and \keyindex{ldsh}
place the same accidentals, but shifted one note head width
to the left. These can be used if a note head has been shifted to the left, or
to avoid collision with other accidentals
in a chord. If you want to shift an accidental by some other amount for
more precise positioning, you could use \keyindex{loffset} with the normal
accidental macro as the second parameter.

 The second way of coding accidentals is to modify the parameter of a
note command. Just put the symbol
\verb|^| for a sharp, \verb|_| for a flat, \verb|=|~for a natural,
\verb|>| for a double sharp, or \verb|<| for a double
flat, right before the letter or number representing the pitch.
For example, \verb|\qb{^g}| yields a
$G\sharp$. This may be used effectively in collective coding, e.g.
\verb|\qu{ac^d}|.

 There are two sizes of accidentals. By default they will be large unless there
is not enough space between notes, in which case they will be made small. Either
size can be forced locally by coding \keyindex{bigfl}, \keyindex{bigsh}, etc., or
\keyindex{smallfl}, \keyindex{smallsh}, etc. If you want all accidentals to
be large, then declare \keyindex{bigaccid} near the top of the input file. For
exclusively small ones use \keyindex{smallaccid}; \keyindex{varaccid} will restore
variable sizes.

 For editorial purposes, small accidentals can be placed \ital{above} note
heads. This is done using \keyindex{uppersh}\pitchp, \keyindex{upperna}\pitchp, or
\keyindex{upperfl}\pitchp:

\begin{music}\nostartrule
\startextract
\NOtes\uppersh l\qa l\en
\NOtes\upperna m\qa m\en
\NOtes\upperfl l\qa l\en
\zendextract
\end{music}

\vspace*{-2ex}
It also possible to introduce \ital{\ixem{cautionary accidental}s},
i.e.,~small accidentals enclosed in parentheses. This is done by preceding
the name of the accidental keyword with ``\verb|c|''\label{cautionary}.
Available cautionary accidentals are \keyindex{csh}, \keyindex{cfl},
\keyindex{cna},
\keyindex{cdfl} and \keyindex{cdsh}, which give

\begin{music}\nostartrule
\startextract
\NOtes\csh g\qa g\en
\NOtes\cfl h\qa h\en
\NOtes\cna i\qa i\en
\NOtes\cdfl j\qa j\en
\NOtes\cdsh k\qa k\en
\zendextract
\end{music}

\vspace*{-2ex}
\noindent The distance between notes and accidentals is controlled by
\keyindex{accshift}\verb|=|\ital{any \TeX\ dimension}, where
positive values shift to the left and negative to right, with a
default of~\verb|0pt|.
For ``big'' cautionary accidentals, use, for example, 
\verb|{|\keyindex{largenotesize}\verb|\csh|\pitchp\verb|}|  or see Section~\ref{brapar}.
