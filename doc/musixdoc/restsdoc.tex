\chapter{Rests}
 \section{Ordinary rests}
 A separate macro is defined for each kind of ordinary rest. They cause
a space after the symbol, just like spacing note commands, but they have
no parameters. A whole rest is coded as
\keyindex{pause}, 
half rest \keyindex{hpause} or \keyindex{hp}, 
quarter rest \keyindex{qp} or \keyindex{soupir},
eighth rest \keyindex{ds}, sixteenth rest \keyindex{qs},
$32$nd~rest \keyindex{hs}, and $64$th rest \keyindex{qqs}.
Dotted rests may be obtained by 
using 
\keyindex{pausep}, 
\keyindex{hpausep} or \keyindex{hpp}, \keyindex{qpp}, \keyindex{dsp}, \keyindex{qsp},
\keyindex{hsp} and \keyindex{qqsp}, 
and double-dotted rests by
\keyindex{pausepp}, 
\keyindex{hpausepp} or \keyindex{hppp}, \keyindex{qppp}, \keyindex{dspp}, \keyindex{qspp},
\keyindex{hspp} and \keyindex{qqspp},.


 Longer rests, normally interpreted as lasting
two or four bars respectively, can be coded as \keyindex{PAuse}
and \keyindex{PAUSe}, which yield:

\begin{music}\nostartrule
\generalmeter{\meterfrac44}
\startextract
\def\atnextbar{\znotes\centerbar\PAuse\en}%
\NOTEs\en\bar
\def\atnextbar{\znotes\centerbar\PAUSe\en}%
\NOTEs\en
\endextract
\end{music}
 %\check


 \section{Raising rests and non-spacing rests}\index{raising rests}
All the
previous rests except \keyindex{pausep} and
\keyindex{hpausep} are \ital{hboxes}, which means that
they can be vertically offset if needed using the
standard \TeX\ command \keyindex{raise}. For example:

 \begin{quote}
 \begin{verbatim}
 \raise 2\Interligne\qp
 \raise 3mm\qqs
 \end{verbatim}
 \end{quote}

\noindent where \keyindex{Interligne} is the distance from one staff line to the
next.
Similarly, non-spacing rests may be coded using the \TeX\ command \keyindex{rlap}, as in

\begin{music}\nostartrule
\startextract
\notes\ibbu0h{-1}\zccl e\qb0h\raise-4\Interligne\rlap\qs\qb0e\en
\notes\raise-4\Interligne\rlap\ds\qb0f\tqu0g\en
\endextract
\end{music}\noindent
for which the coding is
\begin{verbatim}
\notes\ibbu0h{-1}\zccl e\qb0h\raise-4\Interligne\rlap\qs\qb0e\en
\notes\raise-4\Interligne\rlap\ds\qb0f\tqu0g\en
\end{verbatim}

In addition, two macros are available to put a whole or
half rest above or below the staff. The ordinary \verb|\pause| or
\verb|\hpause| cannot be used outside the staff because a short horizontal line
must be added to distinguish between the whole and the half rest. The commands,
which are non-spacing\footnote{Editor's note: The reason for having defined these
as non-spacing is not obvious}, are
 \begin{itemize}\setlength{\itemsep}{0ex}
 \item \keyindex{liftpause}~$n$ to get a
  \hbox to10pt{\liftpause{-2}\hss}
  raised from original position by $n$ staff line intervals,
 \item \keyindex{lifthpause}~$n$ to get
  \hbox to10pt{\lifthpause{-1}\hss} raised the same way.
 \item \keyindex{liftpausep}~$n$ to get a
  \hbox to10pt{\liftpausep{-2}\hss}
  raised from original position by $n$ staff line intervals,
 \item \keyindex{lifthpausep}~$n$ to get
  \hbox to10pt{\lifthpausep{-1}\hss} raised the same way.
 \end{itemize}
 %\check

 \section{Bar-centered rests}\label{barcentered}
Sometimes it is necessary to place a rest (or some other symbols) exactly in the middle
of a bar. This can usually be done with combinations of the commands
\keyindex{atnextbar},
\keyindex{centerbar},
\keyindex{cPAUSe},
\keyindex{cPAuse},
\keyindex{cpause},
\keyindex{chpause},
as demonstrated in the following example:

\begin{music}
\generalmeter\meterC
\setclef1\bass
\setstaffs1{2}
\parindent0pt
\startextract
\NOtes|\qa{cegj}\en
\def\atnextbar{\znotes\centerbar{\cpause}|\en}\bar
\NOTes\ha{Nc}|\en
\def\atnextbar{\znotes|\centerbar{\cpause}\en}\bar
\NOTEs|\en
\def\atnextbar{\znotes\centerbar{\cPAuse}|\centerbar{\cPAuse}\en}%
\endextract
\end{music}
\noindent with the coding
\begin{verbatim}
\generalmeter\meterC
\setclef1\bass
\setstaffs1{2}
\parindent0pt
\startextract
\NOtes|\qa{cegj}\en
\def\atnextbar{\znotes\centerbar{\cpause}|\en}\bar
\NOTes\ha{Nc}|\en
\def\atnextbar{\znotes|\centerbar{\cpause}\en}\bar
\NOTEs|\en
\def\atnextbar{\znotes\centerbar{\cPAuse}|\centerbar{\cPAuse}\en}%
\endextract
\end{verbatim}
The following abbreviations are provided:
\begin{quote}
\begin{tabular}{lcl}
\keyindex{centerpause}& for & \verb|\centerbar{\cpause}|\\
\keyindex{centerhpause}& for & \verb|\centerbar{\chpause}|\\
\keyindex{centerPAuse}& for & \verb|\centerbar{\cPAuse}|\\
\keyindex{centerPAUSe}& for & \verb|\centerbar{\cPAUSe}|
\end{tabular}
\end{quote}
Material other than these rests may be used as arguments to
\verb|\centerbar|, as in:\\ 
\begin{music}
\generalmeter\meterC
\parindent0pt
\startextract
\NOtes\ql{ghij}\en
\def\atnextbar{\znotes\centerbar{\liftpause2}\en}\bar
\NOTes\en
\def\atnextbar{\znotes\centerbar{\duevolte}\en}\bar
\NOTes\en
\def\atnextbar{\znotes\loffset{0.6}{\centerbar{\Fermataup l\wh j}}\en}\bar
\NOTEs\en\def\atnextbar{\znotes\centerbar{\ccn{9}{\meterfont7}}%
\centerbar{\cPAUSe\off{2.5\elemskip}\cPAuse\off{2.5\elemskip}\cpause}\en}\bar
\NOTEs\Hpause4{0.83}\en
\def\atnextbar{\znotes\centerbar{\ccn{9}{\meterfont10}}\en}\endextract
\end{music}
\noindent for which the coding is:
\pagebreak%
\begin{verbatim}
\generalmeter\meterC
\parindent0pt
\startextract
\NOtes\ql{ghij}\en
\def\atnextbar{\znotes\centerbar{\liftpause2}\en}\bar
\NOTes\en
\def\atnextbar{\znotes\centerbar{\duevolte}\en}\bar
\NOTes\en
\def\atnextbar{\znotes\loffset{0.6}{\centerbar{\Fermataup l\wh j}}\en}\bar
\NOTEs\en\def\atnextbar{\znotes\centerbar{\ccn{9}{\meterfont7}}%
\centerbar{\cPAUSe\off{2.5\elemskip}\cPAuse\off{2.5\elemskip}\cpause}\en}\bar
\NOTEs\Hpause4{0.83}\en
\def\atnextbar{\znotes\centerbar{\ccn{9}{\meterfont10}}\en}\endextract
\end{verbatim}
Note that, for the off-center command
\verb|\wh|\ldots,
it has been necessary to use \keyindex{loffset} to correct the centering. 

