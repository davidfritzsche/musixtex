\chapter{Changing Clefs, Keys and Meters}

\section{Introduction}\label{contextintro}

To define clefs, key signatures, or meters at the start of a piece,
or to change one or more of these attributes\footnote{In this section,
\ital{attribute} will refer to any clef, key signature, or meter.}~anywhere
else in a score,
\musixtex\ requires two steps. The first step is to \ital{set} the new values of
the attributes.
Most of the commands for this have the form \verb|\set|...~. They will be
described in the following subsections.

But this alone will
not cause anything to be changed or printed. The second step is to activate the
change. This is done by issuing one of the
following commands (outside \verb|\notes|$\dots$\verb|\en|):
\verb|\startpiece|, \verb|\startextract|, \verb|\contpiece|,
\verb|\Contpiece|,
\verb|\alaligne|, \verb|\alapage|, \verb|\zalaligne|, \verb|\zalapage|,
\verb|\changecontext|,
\verb|\Changecontext|, \verb|\zchangecontext|, \verb|\changesignature|,
\verb|\changeclefs|, or \verb|\zchangeclefs|. Most of these perform
other functions as well, and some may be used even when no attributes
change. Features unrelated to changing attributes are detailed elsewhere.
The first eleven will activate all pending new attributes.
If more than one type is activated by a single command in this manner,
then regardless of the order
they were set, they will always appear in the following order: clef, key signature,
meter.
The last three commands in the above list obviously activate only
the specific type of attribute referred to in the name of the command.

The macros \verb|\changecontext|, \verb|\Changecontext|, \verb|\zchangecontext|
will respectively insert a single, double, or invisible bar
line before printing the attributes.

\section{Key Signatures}

We've already seen in Section~\ref{whatspecify} how to set key
signatures for all instruments with \keyindex{generalsignature}, or for
specific instruments with \keyindex{setsign}. As just noted, these commands only
prepare for the insertion of the signatures into the score; it is
really \verb|\startpiece| that puts them in place at the beginning of the
score.

The commands \verb|\generalsignature| and \verb|\setsign| also serve to set
new key signature(s) anywhere in score. The change can then be activated with one of
the eleven general commands listed above, or with \keyindex{changesignature} if
in the middle of a bar.
While neither \verb|\changesignature| nor \verb|\zchangecontext| prints a bar
line, the differences are that the latter increments the bar number counter and
inserts a horizontal space of \verb|\afterruleskip| after the new signature(s).
All of these options will repost signatures that have not been changed.

Normally, changing a signature from flats to sharps or vice-versa, or
reducing the number of sharps or flats, will produce the appropriate set of
naturals to indicate what has been suppressed. This standard feature can be
temporarily inhibited by the command \keyindex{ignorenats} issued right before
the change-activating command.

Here is an example showing various possibilities for changing key signatures.
Note the comments between the code lines.

\begin{quote}\begin{verbatim}
\instrumentnumber2\setstaffs22%
\setclef1{\bass}\generalsignature2%
\startextract
\Notes\qu K&\qu d|\qu e\en
% Signature change in a single instrument with two staves.
% Naturals appear by default, indicating cancelled sharps.
\setsign20\changesignature
\Notes\qu J&\qu d|\qu e\en
% When changing signature in the middle of a bar and no naturals
% are posted, the new signature can be confused with a simple accidental.
\setsign11\ignorenats\changesignature
\Notes\qu M&\qu d|\qu e\en
% New signatures after a double bar line
\generalsignature{-2}\Changecontext%
\Notes\qu K&\qu d|\qu e\en%
\Notes\qu K&\qu d|\qu e\en%
% New signatures after an invisible bar line. Note the
% difference in spacing compared with beat 3 of the prior measure
\generalsignature{1}\zchangecontext%
\Notes\qu K&\qu d|\qu e\en%
\end{verbatim}\end{quote}

\begin{music}
\instrumentnumber2\setstaffs22%
\setclef1{\bass}\generalsignature2%
\startextract
\Notes\qu K&\qu d|\qu e\en
% Signature changing in a single instrument with 2 staves
% Naturals are allowed to shown there are no more sharps
\setsign20\changesignature
\Notes\qu J&\qu d|\qu e\en
% Signature changing in a single staff without naturals
% if there is no bar line, the signatures are confusing:
% it is not clear if they are for a single note or not
\setsign11\ignorenats\changesignature
\Notes\qu M&\qu d|\qu e\en
% New signatures after a double bar line
\generalsignature{-2}\Changecontext%
\Notes\qu K&\qu d|\qu e\en%
\Notes\qu K&\qu d|\qu e\en%
% New signatures after an invisible bar line
% see the difference in space comparing beat 2
\generalsignature{1}\zchangecontext%
\Notes\qu K&\qu d|\qu e\en%
\endextract
\end{music}


\section{Clefs}\label{treblelowoct}

Macros that define clefs have already been discussed in Section~\ref{whatspecify}. By way of review, here are all of the possible clefs
(applied to the lowest staff):
%avrb
\newcommand{\musicintextsign}[1]{\musicintext#1{\notes\en}}

\begin{center}\vskip-1ex\footnotesize%\small
\begin{tabular}{||c|c|c|c|c||}\hline\hline
&&&&\\
\verb+\setclef1\treble+&&&\verb+\setclef1\alto+&\\
\verb+\setclef10+&\verb+\setclef11+&\verb+\setclef12+
&\verb+\setclef13+&\verb+\setclef14+\\[-2ex]%\hline
\musicintextsign{\treble}&\musicintextsign1&\musicintextsign2
&\musicintextsign{\alto}&\musicintextsign4\\[3ex]\hline
%\end{tabular}
&&&&\\
%\begin{tabular}{ccccc}
&\verb+\setclef1\bass+&&\verb+\setclefsymbol1\empty+\footnotemark&\\
\verb+\setclef15+&\verb+\setclef16+&\verb+\setclef17+
&\verb+\setclef18+&\verb+\setclef19+\\[-2ex]%\hline
\musicintextsign5&\musicintextsign{\bass}&\musicintextsign7
&\musicintextsign8&\musicintextsign9\\[3ex]\hline\hline
%This one uses \setclefsymbol1\empty :
%&\musicintextnoclef{\notes\qu h\en}&\musicintextsign9\\[3ex]\hline\hline
\end{tabular}
\footnotetext{Details of the macro {\tt\Bslash setclefsymbol} will be discussed later}
\end{center}
%avre

Just as with key signatures, these commands only \ital{prepare} for clef changes.
To \ital{activate} them, any of the first eleven  commands listed in Section~\ref{contextintro} could be used. However, one should keep in mind that
according to modern conventions, a clef change at a bar line is posted
before the bar line, while for example \verb|\changecontext| would post it
after the bar line. In part for this reason, we have the special command
\keyindex{changeclefs}. It can be used anywhere outside
\verb+\notes...\en+ to activate a clef change and insert an amount of
horizontal space to accommodate the new clef symbol(s), without printing a
bar line. Sometimes no added
space is required, in which case \keyindex{zchangeclefs} should be used.

Here are some examples of clef changes:

\begin{quote}\begin{verbatim}
\instrumentnumber2\setstaffs22%
\setclef1{\bass}\generalsignature2%
\startextract
% Change in one staff only, with added space
\setclef1\treble\changeclefs%
\Notes\qu k&\qu e|\cu{.d}\ccu{e}\en%
% Combined with signature change, also no extra space needed
% twice the same clef in staff 2 - with the help of a blank clef
\setclef28\zchangeclefs\setclef2\treble%
\setclef1\bass\zchangeclefs\setsign1{-2}\setsign2{-2}%
\ignorenats\changesignature%
\Notes\qu K&\cu{de}|\qu e\en%
% clef change before barline
\setclef1\treble\zchangeclefs\bar%
\Notes\qu k&\cu{de}|\qu e\en%
% clef change after barline
\setclef1\bass\bar\changeclefs%
\Notes\qu K&\cu{de}|\qu e\en%
% clef change after barline with changecontext
\setclef1\treble\changecontext%
\Notes\cu k&\cu d|\qu e\en%
% twice the same clef
\setclef18\zchangeclefs\setclef1\treble\changeclefs%
\Notes\cu k&\cu e |\en%
\zendextract
\end{verbatim}\end{quote}
\begin{music}
\instrumentnumber2\setstaffs22%
\setclef1{\bass}\generalsignature2%
\startextract
% Change in one staff only, with added space
\setclef1\treble\changeclefs%
\Notes\qu k&\qu e|\cu{.d}\ccu{e}\en%
% Combined with signature change, also no extra space needed
% twice the same clef in staff 2 - with the help of a blank clef
\setclef28\zchangeclefs\setclef2\treble%
\setclef1\bass\zchangeclefs\setsign1{-2}\setsign2{-2}%
\ignorenats\changesignature%
\Notes\qu K&\cu{de}|\qu e\en%
% clef change before barline
\setclef1\treble\zchangeclefs\bar%
\Notes\qu k&\cu{de}|\qu e\en%
% clef change after barline
\setclef1\bass\bar\changeclefs%
\Notes\qu K&\cu{de}|\qu e\en%
% clef change after barline with changecontext
\setclef1\treble\changecontext%
\Notes\cu k&\cu d|\qu e\en%
% twice the same clef
\setclef18\zchangeclefs\setclef1\treble\changeclefs%
\Notes\cu k&\cu e |\en%
\endextract
\end{music}

\noindent Of course the examples in the last two bars are contrary
to accepted practice.

Clef changes initiated with the \verb|\setclef| command have several features
in common. When activated after the beginning of the piece, the printed symbol
is smaller than the normal one used at the beginning of the piece. Also,
\musixtex\ automatically adjusts vertical positions of noteheads consistent
with the new clef.

There is an additional group of macros for setting new clefs which does not share
either of these features. In other words, they will always print full sized symbols, and they
won't change the vertical positions of noteheads from what they would have been
before the new symbol was printed. We could call this process ``clef symbol substitution'',
because all it does is print a different symbol (or no symbol at all) in place of
the underlying clef which was set in the normal way.

You'll need to use clef symbol substitution if you want to have a so-called
\index{octave clefs}\label{octclef}octave treble clef or octave bass
clef, i.e., one containing a numeral 8 above or
below the normal symbol. The syntax for setting upper
octaviation
is \keyindex{setbassclefsymbol}\onen\keyindex{bassoct}\\
or \keyindex{settrebleclefsymbol}\onen\keyindex{trebleoct}; for lower octaviation
it is\\ \keyindex{setbassclefsymbol}\onen\keyindex{basslowoct} or
\keyindex{settrebleclefsymbol}\onen\keyindex{treblelowoct}. Because these sequences act to
\ital{replace} normal bass or treble clefs with a different symbol, they
require that the normal clefs be set first. For example

\noindent\begin{minipage}{80mm}
\begin{music}\nostartrule
 \parindent 19mm
 \instrumentnumber{4}
 \generalmeter{\empty}
 \setclef1\bass \setclef2\bass \setclef3\treble \setclef4\treble
 \setbassclefsymbol1\basslowoct
 \setbassclefsymbol2\bassoct
 \settrebleclefsymbol3\treblelowoct
 \settrebleclefsymbol4\trebleoct
\startextract
\Notes\qu{`abcdefghi}&\qu{`abcdefghi}&\qu{abcdefghi}&\qu{abcdefghi}&\en
\zendextract
\end{music}
\end{minipage}
\begin{minipage}{50mm}
\begin{verbatim}
\parindent 19mm
\instrumentnumber{4}
\generalmeter{\empty}
\setclef1\bass \setclef2\bass
\setclef3\treble \setclef4\treble
\setbassclefsymbol1\basslowoct
\setbassclefsymbol2\bassoct
\settrebleclefsymbol3\treblelowoct
\settrebleclefsymbol4\trebleoct
\startextract
\Notes\qu{`abcdefghi}&\qu{`abcdefghi}%
&\qu{abcdefghi}&\qu{abcdefghi}&\en
\zendextract
\end{verbatim}
\end{minipage}

\index{clefs (empty)}
Another application of clef symbol substitution is to cause no clef to be
printed, as for example might be desired in percussion music, This
can be accomplished with \keyindex{setclefsymbol}\onen\verb|\empty|, which
once activated would replace \ital{all} clef symbols in the first (lowest)
staff of instrument $n$ with blanks.

Normal symbols for those clefs that have been substituted can be restored by
\keyindex{resetclefsymbols}.

Four other small clef symbols are available: \verb|\smalltrebleoct|, 
\verb|\smalltreblelowoct|, \verb|\smallbassoct|, and \verb|\smallbasslowoct|. 
They look just like the corresponding normal-sized symbols, and are useful by
clef symbol substitution for clef changes after the beginning of a piece, as
demonstrated in the following example.

The various clef symbol substitution commands can only be used to substitute for
\texttt{treble}, \texttt{alto}, or \texttt{bass} clefs.

In the following example, (1) is two normal clef changes. At (2) the clef is first changed
back to treble and then the \verb|\treblelowoct| symbol is substituted by using
\verb|\settrebleclefsymbol|.
When changing the clef away from treble and then back as at (3), the substitution
symbol is still in force. At (4), \verb|\resetclefsymbols| cancels the symbol substitution.
If using \verb|\setclefsymbol| all available clefs are changed to the same symbol, as you can see
in the three clefs after (5) in comparison with (2). These also illustrate the use of the small
octave clef symbol. Obviously the second clef after (5) is
nonsense; \verb|\resetclefsymbols| puts matters in order at (6) and (7).

\begin{music}
\instrumentnumber1\setclef1\bass
\startpiece
\notes\zchar{-5}{1}\qu H\en\setclef1\treble\changeclefs
\notes\qu i\en\setclef1\bass\changeclefs\bar
\notes\zchar{-5}{2}\qu J\en
\setclef1\treble\settrebleclefsymbol1\treblelowoct\changeclefs
\notes\qu i\en\setclef1\bass\changeclefs\bar
\notes\zchar{-5}{3}\qu I\en\setclef1\treble\changeclefs
\notes\qu i\en\setclef1\bass\changeclefs\bar
\notes\zchar{-5}{4}\qu J\en\resetclefsymbols\setclef1\treble\changeclefs
\notes\qu i\en\setclef1\bass\changeclefs\doublebar
\notes\zchar{-5}{5}\qu J\en
\setclef1\treble\setclefsymbol1\smalltreblelowoct\changeclefs
\notes\qu i\en\setclef1\bass\changeclefs\bar
\notes\qu I\en\setclef1\treble\changeclefs
\notes\zchar{-5}{6}\qu i\en\resetclefsymbols\setclef1\bass\changeclefs\bar
\notes\zchar{-5}{7}\qu J\en\setclef1\treble\changeclefs
\notes\qu i\en
\endpiece
\end{music}

\noindent This is the code:

\begin{quote}\begin{verbatim}
\begin{music}\nostartrule
\instrumentnumber1\setclef1\bass
\startpiece
\notes\zchar{-5}{1}\qu H\en\setclef1\treble\changeclefs
\notes\qu i\en\setclef1\bass\changeclefs\bar
\notes\zchar{-5}{2}\qu J\en
\setclef1\treble\settrebleclefsymbol1\treblelowoct\changeclefs
\notes\qu i\en\setclef1\bass\changeclefs\bar
\notes\zchar{-5}{3}\qu I\en\setclef1\treble\changeclefs
\notes\qu i\en\setclef1\bass\changeclefs\bar
\notes\zchar{-5}{4}\qu J\en\resetclefsymbols\setclef1\treble\changeclefs
\notes\qu i\en\setclef1\bass\changeclefs\doublebar
\notes\zchar{-5}{5}\qu J\en
\setclef1\treble\setclefsymbol1\smalltreblelowoct\changeclefs
\notes\qu i\en\setclef1\bass\changeclefs\bar
\notes\qu I\en\setclef1\treble\changeclefs
\notes\zchar{-5}{6}\qu i\en\resetclefsymbols\setclef1\bass\changeclefs\bar
\notes\zchar{-5}{7}\qu J\en\setclef1\treble\changeclefs
\notes\qu i\en
\endpiece
\end{verbatim}\end{quote}

\section{Meter changes}

As mentioned in Section~\ref{generalmeter}, a common \itxem{meter} for all
staves can be specified by \keyindex{generalmeter}\verb|{|$m$\verb|}|,
where $m$ denotes the meter. On the other hand, meter changes in specific
staves are implemented with
\keyindex{setmeter}\onen\verb|{{|$m1$\verb|}{|$m2$\verb|}{|$m3$\verb|}{|$m4$\verb|}}|,
where $n$ is the number of the instrument, $m1$ specifies the meter
of the first (lowest) staff, $m2$ the second staff, and so forth. (Only enter
as many $m$'s as necessary.)

Since meter changes are meaningful only across bars, there is no special command to
activate a new meter; rather, they are activated with the general commands
\keyindex{changecontext}, etc., listed in Section~\ref{contextintro}.

The next example shows a few methods to get a meter change, in all staves or
in a single staff.

\begin{small}
\begin{quote}\begin{verbatim}
\instrumentnumber2\setstaffs22%
\generalmeter{\meterfrac{4}{4}\meterfrac{2}{4}\meterfrac{1}{4}}%
\setclef1{\bass}\generalsignature2%
\startextract
\setmeter1{{\meterfrac{2}{4}}}%
\setmeter2{{\lower2pt\hbox{\meterfrac{\meterlargefont 2}{}}}%
{\meterfrac{3}{4}}}\changecontext
\Notes\qu K&\cu{de}|\qu e\en
% bar 11
% Meters, clefs, and key signatures.
% All 3 clefs after bar (probably bad form) if no changeclefs
\setmeter1{{\meterfrac{2}{8}}}%
\setmeter2{{\meterfrac{3}{6}}{\meterfrac{3}{8}}}%
\setsign2{-1}%
% How to force showing the bass clef?
\setclef1\bass\setclef2{23}%
\Changecontext
\Notes\qu K&\cu{de}|\qu e\en
% bar 12
% Meters, clefs, and key signatures, with clef before the bar.
% Maybe not best form if signatures are involved
\setmeter1{{\meterfrac{2}{4}}}%
\setmeter2{{\meterfrac{3}{8}}{\meterfrac{3}{6}}}%
\setsign2{-1}%
\setclef1\treble\zchangeclefs\changecontext
\Notes\qu k&\cu{de}|\qu e\en
\end{verbatim}\end{quote}
\end{small}
\begin{music}
\instrumentnumber2\setstaffs22%
\generalmeter{\meterfrac{4}{4}\meterfrac{2}{4}\meterfrac{1}{4}}%
\setclef1{\bass}\generalsignature2%
\startextract
\setmeter1{{\meterfrac{2}{4}}}%
% How big the '2' must be?
\setmeter2{{\lower2pt\hbox{\meterfrac{\meterlargefont 2}{}}}%
{\meterfrac{3}{4}}}\changecontext
\Notes\qu K&\cu{de}|\qu e\en
% bar 11
% Meter Clefs and Key Signatures
% all 3 clefs after bar if no changeclefs
\setmeter1{{\meterfrac{2}{8}}}%
\setmeter2{{\meterfrac{3}{6}}{\meterfrac{3}{8}}}%
\setsign2{-1}%
% How to force showing the bass clef?
\setclef1\bass\setclef2{23}%
\Changecontext
\Notes\qu K&\cu{de}|\qu e\en
% bar 12
% Meter Clefs Keys Signatures all 3 with clef before the bar
% probably not if signatures are involved
\setmeter1{{\meterfrac{2}{4}}}%
\setmeter2{{\meterfrac{3}{8}}{\meterfrac{3}{6}}}%
\setsign2{-1}%
\setclef1\treble\zchangeclefs\changecontext
\Notes\qu k&\cu{de}|\qu e\en
\endextract
\end{music}
