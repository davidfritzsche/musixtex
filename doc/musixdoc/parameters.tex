\chapter{Layout Parameters}\label{LayoutParameters}
 Most layout parameters are set by \musixtex\ to reasonable default values.
However, some projects will require altering one or more of them. In this
section we discuss the most important parameters and how to change them.

 \section{List of layout parameters}
 In the following, the indication ``\nochange'' does not mean that this
parameter cannot be changed at all, but that it should not be modified
directly, e.g,.~by saying something like \verb|\Interligne=14pt|. In other
words, changes in these parameters must be accomplished only by more
comprehensive macros which not only revise them but at the same time perform
other necessary related changes. Even though you cannot \ital{change} these,
you may \ital{refer} to them in your coding if that proves useful.

\begin{description}\setlength{\itemsep}{0ex}
 \item[\keyindex{Interligne} :]vertical distance between the bottoms of
consecutive staff lines of the current instrument, taking no account of a
possible alteration by \keyindex{setsize}; \nochange.
 \item[\keyindex{internote} :]vertical spacing between notes one scale
step apart in the current instrument, taking account of a possible alteration
by \keyindex{setsize}; \nochange.
Note that each distinct instrument may have a different
\verb|\internote| (see Chapter~\ref{staffspacing}).
 \item[\keyindex{Internote} :]vertical spacing between notes one scale
step apart in any instrument whose \keyindex{setsize} has the default
value \keyindex{normalvalue} ($1.0$), equal to \verb|0.5\Interligne|; \nochange.
\item[\keyindex{staffbotmarg} :]margin below the first (lowest) staff of the
first (lowest) instrument. Changes are recognized at the next system.
Default is \verb|3\Interligne|.
 \item[\keyindex{stafftopmarg} :]margin above the last (uppermost) staff of
last (uppermost) instrument. Changes are recognized at the next system.
Default is \verb|3\Interligne|.
 \item[\keyindex{interbeam} :]vertical distance between beams; \nochange.
\item[\keyindex{interstaff} :]a very important macro with a single numerical
argument representing the factor that multiplies \verb|2\internote| to give the
distance between the bottom of one staff and the bottom of the next one. In
fact the macro redefines the parameter \verb|\interfacteur|. Default is 9.
\verb|\interstaff| applies to all the
instruments; to set the inter-staff spacing for a single instrument,
use \keyindex{setinterstaff}\itbrace{n}\itbrace{m},
$n$ is the instrument and $m$ is the replacement factor for that instrument.
Note that
after you have used \keyindex{setinterstaff},
you cannot reset the distances for that instrument with
\keyindex{interstaff}; you must subsequently use
\keyindex{setinterstaff} for that purpose.
\item[\keyindex{interportee} :]distance between the bottom of one staff and
the bottom of the next one. It is always reset to
2 times \keyindex{internote} times the \keyindex{interfacteur} for the current
instrument at the next system.
Therefore, trying to change \verb|\interportee| will have no effect. Change
\verb|\interstaff| or use \keyindex{setinterstaff} instead. 
\item[\keyindex{interinstrument} :]additional vertical distance between
two consecutive instruments. This means that the distance between the lowest
line of the previous instrument and the lowest line of the top staff of the current
instrument is \verb|\interportee+\interinstrument|. The
default value of \verb|\interinstrument| is zero, but sometimes you
may want additional space between distinct instruments. This is a general
dimension register. As usual in \TeX, it can be set using a command such as
\verb|\interinstrument=10pt| or \verb|\interinstrument=6\internote|. Its
value can be overridden for the space above any particular instrument
with the macro \keyindex{setinterinstrument}\itbrace{n}\itbrace{s}, where
$n$ is the instrument and $s$ is the replacement value of the space to
be added. The \verb|\setinterinstrument| macro may be useful in some vocal
scores to provide vertical space for lyrics. 
Note that
after you have used \keyindex{setinterinstrument},
you cannot reset the distances for that instrument with
\keyindex{interinstrument}; you must subsequently use
\keyindex{setinterinstrument} for that purpose.

 \item[\keyindex{systemheight} :]distance from the bottom of the
lowest staff to the top of the highest one. This is the length of any
vertical lines such as repeats that span the full height of a system;
\nochange.
\end{description}

 In addition, when handling notes of a given staff of a given instrument, the
following dimensions are available (note these are not true registers, but
\ital{equivalenced symbols} through a \verb|\def|):

\begin{description}\setlength{\itemsep}{0ex}
 \item[\keyindex{altplancher} :]altitude of the lowest line of the lowest
instrument; \nochange.
 \item[\keyindex{altitude} :]altitude of the lowest line of the lowest
staff of the current instrument; \nochange.
 \item[\keyindex{altportee} :]altitude of the lowest line of the current
staff; \nochange.

 %??? Next perhaps will change to steps of \verb|\internote|.

 \item[\keyindex{stemfactor} :]parameter defining the length of stems on
half, quarter, and beamed eighth notes, in units of \verb|\interbeam|.
Normally a stem has the length of one octave,
i.e.,~\verb|3.5|\keyindex{Interligne}. However, this is not correct for small or tiny
note sizes. Therefore, stem length is defined as a multiple of the
dimension \keyindex{interbeam}, which is chosen because it is automatically
redefined as a
different multiple of \keyindex{Interligne} whenever note size is changed. For
example, with \verb|\normalmusicsize| when \verb|\setsize| is
\verb|\normalvalue|, \verb|\interbeam| is \verb|0.75|\keyindex{Interligne}. This
legislates a default value for \keyindex{stemfactor} of $4.66$ ($=3.5/0.75$).
To change stem length, it is easiest to use e.g.,~\keyindex{stemlength}\verb|{3.5}|, which simply redefines \verb|\stemfactor|.
Subsequently, \verb|\stemfactor| will not automatically be reset to the default,
but keep in mind that if it is changed
inside a notes group, the change will only be effective within that notes group.
\end{description}

 \section{A convenient macro for changing layout parameters}
Of the parameters just described that can be changed, many should only be
changed between the end of one system and the beginning of the next. The
command sequence \verb|\def\atnextline{|\ital{any control sequence}\verb|}|
may be useful for this purpose. It will cause \ital{any control sequence} to
be inserted right before the next new line is begun, provided the line break
is not initiated by a \verb|\startpiece|. Thus this will work with
automatically generated line breaks, with those forced by \verb|\alaligne|, and
with those forced by explicit use of \verb|\endpiece| or \verb|\stoppiece|
followed by \verb|\contpiece|. (Note, however, that in the latter case it
would be just as convenient to enter the parameter changes explicitly as well.)
The control sequence will only be executed once,
after which \verb|\atnextline| is redefined as \verb|\empty|.

 \section{Changing the number of lines per staff}\label{stafflinenumber}
Naturally, the default number of lines per staff is five. But you may want a
different number in some or all staves, for example for gregorian music,
percussion music, guitar tablature, or early baroque keyboard music. To do so,
use the command \keyindex{setlines}\itbrace{n}\itbrace{m}~where $n$ is the
instrument number and $m$ is the number of lines.

 \section{Resetting normal layout parameters}
The general size can only be changed with one of the commands
\keyindex{smallmusicsize},
\keyindex{normalmusicsize}, \keyindex{largemusicsize}, or
\keyindex{Largemusicsize}. Beyond that, the command
\keyindex{resetlayout} will reset the following key layout parameters
to their default values: \keyindex{staffbotmarg} (\verb|3\Interligne|),
\keyindex{stafftopmarg} (\verb|3\Interligne|), \keyindex{interstaff} ($9$),
 number of lines per staff for all instruments ($5$); and will reset all clef
symbols to standard clef symbols.

 \section{Split and Shortened Systems}
It is sometimes desirable to typeset a ``shortened'' system or to split
a system into left and right segments, as in the following (extracted from a 
Geminiani \emph{concerto grosso}):

\begin{music}
\smallmusicsize
\setclef1\bass
\setclef5\treble\settrebleclefsymbol5\trebleoct
\generalsignature{-2}
\instrumentnumber5
\songbottom{1}\songtop{5}
\setinterinstrument4{1\internote}
\nobarnumbers
\generalmeter{\meterfrac38}
\hbox to\hsize\bgroup\vbox\bgroup\hsize=1.7in  
\startpiece\addspace\afterruleskip
\Notes\qa G\sk&\ibu1i{-1}\qb1i\qb1{.g}&\ca d\qa g&\ca n\itieu3n\qa n&\ca g\ds\en
\notes\ds\sk&\sk\tbbu1\tqu1g&\sk\sk&\sk\sk\ttie3&\ds\sk\en
\znotes&&&&\lcn{q}{\bf D.S. al\kern3ex}\coda p\en
\setdoublebar\endpiece
\egroup\hfill\vbox\bgroup\hsize=3.7in         
\generalmeter{\allabreve}
\startpiece\addspace\afterruleskip
\znotes&&&&\coda o\en
\NOtes\qa{.G}&\qa{.k}&\qa{.d}&\qa n&\qa g\en
\notes\sk\sk\qs\cca N&\sk\sk\qs\cca i&\sk\sk\qs\cca k&\qs\ibbl3p{-2}\qb3{p}\qb3{o}\tql3n&\qp\sk\sk\sk\en
\NOtes\qa{.M}&\qa{.h}&\qa{.k}&\qa o&\ha k\en
\Notes\sk&\sk&\sk&\itieu3r\qa r&\sk\en
\notes\qs\cca M&\qs\cca h&\qs\cca k&\sk\sk&\sk\sk\en
\bar
\NOtes\wh L\sk&\qa{g}\qa{g}&\ha{l}\sk&\ttie3\ha r\sk&\ha{g}\sk\en
\NOTes\sk&\ha j&\ha{n}&\tr s\ha q&\ha{n}\en
\bar
\nspace
\NOtesp\ha{.K}&\ha{.h}&\ha{.k}&\ha{.r}&\ha{.^m}\en
\setdoublebar\endpiece
\egroup\egroup                                  
\end{music}

This was coded as follows:
\begin{footnotesize}
\begin{quote}\begin{verbatim}
\setclef1\bass\setclef5\treble\settrebleclefsymbol5\trebleoct
\generalsignature{-2}
\instrumentnumber5\songbottom{1}\songtop{5}
\nobarnumbers
\generalmeter{\meterfrac38}
\hbox to\hsize\bgroup\vbox\bgroup\hsize=1.7in % specify length of left segment
\startpiece\addspace\afterruleskip
\Notes\qa G\sk&\ibu1i{-1}\qb1i\qb1{.g}
  &\ca d\qa g&\ca n\itieu3n\qa n&\ca g\ds\en
\notes\ds\sk&\sk\tbbu1\tqu1g
  &\sk\sk&\sk\sk\ttie3&\ds\sk\en
\znotes&&&&\lcn{q}{\bf D.S. al\kern3ex}\coda p\en
\setdoublebar\endpiece
\egroup\hfill\vbox\bgroup\hsize=3.7in         % specify length of right segment
\generalmeter{\allabreve}
\startpiece\addspace\afterruleskip
\znotes&&&&\coda o\en
\NOtes\qa{.G}&\qa{.k}&\qa{.d}&\qa n&\qa g\en
\notes\sk\sk\qs\cca N&\sk\sk\qs\cca i
  &\sk\sk\qs\cca k
  &\qs\ibbl3p{-2}\qb3{p}\qb3{o}\tql3n
  &\qp\sk\sk\sk\en
\NOtes\qa{.M}&\qa{.h}&\qa{.k}&\qa o&\ha k\en
\Notes\sk&\sk&\sk&\itieu3r\qa r&\sk\en
\notes\qs\cca M&\qs\cca h&\qs\cca k
  &\sk\sk&\sk\sk\en
\bar
\NOtes\wh L\sk&\qa{g}\qa{g}&\ha{l}\sk
  &\ttie3\ha r\sk&\ha{g}\sk\en
\NOTes\sk&\ha j&\ha{n}&\tr s\ha q&\ha{n}\en
\bar
\nspace
\NOtesp\ha{.K}&\ha{.h}&\ha{.k}&\ha{.r}&\ha{.^m}\en
\setdoublebar\endpiece
\egroup\egroup                                % close open \bgroups 
\end{verbatim}\end{quote}
\end{footnotesize}
