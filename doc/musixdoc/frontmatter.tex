\title{\Huge\bfseries\musixtex\\[\bigskipamount]
\LARGE\bfseries Using \TeX{} to write polyphonic\\or
instrumental music\\\Large\itshape Version 1.27}


%\author{\Large\rm Daniel \sc Taupin\\\large\sl
% Laboratoire de Physique des Solides\\\normalsize\sl
% (associ\'e au CNRS)\\\normalsize\sl
% b\^atiment 510, Centre Universitaire, F-91405 ORSAY Cedex\\\medskip
% \\\Large\rm Ross \sc Mitchell\\\large\sl
% CSIRO Division of Atmospheric Research,\\\normalsize\sl
% Private Bag No.1, Mordialloc, Victoria 3195,\\ Australia \\\medskip
%   \\\Large\rm Andreas \sc Egler\\\large\sl
%   (Ruhr--Uni--Bochum)\\ Ursulastr. 32\\ D-44793 Bochum}
\date{Revised \today}
\maketitle
\vfill
\thispagestyle{empty}
\begin{flushright}\it
If you are not familiar with \TeX{} at all, I would recommend\\
to find another software package to do musical typesetting.\\
Setting up \TeX{} and \musixtex\ on your machine and mastering it is\\
an awesome job which gobbles up a lot of your time and disk space.\\[\medskipamount]
But, once you master it\ldots\\[\smallskipamount]
Hans {\sc Kuykens} (ca.~1995)

\vspace*{4ex}
In my humble opinion, that whole statement is obsolete.\\[\smallskipamount]
Christof\/ {\sc Biebricher} (2006) 
\end{flushright}

\clearpage

\pagenumbering{roman}\setcounter{page}{2}


\vspace*{20ex}
\begin{quote}
\musixtex{} may be freely copied, duplicated and used in conformance to
the GNU General Public License (Version 2, 1991, see included file {\tt
copying})\footnote{Thanks to the Free Software Foundation for advice. See
\href{http://www.gnu.org}{\underline{\tt http://www.gnu.org}}}.

You may take it or parts of it to include in other packages, but no packages
called \musixtex{} without specific suffix may be distributed under the name
\musixtex{} if different from the original distribution (except obvious bug
corrections).

 Adaptations for specific implementations (e.g., fonts) should be provided as
separate additional \TeX\ or \LaTeX\ files which override original definitions.
 \end{quote}

\clearpage

\chapter*{Preface}
\addcontentsline{toc}{chapter}{Preface}
\musixtex\ was developed by Daniel Taupin, Ross Mitchell and Andreas
Egler,
building
on earlier work by Andrea Steinbach and Angelika Schofer.
Unfortunately, Daniel Taupin, the main developer, died all too early in a 2003
climbing accident. The \musixtex\ community was shocked by this tragic and unexpected
event. You may read tributes to Daniel Taupin that are archived at the
\href{http://icking-music-archive.org}{\underline{Werner Icking Music Archive}}.

Since then, the only significant update to \musixtex\ has been in version 1.15 (April 2011) which
takes advantage of the greater capacity of the e\TeX\ version of \TeX. 
This manual
is the definitive reference to all features of
\musixtex\ version~1.27.

Novice users need not start here.
Most 
music typesetting tasks can be accomplished entirely by using the \textbf{PMX}
(for instrumental music) or \mbox{\textbf{M-Tx}} (for vocal music)
preprocessors to generate the \musixtex\ input file, relieving the user of
learning any of the commands or syntax of \musixtex\ itself. It is only
for out-of-the-ordinary constructions that one must learn these details in order
to insert the necessary \musixtex\ commands into the preprocessor's input file
as so-called inline \TeX.

Advanced users might
want to use \verb|autosp|, another preprocessor. It simplifies
the production of \musixtex\ scores by automating the choice of
note-spacing commands; see Section~\ref{autosp}.

It is possible to create inputs for \musixtex, \textbf{PMX}, \mbox{\textbf{M-Tx}} or
\texttt{autosp} using any text editor, such as \verb|notepad| or \verb|vi| or
\verb|emacs|. Currently there is no graphical user interface for
generating input files, other than \TeX-oriented integrated
graphical development environments such as \TeX Works.

The \href{http://icking-music-archive.org/software/indexmt6.html}
{\underline{Werner Icking Music Archive}}\ (WIMA) contains excellent and detailed
instructions for installing \TeX, \musixtex{} and the strongly recommended
preprocessors \textbf{PMX}
and \mbox{\textbf{M-Tx}} on
\href{http://icking-music-archive.org/software/htdocs/Getting_Started_Four_Scenar.html#SECTION00022000000000000000}
{\underline{Linux/\unix}}, 
\href{http://icking-music-archive.org/software/htdocs/Getting_Started_Four_Scenar.html#SECTION00021000000000000000}
{\underline{Windows}} and 
\href{http://icking-music-archive.org/software/htdocs/Getting_Started_Four_Scenar.html#SECTION00023000000000000000}
{\underline{Mac OS}}. 
See
\href{http://icking-music-archive.org/software/htdocs/Introduction.html#SECTION00012000000000000000}
{\underline{this}}
page at WIMA for documentation of 
\textbf{PMX} and \mbox{\textbf{M-Tx}}.

\begin{flushright}
Oliver Vogel\\ Don Simons\\ Andre van Ryckeghem\\ Cornelius Noack\\
Hiroaki Morimoto\\ 
Bob Tennent\\ \today
\end{flushright}

\clearpage

\makeatletter
\renewcommand\tableofcontents{%
    \if@twocolumn
      \@restonecoltrue\onecolumn
    \else
      \@restonecolfalse
    \fi
    \chapter*{\contentsname
        \@mkboth{%
           \contentsname}{\contentsname}}%
    \addcontentsline{toc}{chapter}{Contents}  %  added by RDT
    \@starttoc{toc}%
    \if@restonecol\twocolumn\fi
    }
\makeatother


%\begin{small}
\tableofcontents 
%\end{small}

\clearpage
\setcounter{page}{1}
\pagenumbering{arabic}
\renewcommand{\thepage}{\arabic{page}}
