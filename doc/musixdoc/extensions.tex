\chapter{Extension Library}
All of the following files are invoked by saying \keyindex{input} \ital{filename}\ .
Most of them are fully compatible with \musixtex\ in that they do not redefine
any existing macros but rather provide additional functionality. In future
versions of \musixtex\ we may very well incorporate many of them directly into
\verb|musixtex.tex|, but for now we leave them separate.

 \section{musixadd}\ixtt{musixadd.tex}
Increases the number of instruments, slurs and beams from six to nine.

 \section{musixadf}
\input musixadf
This file and the associated fonts (by Hiroaki \textsc{Morimoto})
define some additional and alternative musical accents, as follows.
\begin{description}
\item[small swells, larger sforzatos:]
\ \\\begin{music}
\nostartrule
\startextract
\NOTes
  \ccharnote{-8}{\Bslash ulsw}\ulsw l\ql l%
  \ccharnote{13}{\Bslash llsw}\llsw f\qu f%
\en\bar
\NOtes
  \ccharnote{-8}{\Bslash uSfz}\uSfz l\ql l%
  \ccharnote{13}{\Bslash lSfz}\lSfz f\qu f\qsk%
\en
\endextract
\end{music}
\item[breaths:]
\ \\
\begin{music}
\nostartrule
\startextract
\NOTes
  \zcharnote{-8}{\Bslash cuBreath}\cuBreath \qu f%
  \zcharnote{13}{\Bslash clBreath}\clBreath \qu f%
\en\bar
\NOtes
  \ccharnote{-8}{\Bslash zuBreath}\zuBreath \qu f%
  \ccharnote{13}{\Bslash zlBreath}\zlBreath \qu f%
\en
\endextract
\end{music}
\item[coda with serifs, upright segno:]
\ \\
\begin{music}
\nostartrule
\startextract
\NOTes
  \sk
  \ccharnote{-4}{\Bslash scoda}\scoda n\sk\sk%
  \ccharnote{-4}{\Bslash upsegno}\upsegno n\sk%
\en
\endextract
\end{music}
\end{description}
The following commands will replace the default specified accents by the new ones
defined in \texttt{musixadf.tex}:
  \verb|\replacesforzato|, 
  \verb|\replacecoda|,
  \verb|\replacesegno|.

 \section{musixbm}
This file does nothing; it is retained only for compatibility with 
\musixtex{} version 1.10 or earlier. 
Since version 1.11, \verb|musixtex.tex| itself contains all the functions 
of the older \texttt{musixbm}, namely 
commands for 128th notes, either with flags or with beams:
\keyindex{ibbbbbu},
\keyindex{ibbbbbl},
\keyindex{nbbbbbu},
\keyindex{nbbbbbl},
\keyindex{tbbbbbu},
\keyindex{tbbbbbl},
\keyindex{Ibbbbbu},
\keyindex{Ibbbbbl},
\keyindex{cccccu},
\keyindex{cccccl},
\keyindex{ccccca},
\keyindex{zcccccu} and
\keyindex{zcccccl}.


 \section{musixbbm}\label{musixbbm}
 Provides $256$th notes, but only for use within beams, via the commands
\keyindex{ibbbbbbu},
\keyindex{ibbbbbbl},
\keyindex{nbbbbbbu},
\keyindex{nbbbbbbl},
\keyindex{tbbbbbbu},
\keyindex{tbbbbbbl},
\keyindex{Ibbbbbbu} and
\keyindex{Ibbbbbbl}.

By default \texttt{musixbbm} provides six $256$th beams with 
reference number $0$ to $5$.  
You can specify a larger maximum number 
directly with \keyindex{setmaxcclvibeams}\verb|{|$m$\verb|}| within the
range\footnote{This may require e-\TeX.} $7\leq m\leq 100$.

\section{musixcho}\label{song}
 Enables certain macros intended for choral music\footnote{Remember
that we now recommend using \texttt{musixlyr} for any except the
simplest lyrics. The extension \texttt{musixcho} is only for those
diehards who choose to ignore this advice}. Provides the following commands:
\keyindex{biglbrace}, \keyindex{bigrbrace}, \keyindex{braceheight},
\keyindex{Dtx} and \keyindex{Drtx} for two-line text, \keyindex{Ttx} and
\keyindex{Trtx} for three-line text, \keyindex{Qtx} and \keyindex{Qrtx} for
four-line text.  To eliminate zigzagging lyrics lines, all multiple line texts
are automatically vertically justified with the macro \keyindex{ChroirStrut},
defined as \verb|\vphantom{\^Wgjpqy}|.

The macros \keyindex{tx}\verb|{|$text$\verb|}|,
\keyindex{rtx}\verb|{|$text$\verb|}| cause song text to be left-justified
on the insertion point rather then centered.
 \keyindex{hf}\verb|{|$m$\verb|}| sets a text
continuation rule of length $m$ \verb|\noteskip|.

Consult the source file {\tt musixdoc.tex} to see the coding of the following
example:

\begin{music}
\ninerm
\parindent9mm
\braceheight5.4\Interligne\relax
%\rightline{Arr.: H.~W.~Eichholz}
\instrumentnumber2
\generalsignature{-2}
\generalmeter{\meterfrac22}
\setclef1\bass
\relativeaccid
\setinterinstrument1{11\Interligne}
\songtop2
\songbottom1
\setname2{\vbox{\hsize\parindent\centerline{Sop}\centerline{Alt}}}
\setname1{\vbox{\hsize\parindent\centerline{Ten}\centerline{Bass}}}
\sepbarrules\nobarnumbers
\beforeruleskip-2pt
\startpiece
%% bar 1
\znotes&\rlap{\kernm2em\Qrtx 1.|2.|3.|4.*}\en
\NOTes\sslur ILd1\sslur bNu1\zhl I\hu b%
  &\Qtx\ixhf{Oh}|No|No|There'll*\issluru0f\sslur dad1\zhup f\hl d\en
\Notes\zhl L\hu N&\Qtx ~|more|more|~be*\hl{^c}\en
\NOtes&\tsslur0g\qu g\en
%%% bar 2
\bar
\NOtes\zql I\qu b&\Qrtx\thf freedom,|weepin',|moanin',|singin',*\zqu i\ql d\en
\NOTesp\lpt I\zhl I\hup b&\zhup i\hlp d\en
%%% bar 3
\bar
\nspace
\NOTes\sslur JMd1\sslur bau1\zhl J\hu b%
  &\Qtx\ixhf{oh}|no|no|there'll*\issluru0j\sslur edd1\zhup j\hl{^e}\en
\Notes\zhl M\hu a&\Qtx~|more|more|~be*\hl{=e}\en
\NOtes&\tsslur0k\qu k\en
%%% bar 4
\bar
\NOtes\zql I\qu b&\Qrtx\thf freedom,|weepin',|moanin',|singin',*\zqu j\ql d\en
\NOTesp\lpt I\zhl I\hup b&\zhup i\hlp d\en
%%% bar 5
\bar
\NOTesp\sslur IJd1\sslur bau1\lpt I\zhl I\hup b%
  &\Qtx\ixhf{oh}|no|no|~there'll*\isslurd0f\issluru1j\zhup k\hlp f\en
\NOtes\zql J\qu a&\Qtx~|more|more|~be*\tsslur1f\zqu f\roff{\tsslur0e\ql e}\en
%%% bar 6
\bar
\NOtes\zql K\qu b&\Qrtx\thf freedom,|weepin',|moanin',|singin',*\zqu i\ql f\en
\NOtes\zql J\qu b&\zqu i\ql e\en
\NOtes\zql I\qu b&\rlap{\kernm1\Internote\bigrbrace}\rtx~~~~over*\zqu i\ql d\en
\NOtes\zql G\qu b&\zqu k\ql f\en
%%% bar 7
\bar
\nspace
\NOTes\zhl J\hu b\caesura&\tx ~me,*\zhu j\hl{^e}\caesura\en
\Notes\zcl J\cu b&\rtx ~over*\zcu j\cl e\en
\NOtesp\zqlp J\qup b&\zqup k\lpt e\ql e\en
\endpiece
\end{music}

\section{musixcpt}

Empowers \musixtex\ to run files created under
Musi\textbf{c}TeX, the predecessor of \musixtex, such as some of the examples
provided by Daniel {\sc Taupin}. It is not needed for any files created
under \musixtex, and is included mainly for historical completeness.

\section{musixdat}
Enables the command \keyindex{today}, which sets the current date in one of
several possible languages. The language is selected by an optional
 preparatory command \verb|\date...|.
The default is \keyindex{dateUSenglish}, but this can changed, either at the
end of \ttxem{musixdat.tex} for a permanant change, or right before issuing
\verb|\today|. Available choices and sample results are summarized below:
\smallskip
\begin{quote}\begin{tabular}{ll}\hline
\verb|\dateUSenglish|&\dateUSenglish\today\\
\verb|\dateaustrian|&\dateaustrian\today\\
\verb|\dateenglish|&\dateenglish\today\\
\verb|\datefrench|&\datefrench\today\\
\verb|\dategerman|&\dategerman\today\\\hline
\end{tabular}\end{quote}

\section{musixdbr}
Enables dashed and dotted bar lines (see Section~\ref{musixdbr}).

\section{musixdia}\label{diam}
Enables notes with diamond-shaped heads as follows:
\begin{itemize}\setlength{\itemsep}{0ex}
 \item Solid note heads (\raise.5ex\hbox to .6em{\musixchar37}) are obtained
using the macros
 \keyindex{yqu}, \keyindex{yqup},
\keyindex{yqupp}, \keyindex{yql}, \keyindex{yqlp}, \keyindex{yqlpp},
\keyindex{yzq}, \keyindex{yzqp}, \keyindex{yzqpp}, \keyindex{yqb},
\keyindex{ycu}, \keyindex{yccu}, \keyindex{ycccu}, \keyindex{yccccu},
\keyindex{ycl}, \keyindex{yccl}, \keyindex{ycccl}, \keyindex{yccccl},
\keyindex{ycup}, \keyindex{ycupp}, \keyindex{yclp}, \keyindex{yclpp}.
(Think of d{\it y}\kern.5pt amond). A solid diamond with no stem is obtained
with \keyindex{ynq} (spacing) or \keyindex{yznq} (non-spacing).
 \item Open note heads (\raise.5ex\hbox to .6em{\musixchar38}) are obtained
using the macros
\keyindex{dqu}, \keyindex{dqup},
\keyindex{dqupp}, \keyindex{dql}, \keyindex{dqlp}, \keyindex{dqlpp},
\keyindex{dzq}, \keyindex{dzqp}, \keyindex{dzqpp}, \keyindex{dqb},
\keyindex{dcu}, \keyindex{dccu}, \keyindex{dcccu}, \keyindex{dccccu},
\keyindex{dcl}, \keyindex{dccl}, \keyindex{dcccl}, \keyindex{dccccl},
\keyindex{dcup}, \keyindex{dcupp}, \keyindex{dclp}, \keyindex{dclpp}.
(Think of {\it d}\kern.5pt iamond).
An open diamond with no stem is obtained
with \keyindex{dnq} (spacing) or \keyindex{dznq} (non-spacing).
 \end{itemize}

One use of these note heads is for a string part with \itxem{harmonic notes}.
%(see \ref{othernotes})
% DAS removed description which was misplaced in the samll=notes section 2.18 4/30/06
For example,

\begin{music}
\parindent0pt
\generalsignature{-2}
\generalmeter\allabreve
\startextract
\NOTes\dzq o\zh d\hu h\en
\Notes\ibu0k0\zq g\yqb0k\qb0j\zq e\yqb0i\tbu0\qb0j\en
\bar
\NOTes\dzq g\hu k\en
\NOTes\hpause\en
\bar
\NOTes\dzq o\zh d\hl h\en
\Notes\ibl0e0\zq g\yqb0k\qb0j\zq e\yqb0i\tbl0\qb0j\en
\bar
\NOTes\dzq g\hu k\en
\NOTes\hpause\en
\endextract
\end{music}
\noindent was coded as follows:
\begin{quote}\begin{verbatim}
\generalsignature{-2}
\generalmeter\allabreve
\startextract
\NOTes\dzq o\zh d\hu h\en
\Notes\ibu0k0\zq g\yqb0k\qb0j\zq e\yqb0i\tbu0\qb0j\en
\bar
\NOTes\dzq g\hu k\en
\NOTes\hpause\en
\bar
\NOTes\dzq o\zh d\hl h\en
\Notes\ibl0e0\zq g\yqb0k\qb0j\zq e\yqb0i\tbl0\qb0j\en
\bar
\NOTes\dzq g\hu k\en
\NOTes\hpause\en
\endextract
\end{verbatim}\end{quote}

Another use is for percussion parts. In fact the file \verb|musixdia.tex|
is automatically loaded if you input {\tt musixper.tex} (see Section~\ref{perc}).

\section{musixec}\label{ecfonts}
 This package will replace the \texttt{OT1}-encoded Computer Modern fonts
by \texttt{T1}-encoded EC versions.  
Use as follows:
\begin{quote}\begin{verbatim}
\input musixtex
\input musixec
...
\end{verbatim}\end{quote}


 \section{musixeng}
 This package is provided for music typesetters who are allergic to the default
rest names, which are
taken from French, German or Italian. It does not provide new features, only
new command names:
\medskip
 \begin{quote}\begin{tabular}{ll}\hline
\ital{original}&\ital{alternate}\\\hline
\keyindex{PAUSe}&\keyindex{Qwr}\\
\keyindex{PAuse}&\keyindex{Dwr}\\
\keyindex{liftpause}&\keyindex{liftwr}\\
\keyindex{pausep}&\keyindex{wrp}\\
\keyindex{pause}&\keyindex{wr}\\
\keyindex{lifthpause}&\keyindex{lifthr}\\
\keyindex{hpausep}&\keyindex{hrp}\\
\keyindex{hpause}&\keyindex{hr}\\
\keyindex{qp}&\keyindex{qr}\\
\keyindex{ds}&\keyindex{er}\\
\keyindex{qs}&\keyindex{eer}\\
\keyindex{hs}&\keyindex{eeer}\\
\keyindex{qqs}&\keyindex{eeeer}\\\hline
\end{tabular}\end{quote}

\section{musixext}

This file does nothing; it is retained only for compatibility with 
\musixtex{} version 1.17 or earlier. 
Since version 1.18, \verb|musixtex.tex| itself contains all the functions 
of the older \texttt{musixext}, namely 
definitions of 
\begin{itemize}
\item
\keyindex{raggedstoppiece}~, which inhibits right-justification of the last
line of a score.
\item
\keyindex{slide}\itbrace{p}\itbrace{x}\itbrace{s}~, which provides a glissando
starting at pitch {\it p} and extending for {\it x} \verb|\internote|s with
slope {\it s} (ranging from $-8$ to $8$).
\end{itemize}

 \section{musixf{}l{}l}
 \input musixfll
 Enables modification of \ixem{ledger lines}. Ledger lines normally exceed the
width of a note head by $25$ percent in each direction. If the space between
notes is insufficient, the ledger lines of
consecutive notes may meet, creating visual ambiguities. Therefore,
\musixtex{} shortens the
ledger lines if notes are set so close together that the ledger lines may
meet. But because \musixtex{} does not know whether consecutive notes need
ledger lines, this automatic shortening may be superfluous. The extension
file \ttxem{musixfll.tex} allows this feature to be switched off and on.
Upon inputting \verb|musixfll.tex|, the automatic shortening of ledger lines
is switched off. From then on, it may be switched on again using
\keyindex{autoledgerlines} and switched off again using
\keyindex{longledgerlines}. Both macros have global effect.

 The following example shows that narrowly set scales look better with
\keyindex{autoledgerlines} (the default behavior), whereas single notes
requiring ledger lines look better with \keyindex{longledgerlines}.

\medskip
\begin{music}
\nostartrule
\startextract
\notes\multnoteskip{0.7}\Uptext{autoledgerlines}\autoledgerlines
    \ibbbu0b0\qb0{cba`gfg'a}\tbu0\qb0b%
    \ibbbl0{''b}0\qb0{abcdedc}\tbl0\qb0b%
    \ibbbu0{``b}0\qb0{dad}\tbu0\qb0a%
    \ibbbl0{''b}0\qb0{`g'c`g}\tbl0\qb0{'c}\en
\endextract
\medskip
\startextract
\notes\multnoteskip{0.7}\Uptext{longledgerlines}\longledgerlines
    \ibbbu0b0\qb0{cba`gfg'a}\tbu0\qb0b%
    \ibbbl0{''b}0\qb0{abcdedc}\tbl0\qb0b%
    \ibbbu0{``b}0\qb0{dad}\tbu0\qb0a%
    \ibbbl0{''b}0\qb0{`g'c`g}\tbl0\qb0{'c}\en
\endextract
\autoledgerlines
\end{music}

 \section{musixgre}\label{gregnotes}\index{gregorian chant}

% \section{Gregorian chant: staffs and clefs}
Gregorian chant is often coded using four line staffs
(see
%sections \ref{gregorian} and
Section~\ref{stafflinenumber}) and using special notes called
\itxem{neumes} (which are described later in this section). It also requires
special clefs. One way to substitute them for the modern ones is for example
with commands like

\keyindex{setaltoclefsymbol}\verb|3\gregorianCclef|

\noindent or

\keyindex{setbassclefsymbol}\verb|3\gregorianFclef| ,

\noindent which will cause instrument number~$3$ to display the selected gregorian
clef. The standard clefs can be restored for every instrument with
\keyindex{resetclefsymbols}. Note that when using this method you must
specify whether to substitute for the bass or alto clef
symbol (there is no treble clef in gregorian
chant). The reason is that \musixtex\ selects and
raises the F and C clefs differently, according to the arguments of the
\keyindex{setclef} command. Therefore, if one had substituted any F clef
symbol while saying \verb|\setclef1{1000}|, then an F clef would duly appear
on the staff, but it would be set at the position of an alto clef, thus
seriously misleading the musician.

Another method of clef substitution employs \keyindex{setclefsymbol} (see
Section~ \ref{treblelowoct}), which substitutes the clef given by
the second argument \ital{for all clef symbols} of the instrument given by the first,
regardless of the actual musical meaning of the new clef symbol. This method is
generally appropriate only if you want to change the clef symbol(s) of
an instrument for the whole of the score.

 As an example, the same gregorian scale has been written with a gregorian C
clef on all four lines of the staff:

 \begin{music}\nostartrule
 \parindent 19mm
 \instrumentnumber{4}
 \setname1{1st line} \setname2{2nd line} 
 \setname3{3rd line} \setname4{4th line}
 \setlines1{4}\setlines2{4}\setlines3{4}\setlines4{4}
 \sepbarrules
 \generalmeter{\empty}
 \setclef1{1000} \setclef2{2000} \setclef3{3000} \setclef4{4000}
 \setaltoclefsymbol1\gregorianCclef
 \setaltoclefsymbol2\gregorianCclef
 \setaltoclefsymbol3\gregorianCclef
 \setaltoclefsymbol4\gregorianCclef
\startextract
\Notes\squ{abcdefghi}&\squ{abcdefghi}&\squ{abcdefghi}&\squ{abcdefghi}&\en
\zendextract
\end{music}

 The coding was:
 \begin{verbatim}
 \instrumentnumber{4}
 \setname1{1st line} \setname2{2nd line} 
 \setname3{3rd line} \setname4{4th line}
 \setlines1{4}\setlines2{4}\setlines3{4}\setlines4{4}
 \sepbarrules
 \generalmeter{\empty}
 \setclef1{1000} \setclef2{2000} \setclef3{3000} \setclef4{4000}
 \setaltoclefsymbol1\gregorianCclef
 \setaltoclefsymbol2\gregorianCclef
 \setaltoclefsymbol3\gregorianCclef
 \setaltoclefsymbol4\gregorianCclef
\startextract
\Notes\squ{abcdefghi}&\squ{abcdefghi}&\squ{abcdefghi}&\squ{abcdefghi}&\en
\zendextract
 \end{verbatim}

All of the special gregorian symbols available in \musixtex\ are described
in the following subsections.

\subsection{Clefs}

\begin{itemize}\setlength{\itemsep}{0ex}
 \item Gregorian C clef: \raise 2.5pt\hbox to 1cm{\gregorianCclef\hfil}~=
 \keyindex{gregorianCclef}, normally activated for instrument $n$ with the
command
 \keyindex{setaltoclefsymbol}\itbrace{n}\keyindex{gregorianCclef}
\item Gregorian F clef:  \raise 2.5pt\hbox to 1cm{\gregorianFclef\hfil}~=
\keyindex{gregorianFclef}, normally activated with the command
\keyindex{setbassclefsymbol}\itbrace{n}\keyindex{gregorianFclef}
 \end{itemize}

 \subsection{Elementary symbols}


 \begin{itemize}\setlength{\itemsep}{0ex}

 \item Diamond shaped \itxem{punctum} (This has a different shape compared to the
percussion diamond): \raise 2.5pt\hbox{\xgregchar1}~ =
\keyindex{diapunc}\pitchp\ .
 \item Square \itxem{punctum}: \raise 2.5pt\hbox{\xgregchar5}~ =
\keyindex{squ}\pitchp\ or \keyindex{punctum}\pitchp\ .
 \item Left stemmed \itxem{virga} (not in the 1905 gregorian standard): \raise
2.5pt\hbox{\xgregchar6}~ = \keyindex{lsqu}\pitchp\ .
 \item Right stemmed \itxem{virga}: \raise 2.5pt\hbox{\xgregchar7}~ =
\keyindex{rsqu}\pitchp\ or \keyindex{virga}\pitchp\ .
 \item \ital{Apostropha}\index{apostropha}: \raise 2.5pt\hbox{\xgregchar3}~ =
\keyindex{apostropha}\pitchp\ .
 \item \ital{Oriscus}\index{oriscus}: \raise 2.5pt\hbox{\xgregchar10}~ =
\keyindex{oriscus}\pitchp\ .

 \item \ital{Quilisma}\index{quilisma}: \raise 2.5pt\hbox{\xgregchar125}~ =
\keyindex{quilisma}\pitchp\ .
 \item \ital{Punctum auctum} (up)\index{punctum auctum}: \raise
2.5pt\hbox{\xgregchar9}~ =
\keyindex{punctumauctup}\pitchp\ .
 \item \ital{Punctum auctum} (down)\index{punctum auctum}: \raise
2.5pt\hbox{\xgregchar8}~ =
\keyindex{punctumauctdown}\pitchp\ .
 \item Diamond shaped \ital{punctum auctum} (down)\index{punctum auctum}:
\raise 2.5pt\hbox{\xgregchar2}~ = \keyindex{diapunctumauctdown}\pitchp\ .
 \item \ital{Punctum deminutum}\index{punctum deminutum}: \raise
2.5pt\hbox{\xgregchar4}~ = \keyindex{punctumdeminutum}\pitchp\ .
 \item \ital{Apostropha aucta}\index{apostropha aucta}: \raise
2.5pt\hbox{\xgregchar11}~ = \keyindex{apostropha aucta}\pitchp\ .

\end{itemize}
 All non-\ital{liquescens} symbols have non-spacing variants, namely
\keyindex{zdiapunc}, \keyindex{zsqu}, \keyindex{zlsqu}, \keyindex{zrsqu},
\keyindex{zapostropha} and \keyindex{zoriscus}.

\subsection{Plain complex neumes}
Other \itxem{neumes} can be obtained by combining two or more of these
symbols. Since \itxem{neumes} have a special note head width, an additional
shifting macro is provided, namely \keyindex{groff}. It is similar to
\verb|\roff|, but the offset is smaller. For use with comples neumes, another
shifting macro is provided, namely \keyindex{dgroff}, which causes an
offset twice the offset of \verb|\groff|.

 Since most of these symbols depend on relative pitches of their components,
we cannot provide all possible compact combinations as single symbols. The ones
that are available in \verb|musixgre| are described below. In the following,
$p_1$, $p_2$, $p_3$, and $p_4$ represent pitches specified as usual. Please refer to
the source file \verb|musixtex.tex| if you wish to see the coding of those
examples for which it is not quoted here.

\def\twop{\itbrace{p_1}\itbrace{p_2}}
\def\threep{\twop\itbrace{p_3}}
\def\fourp{\threep\itbrace{p_4}}

\begin{description}\setlength{\itemsep}{0ex}
 \item[\keyindex{bivirga}\twop], for example:

 \begin{music}\nostartrule
 \elemskip 10pt
 \setsize1{\Largevalue}
 \instrumentnumber 1
 \setstaffs 1 1
 \setlines 1 4
 \setclef 1{3000}
 \setaltoclefsymbol 1 \gregorianCclef
 \startextract
 \notes \bivirga ab\en
 \notes \bivirga cc\en
 \zendextract
 \end{music}

  This example was coded as:
  \begin{quote}\begin{verbatim}
 \instrumentnumber 1
 \setstaffs 1 1
 \setlines 1 4
 \setclef 1{3000}
 \setaltoclefsymbol 1 \gregorianCclef
 \startextract
 \notes \bivirga ab\en
 \notes \bivirga cc\en
 \zendextract
  \end{verbatim}\end{quote}

 \item[\keyindex{trivirga}\threep], for example:

 \begin{music}\nostartrule
 \elemskip 10pt
 \setsize1{\Largevalue}
 \instrumentnumber 1
 \setstaffs 1 1
 \setlines 1 4
 \setclef 1{3000}
 \setaltoclefsymbol 1 \gregorianCclef
 \startextract
 \Notes \trivirga abc\en
 \Notes \trivirga cca\en
 \zendextract
 \end{music}

 \item[\keyindex{bistropha}\twop], for example:

 \begin{music}\nostartrule
 \elemskip 10pt
 \setsize1{\Largevalue}
 \instrumentnumber 1
 \setstaffs 1 1
 \setlines 1 4
 \setclef 1{3000}
 \setaltoclefsymbol 1 \gregorianCclef
 \startextract
 \notes \bistropha ab\en
 \notes \bistropha cc\en
 \zendextract
 \end{music}

 \item[\keyindex{tristropha}\threep], for example:

 \begin{music}\nostartrule
 \elemskip 10pt
 \setsize1{\Largevalue}
 \instrumentnumber 1
 \setstaffs 1 1
 \setlines 1 4
 \setclef 1{3000}
 \setaltoclefsymbol 1 \gregorianCclef
 \startextract
 \Notes \tristropha abc\en
 \Notes \tristropha cca\en
 \zendextract
 \end{music}

  \item[\keyindex{clivis}\twop], for example:

 \begin{music}\nostartrule
 \elemskip 10pt
 \setsize1{\Largevalue}
 \instrumentnumber 1
 \setstaffs 1 1
 \setlines 1 4
 \setclef 1{3000}
 \setaltoclefsymbol 1 \gregorianCclef
 \startextract
 \notes \clivis ba\en
 \notes \clivis ca\en
 \zendextract
 \end{music}

 \item[\keyindex{lclivis}\twop], for example:

 \begin{music}\nostartrule
 \setsize1{\Largevalue}
 \elemskip 10pt
 \instrumentnumber 1
 \setstaffs 1 1
 \setlines 1 4
 \setclef 1{3000}
 \setaltoclefsymbol 1 \gregorianCclef
 \startextract
 \notes \lclivis ba\en
 \notes \lclivis ca\en
 \zendextract
 \end{music}

 \item[\keyindex{podatus}\twop], for example:

 \begin{music}\nostartrule
 \elemskip 10pt
 \instrumentnumber 1
 \setsize1{\Largevalue}
 \setstaffs 1 1
 \setlines 1 4
 \setclef 1{3000}
 \setaltoclefsymbol 1 \gregorianCclef
 \startextract
 \notes \podatus ab\en
 \notes \podatus ac\en
 \notes \podatus cf\en
 \zendextract
 \end{music}

 \item[\keyindex{podatusinitiodebilis}\twop], for example:

 \begin{music}\nostartrule
 \elemskip 10pt
 \instrumentnumber 1
 \setsize1{\Largevalue}
 \setstaffs 1 1
 \setlines 1 4
 \setclef 1{3000}
 \setaltoclefsymbol 1 \gregorianCclef
 \startextract
 \notes \podatusinitiodebilis ab\en
 \notes \podatusinitiodebilis ac\en
 \notes \podatusinitiodebilis cf\en
 \zendextract
 \end{music}

 \item[\keyindex{lpodatus}\twop], for example:

 \begin{music}\nostartrule
 \elemskip 10pt
 \setsize1{\Largevalue}
  \instrumentnumber 1
 \setstaffs 1 1
 \setlines 1 4
 \setclef 1{3000}
 \setaltoclefsymbol 1 \gregorianCclef
 \startextract
 \notes \lpodatus ab\en
 \notes \lpodatus ce\en
 \zendextract
 \end{music}

 \item[\keyindex{pesquassus}\twop], for example:

 \begin{music}\nostartrule
 \elemskip 10pt
 \instrumentnumber 1
 \setsize1{\Largevalue}
 \setstaffs 1 1
 \setlines 1 4
 \setclef 1{3000}
 \setaltoclefsymbol 1 \gregorianCclef
 \startextract
 \notes \pesquassus ab\en
 \notes \pesquassus ae\en
 \zendextract
 \end{music}

 \item[\keyindex{quilismapes}\twop], for example:

 \begin{music}\nostartrule
 \elemskip 10pt
 \instrumentnumber 1
 \setsize1{\Largevalue}
 \setstaffs 1 1
 \setlines 1 4
 \setclef 1{3000}
 \setaltoclefsymbol 1 \gregorianCclef
 \startextract
 \notes \quilismapes ab\en
 \notes \quilismapes ae\en
 \zendextract
 \end{music}

  \item[\keyindex{torculus}\threep], for example:

 \begin{music}\nostartrule
 \elemskip 10pt
 \instrumentnumber 1
 \setsize1{\Largevalue}
 \setstaffs 1 1
 \setlines 1 4
 \setclef 1{3000}
 \setaltoclefsymbol 1 \gregorianCclef
 \startextract
 \notes \torculus aba\en
 \notes \torculus cfd\en
 \notes \torculus afc\en
 \zendextract
 \end{music}

  \item[\keyindex{torculusinitiodebilis}\threep], for example:

 \begin{music}\nostartrule
 \elemskip 10pt
 \instrumentnumber 1
 \setsize1{\Largevalue}
 \setstaffs 1 1
 \setlines 1 4
 \setclef 1{3000}
 \setaltoclefsymbol 1 \gregorianCclef
 \startextract
 \notes \torculusinitiodebilis aba\en
 \notes \torculusinitiodebilis cfd\en
 \notes \torculusinitiodebilis afc\en
 \zendextract
 \end{music}

 \item[\keyindex{Porrectus}\threep], for example:

 \begin{music}\nostartrule
 \elemskip 10pt
 \instrumentnumber 1
 \setsize1{\Largevalue}
 \setstaffs 1 1
 \setlines 1 4
 \setclef 1{3000}
 \setaltoclefsymbol 1 \gregorianCclef
 \startextract
 \notes \Porrectus bab\en
 \notes \Porrectus bac\en
 \notes \Porrectus bNd\en
 \notes \Porrectus bMe\en
 \notes \Porrectus bLe\en
 \zendextract
 \end{music}
\noindent coded:
 \begin{quote}\begin{verbatim}
 \notes \Porrectus bab\en
 \notes \Porrectus bac\en
 \notes \Porrectus bNd\en
 \notes \Porrectus bMe\en
 \notes \Porrectus bLe\en
 \end{verbatim}\end{quote}

 \verb|\Porrectus| exists in four different shapes, depending on the
difference between first and second argument. The constraint is that
 $$ p_1-4 \leq p_2 \leq p_1-1 $$ otherwise a diagnostic occurs. Note also that
\keyindex{bporrectus} provides the first curved part of the \verb|porrectus|
command, if you should need it. It has two arguments, the starting pitch and the lower
pitch.

 \item[\keyindex{Porrectusflexus}\fourp], for example:

 \begin{music}\nostartrule
 \elemskip 10pt
 \instrumentnumber 1
 \setsize1{\Largevalue}
 \setstaffs 1 1
 \setlines 1 4
 \setclef 1{3000}
 \setaltoclefsymbol 1 \gregorianCclef
 \startextract
 \notes \Porrectusflexus  bacN\en
 \notes \Porrectusflexus  bNdb\en
 \notes \Porrectusflexus  bMeb\en
 \notes \Porrectusflexus  bLea\en
 \zendextract
 \end{music}
\noindent coded:
 \begin{quote}\begin{verbatim}
 \notes \Porrectusflexus  bacN\en
 \notes \Porrectusflexus  bNdb\en
 \notes \Porrectusflexus  bMeb\en
 \notes \Porrectusflexus  bLea\en
 \end{verbatim}\end{quote}


 \item[\keyindex{climacus}\threep], for example:

 \begin{music}\nostartrule
 \elemskip 10pt
 \instrumentnumber 1
 \setsize1{\Largevalue}
 \setstaffs 1 1
 \setlines 1 4
 \setclef 1{3000}
 \setaltoclefsymbol 1 \gregorianCclef
 \startextract
 \Notes \climacus cbN\en
 \Notes \climacus cba\en
 \Notes \climacus dbN\en
 \zendextract
 \end{music}

 \item[\keyindex{climacusresupinus}\fourp], for example:

 \begin{music}\nostartrule
 \elemskip 10pt
 \instrumentnumber 1
 \setsize1{\Largevalue}
 \setstaffs 1 1
 \setlines 1 4
 \setclef 1{3000}
 \setaltoclefsymbol 1 \gregorianCclef
 \startextract
 \Notes \climacusresupinus cbNa\en
 \Notes \climacusresupinus cbab\en
 \Notes \climacusresupinus dbNb\en
 \zendextract
 \end{music}

 \item[\keyindex{lclimacus}\threep], for example:

 \begin{music}\nostartrule
 \elemskip 10pt
  \setsize1{\Largevalue}
\instrumentnumber 1
 \setstaffs 1 1
 \setlines 1 4
 \setclef 1{3000}
 \setaltoclefsymbol 1 \gregorianCclef
 \startextract
 \notes \lclimacus cbN\en
 \notes \lclimacus cfd\en
 \notes \lclimacus afc\en
 \zendextract
 \end{music}

 \item[\keyindex{scandicus}\threep], for example:

 \begin{music}\nostartrule
 \elemskip 10pt
 \instrumentnumber 1
 \setsize1{\Largevalue}
 \setstaffs 1 1
 \setlines 1 4
 \setclef 1{3000}
 \setaltoclefsymbol 1 \gregorianCclef
 \startextract
 \notes \scandicus abe\en
 \notes \scandicus ceg\en
 \zendextract
 \end{music}

 \item[\keyindex{salicus}\threep], for example:

 \begin{music}\nostartrule
 \elemskip 10pt
 \instrumentnumber 1
 \setsize1{\Largevalue}
 \setstaffs 1 1
 \setlines 1 4
 \setclef 1{3000}
 \setaltoclefsymbol 1 \gregorianCclef
 \startextract
 \Notes \salicus abe\en
 \Notes \salicus ceg\en
 \zendextract
 \end{music}

 \item[\keyindex{salicusflexus}\fourp], for example:

 \begin{music}\nostartrule
 \elemskip 10pt
 \instrumentnumber 1
 \setsize1{\Largevalue}
 \setstaffs 1 1
 \setlines 1 4
 \setclef 1{3000}
 \setaltoclefsymbol 1 \gregorianCclef
 \startextract
 \Notes \salicusflexus abec\en
 \Notes \salicusflexus cegd\en
 \zendextract
 \end{music}

 \item[\keyindex{trigonus}\threep],
%DAS ????????????
%for example\footnote{The second example is in principle irrelevant,
%but it shows the possibilities, in case of.}:
for example:

 \begin{music}\nostartrule
 \elemskip 10pt
 \instrumentnumber 1
 \setsize1{\Largevalue}
 \setstaffs 1 1
 \setlines 1 4
 \setclef 1{3000}
 \setaltoclefsymbol 1 \gregorianCclef
 \startextract
 \Notes \trigonus aaN\en
 \Notes \trigonus cef\en
 \zendextract
 \end{music}

\end{description}

\subsection{Liquescens complex neumes}\index{liquescens neumes}
\begin{description}\setlength{\itemsep}{0ex}
  \item[\keyindex{clivisauctup}\twop], for example:

 \begin{music}\nostartrule
 \elemskip 10pt
 \setsize1{\Largevalue}
 \instrumentnumber 1
 \setstaffs 1 1
 \setlines 1 4
 \setclef 1{3000}
 \setaltoclefsymbol 1 \gregorianCclef
 \startextract
 \notes \clivisauctup ba\en
 \notes \clivisauctup ca\en
 \zendextract
 \end{music}
  \item[\keyindex{clivisauctdown}\twop], for example:

 \begin{music}\nostartrule
 \elemskip 10pt
 \setsize1{\Largevalue}
 \instrumentnumber 1
 \setstaffs 1 1
 \setlines 1 4
 \setclef 1{3000}
 \setaltoclefsymbol 1 \gregorianCclef
 \startextract
 \notes \clivisauctdown ba\en
 \notes \clivisauctdown ca\en
 \zendextract
 \end{music}
 \item[\keyindex{podatusauctup}\twop], for example:

 \begin{music}\nostartrule
 \elemskip 10pt
 \instrumentnumber 1
 \setsize1{\Largevalue}
 \setstaffs 1 1
 \setlines 1 4
 \setclef 1{3000}
 \setaltoclefsymbol 1 \gregorianCclef
 \startextract
 \notes \podatusauctup ab\en
 \notes \podatusauctup ac\en
 \notes \podatusauctup cf\en
 \zendextract
 \end{music}
 \item[\keyindex{podatusauctdown}\twop], for example:

 \begin{music}\nostartrule
 \elemskip 10pt
 \instrumentnumber 1
 \setsize1{\Largevalue}
 \setstaffs 1 1
 \setlines 1 4
 \setclef 1{3000}
 \setaltoclefsymbol 1 \gregorianCclef
 \startextract
 \notes \podatusauctdown ab\en
 \notes \podatusauctdown ac\en
 \notes \podatusauctdown cf\en
 \zendextract
 \end{music}

 \item[\keyindex{pesquassusauctdown}\twop], for example:

 \begin{music}\nostartrule
 \elemskip 10pt
 \instrumentnumber 1
 \setsize1{\Largevalue}
 \setstaffs 1 1
 \setlines 1 4
 \setclef 1{3000}
 \setaltoclefsymbol 1 \gregorianCclef
 \startextract
 \notes \pesquassusauctdown ab\en
 \notes \pesquassusauctdown ae\en
 \zendextract
 \end{music}

 \item[\keyindex{quilismapesauctdown}\twop], for example:

 \begin{music}\nostartrule
 \elemskip 10pt
 \instrumentnumber 1
 \setsize1{\Largevalue}
 \setstaffs 1 1
 \setlines 1 4
 \setclef 1{3000}
 \setaltoclefsymbol 1 \gregorianCclef
 \startextract
 \notes \quilismapesauctdown ab\en
 \notes \quilismapesauctdown ae\en
 \zendextract
 \end{music}

  \item[\keyindex{torculusauctdown}\threep], for example:

 \begin{music}\nostartrule
 \elemskip 10pt
 \instrumentnumber 1
 \setsize1{\Largevalue}
 \setstaffs 1 1
 \setlines 1 4
 \setclef 1{3000}
 \setaltoclefsymbol 1 \gregorianCclef
 \startextract
 \notes \torculusauctdown aba\en
 \notes \torculusauctdown cfd\en
 \notes \torculusauctdown afc\en
 \zendextract
 \end{music}

 \item[\keyindex{Porrectusauctdown}\threep], for example:

 \begin{music}\nostartrule
 \elemskip 10pt
 \instrumentnumber 1
 \setsize1{\Largevalue}
 \setstaffs 1 1
 \setlines 1 4
 \setclef 1{3000}
 \setaltoclefsymbol 1 \gregorianCclef
 \startextract
 \notes \Porrectusauctdown bac\en
 \notes \Porrectusauctdown bNd\en
 \notes \Porrectusauctdown bMe\en
 \notes \Porrectusauctdown bLe\en
 \zendextract
 \end{music}

 \item[\keyindex{climacusauctdown}\threep], for example:

 \begin{music}\nostartrule
 \elemskip 10pt
 \instrumentnumber 1
 \setsize1{\Largevalue}
 \setstaffs 1 1
 \setlines 1 4
 \setclef 1{3000}
 \setaltoclefsymbol 1 \gregorianCclef
 \startextract
 \Notes \climacusauctdown cbN\en
 \Notes \climacusauctdown caM\en
 \Notes \climacusauctdown aNM\en
 \zendextract
 \end{music}

 \item[\keyindex{scandicusauctdown}\threep], for example:

 \begin{music}\nostartrule
 \elemskip 10pt
 \instrumentnumber 1
 \setsize1{\Largevalue}
 \setstaffs 1 1
 \setlines 1 4
 \setclef 1{3000}
 \setaltoclefsymbol 1 \gregorianCclef
 \startextract
 \notes \scandicusauctdown abe\en
 \notes \scandicusauctdown ceg\en
 \zendextract
 \end{music}

 \item[\keyindex{salicusauctdown}\threep], for example:

 \begin{music}\nostartrule
 \elemskip 10pt
 \instrumentnumber 1
 \setsize1{\Largevalue}
 \setstaffs 1 1
 \setlines 1 4
 \setclef 1{3000}
 \setaltoclefsymbol 1 \gregorianCclef
 \startextract
 \Notes \salicusauctdown abe\en
 \Notes \salicusauctdown ceg\en
 \zendextract
 \end{music}

  \item[\keyindex{clivisdeminut}\twop], for example:

 \begin{music}\nostartrule
 \elemskip 10pt
 \setsize1{\Largevalue}
 \instrumentnumber 1
 \setstaffs 1 1
 \setlines 1 4
 \setclef 1{3000}
 \setaltoclefsymbol 1 \gregorianCclef
 \startextract
 \notes \clivisdeminut ba\en
 \notes \clivisdeminut ca\en
 \zendextract
 \end{music}

 \item[\keyindex{podatusdeminut}\twop], for example:

 \begin{music}\nostartrule
 \elemskip 10pt
 \instrumentnumber 1
 \setsize1{\Largevalue}
 \setstaffs 1 1
 \setlines 1 4
 \setclef 1{3000}
 \setaltoclefsymbol 1 \gregorianCclef
 \startextract
 \notes \podatusdeminut ab\en
 \notes \podatusdeminut ac\en
 \notes \podatusdeminut cf\en
 \zendextract
 \end{music}

  \item[\keyindex{torculusdeminut}\threep], for example:

 \begin{music}\nostartrule
 \elemskip 10pt
 \instrumentnumber 1
 \setsize1{\Largevalue}
 \setstaffs 1 1
 \setlines 1 4
 \setclef 1{3000}
 \setaltoclefsymbol 1 \gregorianCclef
 \startextract
 \notes \torculusdeminut aba\en
 \notes \torculusdeminut cfd\en
 \notes \torculusdeminut afc\en
 \zendextract
 \end{music}

  \item[\keyindex{torculusdebilis}\threep], for example:

 \begin{music}\nostartrule
 \elemskip 10pt
 \instrumentnumber 1
 \setsize1{\Largevalue}
 \setstaffs 1 1
 \setlines 1 4
 \setclef 1{3000}
 \setaltoclefsymbol 1 \gregorianCclef
 \startextract
 \notes \torculusdebilis aba\en
 \notes \torculusdebilis cfd\en
 \notes \torculusdebilis afc\en
 \zendextract
 \end{music}

 \item[\keyindex{Porrectusdeminut}\threep], for example:

 \begin{music}\nostartrule
 \elemskip 10pt
 \instrumentnumber 1
 \setsize1{\Largevalue}
 \setstaffs 1 1
 \setlines 1 4
 \setclef 1{3000}
 \setaltoclefsymbol 1 \gregorianCclef
 \startextract
 \notes \Porrectusdeminut bac\en
 \notes \Porrectusdeminut bNd\en
 \notes \Porrectusdeminut bMe\en
 \notes \Porrectusdeminut bLe\en
 \zendextract
 \end{music}

 \item[\keyindex{climacusdeminut}\threep], for example:

 \begin{music}\nostartrule
 \elemskip 10pt
 \instrumentnumber 1
 \setsize1{\Largevalue}
 \setstaffs 1 1
 \setlines 1 4
 \setclef 1{3000}
 \setaltoclefsymbol 1 \gregorianCclef
 \startextract
 \Notes \climacusdeminut cbN\en
 \Notes \climacusdeminut caM\en
 \Notes \climacusdeminut aML\en
 \zendextract
 \end{music}

 \item[\keyindex{scandicusdeminut}\threep], for example:

 \begin{music}\nostartrule
 \elemskip 10pt
 \instrumentnumber 1
 \setsize1{\Largevalue}
 \setstaffs 1 1
 \setlines 1 4
 \setclef 1{3000}
 \setaltoclefsymbol 1 \gregorianCclef
 \startextract
 \notes \scandicusdeminut abe\en
 \notes \scandicusdeminut ceg\en
 \zendextract
 \end{music}
\end{description}

\section{musixgui}\ixtt{musixgui.tex} Provides macros for
typesetting modern style \itxem{guitar chords}. For example:

\begin{music}
\hsize130mm
\tenrm
\parindent0pt
\generalmeter{\meterfrac34}
\generalsignature1
\startbarno0
\def\txh{-6.5}
\def\tx#1*{\zchar\txh{\lrlap{\kern3\Internote#1}}}
\def\rtx#1*{\zchar\txh{\kern-3\Internote#1}}
\stafftopmarg10\Interligne
\raiseguitar{20}
\nostartrule
\startpiece
\addspace{.5\afterruleskip}%
\NOtes\tx We*\qa d\en
\bar
\NOtes\guitar G{}o-----\gbarre3\gdot25\gdot35\gdot44\tx wish*\qa g\en
\Notes\tx you*\ca g\en
\Notes\tx a*\ca h\en
\Notes\zchar\txh{merry}\ca g\en
\Notes\ca f\en
\bar
\NOtes\guitar C5o-----\gbarre4\gdot26\gdot36\gdot45\rtx christmas,*\qa e\en
\NOtes\qa e\en
\NOtes\guitar {e/H}5o-----\gbarre3\gdot35\gdot45\gdot54\tx we*\qa e\en
\bar
\NOtes\guitar {A$\!^7$}5o-----\gbarre1\gdot23\gdot42\tx wish*\qa h\en
\Notes\tx you*\ca h\en
\Notes\tx a*\ca i\en
\Notes\zchar\txh{merry}\ca h\en
\Notes\ca g\en
\bar
\NOtes\guitar D{}xxo---\gdot42\gdot53\gdot62\rtx christmas,*\qa f\en
\NOtes\qa d\en
\zbar
\NOtes\guitar{D/c}{}xo----\gdot23\gdot42\gdot53\gdot62\tx we*\qa d\en
\bar
\NOtes\guitar{B$^7$}{}xo----\gdot22\gdot31\gdot42\gdot62\tx wish*\qa i\en
\Notes\tx you*\ca i\en
\Notes\tx a*\ca j\en
\Notes\tx ~mer-*\ca i\en
\Notes\tx ry*\ca h\en
\bar
\NOtes\guitar e{}xxo---\gdot32\tx ~christ-*\qa g\en
\NOtes\tx mas*\qa e\en
\Notes\guitar {G/d}{}xxo---\gdot63\tx and*\ca d\en
\Notes\tx a*\ca d\en
\bar
\NOtes\guitar{C$^6$}{}xo----\gdot23\gdot32\gdot42\gdot51\tx ~~hap~-*\qa e\en
\NOtes\tx py*\qa h\en
\NOtes\guitar{D$^7$}{}xo---x\gdot25\gdot34\gdot45\gdot45\gdot53\tx new*\qa f\en
\bar
\NOTes\guitar G{}o-----\gbarre3\gdot25\gdot35\gdot44\tx ~year.*\ha g\en
\setdoublebar\endpiece
\end{music}

\medskip
The macro \keyindex{guitar} sets the grid, chord name, barre type,
and on-off indicators for the strings. For example, the first
chord in above example was coded as

\verb|\guitar G{}o-----\gbarre3\gdot25\gdot35\gdot44|

\noindent where the first argument is the text to be placed above the grid, the
second is empty (relative barre), and the next six characters indicate if the string
is played or not with either \verb|x|, \verb|o| or \verb|-|. The dots are set with
\keyindex{gdot}~$sb$ where the $s$ is the string and $b$ is the barre. The
rule is set with \keyindex{gbarre}~$b$ where $b$ indicates the position of the barre.

The whole symbol may be vertically shifted with
\keyindex{raiseguitar}\onen, where $n$ is a number
in units of \keyindex{internote}. When using guitar chords, it might be
useful to reserve additional
space above the chord by advancing \keyindex{stafftopmarg} to
something like \verb|stafftopmarg=10\Interligne|.

For frequently used chords, it might be useful to define your own
macros, e.g.

\verb|\def\Dmajor{\guitar D{}x-----\gdot42\gdot53\gdot62}%|

\section{musixhv}\label{helvetica}


Replaces the default Computer Modern text fonts by Helvetica (sans serif) fonts;
see Section~\ref{UserFonts}.
Use as follows:
\begin{quote}\begin{verbatim}
\input musixtex
\input musixhv
...
\end{verbatim}\end{quote}

The usual ``small'' type commands are supported:
\begin{quote}
\begin{tabular}{lr}
\keyindex{smalltype} & 8pt \\
\keyindex{Smalltype} & 9pt \\
\keyindex{normtype}& 10pt \\
\keyindex{medtype} & 12pt \\
\end{tabular}
\end{quote}
For all of these, the following variants are supported:
\begin{quote}
\begin{tabular}{ll}
\verb|\rm| & ``Roman'' (sans serif) \\
\verb|\bf| & bold \\
\verb|\it| & ``italic'' (oblique) \\
\verb|\bi| & ``bold-italic'' (bold-oblique) \\
\verb|\sc| & small capitals\\
\end{tabular}
\end{quote}
The ``big'' type commands are as follows:
\begin{quote}
\begin{tabular}{lr}
\keyindex{bigtype} & 14pt\\
\keyindex{Bigtype} & 17pt\\
\keyindex{BIgtype}& 20pt\\
\keyindex{BIGtype} & 25pt\\
\end{tabular}
\end{quote}
The default variant for all of these is small-caps;
however, \keyindex{font} commands are defined
for all the usual variants at all of these sizes.
To letter-space a title, use the \verb|\so|
command in the \verb\soul\ package.


The following bold-oblique fonts for dynamic marks are defined:
\begin{quote}\begin{tabular}{lr}
\keyindex{ppfftwelve} & 8pt \\
\keyindex{ppffsixteen} & 10pt\\
\keyindex{ppfftwenty} & 12pt \\
\keyindex{ppfftwentyfour} & 14pt \\
\keyindex{ppfftwentynine} & 17pt \\
\end{tabular}\end{quote}

The default font for typesetting \keyindex{txt} and \keyindex{tuplettxt}
arguments 
is \keyindex{eightit} (\verb\8pt\~oblique).  

 \section{musixlit}\label{litu}\label{otherbars}
 Provides a notation style intermediate between gregorian and baroque, for
example

\makeatletter
\catcodesmusic
%\def\vnotes#1\elemskip{\noteskip#1\@l@mskip \multnoteskip\scal@noteskip
%  \not@s}

%\def\not@s{\def|{\nextstaff}\def&{\nextinstrument}\normaltranspose\transpose
%  \check@nopen\notes@open\@ne
%  \kern\n@skip\advance\x@skip\n@skip \locx@skip\x@skip
%  \n@skip\noteskip \noinstrum@nt\z@ \begininstrument}

%\def\en{\@ndstaff\notes@open\z@
%  \ifx\@ne\V@sw \widthtyp@\z@\t@rmskip \let\V@sw\empty \fi}

\def\double#1{\roffset{1.2}{\advancefalse#1}#1}
\makeatother

 % Don't know if this example correct ??? But looks nice...
\begin{music}
\ninerm
\parindent0pt
\instrumentnumber{2}
\interstaff{11}
\generalsignature2
\setclefsymbol2\oldGclef
\setstaffs1{2}
\setclef1\bass
\setinterinstrument1{-\Interligne}
\startpiece
\shortbarrules
\addspace\afterruleskip
\hardlyrics{Il nous a sign\'es de son
  }\notes\zw d\wh K|\zw f\wh h&\rtx\thelyrics*\Hpause h1\en
\qspace\qspace
\NOTes\zhl N\hu d|\zhl g\hu i&\tx sang*\double{\cnql i}\en
\bar
\hardlyrics{Et nous avons \'e-
  }\notes\zw d\wh K|\zw f\wh h&\rtx\thelyrics~-*\Hpause h1\en
\qspace\qspace
\NOTes\zhl M\hu c|\zhl f\hu h&\rtx t\'e*\double{\cnql i}\en
\NOtes\zql L\qu e|\zql b\qu g&\tx ~pro-*\cnqu g\en
\NOtes\zql b\qu d|\zql d\qu f&\tx ~t\'e-*\cnqu f\en
\NOTes\zhl a\hu c|\zhl e\hu h&\tx g\'es.*\double{\cnqu h}\en
\bar
\NOtes\zql M\qu d|\zql d\qu h&\tx ~~Al~-*\cnqu h\en
\NOtes\zql K\qu a|\zql f\qu k&\tx ~~le~-*\cnql k\en
\NOTes\zhl H\hu a|\zhl e\hu j&\tx ~~lu~-*\double{\cnql j}\en
\bar
\NOTEs\zhl K\hu a|\zhl f\hu k&\tx ~~ia !*\cnhl k\en
\sepbarrules
\endpiece

\startpiece
\interbarrules
\addspace\afterruleskip
\hardlyrics{Il nous a sign\'es de son
  }\notes\zw d\wh K|\zw f\wh h&\rtx \thelyrics*\Hlonga h1\en
\qspace\qspace
\NOTes\zhl N\hu d|\zhl g\hu i&\tx sang*\chl i\en
\bar
\hardlyrics{Et nous avons \'e
  }\notes\zw d\wh K|\zw f\wh h&\rtx \thelyrics~-*\Hlonga h1\en
\qspace\qspace
\NOTes\zhl M\hu c|\zhl f\hu h&\rtx t\'e*\chl i\en
\NOtes\zql L\qu e|\zql b\qu g&\tx ~pro-*\cqu g\en
\NOtes\zql b\qu d|\zql d\qu f&\tx ~t\'e-*\cqu f\en
\NOTes\zhl a\hu c|\zhl e\hu h&\tx g\'es.*\chu h\en
\bar
\NOtes\zql M\qu d|\zql d\qu h&\tx ~~Al~-*\cqu h\en
\NOtes\zql K\qu a|\zql f\qu k&\tx ~~le~-*\cql k\en
\NOTes\zhl H\hu a|\zhl e\hu j&\tx ~~lu~-*\chl j\en
\bar
\NOTEs\zbreve K\breve a|\zbreve f\breve k&\tx ~~ia !*\zbreve k\en
\sepbarrules
\endpiece
\end{music}
\endcatcodesmusic

 This package provides:
 \begin{itemize}\setlength{\itemsep}{0ex}
 \item\keyindex{oldGclef} which replaces the ordinary G clef with an old one,
using (for instrument 2 as an example):
 \verb|\settrebleclefsymbol2\oldGclef|

 \item\keyindex{cqu} $p$ provides a square headed quarter note with stem up at
pitch $p$.

 \item\keyindex{cql} $p$ provides a square headed quarter note with stem down at
pitch $p$.

 \item\keyindex{chu} $p$ provides a square headed half note with stem up at
pitch $p$.

 \item\keyindex{chl} $p$ provides a square headed half note with stem down at
pitch $p$.

 \item\keyindex{cnqu} $p$ and \keyindex{cnql} $p$ provide a stemless square
headed
quarter note at pitch $p$.

 \item\keyindex{cnhu} $p$ and \keyindex{cnhl} $p$ provide a stemless square
headed half note at pitch $p$.

 \item\keyindex{Hpause} $p$ $n$ provides an arbitrary length pause at pitch
$p$ and of length $n$ \keyindex{noteskip}. However, in the first of the above
example, this feature has been used to denote an arbitrary length note rather
than a rest!

 \item\keyindex{Hlonga} $p$ $n$ provides an arbitrary length note at pitch
$p$ and of length $n$ \keyindex{noteskip}.
This feature has been used to denote an arbitrary length note in the second of
the above examples.

 \item\keyindex{shortbarrules} has been used to provide bar rules shorter than
the staff vertical width.

 \item\keyindex{interbarrules} has been used to provide bars between the
staffs, rather that over them. This is an arbitrary question of taste...
 \end{itemize}

 \section{musixlyr} \ixtt{musixlyr.tex}
 Enables the recommended method for adding lyrics to a score (see Section~\ref{musixlyr}).

 \section{musixmad} \ixtt{musixmad.tex} Increases the number of
instruments, slurs and beams up to twelve. When using this extension, it is
not necessary to explicitly input \verb|musixadd.tex|.
If you need greater numbers of these elements, see Sections~\ref{musixmad_setmaxgroups}, 
\ref{musixmad_setmaxinstruments_ccxviiibeams}, 
\ref{musixmad_setmaxoctlines}, 
\ref{musixmad_setmaxslurs} and
\ref{musixmad_setmaxtrills}.


\section{musixper}\label{perc}

Provides special symbols intended for percussion parts. Included are a
\ital{drum clef}---comprising two vertical parallel lines---and notes with
various specially shaped heads. The note symbols that are available are as
follows:

 \begin{itemize}\setlength{\itemsep}{0ex}
 \item The \raise.5ex\hbox{\musixchar113}~{}~symbol which is obtained using the
\verb|\qu|, \verb|\qb|, \verb|\cu|, etc. macros preceded by a
``\verb|dc|'' (think of {\it d}iagonal {\it c}ross).
Available are
\keyindex{dcqu},
\keyindex{dcql},
\keyindex{dcqb},
\keyindex{dczq},
\keyindex{dccu},
\keyindex{dcccu},
\keyindex{dccl} and
\keyindex{dcccl}.

 \item The \raise.5ex\hbox{\musixchar112}~{}~symbol which is obtained using the
\verb|\qu|, \verb|\qb|, \verb|\cu|, etc. macros preceded by a
``\verb|dh|''
(think of {\it d}iagonal cross {\it h}alf open).
Available are
\keyindex{dhqu},
\keyindex{dhql},
\keyindex{dhqb},
\keyindex{dhzq},
\keyindex{dhcu},
\keyindex{dhccu},
\keyindex{dhcl} and
\keyindex{dhccl}.

 \item The \raise.5ex\hbox{\musixchar111}~{}~symbol which is obtained using the
\verb|\qu|, \verb|\qb|, \verb|\cu|, etc. macros preceded by a
``\verb|do|''
(think of {\it d}iagonal cross {\it o}pen).
Available are
\keyindex{doqu},
\keyindex{doql},
\keyindex{doqb},
\keyindex{dozq},
\keyindex{docu},
\keyindex{doccu},
\keyindex{docl} and
\keyindex{doccl}.

 \item The \raise.5ex\hbox{\musixchar114}~{}~symbol which is obtained using the
\verb|\qu|, \verb|\qb|, \verb|\cu|, etc. macros preceeded by
``\verb|x|'' (e.g.\ for spoken text of songs).
Available are
\keyindex{xqu},
\keyindex{xql},
\keyindex{xqb},
\keyindex{xzq},
\keyindex{xcu},
\keyindex{xccu},
\keyindex{xcl} and
\keyindex{xccl}.

 \item The \raise.5ex\hbox{\musixchar115}~{}~symbol which is obtained using the
\verb|\qu|, \verb|\qb|, \verb|\cu|, etc. macros preceeded by
``\verb|ox|'' .
Available are
\keyindex{oxqu},
\keyindex{oxql},
\keyindex{oxqb},
\keyindex{oxzq},
\keyindex{oxcu},
\keyindex{oxccu},
\keyindex{oxcl} and
\keyindex{oxccl}.

 \item The \raise.5ex\hbox{\musixchar118}~{}~symbol which is obtained using the
\verb|\qu|, \verb|\qb|, \verb|\cu|, etc. macros preceeded by
``\verb|ro|'' (think of {\it r}h{\it o}mbus).
Available are
\keyindex{roqu},
\keyindex{roql},
\keyindex{roqb},
\keyindex{rozq},
\keyindex{rocu},
\keyindex{roccu},
\keyindex{rocl} and
\keyindex{roccl}.

 \item The \raise.5ex\hbox{\musixchar116}~{}~symbol which is obtained using the
\verb|\qu|, \verb|\qb|, \verb|\cu|, etc. macros preceeded by
``\verb|tg|'' (think of {\it t}rian{\it g}le).
Available are
\keyindex{tgqu},
\keyindex{tgql},
\keyindex{tgqb},
\keyindex{tgzq},
\keyindex{tgcu},
\keyindex{tgccu},
\keyindex{tgcl} and
\keyindex{tgccl}.

 \item The \raise.5ex\hbox to 1em{\musixchar117\hfil}~{}~symbol which is obtained
using the
\verb|\qu|, \verb|\qb|, \verb|\cu|, etc.~macros preceeded by
``\verb|k|'' .
Available are
\keyindex{kqu},
\keyindex{kql},
\keyindex{kqb},
\keyindex{kzq},
\keyindex{kcu},
\keyindex{kccu},
\keyindex{kcl} and
\keyindex{kccl}.
 \end{itemize}

The diamond shaped noteheads described in Section~\ref{diam} are also
available, because \verb|musixper.tex| inputs \verb|musixdia.tex|.

If any of the foregoing notes need to be dotted, you must use the explicit
dotting macros \verb|\pt|, \verb|\ppt|, or \verb|\pppt| as described in
Section~\ref{dots}.

Since the usage of these note symbols is
not standardized, it would be wise to include in the score a explanation
of which symbol corresponds to which specific percussion instrument.

A special \itxem{drum clef}---comprising two heavy vertical bars---can
be made to replace the normal clef for the $n$-th intrument by saying
\keyindex{setclefsymbol}\onen\keyindex{drumclef} .
To cause this to appear at the right vertical position, the instrument should
previously have been assigned a treble clef (or not explicitly assigned any
clef, thereby giving it a treble clef by default).

Percussion music might be written on a staff with either one or five lines.
If there are several different percussions instruments it may be useful to
use a five-line staff with a drum clef, and differentiate the instruments
by the type of the note heads and the apparent
pitch of the note on the staff. Here is an example of the
latter\footnote{provided by Agusti {\sc Mart\'in Domingo}}:

\medskip
\begin{music}
\generalmeter{\meterfrac44}
\setclefsymbol1\drumclef
\parindent0pt\startpiece
\leftrepeat
\Notes\zql f\rlap\qp\ibu0m0\xqb0{nn}\en
\Notes\kzq d\zql f\zq j\xqb0n\tbu0\xqb0n\en
\Notes\zql f\rlap\qp\ibu0m0\xqb0{nn}\en
\Notes\kzq d\zql f\zq j\xqb0n\tbu0\xqb0n\en
\bar
\Notes\zql f\rlap\qp\ibu0m0\kqb0{nn}\en
\Notes\xzq d\zql f\zq j\kqb0n\tbu0\kqb0n\en
\Notes\zql f\rlap\qp\ibu0m0\kqb0{nn}\en
\Notes\xzq d\zql f\zq j\kqb0n\tbu0\kqb0n\en
\bar
\Notes\zql f\rlap\qp\ibu0m0\oxqb0{nn}\en
\Notes\oxzq d\zql f\zq j\kqb0n\tbu0\oxqb0n\en
\Notes\zql f\rlap\qp\ibu0m0\oxqb0{nn}\en
\Notes\oxzq d\zql f\zq j\kqb0n\tbu0\oxqb0n\en
\setrightrepeat\endpiece
\end{music}
\noindent Its coding is
 \begin{quote}\begin{verbatim}
\begin{music}
\instrumentnumber{1}
\generalmeter{\meterfrac44}
\setclefsymbol1\drumclef
\parindent0pt\startpiece
\leftrepeat
\Notes\zql f\rlap\qp\ibu0m0\xqb0{nn}\en
\Notes\kzq d\zql f\zq j\xqb0n\tbu0\xqb0n\en
\Notes\zql f\rlap\qp\ibu0m0\xqb0{nn}\en
\Notes\kzq d\zql f\zq j\xqb0n\tbu0\xqb0n\en
\bar
\Notes\zql f\rlap\qp\ibu0m0\kqb0{nn}\en
\Notes\xzq d\zql f\zq j\kqb0n\tbu0\kqb0n\en
\Notes\zql f\rlap\qp\ibu0m0\kqb0{nn}\en
\Notes\xzq d\zql f\zq j\kqb0n\tbu0\kqb0n\en
\bar
\Notes\zql f\rlap\qp\ibu0m0\oxqb0{nn}\en
\Notes\oxzq d\zql f\zq j\kqb0n\tbu0\oxqb0n\en
\Notes\zql f\rlap\qp\ibu0m0\oxqb0{nn}\en
\Notes\oxzq d\zql f\zq j\kqb0n\tbu0\oxqb0n\en
\setrightrepeat\endpiece
\end{music}
 \end{verbatim}\end{quote}

Here is an example of a single-line percussion staff using
diamond-shaped note heads:

\begin{music}
\parindent 19mm
\instrumentnumber{3}
\setname1{keyboard} \setname2{drum} \setname3{monks}
\setlines2{1}
\setlines3{4}
\setinterinstrument1{-2\Interligne}% less vertical space above
\setinterinstrument2{-2\Interligne}% and below the percussion
\sepbarrules
\setsign1{-1} % one flat at keyboard
\generalmeter{\meterfrac24}
\setmeter3\empty
\setclef3\alto
\setclef1\bass
\setstaffs12 % 2 staffs at keyboard
\setclefsymbol3\gregorianCclef % gregorian C clef at instrument 3
\setclefsymbol2\drumclef       % cancel G clef at instrument 2
\startextract
\Notes\hu F|\zh c\hu h&\dnq4&\squ{acd}\en\bar
\NOtes\qu I|\zq N\qu d&\qp&\diapunc f\en
\NOtes\qu J|\zq a\qu e&\ynq4&\diapunc e\en\bar
\notes\hu G|\zh b\hu d&\dnq4&\zsqu d\rsqu g\squ{hgh}\en
\endextract
\end{music}
\noindent which is coded as follows:

\begin{verbatim}
\parindent 19mm
\instrumentnumber{3}
\setname1{keyboard} \setname2{drum} \setname3{monks}
\setlines2{1}
\setlines3{4}
\setinterinstrument1{-2\Interligne}% less vertical space above
\setinterinstrument2{-2\Interligne}% and below the percussion
\sepbarrules
\setsign1{-1} % one flat at keyboard
\generalmeter{\meterfrac24}
\setmeter3{\empty}
\setclef3{\alto}
\setclef1{\bass}
\setstaffs1{2} % 2 staffs at keyboard
\setclefsymbol3{\gregorianCclef} % gregorian C clef at instrument 3
\setclefsymbol2{\drumclef}       % cancel G clef at instrument 2
\startextract
\end{verbatim}

% (DAS) Sorry, guys, I couldn't figure out how to get the |'s to work in this
% verbatim.
%
\vskip-11pt
\def\Vert{{\tt\char'174}}
\noindent\verb|\Notes\hu F|\Vert\verb|\zh c\hu h&\dnq4&\squ{acd}\en\bar|\\
\verb|\NOtes\qu I|\Vert\verb|\zq N\qu d&\qp&\diapunc f\en|\\
\verb|\NOtes\qu J|\Vert\verb|\zq a\qu e&\ynq4&\diapunc f\en\bar|\\
\verb|\notes\hu G|\Vert\verb|\zh b\hu d&\dnq4&\zsqu d\rsqu g\squ{hgh}\en|\\
\verb|\endextract|

 \section{musixplt}\index{musixplt@{\tt musixplt.tex}}\label{palatino}
Replaces the default Computer Modern text fonts by Palatino fonts;
see Section~\ref{UserFonts}.
Use as follows:
\begin{quote}\begin{verbatim}
\input musixtex
\input musixplt
...
\end{verbatim}\end{quote}

The usual ``small'' type commands are supported:
\begin{quote}
\begin{tabular}{lr}
\keyindex{smalltype} & 8pt \\
\keyindex{Smalltype} & 9pt \\
\keyindex{normtype}& 10pt \\
\keyindex{medtype} & 12pt \\
\end{tabular}
\end{quote}
For all of these, the following variants are supported:
\begin{quote}
\begin{tabular}{ll}
\verb|\rm| & Roman \\
\verb|\bf| & bold \\
\verb|\it| & italic \\
\verb|\bi| & bold italic \\
\verb|\sc| & small capitals (and old-style figures)\\
\verb|\sl| & oblique \\
\end{tabular}
\end{quote}
The ``big'' type commands are as follows:
\begin{quote}
\begin{tabular}{lr}
\keyindex{bigtype} & 14pt\\
\keyindex{Bigtype} & 17pt\\
\keyindex{BIgtype}& 20pt\\
\keyindex{BIGtype} & 25pt\\
\end{tabular}
\end{quote}
The default variant for all of these is small-caps;
however, \keyindex{font} commands are defined
for all the usual variants at all of these sizes.
To letter-space a title, use the \verb|\so|
command in the \verb\soul\ package.


The following bold-italic fonts for dynamic marks are defined:
\begin{quote}\begin{tabular}{lr}
\keyindex{ppfftwelve} & 8pt \\
\keyindex{ppffsixteen} & 10pt\\
\keyindex{ppfftwenty} & 12pt \\
\keyindex{ppfftwentyfour} & 14pt \\
\keyindex{ppfftwentynine} & 17pt \\
\end{tabular}\end{quote}

The default font for typesetting \keyindex{txt} and \keyindex{tuplettxt}
arguments 
is \keyindex{eightit} (\verb\8pt\~italic).  

 \section{musixpoi}
 Adds definitions of less common singly and doubly dotted notes.
Available are
\keyindex{ccup},
\keyindex{zccup},
\keyindex{cclp},
\keyindex{zcclp},
\keyindex{ccupp},
\keyindex{zccupp},
\keyindex{cclpp},
\keyindex{zcclpp},
\keyindex{cccup},
\keyindex{zcccup},
\keyindex{ccclp},
\keyindex{zccclp},
\keyindex{cccupp},
\keyindex{zcccupp},
\keyindex{ccclpp},
\keyindex{zccclpp},
\keyindex{ccccup},
\keyindex{zccccup},
\keyindex{cccclp},
\keyindex{zcccclp},
\keyindex{ccccupp},
\keyindex{zccccupp},
\keyindex{cccclpp} and
\keyindex{zcccclpp}.

 \section{musixppff}\label{musixppff}\index{musixppff@{\tt musixppff.tex}}

This replaces the default definitions used for dynamic marks to use a ``mini-font''
\verb\xppff10\ designed by Hiroaki \textsc{Morimoto}. It is used as follows:

\begin{quote}\begin{verbatim}
\input musixtex
\input musixppff
...
\end{verbatim}\end{quote}
\noindent The new dynamic marks are as follows:
\font\xppff=xppff10  
\begin{quote}
\xppff
pppp\ ppp\ pp\ p\ mp\ mf\ f\ fp\ sf\ ff\ fff\ ffff\ sfz\ sfzp
\end{quote}
Note that the \verb|xppff10| font has defined glyphs only for the following five characters:  f, m, p, s and~z;
use \keyindex{ppff} etc.\@
for other bold-italic text in music.

 \section{musixps}\label{musixps}\index{musixps@{\tt musixps.tex}}
 Activates type K Postscript slurs, ties,  and hairpins; see Chapter~\ref{PostscriptSlurs}.

 \section{musixstr}\label{musixstr}\index{musixstr@{\tt musixstr.tex}}
 Provides bowing and other symbols for \itxem{string instruments}\footnote{provided
by Werner {\sc Icking}}. The symbol can be posted at the desired position using
 \verb|\zcharnote|\pitchp\verb|{|$command$\verb|}|. The available symbols and
their meanings are as follows:

{\input musixstr
\begin{quote}\begin{description}\setlength{\itemsep}{0ex}

 \item[\hbox to 1em{\AB}~: \keyindex{AB} or  \keyindex{downbow}] down-bow

 \item[\hbox to 1em{\AUF}~: \keyindex{AUF} or \keyindex{upbow}] up-bow

 \item[\hbox to 1em{\SP}~: \keyindex{SP}] at the top of bow

 \item[\hbox to 1em{\FR}~: \keyindex{FR}] at the nut of bow

 \item[\GB\ or \Gb~: \keyindex{GB} or  \keyindex{Gb}] whole bow

 \item[\UH\ or \Uh~: \keyindex{UH} or  \keyindex{Uh}] lower half of bow

 \item[\OH\ or \Oh~: \keyindex{OH} or  \keyindex{Oh}] upper half of bow

 \item[\MI\ or \Mi~: \keyindex{MI} or \keyindex{Mi}] middle of bow

 \item[\UD\ or \Ud~: \keyindex{UD} or  \keyindex{Ud}] lower third of bow

 \item[\OD\ or \Od~: \keyindex{OD} or  \keyindex{Od}] upper third of bow

 \item[\Pizz~: \keyindex{Pizz}] left hand pizzicato or trill

 \end{description}\end{quote}
 }


 %\check
 \section{musixsty}\index{musixsty@{\tt musixsty.tex}}

 Provides certain text-handling facilities for titles, footnotes, and other
items not related to lyrics. It should not be used with \LaTeX. It includes
 \begin{itemize}\setlength{\itemsep}{0ex}
 \item definitions of \keyindex{hsize}, \keyindex{vsize},
\keyindex{hoffset}, \keyindex{voffset} suitable for A$4$ paper; those using
other sizes may wish to modify it once and for all;
 \item a set of text size commands:

 \begin{description}\setlength{\itemsep}{0ex}
  \item[\keyindex{eightpoint}] which sets the usual \keyindex{rm},
\keyindex{bf}, \keyindex{sl}, \keyindex{it} commands to $8$ point font size;
  \item[\keyindex{tenpoint}] which sets the usual \keyindex{rm},
\keyindex{bf}, \keyindex{sl}, \keyindex{it} commands to $10$ point font size;
  \item[\keyindex{twlpoint}] (or \keyindex{twelvepoint}) to get $12$ point font size;
  \item[\keyindex{frtpoint}] to get $14.4$ point font size;
  \item[\keyindex{svtpoint}] to get $17.28$ point font size;
  \item[\keyindex{twtypoint}] to get $20.74$ point font size;
  \item[\keyindex{twfvpoint}] to get $24.88$ point font size;
 \end{description}
 \item commands for creating titles:
  \begin{itemize}\setlength{\itemsep}{0ex} \item \keyindex{author} or
\keyindex{fullauthor} to be put at the right of the first page, below the
title of the piece; the calling sequence is, for example

  \verb|     \author{Daniel TAUPIN\\organiste \`a Gif-sur-Yvette}|

  \noindent where the \verb|\\| causes the author's name to be displayed on
two lines;

  \item \keyindex{shortauthor} to be put at the bottom of each page;
  \item \keyindex{fulltitle} which is the main title of the piece;
  \item \keyindex{subtitle} is displayed below the main title of the piece;
  \item \keyindex{shorttitle} or \keyindex{title}
  which is the title repeated at the bottom of each page;
  \item \keyindex{othermention} which is displayed on the left of the page,
vertically aligned with author's name. It may contain \verb|\\| to display it on
several lines;
  \item \keyindex{maketitle}  which displays all the previous stuff;
  \end{itemize}

 \item  commands for making \itxem{footnotes}:
  \begin{itemize}\setlength{\itemsep}{0ex}
   \item The normal Plain-\TeX\ \keyindex{footnote} command, which has two
arguments---not just one as in \LaTeX\protect\index{LATEX@\LaTeX}---namely
the label of the footnote, which can be any sequence of characters, and
the text of the footnote. This command does not work inside
boxes, so it cannot be issued within music;

 \item The \keyindex{Footnote} command, which counts the footnotes and uses a
number as the label of the footnote (equivalent to \LaTeX's \verb|\footnote|
command). The same restriction as with \verb|\footnote| applies concerning
its use within the music coding;

 \item The \keyindex{vfootnote} command, taken from the Plain-\TeX, which
places a footnote at the bottom of the current page, but does not put
the footnote label at the place the command is entered in the main text. This
also may not be used within music, but if a footnote is needed whose reference
lies inside the music, it can be entered in two steps:
 \begin{enumerate}
  \item manually insert the reference inside the music, using e.g., \verb|zcharnote|;
  \item post the footnote itself with \verb|\vfootnote| outside the music,
either before \keyindex{startpiece} or between \keyindex{stoppiece} and
\keyindex{contpiece} or equivalent commands.
 \end{enumerate}

  \end{itemize}

 \end{itemize}

 \section{musixtmr}\index{musixtmr@{\tt musixtmr.tex}}\label{times}
 Replaces the default Computer Modern text fonts by Times fonts;
 see Section~\ref{UserFonts}.
Use as follows:
\begin{quote}\begin{verbatim}
\input musixtex
\input musixtmr
...
\end{verbatim}\end{quote}

The usual ``small'' type commands are supported:
\begin{quote}
\begin{tabular}{lr}
\keyindex{smalltype} & $8$pt \\
\keyindex{Smalltype} & $9$pt \\
\keyindex{normtype}& $10$pt \\
\keyindex{medtype} & $12$pt \\
\end{tabular}
\end{quote}
For all of these, the following variants are supported:
\begin{quote}
\begin{tabular}{ll}
\verb|\rm| & Roman \\
\verb|\bf| & bold \\
\verb|\it| & italic \\
\verb|\bi| & bold italic \\
\verb|\sc| & small capitals \\
\verb|\sl| & oblique \\
\end{tabular}
\end{quote}
The ``big'' type commands are as follows:
\begin{quote}
\begin{tabular}{lr}
\keyindex{bigtype} & $14$pt\\
\keyindex{Bigtype} & $17$pt\\
\keyindex{BIgtype}& $20$pt\\
\keyindex{BIGtype} & $25$pt\\
\end{tabular}
\end{quote}
The default variant for all of them is Roman;
however, \keyindex{font} commands are defined
for all the usual variants at all of these sizes.
To letter-space a title, use the \verb|\so|
command in the \verb\soul\ package.


The following bold-italic fonts for dynamic marks are defined:
\begin{quote}\begin{tabular}{lr}
\keyindex{ppfftwelve} & 8pt \\
\keyindex{ppffsixteen} & 10pt\\
\keyindex{ppfftwenty} & 12pt \\
\keyindex{ppfftwentyfour} & 14pt \\
\keyindex{ppfftwentynine} & 17pt \\
\end{tabular}\end{quote}

The default font for typesetting \keyindex{txt} and \keyindex{tuplettxt}
arguments 
is \keyindex{eightit} (\verb|8pt| italic).  

\section{musixtnt}\label{musixtnt}
This package (tnt = Transform Notes) provides a macro \keyindex{TransformNotes} which makes 
possible several ``transformations'' of the effect of notes commands such as \keyindex{notes}.
The \verb\musixtnt\ package is distributed separately from \verb\musixtex\.

In general the effect of \verb|\TransformNotes(|\textit{input}\verb|}{|\textit{output}\verb|}|
is that subsequent notes commands in the source will expect their arguments 
to match the \textit{input} pattern but
the notes will be typeset according to the \textit{output} pattern.

For example,
\begin{verbatim}
  \TransformNotes{#2&#3&#4&#5}{#2&#3&#5}
\end{verbatim}
would be appropriate for a four-instrument score (arguments \verb|#2|, \verb|#3|, \verb|#4|, and \verb|#5|, separated by three
\verb\&\s, but the third instrument (\verb|#4|) will be discarded.

The instrument/staff numbers in the first argument must start at $2$ and increase
consecutively, using \verb|&| (or \| for multi-staff instruments) as a separator.  
The reason that the segment identifiers start at $2$ is that the first argument for \keyindex{vnotes}
is a spacing parameter.

It  is
essential  that  every \verb|\znotes|, \verb|\notes|, \verb|\Notes|, \verb|\NOtes|, etc.\@ command in
the score match the pattern of the first argument to \keyindex{TransformNotes} exactly; for example,
too few (or too  many)
note segments  will result in unintentionally discarded material and possibly compilation failure.
Similarly, it is not possible to use \keyindex{nextinstrument}, \keyindex{nextstaff}, \keyindex{selectinstrument} or \keyindex{selectstaff}, or to hide the \verb|&| (or \|) tokens inside
a user-defined macro.
An auxiliary program \verb|msxlint| distributed in the \verb\musixtnt\ package can be used to detect such incompatibilities.

\keyindex{TransformNotes} may be used anywhere between \keyindex{startpiece}
and the command that ends the piece.

\subsection{Extracting single-instrument parts from multi-instrument scores}

To extract 
a single-instrument part from a (copy of a) multi-instrument \musixtex\ score:
\begin{itemize}
\item Set \verb|\nbinstrument| to 1; for example, with command \verb|\instrumentnumber1|.
\item Use \keyindex{TransformNotes} to discard all but one of the note segments in notes
commands.
For example, the following line placed  after
\verb|\startpiece|  (but  before any note commands) would be appropriate for a
four-instrument score and will result in a single-instrument part for the second of
these:
\begin{verbatim}
  \TransformNotes{#2&#3&#4&#5}{#3}%
\end{verbatim}
\end{itemize}
Caveats:
\begin{itemize}
\item
Some additional revisions to the source for the part might be necessary:
\begin{itemize}
\item  adjusting \verb|\setname1|, \verb|\setclef1|, \verb|\setsign1|, 
\verb|\setmeter1| and \verb|\setstaffs1|  commands, as necessary;

\item  ensuring  that  tempo and roadmap markings (\textbf{D.C.}, \textbf{Fine}, etc.) are in
  the appropriate instrument segment;
\end{itemize}
\item
When the extracted part score is compiled and viewed, it  may be seen
that horizontal-spacing commands designed for \emph{multiple} instruments can produce bad
spacing when used for a \emph{single} instrument.  Bad spacing can be corrected  manually  but
this is very tedious; an auxiliary program called \verb\fixmsxpart\
automates this process (for single-staff instruments only).
\end{itemize}

\subsection{Other applications}

The \keyindex{TransformNotes} macro may be used for other purposes. Here are some examples:
\begin{list}{}{}\item
     \verb|\TransformNotes{#2&#3}{#2&\transpose+7#3}%|
\end{list}
will begin transposing\index{transpose@{\Bslash transpose}} just the second instrument (argument \verb|#3|).
\begin{list}{}{}\item
  \verb|\TransformNotes{#2|\|\verb|#3&#4}{#2|\|\verb|#3&\tinynotesize#4}%|
\end{list}
will begin typesetting the notes of the second instrument (\verb|#4|) in tiny size.
\begin{list}{}{}\item
     \verb|\TransformNotes{#2&#3}{#3&#2}%|
\end{list}
will switch the order of the two instruments.
\begin{list}{}{}\item
     \verb|\TransformNotes{#2&#3}{#2&#3}%|
\end{list}
will restore normal two-instrument processing.


The \keyindex{TransformNotes} macro is currently not compatible with the 
\texttt{musixlyr} extension
package for lyrics described in Section~\ref{musixlyr}.  For extracting or omitting parts
in scores with lyrics, use the techniques described in Chapter~\ref{parts}. 

 \section{musixtri}\index{musixtri@{\tt musixtri.tex}}
Provides triply dotted note symbols.
Available are:
\keyindex{lpppt},
\keyindex{whppp},
\keyindex{zwppp},
\keyindex{huppp},
\keyindex{hlppp},
\keyindex{zhppp},
\keyindex{zhuppp},
\keyindex{zhlppp},
\keyindex{quppp},
\keyindex{qlppp},
\keyindex{zquppp},
\keyindex{zqlppp},
\keyindex{zqppp},
\keyindex{cuppp},
\keyindex{zcuppp},
\keyindex{clppp},
\keyindex{zclppp},
\keyindex{qbppp} and
\keyindex{zqbppp}.

