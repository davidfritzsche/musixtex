\chapter{Introduction to \musixtex}

This chapter is not a tutorial on the use of \musixtex, but instead serves as
an overview of some of its capabilities, quirks, and history.

\musixtex\ is a set of macros and fonts which enables music typesetting 
within the \TeX\ system. It requires as a prerequisite a working
installation of \TeX.
\musixtex\ might be regarded as the digital equivalent of a box of type. It
contains symbols for staves, notes, chords, beams, slurs and ornaments,
ready to be arranged to form a sheet of music. But it must be told how to position
those symbols on the page. This could be done by the typesetter
himself, if he or she elects to proceed by entering \musixtex\ commands manually
into an input file. However most users will find it far less taxing to let
such decisions be made largely by the preprocessor
\href{http://icking-music-archive.org/software/indexmt6.html#pmx}
{\underline{\textbf{PMX}}}, which in addition uses a much simpler input
language than \musixtex.

Lyrics can also be handled by \musixtex. There is a set of primitive
commands for this which are described later. But there is also a far
more adaptable set of macros contained in the extension file
\verb|musixlyr.tex|, and there is the preprocessor
\href{http://icking-music-archive.org/software/indexmt6.html#mtx}
{\underline{\mbox{\textbf{M-Tx}}}}\ which provides easy, transparent access to
these macros.

Most users of \TeX\ are familiar with \LaTeX, a set of \TeX\ macros
which eases document layout. In fact many may only use \LaTeX.
Until recently, \LaTeX\ and \musixtex\ coexisted only grudgingly, owing
primarily to the limited availabily of storage registers. But with modern
versions of \TeX\ and with the use of e\LaTeX, only a modest increase in
complexity is incurred with the addition of musical excerpts to a \LaTeX\
document. Still, fortunately, for typesetting a musical score there is
rarely if ever any advantage to using \LaTeX. Only if one wanted to create a
text document with embedded musical examples would there be much use for it.
Even in that case there is a perfectly fine way to avoid using \musixtex\
directly in the document file, namely, by using \musixtex\ to create
\verb|pdf| or \verb|eps| files for
each of the examples, and then embedding references to those in the file for
the larger document. But for anyone who still wants to use both together, there is
no better example than this manual, as generated with the files
\verb|musixdoc.tex| and \verb|musixdoc.sty|. A few further details about such
nonstandard applications are given in Section~\ref{excerpts}.

\section{Primary features of \musixtex}
 \subsection{Music typesetting is two-dimensional} Written music is not
usually a linear sequence of symbols like a literary text. Rather,
except for unaccompanied single-note instruments like clarinets,
trumpets and human voices, it has the form of a two-dimensional matrix.
 Thus, a logical way of coding music consists
in horizontally accumulating a set of \ital{vertical combs} with
\ital{horizontal teeth} as depicted in Table~\ref{readtable}.
%\def\hboxit#1{\boxit{\rlap{#1}\hphantom{note sequence three}}}
 \begin{table}
 \begin{center}
 \small
 \renewcommand{\arraystretch}{1.4}
 \setlength{\arrayrulewidth}{1.5pt}
 \setlength{\tabcolsep}{2ex}
 \begin{tabular}{|ll|ll|}\fbox{note sequence one}
  &\fbox{note seq.\ four}
  &\fbox{note seq.\ seven}
  &\fbox{note seq.\ ten}\\\fbox{note sequence two}
  &\fbox{note seq.\ five}
  &\fbox{note seq.\ eight}
  &\fbox{note seq.\ eleven}\\\fbox{note sequence three}
  &\fbox{note seq.\ six}
  &\fbox{note seq.\ nine}
  &\fbox{note seq.\ twelve}\\
 \end{tabular}
 \end{center}
 \caption{A logical way of coding music}\label{readtable}
 \end{table}
Accordingly, in \musixtex\ the fundamental macro used to represent
one of those vertical combs (or one of the columns in Table~\ref{readtable})
is of the form
\begin{center}
\verb|\notes ... & ... & ... \en|\footnote{The command
\keyindex{en} is an abbreviation for the command \keyindex{enotes}, which may also be used.}
\end{center}\index{"&@{\tt\char'046}}
\noindent where the character \verb|&| is used to separate the notes to be
typeset on respective staves of the various instruments, starting from the
bottom.

In the case of an instrument whose score has to be written with
several staves, they are separated by the
character \|\index{\tt\char'174@{\tt\char'174}}. 

Thus, a score written for a keyboard instrument and
a single-line instrument (e.g., piano and violin)
will be coded as follows:
\begin{center}
\verb=\notes ... | ... & ...\en=
\end{center}
\noindent for each column of simultaneous \ital{groups of notes}.
Each of those groups, represented by a single box in Table~\ref{readtable}\
and by a sequence of three dots in the the two example macros above, may contain
not only chord notes to be played simultaneously, but short sequences of
consecutive
notes or chords. As we'll soon see, this implies the need for two fundamentally different
kinds of basic macros in \musixtex: those whose tokens are automatically followed
by some amount of space (called \ital{spacing} macros), and those that are not.
The former type, for example,
would be used to produce the notes and rests in a single-line score;
the latter would be used, for example, for chord notes and ornaments.

\subsection{Horizontal spacing}
Deciding upon the proper horizontal spacing of notes is a very complicated matter
that we will not address in any detail here. Obviously short-duration notes
should be closer together than longer ones. Almost as obviously, the spacing
cannot be linearly proportional to the duration; otherwise  for example a whole
note would occupy 32 times as much horizontal space as a thirty-second note. And
in polyphonic scores the spacing in one staff is often influenced by the notes
in another. This is a decision that the typesetter or preprocessor
must make. Once the decision is made, \musixtex\ can provide the desired spacing.
The main mechanism is through a set of macros described in
Section~\ref{newspacings}. At this point we shall only mention that to control
spacing, one of those
macros will be selected to replace the symbol \verb|\notes| in the two examples
above, and it will imply that whenever a spacing macro in encountered within
a group of notes, a
certain specific amount of horizontal space will be inserted.

 \subsection{Music tokens}
The tokens provided by \musixtex\ include
\begin{itemize}\setlength{\itemsep}{0ex}
 \item note symbols without stems;
 \item note symbols with stems, and flags for eighth notes and
shorter;
 \item beam beginnings and endings;
 \item beginnings and endings of ties and slurs;
 \item accidentals;
 \item ornaments: arpeggios, trills, mordents, turns, staccatos,
pizzicatos, fermatas, etc.;
 \item bar lines;
 \item meters, key signatures, clefs.
\end{itemize}
For example, \verb|\wh a| produces a whole note at nominal frequency 222.5~Hz,
\verb|\wh h| produces one an octave higher, \verb|\qu c|
produces an up-stemmed quarter note C (250~Hz), and \verb|\cl J| produces a down-stemmed C
eighth note an octave lower.

 To generate chords with solid note heads, the non-spacing macro \keyindex{zq}
can be used. It produces a solid note head at the specified pitch,
the vertical position of which
is memorized and recalled whenever the next stemmed note (possibly with a flag) is
coded. The stem length is automatically adjusted to link all simultaneous notes. Thus, the
\mbox{C-major} chord

\begin{music}\nostartrule
\startextract\NOtes\zq{ceg}\qu j\en\zendextract
\end{music}
 \noindent is coded \verb|\zq c\zq e\zq g\qu j| or more concisely,
\verb|\zq{ceg}\qu j|; the \verb|u| in the spacing note macro
\verb|\qu| is what generates the upstem.

 \subsection{Beams}
Each beam \index{beams} is generated by a pair of macros. The first defines
the beginning horizontal position (implicitly the current position), altitude,
direction (upper or lower), multiplicity (number of lateral bars), slope and
reference number. This latter feature is needed so independent beams
can overlap. The second macro of the pair specifies the termination
location (again implicitly) and the reference number.

 \subsection{Setting anything on the score}
A general macro (\keyindex{zcharnote}) provides a means of
putting any sequence of symbols (possibly contained in an \verb|\hbox{...}|) at any
pitch of any staff of any instrument. This allows any symbol defined in a font
(letters, math symbols, etc.) to be placed in the score at a position keyed
to the music both in time (horizontally) and pitch (vertically) on the staff,
\section{A simple example}
 Before going into more detail, we give below an example of the two first
bars of the sonata in \mbox{C-major} KV545 by {\sc Mozart}\index{Mozart, W.A.@{\sc
Mozart, W.A.}}:

\begin{music}
\parindent11mm
\setname1{Piano}
\setstaffs12
\generalmeter{\meterfrac44}
\startextract
\Notes\ibu0f0\qb0{cge}\tbu0\qb0g|\hl j\en
\Notes\ibu0f0\qb0{cge}\tbu0\qb0g|\ql l\sk\ql n\en
\bar
\Notes\ibu0f0\qb0{dgf}|\qlp i\en
\notes\tbu0\qb0g|\ibbl1j3\qb1j\tbl1\qb1k\en
\Notes\ibu0f0\qb0{cge}\tbu0\qb0g|\hl j\en
\zendextract
\end{music}
 The coding is as follows:
\begin{quote}\begin{verbatim}
\begin{music}
  \parindent10mm
  \instrumentnumber{1}               % a single instrument
  \setname1{Piano}                   % whose name is Piano
  \setstaffs1{2}                     % with two staffs
  \generalmeter{\meterfrac44}        % 4/4 meter chosen
  \startextract                      % starting real score
    \Notes\ibu0f0\qb0{cge}\tbu0\qb0g|\hl j\en
    \Notes\ibu0f0\qb0{cge}\tbu0\qb0g|\ql l\sk\ql n\en
    \bar
    \Notes\ibu0f0\qb0{dgf}|\qlp i\en
    \notes\tbu0\qb0g|\ibbl1j3\qb1j\tbl1\qb1k\en
    \Notes\ibu0f0\qb0{cge}\tbu0\qb0g|\hl j\en
  \zendextract                       % terminate excerpt
\end{music}
 \end{verbatim}\end{quote}
\begin{itemize}\setlength{\itemsep}{0ex}
 \item \verb|\ibu0f0| begins an upper beam, aligned on the
\ital{f}, reference number 0, slope 0
 \item \verb|\tbu0| terminates this beam before writing
the second \ital{g} by means of \verb|\qb0g|
 \item \verb|\qb..| indicates a note belonging to a beam.
 \item \verb|\sk| sets a space between the two quarters in the right
hand, so that the second one is aligned with the third eighth in the left hand.
 \item\verb|\qlp| is a dotted quarter note.
 \item\verb|\ibbl1j3| begins a double beam, aligned on the \ital{C}
(\verb|j| at this pitch) with slope 15\%.
\end{itemize}

\section{The three pass system}\label{threepass}
\subsection{Introduction}
\TeX's line-breaking procedure implicitly assumes
that a normal line of text will contain many words, so that inter-word
glue need not stretch or shrink too much to justify the line.
One might at first consider extending this to music, treating each bar like a
word with no stretchable internal space. But typically this would lead to unsightly
gaps before each bar rule, simply because the number of bars per line
is normally many fewer than the number of words in a line of text.
\musixtex\ needs a more sophisticated horizontal spacing algorithm
than is used in \TeX.

To understand how \musixtex\ solves this problem, we have to
recognize two different kinds of horizontal space, \ital{hard}\ and
\ital{scalable}. Hard space is fixed and always represents the same
physical distance.
Examples of hard space are the widths of bar rules, clefs,
and key signatures. Scalable space can be stretched as needed. It
is what is normally used for the space after notes or rests. At the
outset it is only defined in a relative sense. In other words,
scalable spaces are defined as multiples of \verb|\elemskip|, an
initially undefined basic spacing unit. For example, in \textbf{PMX}\
all sixteenth notes
are typically assigned a scalable width of \verb|1.41\elemskip|.
One main job of \musixtex\ is to compute the physical value of
\verb|\elemskip|, often expressed in points (72nds of an inch).
The correct value is that which makes all the scalable space on
a line just fill up what's not occupied by hard space. Obviously
it may vary from line to line.

To this end a three pass system was developed. To start the
first pass on the file \verb|jobname.tex|, you would enter\footnote{
In many cases it will suffice to run {\tt tex}, but with \musixtex~1.15,
it is possible to create a file requiring the enhanced capacity of 
{\tt etex}. Since {\tt etex} can handle any file {\tt tex} can, it's best
always to use {\tt etex}.} 
\verb|etex jobname|.
Information about each bar is written to
an external file named \verb|jobname.mx1|.
This file begins with a header containing parameters
such as line width and paragraph indentation. Then the hard and
scalable space is listed for each bar.
\index{scalable width}\index{hard width}

The second pass, which is started with \verb|musixflx jobname|,
determines optimal values of the basic spacing unit \verb|\elemskip|
for each line, so as to properly fill each line, and to spread
the piece nicely over an integral number of full lines. This routine was originally written
in \textsc{fortran}  rather than \TeX, the main reason
being the lack of an array handling capability in \TeX; the current version
of \verb|musixflx| is a \textsc{lua} script, which may be executed without
compilation in any standard \TeX\ system.

\verb|musixflx| reads in the file \verb|jobname.mx1|, and writes its
output to \verb|jobname.mx2|. The latter file contains a single entry for
each line of music in the reformatted output. The key piece of information
is the revised value of \keyindex{elemskip} for each line.

Next, the file is \TeX-ed again, by entering \verb|etex jobname|. On this third pass,
the \verb|jobname.mx2| file is read in, and the information
is used to physically define the final score and embed the
page descriptions into a \verb|dvi|\ file.

You may wonder how the same command can
cause something different to happen the second time it is issued. The explanation
is that \musixtex\ checks for the presence of \verb|jobname.mx2|. If it's
not present, the first pass is executed; if it is, the third pass. Obviously if
you've made some corrections that affect any horizontal spacing, you must remember
to delete the old \verb|jobname.mx2| and then rerun all three passes, or
build or use a script that does so.

\subsection{An example}

Here's an example. Here, no value for \verb|\elemskip| is explicitly specified,
so \musixtex\ assigns a single, default value. After the first pass you might get
the following output:

\begin{music}
\hsize=100mm
\generalmeter{\meterfrac24}%
\parindent 0pt
\setsign1{-3}
\nostartrule
\startpiece\bigaccid
\NOtes\qu{ce}\en
\bar
\NOtes\qu{gh}\en
\bar
\NOtes\qu{=b}\en
\Notes\ds\cu g\en
\bar
\NOtes\qu{^f=f}\en
\raggedstoppiece\contpiece% this is cheated, the problem was to get
%                           pass 1 and pass 3 at the same time
\NOtes\qu{=e}\itied0e\qu{_e}\en
\bar
\Notes\ttie0\Qqbu ed{_d}c\en
\bar
\Notes\ibu0b{-2}\qb0{=b}\en
\notes\nbbu0\qb0{=a}\tqh0N\en
\Notes\Dqbu cf\en
\raggedstoppiece\contpiece
\NOtes\uptext{\it tr}\qu e\uptext{\it tr}\qu d\en
\bar
\NOtes\qu c\qp\en
\setdoubleBAR\raggedstoppiece
\end{music}

\noindent Note that the space after every quarter note is the same, and
that lines are not justified.
After running \verb|musixflx| and \TeX-ing the second time you'll
get:

\begin{music}
\hsize=100mm
\generalmeter{\meterfrac24}%
\parindent 0pt
\generalsignature{-3}
\nostartrule
\startpiece\bigaccid
\NOtes\qu{ce}\en
\bar
\NOtes\qu{gh}\en
\bar
\NOtes\qu{=b}\en
\Notes\ds\cu g\en
\bar
\NOtes\qu{^f=f}\en
\bar
\NOtes\qu{=e}\itied0e\qu{_e}\en
\bar
\Notes\ttie0\Qqbu ed{_d}c\en
\bar
\Notes\ibu0b{-2}\qb0{=b}\en
\notes\nbbu0\qb0{=a}\tqh0N\en
\Notes\Dqbu cf\en
\bar
\NOtes\uptext{\it tr}\qu e\uptext{\it tr}\qu d\en
\bar
\NOtes\qu c\qp\en
\Endpiece
\end{music}

\noindent Now \musixtex\ has determined a number of lines
(which is different from the original number), the lines are justified,
and if you look carefully you
can see that the space after quarters in the first line
is smaller than in the second. This example was coded as:
\begin{verbatim}
\hsize=100mm
\generalmeter{\meterfrac24}%
\parindent 0pt
\generalsignature{-3}
\nostartrule
\startpiece\bigaccid
\NOtes\qu{ce}\en\bar
\NOtes\qu{gh}\en\bar
\NOtes\qu{=b}\en
\Notes\ds\cu g\en\bar
\NOtes\qu{^f=f}\en\bar
\NOtes\qu{=e}\itied0e\qu{_e}\en\bar
\Notes\ttie0\Qqbu ed{_d}c\en\bar
\Notes\ibu0b{-2}\qb0{=b}\en
\notes\nbbu0\qb0{=a}\tqh0N\en
\Notes\Dqbu cf\en\bar
\NOtes\uptext{\it tr}\qu e\uptext{\it tr}\qu d\en\bar
\NOtes\qu c\qp\en\Endpiece
\end{verbatim}

\subsection{Adjusting the layout}

One benefit of the 3-pass system is the quick and easy
alteration to the layout which can be achieved
by changing only one parameter, namely \keyindex{mulooseness}. This value
acts analogously to \TeX's \verb|\looseness| command. For non-\TeX-perts: if you
state \verb|\looseness=-1| somewhere inside any paragraph, then
\TeX\ will try to make the paragraph one line shorter than it normally would.
With \verb|\mulooseness|, \musixtex\ does the same, but for \ital{systems}
and \ital{sections} rather than lines and paragraphs. A system is just a group of
staves treated as a unit, and in this discussion is analogous to a line of text.
What is a \ital{section}? It's any chunk of coding not containing a
forced system break, System breaks can be forced with
\keyindex{stoppiece}, \keyindex{endpiece}, \keyindex{zstoppiece},
\keyindex{Stoppiece}, \keyindex{Endpiece}, \keyindex{alaligne}
\keyindex{zalaligne}, \keyindex{alapage} or \keyindex{zalapage}. If none
of these is present, the section comprises the whole piece.
Somewhere\footnote{Advisably, at the beginning or
at the end of the section, for the sake of clarity.}
before the end of the section, you can change the value of
\keyindex{mulooseness} to something different from the default of 0, and
\musixtex\ will typeset that section with a different number of systems.

To give an easy example, changing the last line in the previous example to:
\begin{verbatim}
\NOtes\qu c\qp\en\mulooseness=1\Endpiece
\end{verbatim}
yields:

\begin{music}
\hsize=100mm
\generalmeter{\meterfrac24}%
\parindent 0pt
\generalsignature{-3}
\nostartrule
\startpiece\bigaccid
\NOtes\qu{ce}\en
\bar
\NOtes\qu{gh}\en
\bar
\NOtes\qu{=b}\en
\Notes\ds\cu g\en
\bar
\NOtes\qu{^f=f}\en
\bar
\NOtes\qu{=e}\itied0e\qu{_e}\en
\bar
\Notes\ttie0\Qqbu ed{_d}c\en
\bar
\Notes\ibu0b{-2}\qb0{=b}\en
\notes\nbbu0\qb0{=a}\tqh0N\en
\Notes\Dqbu cf\en
\bar
\NOtes\uptext{\it tr}\qu e\uptext{\it tr}\qu d\en
\bar
\NOtes\qu c\qp\en
\mulooseness1\Endpiece
\end{music}

\noindent On the other hand,
\begin{verbatim}
\NOtes\qu c\qp\en\mulooseness=-1\Endpiece
\end{verbatim}
yields

\begin{music}
\hsize=100mm
\generalmeter{\meterfrac24}%
\parindent 0pt
\generalsignature{-3}
\nostartrule
\startpiece\bigaccid
\NOtes\qu{ce}\en
\bar
\NOtes\qu{gh}\en
\bar
\NOtes\qu{=b}\en
\Notes\ds\cu g\en
\bar
\NOtes\qu{^f=f}\en
\bar
\NOtes\qu{=e}\itied0e\qu{_e}\en
\bar
\Notes\ttie0\Qqbu ed{_d}c\en
\bar
\Notes\ibu0b{-2}\qb0{=b}\en
\notes\nbbu0\qb0{=a}\tqh0N\en
\Notes\Dqbu cf\en
\bar
\NOtes\uptext{\it tr}\qu e\uptext{\it tr}\qu d\en
\bar
\NOtes\qu c\qp\en
\mulooseness-1\Endpiece
\end{music}

\noindent which is tighter than you would ever want, but serves to further
demonstrate the use of \keyindex{mulooseness}.

If you want to build up a \musixtex\ input file
manually (which in fact very few users will ever need to do, considering the
availability of \textbf{PMX}), here is a roadmap for one way to proceed:
\begin{enumerate}
 \item Enter the data one \verb|\notes| group at a time, taking care to select the
relative horizontal spacing for each group (via the macros discussed in Section~\ref{newspacings})
so as to maintain a consistent relationship between scalable space and note durations.
This will be discussed in a great deal more detail in Chapter~\ref{preparing}.
 \item \TeX~$\Longrightarrow$ {\tt musixflx} $\Longrightarrow$ \TeX.
 \item Look at the output and decide if you want to have more or fewer
systems, e.g.,~to fill the page or to get an even number of
pages.\index{musixflx@{\tt musixflx}}
 \item If you want to change the number of systems in a section, adjust
\keyindex{mulooseness} accordingly. Keep in mind that each section cannot have fewer
systems than bars.
\item Delete \verb|jobname.mx2| and repeat the process\\
\TeX~$\Longrightarrow$ \verb|musixflx| $\Longrightarrow$ \TeX.

\end{enumerate}

There is an alternate way to proceed if you know at the outset
how many systems
you want in a section. You can specify it directly by assigning a
positive number to \keyindex{linegoal} somewhere within the
section (\verb|\linegoal| requires
version 0.83 or later of \verb|musixflx|).
\keyindex{mulooseness} must be zero
for \keyindex{linegoal} to work. Both are automatically reset to
zero after processing the end of a section e.g., as defined by
\keyindex{stoppiece}.

Finally, for large scores (more than 4 pages or so), having
only one section and an overall value of \verb|\mulooseness| becomes
impractical since one wants not only to have nicely spaced systems, but also
completely filled pages without empty top and
bottom margins on the last page. It is then wise to force the total number of
pages and possibly the line breaks in each page, which can be done using
\keyindex{alapage} and/or \keyindex{alaligne}\footnote{This is the technique
always used by \textbf{PMX} in constructing a \musixtex\ input file.}.
%avrb
%, or more automatically by
%means of the \keyindex{autolines} command borrowed from \musictex\ and
%implemented in the additional {\ttxem{musixcpt.tex}} file
%avre

There is another advantage to \musixtex's way of using scalable space and
the three pass system. In
\TeX nical terms it eliminates the need for \ital{glue}, and enables
every horizontal position in a line to be computed. This in turn enables certain
variable length symbols such as slurs to be specified by macros entered at
their beginning and ending points, rather than having to estimate the
length of the symbol and enter it at the starting point.

 \subsection{Using \texttt{musixflx}}

On most systems, \verb|musixflx|  can be invoked by typing the name of the program
and the name of the file to be acted upon; i.e.,
\begin{quote}
\verb|musixflx jobname.mx1|
\end{quote}
Optionally, you can add a letter to indicate one of the debug modes, which are:
\begin{quote}
\verb|d | for debug information to screen\\
\verb|f | for debug information to file \verb|jobname.mxl|\\
\verb|s | to get the computed lines immediately on screen
\end{quote}
To allow for ease of use with scripts, \verb|musixflx| can be fed
with either \verb|jobname.mx1|, \verb|jobname.tex| or only
\verb|jobname|, any one of which will open \verb|jobname.mx1|.

 \subsection{Unrecorded spaces: the novice's bugaboo}

Because of the way \musixtex\ accounts for hard and scalable space and avoids
using glue, it is absolutely essential that every horizontal space be
properly entered into the input file. The most common error in this regard
is including a blank space in the midst \ital{or at the end} of an input line.
Such a blank space, or for that matter any stray character not entered with
an appropriate \musixtex\ macro, will not be properly
recognized and recorded by \musixtex, but it will still be treated like
ordinary text by \TeX. The symptoms of such a transgression will be an
{\bf Overfull hbox} warning during the third pass, and the appearance of
either excessive blank horizontal space or thick black vertical lines in
the page image.

\medskip
\centerline{\fbox{Considerable discipline is needed to avoid this problem!}}
\smallskip

The best technique for avoiding such unpleasantness is to avoid entering
any nonessential blanks within input lines, and to end every
input line with either \verb|%|\index{%@{\tt\char37}} or
\keyindex{relax}, unless it already ends with a control sequence
ending in a letter.

There are other ways to enter unrecorded space which should be
avoided. Between \verb|\startpiece| and \verb|\stoppiece| or
\verb|\endpiece|, never use \verb|\hskip| or \verb|\kern| except
within \ital{zero}-boxes like \verb|\rlap|, \verb|\llap|,
\verb|\zcharnote|, \verb|\uptext|, etc., and never assign hard
values to scalable dimensions like \verb|\noteskip|,
\verb|\elemskip|, \verb|\afterruleskip| and \verb|\beforeruleskip|\footnote{Note that {\Bslash hardspace} does not fall in this
category; it is specifically designed to enter hard space in a way
that \musixtex\ can properly record it}.

Here is a checklist of tips related to spacing issues. Because the
foregoing several paragraphs are so important, some of their content
is repeated in the list.

\begin{enumerate}
 \item
Between \verb|\startpiece| and \verb|\stoppiece| or \verb|\endpiece|,
end every input line with a either \verb|%| or a command with no arguments
(including \verb|\relax| and \verb|\en|).
 \item \verb|\off| must only be used with scalable values, e.g.,
\verb|\noteskip|, \verb|\elemskip|, \verb|\afterruleskip|, and
 \verb|\beforeruleskip|.
 \item Remember that \verb|\qqsk|, \verb|\qsk| and \verb|\hqsk| are scalable, i.e., \verb|\qsk|
  doesn't necessarily mean exactly one note head width (it depends on \verb|\elemskip|).
 \item Lyrics and any other non-\musixtex\ text must be put in zero-width boxes
such as
  \verb|\zcharnote|, \verb|\zchar|, \verb|\rlap|, \verb|\lrlap|, \verb|\llap|,
\verb|\uptext|, or \verb|\zsong|. Additional specific capabilities for entering
lyrics are provided by
 \verb|\hardlyrics|, \verb|\hsong|, and---most significantly---the
text-placement macros defined in \verb|musixlyr.tex| (see Chapter~\ref{lyrics},
p.~\pageref{lyrics}).
 \item
Between \verb|\startpiece| and \verb|\stoppiece| or \verb|\endpiece|,
don't assign hard values to \verb|\noteskip|, \verb|\beforeruleskip|,
or \verb|\afterruleskip|.
\end{enumerate}


\section{Further highlights}
 %\check
 \subsection{Key signatures}
\begin{sloppypar}
A single key signature can be assigned~to~all~instruments,~for~example~by 
\keyindex{generalsignature}\verb|{-2}|
which sets two flats on each staff. \keyindex{generalsignature}
can be overridden for selected instruments, for example by
\keyindex{setsign}\verb|2{1}| which puts one sharp on each staff
of instrument number 2. Of course, the current signature as well
as meters and clefs may change at any time.\end{sloppypar}
\subsection{Transposition}
With some extra attention, a score can be input in such a way that it is
fairly easily transposed. There is an internal register
called \keyindex{transpose}, the default value of which is zero, but which may be
set to any reasonable positive or negative value. It defines a number of pitch
steps (lines or spaces on a staff) by which all pitched symbols with be offset, provided
they have been entered with letter values to represent their pitch.
However,
it will neither change the local accidentals nor the key signature.

For
example, suppose a piece were originally input in F major, and it contained a
B natural, and you wanted to transpose it to G. If you simply set
\verb|\transpose| to 1 with no other special considerations, then
the key signature would not change, and
the B~natural would appear as a C~natural, whereas it should be a C~sharp.
So first you must explicitly change the key signature. Then, to
solve the problem with accidentals, you should declare
\verb|\relativeaccid|, which will cause the actual appearance of any
accidental to depend on the pitch of
the accidental as well as the current key signature. But the use of this
facility requires the typesetter to have entered the original set of accidentals
according to a nonstandard convention wherein an accidental does not
specifically refer to the black or white keys on a piano, but to the
amount by which the pitch is altered up or down from what it would naturally
have been, taking the key signature into account.\index{relative accidentals} This
is discussed in more detail in Section~\ref{transposeaccids}. Now, finally, you
can enter \verb|\transpose=1| with more or less the desired effect.

More or less, because there is also an issue with stem
and beam directions. Normally a typesetter would want full control over them,
and would exercise that control by entering them with macros that explicitly
assign the direction. Naturally the assigned directions would persist
even after changing \verb|\transpose|. With respect to stems of unbeamed notes,
this matter can also be addressed at
the input level, by using special macros for notes that leave the
decision about stem directions up to \musixtex. These macros do the right
thing in the face of transposition. They are discussed in
Chapter~\ref{autostemdirections}. Unfortunately there is no corresponding
such facility for beams, so the typesetter will have to edit the transposed
score to adjust beam directions as required\footnote{\textbf{PMX}
will automatically adjust both stem and beam directions when transposing, which
is yet another reason to use it. However
if a piece is to be transposed, the typesetter must still explicitly activate relative accidentals
and enter accidentals according to the relative-accidental convention.}.

 %\check
 \subsection{Extracting parts from a score} 
It is possible to extract separate scores for
each individual instrument of a multi-instrument score.
If you use \textbf{PMX}, it is very easy;
see the documentation for \textbf{PMX}.
If you use \musixtex\ without a preprocessor, 
see Chapter~\ref{parts}.

%After assigning symbolic numbers to instruments,
%there are macros (see \ref{instrum-inhibit}) that permit
% \begin{itemize}\setlength{\itemsep}{0ex}
% \item choosing which instrument the following source code is attached to,
% \item choosing which staff of an instrument the following source code is
%attached to,
% \item hiding one or several instruments by zeroing out their staff sizes and staff
%numbers.
% \end{itemize}

 %\check
 \subsection{Staff and note sizes}
 Although the standard staff size is 20pt, \musixtex\ allows scores with
sizes of 16, 24, or 29pt. Furthermore, any instrument may be assigned its own special staff
size (usually smaller than the overall staff size), and there are special macros
(e.g., \verb|\smallnotesize|, \verb|\tinynotesize|) that cause notes, beams, and
accidentals all to take a different size.

\subsection{Add-in macro libraries}
During the early stages of \musixtex's development, common versions of \TeX\ itself were
very limited in capacity, especially in terms of the numbers of registers that could
be defined for use in macros. For this historical reason, many important enhancements
to \musixtex\ are available only via add-in libraries. The user can thus pick and choose
which to include for any particular compilation. Most of these are included in
a \musixtex\ distribution, and their uses are discussed in this manual. The
libraries have names like \verb|blabla.tex|, and are activated by including a line
like \verb|\input blabla|\ within the the input file. The most common such files
are \ttxem{musixadd.tex} and \ttxem{musixmad.tex} which respectively increase the
number of instruments from the default 6 to 9 or 12 as well as increasing available numbers
of other features; \verb|musixps.tex| which enables Type~K Postscript slurs; and
\verb|musixlyr.tex| which greatly eases typesetting lyrics. In fact the latter two,
while now included in \musixtex\ distributions, are not documented in this manual but
in separate files, namely \verb|musixps.tex| itself and
\verb|mxlyrdoc.pdf| respectively.


\section{Installing and Using \musixtex}\label{using}

As stated in the preface, 
the \href{http://icking-music-archive.org/software/indexmt6.html}
{\underline{Werner Icking Music Archive}}\ (WIMA) contains excellent and detailed
instructions for installing \TeX, \musixtex{} and the strongly recommended
preprocessors \textbf{PMX}
(for instrumental music) and \mbox{\textbf{M-Tx}} (for vocal music) under
\href{http://icking-music-archive.org/software/htdocs/Getting_Started_Four_Scenar.html#SECTION00022000000000000000}
{\underline{Linux/\unix}}, 
\href{http://icking-music-archive.org/software/htdocs/Getting_Started_Four_Scenar.html#SECTION00021000000000000000}
{\underline{Windows}} and 
\href{http://icking-music-archive.org/software/htdocs/Getting_Started_Four_Scenar.html#SECTION00023000000000000000}
{\underline{Mac OS}}. 

As stated in Section~\ref{threepass}, to compile a \musixtex-formatted score, you must run 
in sequence \verb|etex|, \verb|musixflx|,
and \verb|etex| again. This will produce a \verb|.dvi| file. From this point
forward, this file can be processed like the device-independent output file
for any \TeX\ document. The most common approach and the only one to 
be discussed here is to next generate a Postscript file. This is accomplished
by running \verb|dvips|, which is included in every \TeX\ system. To view the 
Postscript, most users run {\tt GSview} or {\tt gv}, a front-end to {\tt ghostscript}, which must
be separately downloaded and installed. Finally, you may produce a PDF file 
using \verb|ps2pdf| or GSview, and view that with any PDF viewer, such as
the Adobe Reader or {\tt GSview}. 

You also can view the \verb|.dvi| file directly using a DVI viewer such as Yap or \verb|xdvi|, 
but, if you are using type K Postscript slurs, some DVI viewers will not display them.

Rather than executing all the foregoing programs separately, most users will 
find it more convenient to use the \verb|musixtex| script that comes in most
distributions of \musixtex. For example,

\verb|musixtex jobname|

\noindent will run the three passes described in Section~\ref{threepass},
convert the result to PDF and delete intermediate files.  
There are many options available to
change the behavior; run

\verb|musixtex -h|

\noindent to get a summary.


\section{Where to get help using \musixtex}

The home base for all matters related to \musixtex\ is the
Werner Icking Music Archive, at
\href{http://icking-music-archive.org}{\underline{http://icking-music-archive.org}}.
There is a  
\href{http://tug.org/mailman/listinfo/tex-music}{\underline{mailing list}}, 
hosted by TUG, the \TeX\ Users Group, where you will always find
someone willing to answer questions and help solve problems.

  \section{A very brief history of \musixtex}
The idea of using \TeX\ to typeset music appears to have originated
around 1987 with the master's thesis of Andrea
{\sc Steinbach} and Angelika {\sc Schofer}\footnote{Steinbach A. \& Schofer
A., \ital{Automatisierter Notensatz mit \TeX}, master's thesis,
Rheinische Friedrich-Wilhelms Universit\"at,
Bonn, Germany, 1987}. They called their package
\mutex\index{mutex@\protect\mutex}. It was limited to a single staff.
It introduced two key concepts: (1) using
a large number of font characters to construct beams and slurs, and (2) using
\TeX\ glue to help control horizontal spacing and justification.

The next major step came around 1991 when Daniel {\sc Taupin} created \musictex.
Its major enhancement was to allow multiple staves. But this came at a
price: some flexibility was lost in controlling horizontal spacing and
a great deal of trial and error became necessary to avoid excessive or
insufficient gaps before and after bar lines.

\musictex\ was a single-pass system. To remedy its shortcomings it became clear
that a multi-pass system would be required. Around 1997 Dr.~Taupin along with Ross {\sc Mitchell}
and Andreas {\sc Egler} created the first version of \musixtex. At last a fully
automatic procedure was coded so as to provide pleasing horizontal spacing in
multi-staff scores.

Significant enhancements to \musixtex, which have already been mentioned, have been provided
by Stanislav {\sc Kneifl} (Type~K Postscript slurs) and
Rainer {\sc Dunker} (Lyrics handling via \verb|musixlyr.tex|).

Since Dr.~Taupin passed away in 2003, \musixtex\ has been maintained by a 
varying cast including Olivier {\sc Vogel}, Hiroaki {\sc Morimoto}, 
Bob {\sc Tennent}, Andre {\sc Van Ryckeghem}, 
Cornelius {\sc Noack}, and Don {\sc Simons}.

No discussion of the history of \musixtex\ would be complete without mentioning
the contributions of Werner {\sc Icking}. From the early days of \mutex\ until
his untimely death in 2001, he served this line of software as its most prominent
proponent, beta tester, web site and mailing list editor, consultant,
problem solver, and inspiration for
many third-party enhancements including \textbf{PMX}. In fact he founded the
mailing list and the archive that now is named in his honor. The web site is
currently edited by Christian {\sc Mondrup}, the software page by Bob {\sc Tennent},
and the mailing list by Maurizio {\sc Codogno}.


