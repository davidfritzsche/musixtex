\chapter{Miscellaneous Other Notations}
%avre
 \section{Metronomic indications}
%Metronomic indication deserves a special macro. The mention:
%
%\medskip
%\centerline{\def\nbinstruments{0}\metron{\hup}{60}}
%
%\smallskip\noindent
%is coded by \keyindex{metron}\verb|{\hup}{60}| (normally embedded in
%\keyindex{Uptext} which is in turn embedded within \verb|\notes...\en|).
%
% On the other hand, music writers sometimes want to specify that the duration
%of a previous note is equal to a distinct furthernote. Thus
%
% \smallskip
%\centerline{\def\nbinstruments{0}\metronequiv{\qup}{\qu}}
%
%\smallskip\noindent
%is coded by \keyindex{metronequiv}\verb|{\qup}{\qu}|.

By way of example, the notations 
\begin{quote}
\def\nbinstruments{0}
\metron{\qu}{60}
\quad and\quad
\def\nbinstruments{0}\metronequiv{\qup}{\qu}
\end{quote}
are
respectively coded as
\keyindex{metron}\verb|{\qu}{60}| and \keyindex{metronequiv}\verb|{\qup}{\qu}|,
which are normally placed using \keyindex{Uptext}.
Also, you can set a smaller metronomic indication like
\begin{quote}
  \def\nbinstruments{0}\metron{\smallnotesize\lqu}{ca.~72}
\quad or \quad
\def\nbinstruments{0}\metron{\tinynotesize\lqu}{ca.~72}
\end{quote}
by writing 
\begin{quote}
\verb|\metron{\smallnotesize\lqu}{ca.~72}| 
\end{quote}
or 
\begin{quote}
\verb|\metron{\tinynotesize\lqu}{ca.~72}|
\end{quote}

 \section{Accents}

You may use\footnote{Note from the editor: the reason the names of some of
these macros don't seem to be constructed to suggest the terms used in the
descriptions is that whoever originally defined the macros had in mind terms
that did not agree with normal English usage.}
\begin{itemize}\setlength{\itemsep}{0ex}
 \item \keyindex{upz}\pitchp~(upper \itxem{staccato}) to put a dot above a note
head at pitch $p$,
 \item \keyindex{lpz}\pitchp~(lower \ital{staccato}) to put a dot below a note
head at pitch $p$,
 \item \keyindex{usf}\pitchp~(upper \itxem{sforzando}) to put a $>$ accent above
a note head at pitch $p$,
 \item \keyindex{lsf}\pitchp~(lower \itxem{sforzando}) to put a $>$ accent below
a note head at pitch $p$,
 \item \keyindex{ust}\pitchp~(upper \itxem{tenuto}) to put a
hyphen above a note head at pitch $p$,
 \item \keyindex{lst}\pitchp~(lower \ital{tenuto}) to put a
hyphen below a note head at pitch $p$,
 \item \keyindex{uppz}\pitchp~(upper \itxem{staccatissimo}) to put a solid vertical
wedge above a note head at pitch $p$,
 \item \keyindex{lppz}\pitchp~(lower \ital{staccatissimo}) to put an inverted
solid vertical wedge below a note head at pitch $p$,
 \item \keyindex{usfz}\pitchp~(upper \itxem{forzato}) to put a
``dunce cap'' above a note head at pitch $p$,
 \item \keyindex{lsfz}\pitchp~(lower \ital{forzato}) to put an inverted
``dunce cap'' below a note head at pitch $p$,
 \item \keyindex{upzst}\pitchp~(upper \itxem{staccato/tenuto}) to put a
dot and a hyphen above a note head at pitch $p$,
 \item \keyindex{lpzst}\pitchp~(lower \ital{staccato/tenuto}) to put a
dot and a hyphen below a note head at pitch $p$,
 \item \keyindex{flageolett}\pitchp~to put a
small circle above a note head at pitch $p$.
\end{itemize}
These marks are horizontally centered relative to solid note heads. 
To
compensate for the fact that whole notes are wider, you should use
\keyindex{wholeshift}\verb|{|\ital{Any nonspacing macro}\verb|}| to center accents
and other items (e.g.,~\verb|\Fermataup|) above a whole note.

There are also variants of the most common accents\footnote{Thanks to Klaus
{\sc Bechert}'s corrections.} which will be automatically positioned
above or below a beam. They are spelled like the corresponding normal
accent, but preceded with the letter \verb|b|, and their argument, instead of
the pitch, is the beam reference number . Thus

\medskip
\begin{music}\nostartrule
\startextract
\Notes\ibu0f3\busfz0\qb0f\bupz0\qb0g\bust0\qb0h%
  \buppz0\qb0i\busf0\qb0j\butext0\tqh0k\en
\Notes\Ibl0lg5\blsfz0\qb0l\blpz0\qb0k\blst0\qb0j%
  \blppz0\qb0i\blsf0\qb0h\bltext0\tqb0g\en
\zendextract
\end{music}
\noindent was coded as
\begin{quote}\begin{verbatim} \startextract
\Notes\ibu0f3\busfz0\qb0f\bupz0\qb0g\bust0\qb0h%
  \buppz0\qb0i\busf0\qb0j\butext0\tqh0k\en
\Notes\Ibl0lg5\blsfz0\qb0l\blpz0\qb0k\blst0\qb0j%
  \blppz0\qb0i\blsf0\qb0h\bltext0\tqb0g\en
\zendextract
\end{verbatim}\end{quote}
\zkeyindex{busfz}
\zkeyindex{bupz}
\zkeyindex{bust}
\zkeyindex{buppz}
\zkeyindex{busf}
\zkeyindex{butext}
\zkeyindex{blsfz}
\zkeyindex{blpz}
\zkeyindex{blst}
\zkeyindex{blppz}
\zkeyindex{blsf}
\zkeyindex{bltext}

The macros \verb|\bltext| and \verb|\butext| are detailed in the next section,
where the mystery of why they produce the number~$3$ is resolved.


\section{Numbers and brackets for xtuplets}\label{tuplet}

The following table lists all the special macros that place a number indicating
an xtuplet. Some also place a bracket above or below the notes, and are
intended for use with unbeamed notes.
In the table, $p$ is a pitch, $k$ is a number, $n$ is a beam number,
$w$ is a bracket width in \verb|\internote|s, and $s$ is the bracket slope as
a multiple of $1.125$~degrees. The macro \verb|\txt| contains a default number
which will be placed by the first and third through sixth macros,
and is initially defined as \verb|\def\txt{3}|. 
The macro
\verb|\tuplettxt| serves the same role for the last two macros. 
The font used is
\verb|\txtfont| which is by default \verb|\eightit| (in normal music size) but may be re-defined. 
The first four
are to be used with beamed xtuplets. As indicated in the
last column, the last four produce a sloping bracket and are to be used with
unbeamed xtuplets. 


\begin{center}
  \begin{tabular}{lclcc}
    \hline
    ~ & Number & ~ & Needs & ~ \\
    Macro and arguments & printed & Where invoked & Bracket \\
    \hline
    \verb|\triolet|\pitchp & \verb|\txt| & before beam & none \\
    \verb|\xtuplet|\itbrace{k}\pitchp & $k$ & before beam & none \\
    \verb|\butext|\itbrace{n} & \verb|\txt| & before note at number & none \\
    \verb|\bltext|\itbrace{n} & \verb|\txt| & before note at number & none \\
    \verb|\uptrio|\pitchp\itbrace{w}\itbrace{s} & \verb|\txt| & before first note & solid \\
    \verb|\downtrio|\pitchp\itbrace{w}\itbrace{s} & \verb|\txt| & before first note &  solid \\
    \verb|\uptuplet|\pitchp\itbrace{w}\itbrace{s} & \verb|\tuplettxt| & before first note & with gap \\
    \verb|\downtuplet|\pitchp\itbrace{w}\itbrace{s} & \verb|\tuplettxt| & before first note & with gap \\
    \hline
  \end{tabular}
\end{center}
\medskip

Here are some examples of the first six macros in the table:\\
\vskip-3pt
\begin{music}
\parindent0mm
\generalmeter{\meterfrac24}
\nostartrule
\startpiece
\addspace\afterruleskip
\notesp\xtuplet6n\isluru0l\ibl0l0\qb0{lllll}\tslur0l\tqb0l\en\bar
\notesp\triolet n\isluru0l\Ibl0ln2\qb0{lm}\tslur0n\tqb0n\en
\notesp\ibslurd0k\Ibl0km2\qb0k\bltext0\qb0l\tdbslur0m\tqb0m\en\bar
\Notesp\triolet o\isluru0l\ql{lm}\tslur0n\ql n\en\bar
\Notesp\uptrio o16\ql l\en\notesp\cl n\en
%avrb
\Notesp\downtrio O16\qu e\en\notesp\cu g\en
%avre
\endpiece
\end{music}
whose coding is
\begin{quote}\begin{verbatim}
\notesp\xtuplet6n\isluru0l\ibl0l0\qb0{lllll}\tslur0l\tqb0l\en\bar
\notesp\triolet n\isluru0l\Ibl0ln2\qb0{lm}\tslur0n\tqb0n\en
\notesp\ibslurd0k\Ibl0km2\qb0k\bltext0\qb0l\tdbslur0m\tqb0m\en\bar
\Notesp\triolet o\isluru0l\ql{lm}\tslur0n\ql n\en\bar
\Notesp\uptrio o16\ql l\en\notesp\cl n\en
\Notesp\downtrio O16\qu e\en\notesp\cu g\en
\end{verbatim}\end{quote}
\zkeyindex{xtuplet}\zkeyindex{triolet}

\medskip
Next are examples using \keyindex{uptuplet} and \keyindex{downtuplet}. 
A re-definition of \verb|\tuplettxt| inside a notes group only applies
inside that group and leaves the default definition intact.
To get the
number properly centered in the gap, you must include some extra space after
the number in the definition of \verb|\tuplettxt|. As you can see in
the last bar, if the span becomes too small, the macros still won't provide
enough room for the number in the gap; in this case it would be better to
use the macros without gaps.

%   Examples:
%
%       Draw a triplet bracket over the notes, starting at pitch "p",
%       1.25\noteskip wide, with a slope up of 3-3/8 degrees:
%
%           \uptuplet p{1.25}3
%
%       Draw a sextuplet bracket under the notes, starting at pitch "a",
%       2.5\noteskip wide, with a slope down of 4-1/2 degrees:
%
%           \def\tuplettxt{\smallppff 6\/\/}
%           \downtuplet{a}{2.5}{-4}

\begin{music}
%\input tuplet
\parindent0mm
\generalmeter{\meterfrac24}
%\startpiece
\nostartrule
\startextract
\addspace\afterruleskip
\notesp\triolet o\Ibl0ln2\qb0{lm}\tqb0n\en
\notesp\Ibl0km2\qb0k\bltext0\qb0l\tqb0m\en\bar
\notesp\def\tuplettxt{5\/\/}\uptuplet o{4.1}2\cl{jklmn}\en\bar
\Notesp\uptuplet o16\ql l\en\notesp\cl n\en
\Notesp\downtuplet O16\qu e\en\notesp\cu g\en\bar
\notesp\uptuplet o16\ccl l\en\notesp\cccl n\en
\notesp\downtuplet O16\ccu e\en\notesp\cccu g\en
%\endpiece
\endextract
\end{music}

\noindent with coding

\begin{quote}\begin{verbatim}
\parindent0mm
\generalmeter{\meterfrac24}
\startextract
\addspace\afterruleskip
\notesp\triolet o\Ibl0ln2\qb0{lm}\tqb0n\en
\notesp\Ibl0km2\qb0k\bltext0\qb0l\tqb0m\en\bar
\notesp\def\tuplettxt{5\/\/}\uptuplet o{4.1}2\cl{jklmn}\en\bar
\Notesp\uptuplet o16\ql l\en\notesp\cl n\en
\Notesp\downtuplet O16\qu e\en\notesp\cu g\en\bar
\notesp\uptuplet o16\ccl l\en\notesp\cccl n\en
\notesp\downtuplet O16\ccu e\en\notesp\cccu g\en
\end{verbatim}\end{quote}

 \section{Ornaments}


 \subsection{Simple ornaments}\index{ornaments}

The argument $p$ in the following macros for ordinary ornaments is the
pitch at which the ornament itself appears. They are all nonspacing macros.
You may use
\begin{itemize}\setlength{\itemsep}{0ex}
 \item \keyindex{mordent}\pitchp\ for \hbox to .75em{\mordent 0\hss}
 \item \keyindex{Mordent}\pitchp\ for \hbox to 1.5em{\kern 0.6em\Mordent 0\hss}
 \item \keyindex{shake}\pitchp\ for \hbox to .75em{\shake 0\hss}
 \item \keyindex{Shake}\pitchp\ for \hbox to 1.5em{\kern 0.6em\Shake 0\hss}
 \item \keyindex{Shakel}\pitchp\ for \hbox to 1.5em{\kern 0.6em\Shakel 0\hss}
 \item \keyindex{Shakesw}\pitchp\ for \hbox to 1.5em{\kern 0.6em\Shakesw 0\hss}
 \item \keyindex{Shakene}\pitchp\ for \hbox to 1.5em{\kern 0.6em\Shakene 0\hss}
 \item \keyindex{Shakenw}\pitchp\ for \hbox to 1.5em{\kern 0.6em\Shakenw 0\hss}
 \item \keyindex{turn}\pitchp\ for \hbox to 1.5em{\kern 0.6em\turn 0\hss}
 \item \keyindex{backturn}\pitchp\ for \hbox to 1.5em{\kern 0.6em\backturn 0\hss}
 \item \keyindex{tr}\pitchp\ for \hbox to 1.5em{\kern 0.6em\tr 0\hss}
 \item \keyindex{trt}\pitchp\ for \hbox to 1.5em{\kern 0.6em\trt 0\hss}\quad (J. S.~Bach's \textit{trillo und mordant})
 \item \keyindex{xtr}\pitchp\ for \hbox to 1.5em{\kern 0.5em\xtr 0\hss}
 \item \keyindex{ptr}\pitchp\ for \hbox to 1.5em{\kern 0.5em\ptr 0\hss}
 \end{itemize}
To
compensate for the fact that whole notes are wider, you may use
\begin{quote}
\keyindex{wholeshift}\verb|{|\mbox{\ital{Any nonspacing macro}}\verb|}| 
\end{quote}
to center any of these ornaments
above a whole note.


In the following macros for fermatas, the argument $p$ is the pitch of the
notehead on which the fermata rests, assuming no additional vertical
adjustments are needed for stems or intervening staff lines. They are all
nonspacing macros. You may use
\begin{itemize}\setlength{\itemsep}{0ex}
 \item \keyindex{fermataup}\pitchp\ for
\raisebox{0ex}[0ex][0ex]{\notesintext{\notes\fermataup1\en}},
\end{itemize}
\begin{itemize}\setlength{\itemsep}{3ex}
 \item \keyindex{fermatadown}\pitchp\ for
\raisebox{0ex}[0ex][0ex]{\notesintext{\notes\fermatadown2\en}},
 \item \keyindex{Fermataup}\pitchp\ for
\raisebox{-1ex}[-1ex][0ex]{\musicintextnoclef{\notes\Fermataup 7\zwh{'c}\en}}~,
centered over a whole note,
 \item \keyindex{Fermatadown}\pitchp\ for
\raisebox{-1ex}[0ex][0ex]{\musicintextnoclef{\notes\Fermatadown1\zwh{'c}\en}}~,
centered under a whole note.
 \end{itemize}
\bigskip

A \itxem{breath} mark
\raisebox{-5ex}[0ex][0ex]{\notesintext{\notes\zbreath\en}}
can be put just above the staff with \keyindex{zbreath}. This is a nonspacing
macro. On the other hand, \keyindex{cbreath} will cause a space of one
\verb|\noteskip| and place the comma midway through the space.

The \keyindex{caesura} command inserts a slash \verb|0.5\noteskip| before
the place it is entered, while adding no space:

\begin{music}\nostartrule
\startextract
\NOTes\zhu j\hl{^e}\caesura\en
\Notes\zcu j\cl e\en
\zendextract
\end{music}


 \subsection{Arpeggios}
\ital{Arpeggios} (i.e.~~~\arpeggio{-2}3) can be coded with the macro
\keyindex{arpeggio}\pitchp\itbrace{m}
where $p$ is the pitch of
the base of the arpeggio symbol and $m$ is its height in units of
\verb|\interligne|, the distance from one staff line to the next. It should be
issued before the affected chord. It is
automatically positioned to the left of the chord, but inserts no spacing.
Its variant \keyindex{larpeggio} sets the arpeggio symbol roughly
one note head width to the left of the default position, and is intended to
avoid collision with single accidentals on chord notes.

 \subsection{Arbitrary length trills}
 There are two styles of arbitrary length trills\index{trills}, each with two
variants.
For a trill with preassigned length, use \keyindex{trille}\pitchp\itbrace{l}
for \hbox to 1cm{\noteskip1cm\trille11\hss} or
\keyindex{Trille}\pitchp\itbrace{l} for~~\hbox to
2cm{\noteskip1cm\Trille12\hss}, where $p$ is the pitch and $l$ the length in
current \verb|\noteskips|.

To let \musixtex\ compute the length of the trill, or if it extends across a
system break, you can use \keyindex{Itrille}$n$\pitchp\ to start a plain
trill, where $n$ is a trill reference number between 0 and 5; then
\keyindex{Ttrille}$n$ to terminate it. To get the {\it tr} at the
beginning, use \keyindex{ITrille}$n$\pitchp\ to start the trill and
\verb|\Ttrille|$n$ to close it.

\begin{sloppypar}
As with other elements, you can specify the maximum number 
directly with \keyindex{setmaxtrills}\verb|{|$m$\verb|}|, where $7<m\leq 100$%
\footnote{This may require e-\TeX.}; 
the reference number $n$ will be in the range between $0$ and $m-1$.
\label{musixmad_setmaxtrills}
\end{sloppypar}


For example:

\medskip
 \begin{music}
\instrumentnumber{1}
\setstaffs12
\setclef1{6000}
%
\startextract
\notes\qu{CDEFGH}|\hu k\sk\ITrille 1p\itenu1l\wh l\en
\bar
\notes\Itrille 2A\itenu2E\whp E|\tten1\hlp l\sk\Ttrille 1\qu {mno}\en
\bar
\Notes\tten2\itenu2E\whp E|\ql{nmlkji}\en
\bar
\Notes\tten2\whp E\sk\sk\sk\sk\Ttrille2|\qu{hgfedc}\en
\endextract
\end{music}
whose coding is
\begin{quote}\begin{verbatim}
\begin{music}\nostartrule
\instrumentnumber{1}
\setstaffs12
\setclef1{6000}
%
\startextract
\notes\qu{CDEFGH}|\hu k\sk\ITrille 1p\itenu1l\wh l\en
\bar
\notes\Itrille 2A\itenu1e\whp E|\tten1\hlp l\sk\Ttrille 1\qu {mno}\en
\bar
\Notes\tten1\itenu1E\whp E|\ql{nmlkji}\Toctfin1\en
\bar
\Notes\tten1\whp E\sk\sk\sk\sk\Ttrille2|\qu{hgfedc}\en
\endextract
\end{verbatim}\end{quote}


 \subsection{Tremolos}\label{tremolos}

The command \keyindex{trml}\verb|{|$p$\verb|}|~is used to indicate that the immediately
following 
quarter or half note (normally at pitch $p$ and with a lower stem) is to be repeated in eighth notes.
For example,
\verb|\trml j\ha j|
generates

\begin{music}\nostartrule
\startextract
\NOTes\trml j\ha j\en
\zendextract
\end{music}\noindent
and is an abbreviation for

\begin{music}\nostartrule
\startextract
\notes\ibl0j0\qb0{jjj}\tql0j\en
\zendextract
\end{music}\noindent
Similarly, \keyindex{trmu}\verb|{|$p$\verb|}|~generates an eighth-note tremolo
on a quarter or half note with an \emph{upper} stem:

\begin{music}\nostartrule
\startextract
\NOTes\trmu h\ha h\en
\zendextract
\end{music}\noindent
16th-note and 32-note tremolos are similarly generated by
commands \keyindex{trrml}, \keyindex{trrmu}, \keyindex{trrrml}
and \keyindex{trrrmu}, as in 

\begin{music}
\nostartrule
\startextract
\NOtesp\trrml h\zqlp h\trrmu j\qup j\en
\Notes\stemlength{6.4}\trrml j\zcl j\trrmu l\cu l\en
\NOtes\stemlength{6.4}\trrrml h\zql h\trrrmu j\qu j\en
\endextract
\end{music}\noindent
which is coded as
\begin{quote}\begin{verbatim}
\NOtesp\trrml h\zqlp h\trrmu j\qup j\en
\Notes\stemlength{6.4}\trrml j\zcl j\trrmu l\cu l\en
\NOtes\stemlength{6.4}\trrrml h\zql h\trrrmu j\qu j\en
\end{verbatim}\end{quote}
The \keyindex{stemlength} command is used to adjust 
stem lengths;
the new length remains effective to the end of the current notes group. 
If necessary, the usual stem length can be restored by using
\verb|\stemlength{\DefaultStemlength}|.


For \emph{whole} notes, tremolo indications
should be wider and centered directly over or under the note. These 
are generated by commands \keyindex{Trml}, \keyindex{Trmu}, \keyindex{Trrml}, 
\keyindex{Trrmu}, \keyindex{Trrrml} and  \keyindex{Trrrmu}; for example,

\begin{music}
\nostartrule
\startextract
\NOTEs\Trml h\zw h\Trmu j\wh j\en
\bar%
\NOTEs\Trrml h\zw h\Trrmu j\wh j\en
\bar%
\NOTEs\Trrrml h\zw h\Trrrmu j\wh j\en
\endextract
\end{music}\noindent
which is coded as
\begin{quote}\begin{verbatim}
\NOTEs\Trml h\zw h\Trmu j\wh j\en\bar
\NOTEs\Trrml h\zw h\Trrmu j\wh j\en\bar
\NOTEs\Trrrml h\zw h\Trrrmu j\wh j\en
\end{verbatim}\end{quote}

For all of the commands described in this section, the pitch parameter may be used 
if it is necessary to raise or lower the tremolo indication.
%however, note values
%must be alphabetic and not numeric,
%because transposition is used to adjust the vertical position of the mini-beams.

 \section{Piano pedal commands}
The macro \keyindex{PED} inserts a piano-pedal command below the staff;
\keyindex{DEP}, a pedal release. Alternate symbols, the first of which
occupies less space, are invoked with \keyindex{sPED} and \keyindex{sDEP}.
For example,

\begin{music}
\setclef1\bass
\setstaffs1{2}
\startextract
\NOtes\PED\wh J|\qu h\en
\NOtes|\qu g\en
\NOtes|\hu k\en
\Notes\DEP\en
\bar
\NOtes\sPED\wh J|\qu h\en
\NOtes|\qu g\en
\NOtes|\hu k\en
\Notes\sDEP\en
\endextract
\end{music}
 \noindent was coded as \begin{quote}\begin{verbatim}
 \NOtes\PED\wh J|\qu h\en
 \NOtes|\qu g\en
 \NOtes|\hu k\en
 \Notes\DEP\en \bar
 \NOtes\sPED\wh J|\qu h\en
 \NOtes|\qu g\en
 \NOtes|\hu k\en
 \Notes\sDEP\en
 \end{verbatim}\end{quote}
The vertical position of
\keyindex{PED}, \keyindex{sPED}, \keyindex{DEP} and \keyindex{sDEP} can be
globally changed by redefining its elevation, which has the default
definition \verb|\def|\keyindex{raiseped}\verb|{-5}|.
To locally change the vertical position of a pedal symbol, use one of the more
fundamental
macros \keyindex{Ped}, \keyindex{sPed}, \keyindex{Dep} and \keyindex{sDep}
in combination with \keyindex{zchar} or \keyindex{zcharnote}.
Since the \hbox to .7cm{ \Ped\hss} symbol is rather wide, it might collide
with adjacent bass notes. To shift it horizontally, you could use \verb|\loff{\PED}|.

The following excerpt from \textit{Liebestr\"aume} by Franz Liszt illustrates
a more modern approach to specifying piano-pedal usage:

\begin{music}
\smallmusicsize
\relativeaccid
\parindent=0pt
\generalmeter{\meterfrac64}\generalsignature{-4}
\nobarnumbers
\setstaffs12\setclef1{\bass}
\def\raiseped{-11}
\def\qpl{\raise-2\Interligne\qp}
\def\qpu{\raise2\Interligne\qp}
\def\hppl{\lifthpausep{-2}}
\startpiece\addspace\afterruleskip%
\NOtes\zqu L\qpl|\qp\en
\bar%
\Notes\loff{\pdld}\ql H\sk\qpl\sk\qpl|%
  \zhl{.c}\ds\ibu1i0\qb1{ehjh}\tqu1e\en
\Notes\zhu{.c}\hppl|%
  \ds\ibu1i0\qb1{ehjh}\tqu1e\en
\bar%
\Notes\pdlud\ql G\sk\qpl\sk\qpl|%
  \itied1c\zhl{.c}\ds\ibu1i0\qb1{^eiji}\tqu1e\en
\Notes\hppl\sk\sk\sk\qu c|%
  \ttie1\zhl{c}\ds\ibu1i0\qb1{eiji}\tqu1e\en
\stoppiece%
\def\raiseped{-15}
\contpiece%%
\Notes\pdlud\ql F\sk\qpl\sk\zqu c\qpl|%
  \zhl{c}\ds\ibu1i0\qb1{=e^hjh}\tqu1e\en
\Notes\hppl\sk\sk\sk\qu c|%
  \zhl{d}\ds\ibu1i0\qb1{ehjh}\tqu1e\en
\bar%
\Notes\pdlud\ql B\sk\qpl\sk\qpl|%
  \zhl{.c}\ds\ibu1i0\qb1{^d^hjh}\tqu1d\en
\Notes\zqu{M}\hppl\sk\sk\qpu\sk\qu M|%
  \ds\ibu1i0\qb1{dhjh}\tqu1d\en
\bar%
\Notes\zmidstaff{$\cdots$}\sk\pdlu\sk|\en
\endpiece%
\end{music}
\vspace*{2ex}\par\noindent
This has been coded using the following commands:
\begin{quote}
\begin{tabular}{lll}
\keyindex{pdld}& pedal down & \huge$\llcorner$\\
\keyindex{pdlu}& pedal up & \huge$\lrcorner$\\
\keyindex{pdlud} & pedal up and immediately down & \_\raisebox{2pt}{$\!\bigwedge\!$}\_ 
\end{tabular}
\end{quote}
The elevation is determined by the value of \keyindex{raiseped}. 
Here is how the excerpt may be coded:
\begin{quote}
\begin{footnotesize}
\begin{verbatim}
\smallmusicsize\relativeaccid\nobarnumbers
\generalmeter{\meterfrac64}
\generalsignature{-4}
\setstaffs12\setclef1{\bass}
\def\raiseped{-11}
\def\qpl{\raise-2\Interligne\qp}
\def\qpu{\raise2\Interligne\qp}
\def\hppl{\lifthpausep{-2}}
\startpiece\addspace\afterruleskip%
\NOtes\zqu L\qpl|\qp\en
\bar%
\Notes\loff{\pdld}\ql H\sk\qpl\sk\qpl|%
  \zhl{.c}\ds\ibu1i0\qb1{ehjh}\tqu1e\en
\Notes\zhu{.c}\hppl|%
  \ds\ibu1i0\qb1{ehjh}\tqu1e\en
\bar%
\Notes\pdlud\ql G\sk\qpl\sk\qpl|%
  \itied1c\zhl{.c}\ds\ibu1i0\qb1{^eiji}\tqu1e\en
\Notes\hppl\sk\sk\sk\qu c|%
  \tltie1\zhl{c}\ds\ibu1i0\qb1{eiji}\tqu1e\en
\stoppiece%
\def\raiseped{-15}
\contpiece%
\Notes\pdlud\ql F\sk\qpl\sk\zqu c\qpl|%
  \zhl{.c}\ds\ibu1i0\qb1{=e^hjh}\tqu1e\en
\Notes\hppl\sk\sk\sk\qu c|%
  \zhl{d}\ds\ibu1i0\qb1{ehjh}\tqu1e\en
\bar%
\Notes\pdlud\ql B\sk\qpl\sk\qpl|%
  \zhl{c}\ds\ibu1i0\qb1{^d^hjh}\tqu1d\en
\Notes\zqu{M}\hppl\sk\sk\qpu\sk\qu M|%
  \ds\ibu1i0\qb1{dhjh}\tqu1d\en
\bar%
\Notes\zmidstaff{$\cdots$}\sk\pdlu\sk|\en
\endpiece
\end{verbatim}
\end{footnotesize}
\end{quote}

\pagebreak
\noindent
The following ``mixed'' patterns are also supported:
\begin{quote}
\begin{verbatim}
\PED ... [\pdlud].. ... \pdlu

\sPED ... [\pdlud].. ... \pdlu
\end{verbatim}
\end{quote}

\section{Alphabetic dynamic marks}
 Conventional dynamic symbols
\pppp, \ppp, \pp,
\p,
\mezzopiano,
\mf,
\f,
\fp,
\sF,
\ff,
\fff,
\ffff,
\sfz\ and
\sfzp\  can be posted using the macros
\keyindex{pppp},
\keyindex{ppp},
\keyindex{pp},
\keyindex{p},
\keyindex{mp},
\keyindex{mf},
\keyindex{f},
\keyindex{fp},
\keyindex{sF},
\keyindex{ff},
\keyindex{fff},
\keyindex{ffff},
\keyindex{sfz} and
\keyindex{sfzp},  respectively, as the second argument of \verb|\zcn| (\verb|\zcharnote|) or
\verb|\ccn| (\verb|\ccharnote|).
See also Section~\ref{musixppff} for an alternative set of dynamic marks.


 \section{Length of note stems}

The default length of note stems is the distance of one octave, i.e.,~\verb|7\internote| or \texttt{4.66}\keyindex{interbeam}. The default may be changed
with the macro \keyindex{stemlength}\itbrace{b}\ where $b$ is the length in
\verb|\interbeam|s.

The command \keyindex{stemcut} causes stems that extend outside the staff
to be shortened depending on the pitch of the notes. It is the default
behavior. To suppress this adjustment, issue the command
\keyindex{nostemcut}.

Normally, down stems never end above the middle line of the staff and up stems
never below that line. The command \keyindex{stdstemfalse} will inhibit this
adjustment, but only for the next stem generated.  There is no command to globally
suppress this behavior.

 \section{Brackets, parentheses, oblique lines and slides}\label{brapar}

Several varieties of brackets\index{brackets}, parentheses\index{parentheses} and oblique lines\index{oblique lines} are provided for
use within a score.
 \begin{itemize}\setlength{\itemsep}{0ex}
 \item \keyindex{lpar}\verb|{|$p$\verb|}| and
\keyindex{rpar}\verb|{|$p$\verb|}| yield left and right
parentheses at pitch $p$. They can be used to enclose notes or to build
``big'' cautionary\index{cautionary accidental} accidentals.  For example,
\\
\begin{center}
\begin{tabular}{ll}
\raisebox{-1.5ex}[0ex][0ex]{\musicintextnoclefn{\NOtes\qsk\lpar g\rpar g\hu g\sk%
\loffset{1.5}{\lpar g\rpar g}\loffset{.4}{\sh g}\hu g\sk%
\loffset{2.1}{\lpar g}\loffset{1.5}{\rpar g}\loffset{.4}{\dfl g}\hu g\en}}
\end{tabular}
\end{center}
is coded as
\end{itemize}
\begin{center}\small
\begin{tabular}{l}
\verb+\NOtes\lpar g\rpar g\hu g\sk%+\\
\verb+\loffset{1.5}{\lpar g\rpar g}\loffset{.4}{\sh g}\hu g\sk%+\\
\verb+\loffset{2.1}{\lpar g}\loffset{1.5}{\rpar g}\loffset{.4}{\dfl g}\hu g\en+
\end{tabular}
\end{center}
\begin{itemize}
 \item \keyindex{bracket}\verb|{|$p$\verb|}{|$n$\verb|}| posts a
square bracket to the left of a chord, vertically spanning $n$
\verb|internote|s.
 \item \keyindex{doublethumb}\verb|{|$p$\verb|}| indicates a bracket as
above spanning 2\verb|\internote|s.
 \item \keyindex{ovbkt}\verb|{|$p$\verb|}{|$n$\verb|}{|$s$\verb|}| and
\keyindex{unbkt}\verb|{|$p$\verb|}{|$n$\verb|}{|$s$\verb|}|
draw a sloped bracket starting at the current position at pitch $p$,
with horizontal extent $n$ \verb|noteskip|s and slope $s$ in multiples of
$1.125$~degrees.
 \item \keyindex{uptrio}\verb|{|$p$\verb|}{|$n$\verb|}{|$s$\verb|}| and
\keyindex{downtrio}\verb|{|$p$\verb|}{|$n$\verb|}{|$s$\verb|}| are like
\verb|\ovbkt| but with freely definable \keyindex{txt} centered inside.

 \item \keyindex{varline}\itbrace{h}\itbrace{\ell}\itbrace{s}\ builds an
oblique line starting at the current horizontal position. It must be used
inside a zero-width box, such as for example as the second argument of
\verb|\zcharnote|. $h$ is the height of the starting point, $\ell$ is the
length, and $s$ is the slope. It is used in the definitions of some of
the foregoing macros, and could be used, for example, to construct various
obscure baroque ornaments made up of diagonal lines.

\item \keyindex{slide}\itbrace{p}\itbrace{x}\itbrace{s}~ provides a \emph{glissando}
starting at pitch {\it p} and extending for {\it x} \verb|\internote|s with
slope {\it s} (ranging from $-8$ to $8$).

\end{itemize}

For example,

\begin{music}
\setstaffs1{2}
\setclef1\bass
\startextract
\NOtes\bracket C8\zq C\qu J\en
\NOtes|\doublethumb g\rq h\qu g\en
\NOtes\lpar c\rpar c\qu c\en
\NOtes\unbkt C15\qu {FH}|\ovbkt n14\ql{kl}\en
\NOtes\downtrio C16\qu {FH}|\uptrio o14\ql{lm}\en
\NOtesp\roffset{1.5}{\slide J33}\qa{JM}|\roffset{1.5}{\slide j33}\qa{jm}\en
\endextract
\end{music}
\noindent is coded as
\begin{quote}\small\begin{verbatim}
\begin{music}
\setstaffs1{2}
\setclef1{\bass}
\startextract
\NOtes\bracket C8\zq C\qu J\en
\NOtes|\doublethumb g\rq h\qu g\en
\NOtes\lpar c\rpar c\qu c\en
\NOtes\unbkt C15\qu {FH}|\ovbkt n14\ql{kl}\en
\NOtes\downtrio C16\qu {FH}|\uptrio o14\ql{lm}\en
\NOtesp\roffset{1.5}{\slide J33}\qa{JM}|\roffset{1.5}{\slide j33}\qa{jm}\en
\zendextract
\end{music}
\end{verbatim}\end{quote}

 \section{Forcing activity at the beginning of systems}
A macro named \keyindex{everystaff} is executed each time a new system
begins. It is normally void, but it can be defined (simply by \verb|\def|%
\keyindex{everystaff}\verb|{...}|) to cause \musixtex\ to post anything
reasonable at the beginning of each system. For it to affect the first system
in a score, it must be defined \ital{before}
\verb|\startpiece|.

If a macro named \keyindex{atnextline} is defined at any point in a score,
it will be executed just once, viz., at the next computed or forced system
break. More
precisely, it is executed after the break and before the next system begins.
Thus it is suitable for redefining layout parameters.

 \index{octave treble clef}In some scores, tenor parts are not coded using the
\ital{bass} clef, but using rather the \ital{octave treble clef}, which is
subscripted by a numeral {\tt8}.
This clef is supported by the clef substitution command
\keyindex{settrebleclefsymbol}\onen\keyindex{treblelowoct}, as already
explained in Section~\ref{treblelowoct}. However, if for some reason you
aren't happy with the height of the ``8'', it can be posted on selected
staves at the beginning of every system using \keyindex{everystaff} and
\keyindex{zcharnote} as follows:

\begin{music}
\instrumentnumber{4}\setclef1\bass
\def\everystaff{\znotes&\zchar{-6}{\eightrm\kern-2\Interligne8}%
  &\zchar{-6}{\eightrm\kern-2\Interligne8}\en}%
\startextract
\NOTes\ha{HIJK}&\ha{efgh}&\ha{hijk}&\ha{hmlk}\en
\zendextract
\end{music}
\noindent
The coding is
\begin{quote}\begin{verbatim}
\instrumentnumber{4}\setclef1\bass
\def\everystaff{\znotes&\zchar{-6}{\eightrm \kern -2\Interligne 8}%
  &\zchar{-6}{\eightrm \kern -2\Interligne 8}\en}%
\startextract
\NOTes\ha{HIJK}&\ha{efgh}&\ha{hijk}&\ha{hmlk}\en
\endextract
\end{verbatim}\end{quote}
