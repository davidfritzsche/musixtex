%% PostScript slurs, ties and crescendos
%   Coded by Stanislav Kneifl

\ifx\undefined\liftcresc\else\endinput\fi

\immediate\write16{MusiXPS PostScript slurs, ties and crescendos 0.92 (12.5.2002)}%

\catcode`\@=11


%%%% remove the next line if you include this file into a format (see endnotes!).

\special{header=psslurs.pro}%

%%%%%%%%%%%%%%%%%%%%%%%%%%%%%%%%%%%%%%%

\font\mxsk=mxsk

% you can change these values if you want...
% Offsets are in \internotes measured vertically from the center and
%   horizontally from the left edge of the notehead.
% see the doc at the end for (short) explanation

\def\psslurraise{1.6}      % vertical offset of the slur beginnings/endings
\def\psbslurraise{2}       % vert. offset for slurs on beams

\def\psLslurskip{-0.2}     % horiz. offset of slurs shifted more to the left
\def\pslslurskip{0.4}      % horiz. offset of slurs shifted to the left
\def\psislurskip{1.4}      % horiz. offset of the slur beginnings
\def\pstslurskip{1.2}      % horiz. offset of the slur endings
\def\psrslurskip{2.8}      % horiz. offset of slurs shifted to the right

\def\psfslurhgt{0.7}       % height coeficient of the 'f'lat slurs
\def\psslurhgt{1}          % height coeficient of normal slurs
\def\pshslurhgt{1.6}       % height coeficient of 'h'igh slurs
\def\psHslurhgt{2.2}       % height coeficient of 'H'igh slurs
\def\psHHslurhgt{2.8}      % height coeficient of 'HH'igh slurs
\def\psslurmaxhgt{40}      % maximal height of the slur
                           %  (before multiplication by \...slurhgt !!)

\def\psslurangul{0.17}     % angularity of normal, flat and high slurs
\def\psHslurangul{0.15}    % angularity of High slurs
\def\psHHslurangul{0.13}   % angularity of HHigh slurs

\def\pstieraise{1.3}       % vertical offset of the ties
\def\psitieskip{2.6}       % horiz. offset of the tie beginnings
\def\psttieskip{0.3}       % horiz. offset of the tie endings
\def\psltieskip{-0.2}      % horiz. offset of ties shifted to the left
\def\psrtieskip{3}         % horiz. offset of ties shifted to the right

\def\pstiehgt{0.7}         % height coeficient of the ties
\def\pstiemaxhgt{30}       % maximal height if the ties
\def\pstieflathgt{20}      % maximal height if the ties between staff lines
\def\pstieangul{0.14}      % angularity of the ties

%%%%

\def\d@nh{-5}              % default altitude of the crescendos in \internote
\def\setcrescheight{\def@it\d@nh}

\def\sl@space{-3pt}			% neg. space before continued slurs, ties and crescs.
\def\h@tielim{15pt}			% limit for drawing half continued ties

\def\nohalfties{\def\h@tielim{\z@}}
\def\halfties{\def\h@tielim{15pt}}
\newif\ifslopebrkslurs
\slopebrkslursfalse

\def\dotted{\def\d@sh{1}}  % the '1' is important!
\def\solid{\def\d@sh{0}}
\def\Dotted{\def\D@sh{1}}
\def\Solid{\def\D@sh{0}}
\Solid\solid

\def\nosluradjust{\gdef\s@adj{0}}
\def\sluradjust{\gdef\s@adj{1}}
\def\Nosluradjust{\gdef\S@adj{0}\gdef\s@adj{0}}
\def\Sluradjust{\gdef\S@adj{1}\gdef\s@adj{1}}
\Sluradjust

\def\notieadjust{\gdef\t@adj{0}}
\def\tieadjust{\gdef\t@adj{1}}
\def\Notieadjust{\gdef\T@adj{0}\gdef\t@adj{0}}
\def\Tieadjust{\gdef\T@adj{1}\gdef\t@adj{1}}
\Notieadjust

\newif\ifdangermessages\dangermessagestrue

\def\@clr#1{\let#1\empty}
\def\def@it#1#2{\def#1{#2}}
\def\slurtext{\def@it\s@tx}
\let\s@tx\empty
\def\lr@lap#1{\rlap{\hss#1}}

% how many slurs will be stored into the TeX's registers?
% => how many slurs can go across the page break?
% determines also the needed amount of the following registers.
\let\max@slurs\t@n

\let\slur@alt=\s@xi     \let\slur@hgt=\s@Yv     \let\slur@lift=\s@yi
\let\slur@alti=\s@xii   \let\slur@hgti=\s@Yvi   \let\slur@lifti=\s@yii
\let\slur@altii=\s@xiii \newdimen\slur@hgtii    \let\slur@liftii=\s@yiii
\let\slur@altiii=\s@xiv \newdimen\slur@hgtiii   \let\slur@liftiii=\s@yiv
\let\slur@altiv=\s@xv   \newdimen\slur@hgtiv    \let\slur@liftiv=\s@yv
\let\slur@altv=\s@Y     \newdimen\slur@hgtv     \let\slur@liftv=\s@yvi
\let\slur@altvi=\s@Yi   \newdimen\slur@hgtvi    \let\slur@liftvi=\s@Ni
\let\slur@altvii=\s@Yii \newdimen\slur@hgtvii   \let\slur@liftvii=\s@Nii
\let\slur@altviii=\s@Yiii \newdimen\slur@hgtviii  \let\slur@liftviii=\s@Niii
\let\slur@altix=\s@Yiv  \newdimen\slur@hgtix    \let\slur@liftix=\s@Niv

% in the 'lift' register is coded the altitude of the ending point of the slur
% at the end of a line (y1) and the relative alt. of the starting point at the
% beginning of a line (y2): y1=(reg div 65536)-16384, y2=(reg mod 65536)-16384

\newdimen\tie@alt       \newdimen\tie@hgt
\newdimen\tie@alti      \newdimen\tie@hgti
\newdimen\tie@altii     \newdimen\tie@hgtii
\newdimen\tie@altiii    \newdimen\tie@hgtiii
\newdimen\tie@altiv     \newdimen\tie@hgtiv
\newdimen\tie@altv      \newdimen\tie@hgtv
\newdimen\tie@altvi     \newdimen\tie@hgtvi
\newdimen\tie@altvii    \newdimen\tie@hgtvii
\newdimen\tie@altviii   \newdimen\tie@hgtviii
\newdimen\tie@altix     \newdimen\tie@hgtix

\newdimen\cresc@alt
\newdimen\cresc@alti
\newdimen\cresc@altii
\newdimen\cresc@altiii
\newdimen\cresc@altiv
\newdimen\cresc@altv
\newdimen\cresc@altvi
\newdimen\cresc@altvii
\newdimen\cresc@altviii
\newdimen\cresc@altix

\newcount\pss@rega \pss@rega=\z@   % PS slurs
\newcount\pss@regb \pss@regb=\z@
\newcount\pst@rega \pss@rega=\z@   % PS ties
\newcount\pst@regb \pss@regb=\z@
\newcount\psc@rega \pss@rega=\z@   % PS [de]crescendos
\newcount\psc@regb \pss@regb=\z@

\newif\if@up
\newif\if@dash

% information in these registers is binary coded:
% a: lower bit is set ( 2^2i ): there is a pending slur number i
%    upper bit is set ( 2^(2i+1) ): slur number i already went across line break
% b: lower bit is set: slur/tie is dotted
%    upper bit is set: slur/tie is lower / decrescendo

% everything else is stored in PostScript variables. If you run out
% of the PostScript dictionary used for slurs, you can increase its size
% by changing the number 200 in the first line of psslurs.pro to anything
% reasonable. (The number says how many tokens can be defined.)

%%%%%%%

% input is i for 2^i and register to extract the bits from
% in reg. n@i will be only the bits 2^2i and 2^(2i+1)
% in n@vi will be these bits shifted to 2^0 and 2^1,
% n@ii will be set to 2^2i
\def\get@bits#1#2{\n@i=#1\relax\n@ii\@ne
	\loop\ifnum\n@i>\z@\multiply\n@ii\f@ur\advance\n@i\m@ne\repeat
	\n@i=\n@ii\multiply\n@i\f@ur\n@iii #2\relax
	\divide\n@iii\n@i\multiply\n@iii\n@i\n@i=#2\relax
	\advance\n@i-\n@iii\divide\n@i\n@ii\n@vi\n@i\multiply\n@i\n@ii\relax}

\def\s@l@slur#1{\xdef\th@salt{\csname slur@alt\romannumeral#1\endcsname}%
  \xdef\th@shgt{\csname slur@hgt\romannumeral#1\endcsname}%
  \xdef\th@slift{\csname slur@lift\romannumeral#1\endcsname}}

\def\s@l@tie#1{\xdef\th@salt{\csname tie@alt\romannumeral#1\endcsname}%
  \xdef\th@shgt{\csname tie@hgt\romannumeral#1\endcsname}}

\def\s@l@cresc#1{\xdef\th@salt{\csname cresc@alt\romannumeral#1\endcsname}}

\def\gt@pssb#1#2{\s@l@ctbeam#1\relax\getcurpos\advance\y@v#2\relax	% \y@v = hpos
	\advance\y@v-\b@x
	\y@ii\b@p\y@v \divide\y@ii\tw@nty \advance\y@ii\b@z
	\advance\y@ii-\altportee\n@ii=\internote \divide\n@ii6553
	\n@i=\y@ii\divide\n@i\n@ii\divide\n@i6554\relax}

\def\t@bslur#1#2#3#4{\n@vii\n@i\ifnum#3<\fiv@t@@n\get@bits{#3}\pss@regb\n@i\n@vii
		\ifnum\n@vi>\@ne\gt@pssb{#4}\z@\T@slur{#3}{\psbslurraise}{0}{#1}{#2}%
		\else\gt@pssb{#4}{\tw@\qn@width}\T@slur{#3}{\psbslurraise}{2.4}{#1}{#2}\fi
	\else\gt@pssb{#4}\z@\T@slur{#3}{1.7}{0}{#1}{#2}\fi}


%%%%%%  slurs  %%%%%%%%%%%%%%%%%%%

% ID, pitch, voffset, hoffset
% all offsets are in \internote, the slur direction is determined
%   by the sign of the vertical offset.

\def\iSlur#1#2{\inhgetn@i#2\relax\I@slur{#1}}

% ID, voffset, hoffset; \n@i = pitch
\def\I@slur#1#2#3{\y@i=\n@i\internote
	\ifdim#2\internote<\z@\@upfalse\else\@uptrue\fi
	\advance\y@i#2\internote\raise\y@i\rlap{\hskip #3\internote
		\special{ps: slur@Dict begin CP /S#1y1 ED /S#1x1 ED
		/S#1hgt (\the\y@i) GD N /S#1brk S#1hgt (\the\internote) GD
		\ifslopebrkslurs 3 \else 1 \fi\if@up 1 \else -1 \fi M M A N /S#1lift 0 N
		/S#1alt (\the\altportee) GDAR N /S#1dir \if@up1\else-1\fi\space N
		/S#1dash \d@sh\space N /S#1adj \s@adj\space N end}}%
	\ifnum#1>\fourt@@n
		\ifnum#1=99\relax\else
			\ifdangermessages\message{Danger! Slur with ID >14.}\fi\fi
	\else\n@iv\n@i\get@bits{#1}\pss@rega
		\ifcase\n@i\else\message{Slur #1 was re-opened.}%
			\advance\pss@rega-\n@i\fi  % clear the bits
		\global\advance\pss@rega\n@ii % and set the lower bit
		\get@bits{#1}\pss@regb\global\advance\pss@regb-\n@i  % clear the bits
		\ifnum\d@sh=1\relax\global\advance\pss@regb\n@ii\fi
		\if@up\else\multiply\n@ii\tw@\global\advance\pss@regb\n@ii\fi
		\ifnum#1<\max@slurs\s@l@slur{#1}%
			\global\th@salt\altportee\global\th@shgt\y@i\th@slift\n@iv
			\if@up\advance\th@slift\fiv@\else\advance\th@slift-\fiv@\fi
			\advance\th@slift 16384\relax\multiply\th@slift65536\relax
			\global\advance\th@slift16384\relax
		\fi\fi\gdef\d@sh{\D@sh}\gdef\s@adj{\S@adj}}

% ID, height, voffset, hoffset, curvature, angularity
\def\tSlur#1#2{\inhgetn@i#2\relax\T@slur{#1}}

\def\T@slur#1#2#3#4#5{\n@vii\n@i\n@ii\@ne
	\ifnum#1<\fiv@t@@n
		\get@bits{#1}\pss@rega
		\global\advance\pss@rega-\n@i
		\ifcase\n@i\message{Not opened slur #1 was closed.}\n@ii\z@
		\else
			\ifnum\n@vi>\@ne             % the slur was breaked
				\get@bits{#1}\pss@regb    % it's not necessary to clear this register
				\ifnum\n@vi>\@ne\@upfalse\else\@uptrue\fi
				\ifodd\n@vi\@dashtrue\else\@dashfalse\fi
				\y@ii=\lin@pos\advance\y@ii\locx@skip\advance\y@ii#3\internote
				\advance\y@ii\sl@space
				\y@i=\n@vii\internote\advance\y@i\if@up\else-\fi#2\internote
				\ifnum#1<\max@slurs
					\s@l@slur{#1}\n@ii\th@slift\divide\n@ii65536\relax
					\multiply\n@ii65536\relax\n@iii\th@slift\advance\n@iii-\n@ii
					\advance\n@iii-16384\y@iii=\n@iii\internote\advance\y@iii\th@shgt
					\raise\y@i\rlap{\hskip #3\internote\special{ps: slur@Dict begin
						CP S (\the\y@ii) GDAR SB /S#1x1 ED
						(\the\y@i) GD (\the\y@iii) GD SB 4 AR M A /S#1y1 ED
						(\the\internote) GD (\the\th@shgt) GD (\the\y@i) GD
						S#1x1 S#1y1 CP #4 \if@up\else neg \fi #5 30
						\if@dash1\else0\fi\space \s@adj\space 90 DS end}}%
				\else
					\raise\y@i\rlap{\hskip #3\internote\special{ps: slur@Dict begin
						CP S (\the\y@ii) GDAR SB /S#1x1 ED
						S#1hgt S#1lift (\the\internote) GD M A 4 AR M
						(\the\y@i) GDAR SB SB /S#1y1 ED
						(\the\internote) GD S#1hgt (\the\y@i) GD
						S#1x1 S#1y1 CP #4 \if@up\else neg \fi #5 30
						S#1dash \s@adj\space 90 DS end}}%
				\fi\n@ii\z@
			\fi
		\fi
	\fi
	\ifcase\n@ii\else\y@ii=#2\internote\setbox\z@=\hbox{\s@tx}%
		\y@i=\n@vii\internote\raise\y@i\rlap{\hskip #3\internote
		\special{ps: slur@Dict begin (\the\internote) GD S#1hgt (\the\y@i) GD
			#2 (\the\internote) GD M S#1dir M A
			S#1x1 S#1y1 CP (\the\y@ii) GDAR S#1dir M SB 
			#4 S#1dir M #5 \psslurmaxhgt\space S#1dash S#1adj 90 DS
			/ay1 ay1 (\the\ht\z@) GDAR -10 AR add dir mul add def}%
		\lr@lap{\box\z@}\special{ps: end}}%
	\fi\@clr\s@tx}

\def\liftslur#1#2{\special{ps: slur@Dict begin /S#1lift #2 N end}%
	\ifnum#1<\t@n\s@l@slur{#1}\divide\th@slift65536\relax
		\multiply\th@slift65536 \advance\th@slift16384\relax
		\advance\th@slift#2\relax\fi}

\def\breakslur#1#2{\inhgetn@i#2\relax\y@i=\n@i\internote\special
	{ps: slur@Dict begin /S#1brk (\the\y@i) GD N end}%
	\ifnum#1<\t@n\s@l@slur{#1}\n@ii\th@slift\divide\n@ii65536\relax
	\multiply\n@ii65536 \advance\th@slift-\n@ii\advance\n@i16384\relax
	\multiply\n@i65536 \global\advance\th@slift\n@i\fi}

%%%%%  [de]crescendos  %%%%%%%%%%%%%%%%%%%%%%%%%%%%%%%

% ID, height, hoffset

\def\Icresc{\i@cresc0}
\def\Idecresc{\i@cresc2}

\def\i@cresc#1#2#3#4{\y@i=#3\internote\raise\y@i\rlap
	{\hskip#4\internote\advance\y@i\altportee\special{ps: slur@Dict begin
		CP /C#2y1 ED /C#2x1 ED /C#2alt (\the\y@i) GD N
		/C#2dir #1 N end}}%
	\ifnum#2>\fourt@@n
		\ifdangermessages\message{Danger! [De]crescendo with ID >14.}\fi
	\else\get@bits{#2}\psc@rega
		\ifcase\n@i\else\message{[De]crescendo #2 was re-opened.}%
			\advance\psc@rega-\n@i\fi
		\global\advance\psc@rega\n@ii
		\get@bits{#2}\psc@regb\advance\psc@regb-\n@i
		\multiply\n@ii#1\relax
		\global\advance\psc@regb\n@ii\relax
		\ifnum#2<\max@slurs\s@l@cresc{#2}%
			\advance\y@i\altportee\global\th@salt\y@i\fi\fi}

\def\Tcresc#1#2#3{\n@ii\@ne\n@i=#1\relax
	\ifnum\n@i<\fiv@t@@n
		\get@bits{#1}\psc@rega\global\advance\psc@rega-\n@i
		\ifcase\n@i\message{Not opened crescendo #1 was closed.}\n@ii\z@
		\else
			\ifnum\n@vi>\@ne                    % the crescendo was breaked
				\get@bits{#1}\psc@regb    % it's not necessary to clear this register
				\y@ii=\lin@pos\advance\y@ii\locx@skip\advance\y@ii#3\internote
				\advance\y@ii\sl@space
				\zchar{#2}{\hskip #3\internote\special{ps: slur@Dict begin
					CP S (\the\y@ii) GDAR SB S CP 4 \the\n@vi\space R
					\the\n@vi\space 0 eq { DHC } { DC } ifelse end}}%
				\global\n@ii\z@\fi
	\fi\fi
	\ifcase\n@ii\else\zchar{#2}{\hskip#3\internote\special{ps: slur@Dict begin
			C#1x1 C#1y1 CP 4 C#1dir R DC end}}\fi}

\def\liftcresc#1#2{\y@i=#2\internote\ifnum#1<\max@slurs\s@l@cresc{#1}%
	\global\advance\th@salt\y@i\else\special{ps: slur@Dict begin
	/C#1alt (\the\y@i) GDAR C#1alt A N end}\fi}


%%%%  ties  %%%%%%%%%%%%%%%%%%%%%%

\def\iTieu{\i@tie{1}}
\def\iTied{\i@tie{-1}}

% 1/-1, ID, height, hoffset
\def\i@tie#1#2#3#4{\inhgetn@i#3\relax\y@i=\pstieraise\internote
	\multiply\y@i#1\relax\advance\y@i\n@i\internote\raise\y@i\rlap
		{\hskip #4\internote\special{ps: slur@Dict begin
		/S#1alt (\the\altportee) GDAR N
		CP /T#2y1 ED	/T#2x1 ED /T#2hgt (\the\y@i) GD N
		/T#2dir #1 N /T#2dash \d@sh\space N
		/T#2adj \t@adj\space N end}}%
	\ifnum #2>\fourt@@n
		\ifnum#2=99\relax\else
			\ifdangermessages\message{Danger! Tie with ID >14.}\fi\fi
	\else\get@bits{#2}\pst@rega
		\ifcase\n@i\else\message{Tie #2 was re-opened.}%
			\global\advance\pst@rega-\n@i\fi
		\global\advance\pst@rega\n@ii
		\get@bits{#2}\pst@regb\global\advance\pst@regb-\n@i  % clear the bits
		\ifnum\d@sh=1\relax\global\advance\pst@regb\n@ii\fi
		\ifnum#1=1\relax\else\multiply\n@ii\tw@\global\advance\pst@regb\n@ii\fi
		\ifnum#2<\max@slurs\s@l@tie{#2}%
			\global\th@salt\altportee\global\th@shgt\y@i\fi
	\fi\gdef\d@sh{\D@sh}\gdef\t@adj{\T@adj}}

% ID, hoffset
\def\tTie#1#2{\n@ii\@ne
	\ifnum#1<\fiv@t@@n
		\get@bits{#1}\pst@rega
		\global\advance\pst@rega-\n@i
		\ifcase\n@i\message{Not opened tie #1 was closed.}\n@ii\z@
		\else
			\ifnum\n@vi>\@ne             % the tie was breaked
				\get@bits{#1}\pst@regb    % it's not necessary to clear this register
				\ifnum\n@vi>\@ne\@upfalse\else\@uptrue\fi
				\ifodd\n@vi\@dashtrue\else\@dashfalse\fi
				\y@ii=\lin@pos\advance\y@ii\locx@skip\advance\y@ii#2\internote
				\advance\y@ii\sl@space
				\ifnum#1<\max@slurs\s@l@tie{#1}%
					\ifdim\y@ii<\h@tielim
						\raise\th@shgt\rlap{\hskip-0.3\internote
							\ifdim\internote>.9\Internote\n@i=\z@\else
							\ifdim\internote<.7\Internote \n@i=\@cxxviii
							\else\n@i=64\fi\fi\if@up\else\advance\n@i\@ne\fi
							\mxsk\char\n@i}%
					\else
						\rlap{\hskip #2\internote\special{ps: slur@Dict begin
						(\the\internote) GD (\the\th@shgt) GD DP
						CP S (\the\y@ii) GDAR SB S (\the\th@shgt)
						GDAR SB DP CP P S \if@up \pstiehgt\space
						\else -\pstiehgt\space \fi \pstieangul\space \pstiemaxhgt\space
						\if@dash 1 \else 0 \fi \t@adj\space 40 DS end}}%
					\fi
				\else
					\rlap{\hskip #2\internote\special{ps: slur@Dict begin
					(\the\internote) GD T#1hgt GD DP
					CP S (\the\y@ii) GDAR SB S T#1hgt GDAR SB DP CP P S
					\if@up -\pstiehgt\space \else \pstiehgt\space \fi
					\pstieangul\space \pstiemaxhgt\space \if@dash 1 \else 0 \fi
					\t@adj\space 40 DS end}}%
				\fi
				\global\n@ii=\z@
			\fi
		\fi
	\fi
	\ifcase\n@ii\else
		\rlap{\hskip#2\internote
		\special{ps: slur@Dict begin (\the\internote) GD T#1hgt DP
				T#1x1 T#1y1 CP P T#1y1
				\pstiehgt\space T#1dir M \pstieangul\space \pstiemaxhgt\space T#1dash T#1adj 40 DS end}}%
	\fi}


%%%%%%%  simple slurs  %%%%%%%%%%%%%%%%%

% #1 start pitch #2 final pitch #3 sense(u,d) #4 length[noteskip]

% \s@Y = vert. offset, \y@ii = horiz. offset at the end, \y@iii add to length
% \y@v = length

\def\C@sl#1#2#3#4{\inhgetn@i#1\relax \n@ii\n@i\inhgetn@i#2\relax
	\if d#3 \s@vy-\s@vy \ifx\@ne\nxt \y@ii\z@\fi \fi
	\y@v#4\noteskip\advance\y@v\y@iii\advance\y@ii-\y@iii
	\rlap{\hskip\y@ii\y@i=\n@ii\internote\advance\y@i\s@vy
		\y@ii=\n@i\internote\advance\y@ii\s@vy\setbox\z@=\hbox{\s@tx}%
		\ifx\s@tx\empty\def\tx@sp{1}\else\ifdim\y@v<\tw@nty\internote\def\tx@sp{1.5}%
			\else\ifdim\y@v<\@l\internote\def\tx@sp{1.2}\else\def\tx@sp{1}\fi\fi\fi
		\special{ps: slur@Dict begin (\the\internote) GD
			(\the\y@i) GD (\the\y@ii) GD CP (\the\y@i) GDAR swub
			CP exch (\the\y@v) GDAR add exch (\the\y@ii) GDAR sub
			\if d#3 -1 \else 1 \fi \tx@sp\space mul \psslurangul\space
			\psslurmaxhgt\space \d@sh\space \s@adj\space 90 DS
			/ax1 ax1 (\the\y@v) GDAR add def /ay1 ay1 (\the\y@ii) GDAR sub
			(\the\ht\z@) GDAR dir mul add def}\lr@lap{\unhbox\z@}\special{ps: end}}%
	\@clr\s@tx\gdef\d@sh{\D@sh}\gdef\s@adj{\S@adj}}


%%%%%%%  breaking the line  %%%%%%%%%%%%%%%%%

\def\br@slur{\divide\n@v\tw@
	\ifodd\n@v        % the slur was breaked again
		\y@i=\lin@pos\advance\y@i\x@skip\advance\y@i-\tw@\p@
		\advance\y@i\sl@space
		\ifnum\n@iv<\max@slurs\s@l@slur\n@iv\n@i\th@slift\divide\n@i65536%
			\n@ii\n@i\advance\n@ii-16384\y@iii=\n@ii\internote
			\multiply\n@i65536 \n@ii\th@slift\advance\n@ii-\n@i
			\advance\n@ii-16384\y@ii=\n@ii\internote\advance\y@ii\th@shgt
			\ifnum\n@vi>\@ne\@upfalse\else\@uptrue\fi
			\raise\th@salt\rlap{\hskip -2pt\special
				{ps: slur@Dict begin /in (\the\internote) GD N
				in (\the\y@ii) GD (\the\y@iii) GD	CP S (\the\y@i) GD
				4 AR M SB S (\the\y@ii) GDAR SB
				CP (\the\y@iii) GDAR SB \if@up1\else-1\fi\space
				0.13 30 \ifodd\n@vi 1 \else 0 \fi \s@adj\space 90 DS end}}%
		\else
			\rlap{\hskip -2pt\special{ps: slur@Dict begin
				/in (\the\internote) GD N in S\the\n@iv hgt
				S\the\n@iv lift in M A S\the\n@iv brk
				CP S (\the\y@i) GDAR SB in 10 AR M A S
				S\the\n@iv hgt S\the\n@iv lift in M A 4 AR M SB
				S\the\n@iv alt 4 AR M SB CP S\the\n@iv brk 
				S\the\n@iv alt A 4 AR M SB S\the\n@iv dir
				0.13 30 S\the\n@iv dash S\the\n@iv adj 90 DS end}}%
		\fi
	\else               % the slur was breaked the first time
		\rlap{\hskip-2pt\special{ps: slur@Dict begin (\the\internote) GD
			S\the\n@iv hgt S\the\n@iv brk S\the\n@iv x1 S\the\n@iv y1 CP P
			S\the\n@iv y1 S\the\n@iv hgt S\the\n@iv brk SB 4 AR M A
			S\the\n@iv dir 0.17 \psslurmaxhgt\space S\the\n@iv dash S\the\n@iv adj 90 DS end}}%
		\get@bits\n@iv\pss@rega
		\multiply\n@ii\tw@\global\advance\pss@rega\n@ii\fi}

		
\def\br@cresc{\divide\n@v\tw@
	\ifodd\n@v        % the crescendo was breaked again
		\y@i=\lin@pos\advance\y@i\x@skip\advance\y@i-\tw@\p@
		\advance\y@i\sl@space
		\ifnum\n@iv<\max@slurs\s@l@cresc\n@iv
			\rlap{\hskip-2pt\divide\n@vi\tw@\special{ps: slur@Dict begin CP S
				(\the\y@i) GDAR SB S (\the\th@salt) GDAR SB DP CP P S
				4 \ifodd\n@vi 2 \else 0 \fi R DHC end}}%
		\else
			\rlap{\hskip-2pt\divide\n@vi\tw@\special{ps: slur@Dict begin CP S
				(\the\y@i) GDAR SB (\the\internote) GD 10 AR M A S
				C\the\n@iv alt GDAR SB DP CP P S
				4 C\the\n@iv dir R DHC end}}\fi
	\else             % the crescendo was breaked the first time
		\rlap{\hskip-2pt\special{ps: slur@Dict begin
			C\the\n@iv x1 C\the\n@iv y1 DP CP P S
			4 C\the\n@iv dir R C\the\n@iv dir 0 eq
			{ DC } { DHC } ifelse end}}%
		\get@bits\n@iv\psc@rega
		\multiply\n@ii\tw@\global\advance\psc@rega\n@ii\fi}


\def\br@tie{\divide\n@v\tw@
	\ifodd\n@v        % the tie was breaked again
		\y@i=\lin@pos\advance\y@i\x@skip\advance\y@i-\tw@\p@
		\advance\y@i\sl@space
		\ifnum\n@iv<\max@slurs\s@l@tie\n@iv
			\ifnum\n@vi>\@ne\@upfalse\else\@uptrue\fi
			\y@ii\th@salt\advance\y@ii\th@shgt
			\raise\y@ii\rlap{\hskip-2pt\special{ps: slur@Dict begin 
				(\the\internote) GD (\the\th@shgt) GD DP CP S (\the\y@i) GDAR SB S
				CP \if@up \pstiehgt\space \else -\pstiehgt\space \fi \pstieangul\space
				\pstiemaxhgt\space \ifodd\n@vi 1 \else 0 \fi \t@adj\space 40 DS end}}%
		\else
			\rlap{\hskip-2pt\special{ps: slur@Dict begin 
				(\the\internote) GD T\the\n@iv hgt DP CP S (\the\y@i)
				GDAR SB (\the\internote) GD 10 AR M A S
				T\the\n@iv alt T\the\n@iv hgt A 4 AR M SB DP
				CP P S \pstiehgt\space T\the\n@iv dir M
				\pstieangul\space \pstiemaxhgt\space T\the\n@iv dash T\the\n@iv adj 40 DS end}}\fi
	\else        % the tie was breaked the first time
		\rlap{\hskip-2pt\special{ps: slur@Dict begin
			(\the\internote) GD T\the\n@iv hgt DP
			T\the\n@iv x1 T\the\n@iv y1 DP CP P S
			\pstiehgt\space T\the\n@iv dir M \pstieangul\space \pstiemaxhgt\space
			T\the\n@iv dash 40 T\the\n@iv adj DS end}}%
		\get@bits\n@iv\pst@rega
		\multiply\n@ii\tw@\global\advance\pst@rega\n@ii\fi}

\let\mxps@oldsuspend=\z@suspend

\def\z@suspend{%
% cut PostScript slurs
	{\n@iv\z@\n@v\pss@rega\n@vi\pss@regb
	\loop\ifnum\n@v>\z@
		\ifodd\n@v{\br@slur}\fi\divide\n@v\f@ur
		\advance\n@iv\@ne\divide\n@vi\f@ur
	\repeat}%
% cut PostScript [de]crescendos
	{\n@iv=\z@\n@v=\psc@rega\n@vi\psc@regb
	\loop\ifnum\n@v>\z@
		\ifodd\n@v{\br@cresc}\fi\divide\n@v\f@ur
		\advance\n@iv\@ne\divide\n@vi\f@ur
	\repeat}%
% cut PostScript ties
	{\n@iv=\z@\n@v=\pst@rega\n@vi\pst@regb
	\loop\ifnum\n@v>\z@
		\ifodd\n@v{\br@tie}\fi\divide\n@v\f@ur
		\advance\n@iv\@ne\divide\n@vi\f@ur
	\repeat}%
	\special{ps: slur@Dict begin /pss@rega \the\pss@rega\space N
		/psc@rega \the\psc@rega\space N /pst@rega \the\pst@rega\space N end }%
% do all other mxtex stuff
	\mxps@oldsuspend}

% checking for lost slurs

\def\checkslurs{
	\ifcase\pss@rega\else\message{Warning! Opened slur(s) at checkpoint!}\fi
	\ifcase\psc@rega\else\message{Warning! Opened crescendo(s) at checkpoint!}\fi
	\ifcase\pst@rega\else\message{Warning! Opened tie(s) at checkpoint!}\fi}


%%%%%%  common abbreviations  %%%%%%%%%%%%%%%%%%%%

\def\isluru#1#2{\iSlur{#1}{#2}{\psslurraise}{\psislurskip}}
\def\islurd#1#2{\iSlur{#1}{#2}{-\psslurraise}{\psislurskip}}
\def\irsluru#1#2{\iSlur{#1}{#2}{\psslurraise}{\psrslurskip}}
\def\irslurd#1#2{\iSlur{#1}{#2}{-\psslurraise}{\psrslurskip}}
\def\ilsluru#1#2{\iSlur{#1}{#2}{\psslurraise}{\pslslurskip}}
\def\ilslurd#1#2{\iSlur{#1}{#2}{-\psslurraise}{\pslslurskip}}
\def\iLsluru#1#2{\iSlur{#1}{#2}{\psslurraise}{\psLslurskip}}
\def\iLslurd#1#2{\iSlur{#1}{#2}{-\psslurraise}{\psLslurskip}}

\def\tfslur#1#2{\tSlur{#1}{#2}{\psslurraise}{\pstslurskip}{\psfslurhgt}{\psslurangul}}
\def\tslur#1#2{\tSlur{#1}{#2}{\psslurraise}{\pstslurskip}{\psslurhgt}{\psslurangul}}
\def\thslur#1#2{\tSlur{#1}{#2}{\psslurraise}{\pstslurskip}{\pshslurhgt}{\psslurangul}}
\def\tHslur#1#2{\tSlur{#1}{#2}{\psslurraise}{\pstslurskip}{\psHslurhgt}{\psHslurangul}}
\def\tHHslur#1#2{\tSlur{#1}{#2}{\psslurraise}{\pstslurskip}{\psHHslurhgt}{\psHHslurangul}}

\def\trfslur#1#2{\tSlur{#1}{#2}{\psslurraise}{\psrslurskip}{\psfslurhgt}{\psslurangul}}
\def\trslur#1#2{\tSlur{#1}{#2}{\psslurraise}{\psrslurskip}{\psslurhgt}{\psslurangul}}
\def\trhslur#1#2{\tSlur{#1}{#2}{\psslurraise}{\psrslurskip}{\pshslurhgt}{\psslurangul}}
\def\trHslur#1#2{\tSlur{#1}{#2}{\psslurraise}{\psrslurskip}{\psHslurhgt}{\psHslurangul}}
\def\trHHslur#1#2{\tSlur{#1}{#2}{\psslurraise}{\psrslurskip}{\psHHslurhgt}{\psHHslurangul}}

\def\tlfslur#1#2{\tSlur{#1}{#2}{\psslurraise}{\pslslurskip}{\psfslurhgt}{\psslurangul}}
\def\tlslur#1#2{\tSlur{#1}{#2}{\psslurraise}{\pslslurskip}{\psslurhgt}{\psslurangul}}
\def\tlhslur#1#2{\tSlur{#1}{#2}{\psslurraise}{\pslslurskip}{\pshslurhgt}{\psslurangul}}
\def\tlHslur#1#2{\tSlur{#1}{#2}{\psslurraise}{\pslslurskip}{\psHslurhgt}{\psHslurangul}}
\def\tlHHslur#1#2{\tSlur{#1}{#2}{\psslurraise}{\pslslurskip}{\psHHslurhgt}{\psHHslurangul}}

\def\tLfslur#1#2{\tSlur{#1}{#2}{\psslurraise}{\psLslurskip}{\psfslurhgt}{\psslurangul}}
\def\tLslur#1#2{\tSlur{#1}{#2}{\psslurraise}{\psLslurskip}{\psslurhgt}{\psslurangul}}
\def\tLhslur#1#2{\tSlur{#1}{#2}{\psslurraise}{\psLslurskip}{\pshslurhgt}{\psslurangul}}
\def\tLHslur#1#2{\tSlur{#1}{#2}{\psslurraise}{\psLslurskip}{\psHslurhgt}{\psHslurangul}}
\def\tLHHslur#1#2{\tSlur{#1}{#2}{\psslurraise}{\psLslurskip}{\psHHslurhgt}{\psHHslurangul}}

% slur ID, beam ID
\def\iBsluru#1#2{\gt@pssb{#2}{\tw@\qn@width}\I@slur{#1}{\psbslurraise}{2.4}}
\def\iBslurd#1#2{\gt@pssb{#2}\z@\I@slur{#1}{-\psbslurraise}0}
\def\tBfslur{\t@bslur\psfslurhgt\psslurangul}
\def\tBslur{\t@bslur\psslurhgt\psslurangul}
\def\tBhslur{\t@bslur\pshslurhgt\psslurangul}
\def\tBHslur{\t@bslur\psHslurhgt\psHslurangul}
\def\tBhHslur{\t@bslur\psHHslurhgt\psHHslurangul}


\def\itieu#1#2{\iTieu{#1}{#2}{\psitieskip}}
\def\itied#1#2{\iTied{#1}{#2}{\psitieskip}}
\def\iltieu#1#2{\iTieu{#1}{#2}{\pslslurskip}}
\def\iltied#1#2{\iTied{#1}{#2}{\pslslurskip}}
\def\irtieu#1#2{\iTieu{#1}{#2}{\psrslurskip}}
\def\irtied#1#2{\iTied{#1}{#2}{\psrslurskip}}
\def\ttie#1{\tTie{#1}{\psttieskip}}
\def\tltie#1{\tTie{#1}{\psltieskip}}
\def\trtie#1{\tTie{#1}{\psrtieskip}}

\def\icresc#1{\Icresc{#1}{\d@nh}{0}}
\def\idecresc#1{\Idecresc{#1}{\d@nh}{0}}
\def\tcresc#1{\Tcresc{#1}{\d@nh}{2}}
\let\tdecresc\tcresc

\def\ilcresc#1{\Icresc{#1}{\d@nh}{-4}}
\def\ildecresc#1{\Idecresc{#1}{\d@nh}{-4}}
\def\tlcresc#1{\Tcresc{#1}{\d@nh}{-2}}
\let\tldecresc\tlcresc

\def\ircresc#1{\Icresc{#1}{\d@nh}{4}}
\def\irdecresc#1{\Idecresc{#1}{\d@nh}{4}}
\def\trcresc#1{\Tcresc{#1}{\d@nh}{-6}}
\let\trdecresc\trcresc

% for compatibility with old MusiXTeX slurs ...
\let\Tslur\tslur
\let\Ttie\ttie
\let\issluru\irsluru
\let\isslurd\irslurd
\let\tsslur\tLslur
\def\ibsluru#1#2{\iSlur{#1}{#2}8{\psrslurskip}}
\def\ibslurd#1#2{\iSlur{#1}{#2}{-8}{\psrslurskip}}
\def\tubslur#1#2{\tSlur{#1}{#2}8{\psrslurskip}{\psslurhgt}{\psslurangul}}
\def\tdbslur#1#2{\tSlur{#1}{#2}8{\pslslurskip}{\psslurhgt}{\psslurangul}}
\let\tbsluru=\tubslur
\let\tbslurd=\tdbslur
\let\istieu\irtieu
\let\istied\irtied
\let\itenu\irtieu
\let\itenl\irtied
\let\tstie\tltie
\let\tten\tltie
\let\Itenu\istieu
\let\Itenl\istied
\let\Liftslur\liftslur
% p1 p2 sense len
\def\slur#1#2#3#4{\csname islur#3\endcsname{99}{#1}%
	\rlap{\hskip#4\noteskip\tslur{99}{#2}}}
\def\tie#1#2#3{\csname itie#2\endcsname{99}{#1}%
	\rlap{\hskip#3\noteskip\ttie{99}}}

\catcode`\@=\catcodeat

\endinput

%%%%%%%%%%%%%%%%%%%%%%%%%%%%%%%%%%%%%%%%%%%%%%%%%%%%%


USAGE:
~~~~~~
Basic usage is:

\isluru0g - starts upper slur with ID 0 above the note 'g'
\tslur0i - terminates slur with ID 0. It will end above or below
           the note 'i' with respect to the slur direction

You can shift the starting or ending point slightly to the left
or right using \ilsluru, \ilslurd, \irsluru, \irslurd, \trslur and \tlslur.

And finally you can make the slur a bit more flat than default
( \tfslur0f ), higher ( \thslur0f ), more higher ( \tHslur0f ) or
even more higher ( \tHHslur0f ). All combinations with shifting are
allowed (see the abbreviations above). These commands act like
\midslur for standard slurs.

If this is not sufficient, you can use the \iSlur and \tSlur and adjust the
parameter as you want (see comments above their definitions) or even change
the default parameters.

NOTE 1: The identification parameter should be a number greater
        or equal to 0 and lower than 10. You can use even higher
        numbers, but slurs, ties and crescendos with ID >10
        can not go over page break but can go across line break.
        Slurs, ties and crescendos with ID >14 can't be breaked at all
        (these slurs also can not be exactly positioned at a beam, so
        don't use for example \tbslur{16}0; \ibslur{16}0 is safe).
        If you know that a line break can not occur inside the slur
        (probably inside a bar) you can happily use numbers
        up to 2^31-1, which is also the maximal number of
        simultaneously opened slurs :-). Such slurs will be also
        a bit faster...

NOTE 2: You can start a slur, tie and crescendo with the same ID,
		  the ID's are independent. This is not true for crescendos
        and decrescendos. That means you can say \isluru0f\icresc0,
        but not \icresc0\idecresc0, in which case you'll get a warning
        message and the first crescendo will be forgotten.

There are also macros for slurs begining or ending at a beam:

\iBsluru{slur ID}{beam ID}
\iBslurd{slur ID}{beam ID}
\tBslur{slur ID}{beam ID}

They are NOT THE SAME as for the original slurs, but a bit smarter...

for example:

\Notes\ibu0i0\iBsluru00\qb0{eh}\tbu0\qb0i\ibu0j0\qb0{jl}\tbu0%
  \setslurtext{6}\tBslur00\qb0e\en

The \slurtext works for PostScript slurs exactly the same way as
for the bitmapped ones. When you say '\slurtext3', a number '3' will
be typeset in the middle of the next closed slur.
The position of the reference point of the box with text will be
exactly in the middle horizontally and adjusted according to the
height of the box in case of upper slur.

This works only for non-breaked slurs, if the slur is breaked,
the text disappears.

The placing of the slur text is done with a very dirty PostScript
hack, so I am not really sure that everything you want to typeset
will be placed at the correct position (if you are interested, see
the end of psslurs.pro for details). If you find something that won't
work, let me know.

There is also an unpleasant side-effect: when you look at the
DVI-file with some standard (non-PostScript) driver, the figure will
be placed at the end of the slur instead of the middle. From obvious
reasons this can't be fixed --- TeX can't know where the middle is,
as the slur shape is defined by a PostScript program.


General coding is very simple:

\i[h.shift]slur[u|d]{slur ID}{note height}
\t[h.shift][slur height]slur{slur ID}{note height}

\iBslur[u|d]{slur ID}{beam ID}
\tB[slur height]slur{slur ID}{beam ID}

where h. shift can be 'l', 'r' or nothing
and slur height can be 'f', nothing, 'h', 'H' or 'HH'

Mnemonics: \tlfslur means 'terminate left flat slur'.


There are also simple slurs with same invocation and parameters as the
original ones.


Similarly the ties:
~~~~~~~~~~~~~~~~~~~
\itieu0e
\ttie0

\i[h.shift]tie[u|d]{ID}{note height}
\t[h.shift]tie{ID}


And crescendos:
~~~~~~~~~~~~~~~
\icresc0
\tcresc0   draws a crescendo at a default position (defined with
           \setcrescheight)

NOTE 1: \tcresc is the same as \tdecresc.

NOTE 2: The coding of crescendos is incompatible with original crescendos.
        The difference (as you notices) is in defining the type
        (crescendo/decrescendo) at the beginning instead of the end.
        This can't be fixed, because the TeX must know this at the end
        of the line when the crescendo is breaked.


You can shift the starting and ending point horizontally with

\ilcresc    \ildecresc
\ircresc    \irdecresc
\tlcresc    \tldecresc
\trcresc    \trdecresc

to get some space for \mf, \ff etc.


Furthermore, you can say:

\Icresc0{-6}0
\Tcresc0{-8}2

and you will get crescendo with ending point lower than the begining
one. (The '0' and '2' at the end is horizontal offset in \internote
from the current point.) Due to some implemenational reasons, you must
use numbered height when specifying the altitude of the crescendos.



Dotted slurs and ties
~~~~~~~~~~~~~~~~~~~~~
The slurs and ties can be dotted, when you say \dotted anywhere before
the begining of the slur or tie. The first slur or tie following this
command will then be drawed with dotted line.

[dangerous bend]  When you say \dotted, then ALL slurs and ties
                  from this point will be dotted until you say
\Solid. Furthermore, inside the \dotted...\Solid you can make some slur or
tie solid saying \solid before its begining.

NOTE: These macros are taken into account only at the begining
      of the slur or tie.


Collisions of slurs/ties and staff lines
~~~~~~~~~~~~~~~~~~~~~~~~~~~~~~~~~~~~~~~~
In PostScript it is possible to do some computations, which are very hard
to implement in TeX. One of neat features of the PS slurs and ties is
checking, if the curve passes close the staff line and if so, adjusting
the altitude of the curve to avoid collision ("black triangles" or narrow
white space). You can turn this feature on globally (\Sluradjust,
\Tieadjust) or locally (\sluradjust, \tieadjust), or disable it either
globally (\Nosluradjust, \Notieadjust) or locally (\nosluradjust,
\notieadjust). Default is \Sluradjust and \Notieadjust. "Locally"
means before the next opening command of the slur or tie.


Line breaking:
~~~~~~~~~~~~~~
The slurs, ties and crescendos going across the line breaks are handled
automatically. Of course they can go over more lines than two (this is
true also for ties, though it would be somewhat strange).

There is the classic \breakslur implemented, which sets the height of the
ending point slur at the end of a line to given number of \internotes.

There is also a switch, \ifslopebrkslurs, indicating whether the ending point
of the first segment of the breaked slur (at the end of the line) should
be automatically raised by 3 \internote. You can turn this feature on by
the \slopebrkslurstrue and off by \slopebrkslursfalse commands, default
is false.

You can raise or lower the starting point of the slur at the begining of
the next line with the command \liftslur with ID and the relative offset in
\internotes measured from the slur beginning height. Its usage and behavior
is the same as of the original \liftslur, only it is not necessary to code
it in \atnextstaff{}, just anywhere inside the slur.

Similar is \liftcresc, but it works only for the middle part of crescendos
broken into three parts. The height of the last part of a broken crescendo
is defined with the height parameter of \Tcresc.

As said before, anything with ID <10 is breaked fully automatically, but
you should be careful about the slurs, ties and crescendos with 10 <= ID < 15.
These can not go acress PAGE break, otherwise you'll get a PostScript error
like 's11hgt undefined'. Of course the LINE breaking works for them
automatically too.

When a tie is continuead after breaking at the very beginning of a line,
only a half of the tie is drawn (the 'very beginning' means that the tie
would be less than 15pt long). To forbid this feature, say \nohalfties
(default is \halfties). A font called mxsk is necessary for \halfties.


Backwards compatibility
~~~~~~~~~~~~~~~~~~~~~~~

There are several "aliases" which allow to use the old, bitmapped slur
commands for PS slurs without any change. There are however a few differences:
- the \invertslur is not implemented yet
- the \curve and \midslur macros have no effect




Memory requrements
~~~~~~~~~~~~~~~~~~

I have tried to make the macros as small and fast as possible. The problem
about them is not in their size, but in their output. Each \i... and \t...
produces a \special command, which must be stored into the TeX's main memory.
Therefore, if too many slur occurs in one page, some memory troubles could
occur. In that case either use BigTeX (but I suppose you already use it) or
use the bitmapped slurs. I will try to reduce the memory requirements, but
do not expect any miracles. There is a lot of things to be passed to the
PostScript part...



About the PS header:
~~~~~~~~~~~~~~~~~~~~
The slurs need the PS header file psslurs.pro to be included into the
output PostScript file. You have to say this to dvips with command

\special{header=psslurs.pro}

placed anywhere inside the piece. As default, this is done when \inputting
the MusiXpss, but if you include the MusiXpss.tex into a format, the first
page of your document will be always empty. If you want the macros included
in your format file, comment the line 13 of this file and don't forget
to copy it into your source file.

Warning to dvidvi users: If you select a set of pages from your document,
don't forget to include the page in which the \special command was used
(default is first page). Otherwise the dvips won't find this command and
won't include the header file, putting the PostScript device into a serious
problems. You can also use the "-h psslurs.pro" option to tell dvips to add
the header file manually.


About the slur shape:
~~~~~~~~~~~~~~~~~~~~~
The shape is defined via trapezium:

       c
    <---->               b
    <----------------------------------------->
     _ _ _________________________________ _ _
    |    /^1 _____/  ^2  ^3  ^4\_____  5^\    | ^
        / __/                        \___ \     |
    |  /_/  \                            \_\  | |
      //     this is the 'slur'            \\   | a
    |/                                      \\| |
    |_________________________________________| v
    0                                         6

Points 0 and 6 are the starting and ending points. When the slope is not
zero, the trapezium is simply rotated, NOT slanted.

'a' is computed as sqrt(b)*2, bounded with <3,\psslurmaxhgt> and multiplied
by the given height coeficient (see \tSlur and \t..slur, defaults to 1)
The units are not equal to anything common, the values were found
after a lot of testing.

'c' is computed as angularity*b. Default angularity is 0.17, reasonable
values should lie in <0.1, 0.3>.

Then the shape is corrected when the slur is so steep that its extreme
in horizontal direction lies inside the slur and shifted up or down
if it's top would be too close to nearest staff line (!!!!!!!!).

The slur then consists of two bezier curves:
from 0 to 3 with control points 1 and 2 and
from 3 to 6 with control points 4 and 5.

The shading is made according to its length and current \internote.



If you are interested, see a slightly commented PS code in psslurs.pro.

Bug reports and complainments are welcomed at:

	Stanislav Kneifl,

	standa@hiero.cz



HISTORY
~~~~~~~

0.92 (12.5.2002)
      added: \nosluradjust, \sluradjust, \notieadjust, \tieadjust
      added: \slopebrkslurtrue, \slopebrkslurfalse
      fixed: shaded slurs now do not break up to separate curves
             at high resolutions

0.91 (10.1.2002)
     First public release, adopted from OpusTeX version.
